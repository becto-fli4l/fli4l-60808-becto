% Last Update: $Id$

\section{HWSUPP - Geräteabhängige Einstellungen}

\marklabel{sec:leddevices}{
\subsection{Verfügbare LED-Devices}}
  Je nach \var{HWSUPP\_TYPE} sind verschiedene LED-Devices verfügbar.
  Bei nicht aufgeführter Hardware sind die PC-Tastatur LEDs wie bei  
  \smalljump{sec:leddevices_keyboard}{generic-pc} verfügbar.

  Zusätzlich LED-Devices können z.B. auf WLAN-Karten verfügbar sein.
  Die gültigen Namen der LED-Devices ermittelt man mittels Eingabe von
  \var{ls /sys/class/leds/} z.B. per ssh auf der Router-Console.
 

\subsubsection{sim}
LED Simulation, erzeugt Eintrag im \var{syslog}:
\begin{itemize}
  \item \var{simu::1}
  \item ...
  \item \var{simu::8}
\end{itemize}

\marklabel{sec:leddevices_keyboard}{
\subsubsection{generic-pc}}
PC-Tastatur LEDs:
\begin{itemize}
  \item \var{inputX::capslock}
  \item \var{inputX::numlock}
  \item \var{inputX::scrolllock}
\end{itemize}

\subsubsection{generic-acpi}
PC-Tastatur-LEDs, wie \smalljump{sec:leddevices_keyboard}{generic-pc}

\subsubsection{generic-acpi-coretemp}
PC-Tastatur-LEDs, wie \smalljump{sec:leddevices_keyboard}{generic-pc}

\subsubsection{pcengines-alix}
\begin{itemize}
  \item \var{alix::1}
  \item \var{alix::2}
  \item \var{alix::3}
\end{itemize}

\subsubsection{pcengines-apu}
\begin{itemize}
  \item \var{apu::1}
  \item \var{apu::2}
  \item \var{apu::3}
\end{itemize}

\subsubsection{pcengines-apu2}
\begin{itemize}
  \item \var{apu::1}
  \item \var{apu::2}
  \item \var{apu::3}
\end{itemize}

\subsubsection{pcengines-wrap} 
\begin{itemize}
  \item \var{wrap::1}
  \item \var{wrap::2}
  \item \var{wrap::3}
\end{itemize}

\subsubsection{rpi}
\begin{itemize}
  \item \var{led0}
\end{itemize}

\subsubsection{soekris-net4801}
\begin{itemize}
  \item \var{net48xx::error}
\end{itemize}

\subsubsection{soekris-net5501}
\begin{itemize}
  \item \var{net5501::error}
\end{itemize}

\marklabel{sec:buttondevices}{
\subsection{Verfügbare Button-Devices}}
Je nach \var{HWSUPP\_TYPE} sind verschiedene GPIO-Devices für Taster vorbelegt.

\subsubsection{generic-pc und generic-acpi}
\begin{itemize}
  \item \var{evdev:isa0060/serio0/input0}, beliebige (verfügbare) Taste
\end{itemize}

Zu beachten ist, dass Tasten\emph{kombinationen} nicht möglich sind. Man kann
also den Router nicht via \textsc{Alt+F4} o.ä. herunterfahren.

\subsubsection{pcengines-alix}
\begin{itemize}
  \item \var{evdev:gpio-keys-polled/input0}, Taste 408 (Restart)
\end{itemize}

\subsubsection{pcengines-apu}
\begin{itemize}
  \item \var{evdev:gpio-keys-polled/input0}, Taste 408 (Restart)
\end{itemize}

\subsubsection{pcengines-apu2}
\begin{itemize}
  \item \var{evdev:gpio-keys-polled/input0}, Taste 408 (Restart)
\end{itemize}

\subsubsection{pcengines-wrap} 
\begin{itemize}
  \item \var{evdev:gpio-keys-polled/input0}, Taste 408 (Restart)
\end{itemize}

\subsubsection{soekris-net5501}
\begin{itemize}
  \item \var{evdev:gpio-keys-polled/input0}, Taste 408 (Restart)
  \newline Der Taster ist am Soekris-Gehäuse mit ``Reset'' beschriftet. 
  \newline Achtung: Der Taster muss im BIOS freigeschaltet werden.
\end{itemize}

\marklabel{sec:notes}{
\subsection{Hinweise zu spezieller Hardware}}

\subsubsection{pcengines-alix}
Beim Alix führt ein fehlerhafter Treiber für den lm90 Temperatursensor nach
einiger Zeit zum Ausfall der Temperaturanzeige. 

Als Workaraound wird der lm90 Treiber entladen und wieder neu geladen.
Dies geschieht automatisch per cron-Job. Dazu muss das Paket easycron geladen
werden (\var{OPT\_EASYCRON='yes'}).

\marklabel{sec:examples}{
\section{HWSUPP - Konfigurations-Beispiele}}

\subsection{generic-pc}

\begin{verbatim}
OPT_HWSUPP='yes'
HWSUPP_TYPE='generic-pc'

HWSUPP_WATCHDOG='no'
HWSUPP_CPUFREQ='no'

HWSUPP_LED_N='3'
HWSUPP_LED_1='ready'
HWSUPP_LED_1_DEVICE='input1::numlock'
HWSUPP_LED_2='online'
HWSUPP_LED_2_DEVICE='input1::capslock'
HWSUPP_LED_3='wlan'
HWSUPP_LED_3_DEVICE='input1::scrolllock'
HWSUPP_LED_3_WLAN='wlan0'

HWSUPP_BUTTON_N='0'

\end{verbatim}

\subsection{pcengines-apu}

\begin{verbatim}
OPT_HWSUPP='yes'
HWSUPP_TYPE='pcengines-apu'

HWSUPP_WATCHDOG='yes'
HWSUPP_CPUFREQ='yes'
HWSUPP_CPUFREQ_GOVERNOR='ondemand'

HWSUPP_LED_N='3'
HWSUPP_LED_1='ready'
HWSUPP_LED_1_DEVICE='apu::1'
HWSUPP_LED_2='wlan'
HWSUPP_LED_2_DEVICE='apu::2'
HWSUPP_LED_2_WLAN='wlan0'
HWSUPP_LED_3='online'
HWSUPP_LED_3_DEVICE='apu::3'

HWSUPP_BUTTON_N='1'
HWSUPP_BUTTON_1='wlan'
HWSUPP_BUTTON_1_DEVICE='evdev:gpio-keys-polled/input0'
HWSUPP_BUTTON_1_DEVICE_KEY='408'
HWSUPP_BUTTON_1_PARAM='wlan0'
\end{verbatim}

\subsection{pcengines-apu mit GPIOs}

\begin{verbatim}
OPT_HWSUPP='yes'
HWSUPP_TYPE='pcengines-apu'

HWSUPP_WATCHDOG='yes'
HWSUPP_CPUFREQ='yes'
HWSUPP_CPUFREQ_GOVERNOR='ondemand'

HWSUPP_LED_N='5'
HWSUPP_LED_1='ready'
HWSUPP_LED_1_DEVICE='apu::1'
HWSUPP_LED_2='wlan'
HWSUPP_LED_2_DEVICE='apu::2'
HWSUPP_LED_2_WLAN='wlan0'
HWSUPP_LED_3='online'
HWSUPP_LED_3_DEVICE='apu::3'
HWSUPP_LED_4='trigger'
HWSUPP_LED_4_PARAM='phy0rx'
HWSUPP_LED_4_DEVICE='gpio::237'
HWSUPP_LED_5='trigger'
HWSUPP_LED_5_PARAM='phy0tx'
HWSUPP_LED_5_DEVICE='gpio::245'

HWSUPP_BUTTON_N='2'
HWSUPP_BUTTON_1='wlan'
HWSUPP_BUTTON_1_DEVICE='evdev:gpio-keys-polled/input0'
HWSUPP_BUTTON_1_DEVICE_KEY='408'
HWSUPP_BUTTON_1_PARAM='wlan0'
HWSUPP_BUTTON_2='online'
HWSUPP_BUTTON_2_DEVICE='gpio::236' # für Linux-Kernel < 3.18
HWSUPP_BUTTON_2_DEVICE='gpio::492' # für Linux-Kernel >= 3.18 (x+256)
\end{verbatim}

\marklabel{sec:blink}{
\section{HWSUPP - Blinkfolge}}
  Die folgenden Blinkfolgen werden während des Bootvorgangs angezeigt:
  
  {\def\led{\begin{math}\otimes\quad\quad\end{math}}
  \begin{tabular}{c l l l l l l l l l}
    \quad\\
    1. & \led & & & & \led & & & & ... \\ 
    2. & \led & \led & & & \led & \led & & & ... \\ 
    3. & \led & \led & \led & & \led & \led & \led & & ... \\ 
    4. & \led & \led & \led & \led & \led & \led & \led & \led & ... \\ 
    \quad 
  \end{tabular}}

  Während der Abarbeitung von  \var{rc002.*} bis  \var{rc250.*}
  wird die erste Folge angezeigt\newline(1 * Blinken - Pause),\newline
  von \var{rc250.*} bis \var{rc500.*} die zweite (2 * Blinken - Pause),\newline
  von \var{rc500.*} bis \var{rc750.*} die 3. und \newline
  von \var{rc750.*} bis zum Ende des Bootvorgangs die 4. Folge (Dauerblinken).
    
\marklabel{sec:hwsupp:evdev-keys}{
\section{HWSUPP - Tasten-Codes}}

Die folgende Tabelle stellt die verschiedenen Tasten-Codes zusammen mit der
jeweiligen Taste bei der deutschen, englischen und französischen Tastenbelegung
dar. Die Abkürzung ``NB'' steht für ``Nummernblock'' und bezeichnet eine Taste
im rechten abgesetzten Bereich der Tastatur. Ein Eintrag ``---'' bezeichnet eine
Taste, die in der jeweiligen Belegung nicht existiert. Tasten-Codes, die bei
allen drei dargestellten Tastaturbelegungen gar nicht vorkommen (dies gilt
z.\,B. für diverse Tasten japanischer oder chinesischer Tastaturen), wurden
gleich komplett weggelassen.

Zuerst folgt eine Tabelle mit den Standard-Tasten der IBM AT-Tastaturen mit 101
bzw. 102 Tasten.

\newcommand\twoheaduparrow{\mathrel{\rotatebox[origin=c]{90}{$\twoheadrightarrow$}}}
\newcommand\twoheaddownarrow{\mathrel{\rotatebox[origin=c]{270}{$\twoheadrightarrow$}}}

\begin{longtable}{|l|l|l|l|}
  \hline
  \textbf{Code} & \textbf{dt.} & \textbf{engl.} & \textbf{frz.} \\
  \endhead
  \hline
  1 & \texttt{Esc} & \texttt{Esc} & \texttt{Esc} \\ \hline
  2 & \texttt{1}   & \texttt{1}   & \texttt{\&} \\ \hline
  3 & \texttt{2}   & \texttt{2}   & \texttt{é} \\ \hline
  4 & \texttt{3}   & \texttt{3}   & \texttt{"} \\ \hline
  5 & \texttt{4}   & \texttt{4}   & \texttt{'} \\ \hline
  6 & \texttt{5}   & \texttt{5}   & \texttt{(} \\ \hline
  7 & \texttt{6}   & \texttt{6}   & \texttt{-} \\ \hline
  8 & \texttt{7}   & \texttt{7}   & \texttt{è} \\ \hline
  9 & \texttt{8}   & \texttt{8}   & \texttt{\_} \\ \hline
  10 & \texttt{9}   & \texttt{9}   & \texttt{ç} \\ \hline
  11 & \texttt{0}   & \texttt{0}   & \texttt{à} \\ \hline
  12 & \texttt{ß}   & \texttt{-}   & \texttt{)} \\ \hline
  13 & \texttt{?}   & \texttt{=}   & \texttt{=} \\ \hline
  14 & \multicolumn{3}{|l|}{Rückschritt $\longleftarrow$} \\ \hline
  15 & \multicolumn{3}{|l|}{Tabulator} \\ \hline
  16 & \texttt{Q} & \texttt{Q} & \texttt{A} \\ \hline
  17 & \texttt{W} & \texttt{W} & \texttt{Z} \\ \hline
  18 & \texttt{E} & \texttt{E} & \texttt{E} \\ \hline
  19 & \texttt{R} & \texttt{R} & \texttt{R} \\ \hline
  20 & \texttt{T} & \texttt{T} & \texttt{T} \\ \hline
  21 & \texttt{Z} & \texttt{Y} & \texttt{Y} \\ \hline
  22 & \texttt{U} & \texttt{U} & \texttt{U} \\ \hline
  23 & \texttt{I} & \texttt{I} & \texttt{I} \\ \hline
  24 & \texttt{O} & \texttt{O} & \texttt{O} \\ \hline
  25 & \texttt{P} & \texttt{P} & \texttt{P} \\ \hline
  26 & \texttt{Ü} & \texttt{[} & \texttt{\^} \\ \hline
  27 & \texttt{+} & \texttt{]} & \texttt{\$} \\ \hline
  28 & \multicolumn{3}{|l|}{Enter} \\ \hline
  29 & \multicolumn{3}{|l|}{linke Strg-Taste} \\ \hline
  30 & \texttt{A} & \texttt{A} & \texttt{Q} \\ \hline
  31 & \texttt{S} & \texttt{S} & \texttt{S} \\ \hline
  32 & \texttt{D} & \texttt{D} & \texttt{D} \\ \hline
  33 & \texttt{F} & \texttt{F} & \texttt{F} \\ \hline
  34 & \texttt{G} & \texttt{G} & \texttt{G} \\ \hline
  35 & \texttt{H} & \texttt{H} & \texttt{H} \\ \hline
  36 & \texttt{J} & \texttt{J} & \texttt{J} \\ \hline
  37 & \texttt{K} & \texttt{K} & \texttt{K} \\ \hline
  38 & \texttt{L} & \texttt{L} & \texttt{L} \\ \hline
  39 & \texttt{Ö} & \texttt{;} & \texttt{M} \\ \hline
  40 & \texttt{Ä} & \texttt{'} & \texttt{ù} \\ \hline
  41 & \texttt{\^} & \texttt{`} & $^2$ \\ \hline
  42 & \multicolumn{3}{|l|}{linke Umschalttaste} \\ \hline
  43 & \texttt{\#} & \texttt{\textbackslash} & \texttt{*} \\ \hline
  44 & \texttt{Z} & \texttt{Z} & \texttt{W} \\ \hline
  45 & \texttt{X} & \texttt{X} & \texttt{X} \\ \hline
  46 & \texttt{C} & \texttt{C} & \texttt{C} \\ \hline
  47 & \texttt{V} & \texttt{V} & \texttt{V} \\ \hline
  48 & \texttt{B} & \texttt{B} & \texttt{B} \\ \hline
  49 & \texttt{N} & \texttt{N} & \texttt{N} \\ \hline
  50 & \texttt{M} & \texttt{M} & \texttt{,} \\ \hline
  51 & \texttt{,} & \texttt{,} & \texttt{;} \\ \hline
  52 & \texttt{.} & \texttt{.} & \texttt{:} \\ \hline
  53 & \texttt{-} & \texttt{/} & \texttt{!} \\ \hline
  54 & \multicolumn{3}{|l|}{rechte Umschalttaste} \\ \hline
  55 & \texttt{*} (NB) & \texttt{*} (NB) & \texttt{*} (NB) \\ \hline
  56 & \multicolumn{3}{|l|}{linke Alt-Taste} \\ \hline
  57 & \multicolumn{3}{|l|}{Leertaste} \\ \hline
  58 & \multicolumn{3}{|l|}{Feststelltaste} \\ \hline
  59 & \texttt{F1} & \texttt{F1} & \texttt{F1} \\ \hline
  60 & \texttt{F2} & \texttt{F2} & \texttt{F2} \\ \hline
  61 & \texttt{F3} & \texttt{F3} & \texttt{F3} \\ \hline
  62 & \texttt{F4} & \texttt{F4} & \texttt{F4} \\ \hline
  63 & \texttt{F5} & \texttt{F5} & \texttt{F5} \\ \hline
  64 & \texttt{F6} & \texttt{F6} & \texttt{F6} \\ \hline
  65 & \texttt{F7} & \texttt{F7} & \texttt{F7} \\ \hline
  66 & \texttt{F8} & \texttt{F8} & \texttt{F8} \\ \hline
  67 & \texttt{F9} & \texttt{F9} & \texttt{F9} \\ \hline
  68 & \texttt{F10} & \texttt{F10} & \texttt{F10} \\ \hline
  69 & \texttt{Num $\Downarrow$} & \texttt{Num Lock} & \texttt{Verr num} \\ \hline
  70 & \texttt{Rollen $\Downarrow$} & \texttt{Scroll Lock} & \texttt{Arrêt Défil} \\ \hline
  71 & \texttt{7} (NB) & \texttt{7} (NB) & \texttt{7} (NB) \\ \hline
  72 & \texttt{8} (NB) & \texttt{8} (NB) & \texttt{8} (NB) \\ \hline
  73 & \texttt{9} (NB) & \texttt{9} (NB) & \texttt{9} (NB) \\ \hline
  74 & \texttt{-} (NB) & \texttt{-} (NB) & \texttt{-} (NB) \\ \hline
  75 & \texttt{4} (NB) & \texttt{4} (NB) & \texttt{4} (NB) \\ \hline
  76 & \texttt{5} (NB) & \texttt{5} (NB) & \texttt{5} (NB) \\ \hline
  77 & \texttt{6} (NB) & \texttt{6} (NB) & \texttt{6} (NB) \\ \hline
  78 & \texttt{+} (NB) & \texttt{+} (NB) & \texttt{+} (NB) \\ \hline
  79 & \texttt{1} (NB) & \texttt{1} (NB) & \texttt{1} (NB) \\ \hline
  80 & \texttt{2} (NB) & \texttt{2} (NB) & \texttt{2} (NB) \\ \hline
  81 & \texttt{3} (NB) & \texttt{3} (NB) & \texttt{3} (NB) \\ \hline
  82 & \texttt{0} (NB) & \texttt{0} (NB) & \texttt{0} (NB) \\ \hline
  83 & \texttt{,} (NB) & \texttt{.} (NB) & \texttt{.} (NB) \\ \hline
  86 & \texttt{<} & --- & \texttt{<} \\ \hline
  87 & \texttt{F11} & \texttt{F11} & \texttt{F11} \\ \hline
  88 & \texttt{F12} & \texttt{F12} & \texttt{F12} \\ \hline
  96 & \multicolumn{3}{|l|}{Enter (NB)} \\ \hline
  97 & \multicolumn{3}{|l|}{rechte Strg-Taste} \\ \hline
  98 & $\div$ (NB) & \texttt{/} (NB) & \texttt{/} (NB) \\ \hline
  99 & \texttt{Druck} & \texttt{Print Screen} & \texttt{Impr écran} \\ \hline
  100 & \multicolumn{3}{|l|}{rechte Alt-Taste (AltGr)} \\ \hline
  102 & \texttt{Pos 1} & \texttt{Home} & $\nwarrow$ \\ \hline
  103 & $\uparrow$ & $\uparrow$ & $\uparrow$ \\ \hline
  104 & \texttt{Bild} $\uparrow$ & \texttt{Page Up} & $\twoheaduparrow$ \\ \hline
  105 & $\leftarrow$ & $\leftarrow$ & $\leftarrow$ \\ \hline
  106 & $\rightarrow$ & $\rightarrow$ & $\rightarrow$ \\ \hline
  107 & \texttt{Ende} & \texttt{End} & \texttt{Fin} \\ \hline
  108 & $\downarrow$ & $\downarrow$ & $\downarrow$ \\ \hline
  109 & \texttt{Bild} $\downarrow$ & \texttt{Page Down} & $\twoheaddownarrow$ \\ \hline
  110 & \texttt{Einfg} & \texttt{Insert} & \texttt{Inser} \\ \hline
  111 & \texttt{Entf} & \texttt{Delete} & \texttt{Suppr} \\ \hline
  119 & \texttt{Pause} & \texttt{Pause} & \texttt{Pause} \\ \hline
\end{longtable}

Es schließt sich eine Tabelle mit zusätzlichen Tasten an. Nicht alle
Tastaturen bieten diese Tasten an. Einige werden typischerweise nur von
speziellen Hardware-Treibern generiert, etwa die \texttt{Restart}-Taste, die
einen Druck auf den Reset-Schalter darstellt.

\begin{longtable}{|l|l|l|l|}
  \hline
  \textbf{Code} & \textbf{dt.} & \textbf{engl.} & \textbf{frz.} \\
  \endhead
  \hline
  116 & \texttt{Power} & \texttt{Power} & \texttt{Power} \\ \hline
  183 & \texttt{F13} & \texttt{F13} & \texttt{F13} \\ \hline
  184 & \texttt{F14} & \texttt{F14} & \texttt{F14} \\ \hline
  185 & \texttt{F15} & \texttt{F15} & \texttt{F15} \\ \hline
  186 & \texttt{F16} & \texttt{F16} & \texttt{F16} \\ \hline
  187 & \texttt{F17} & \texttt{F17} & \texttt{F17} \\ \hline
  188 & \texttt{F18} & \texttt{F18} & \texttt{F18} \\ \hline
  189 & \texttt{F19} & \texttt{F19} & \texttt{F19} \\ \hline
  190 & \texttt{F20} & \texttt{F20} & \texttt{F20} \\ \hline
  191 & \texttt{F21} & \texttt{F21} & \texttt{F21} \\ \hline
  192 & \texttt{F22} & \texttt{F22} & \texttt{F22} \\ \hline
  193 & \texttt{F23} & \texttt{F23} & \texttt{F23} \\ \hline
  194 & \texttt{F24} & \texttt{F24} & \texttt{F24} \\ \hline
  408 & \texttt{Restart} & \texttt{Restart} & \texttt{Restart} \\ \hline
\end{longtable}

\marklabel{sec:developer}{
\section{HWSUPP - Hinweise für Paket-Entwickler}}
  Im folgenden ist beschrieben was ein Paket-Entwickler zu tun hat, um Button- 
  oder LED-Funktionalität zu einem Paket hinzuzufügen\footnotemark.

  \footnotetext{Wenn man im WLAN-Paket nach \var{\#\#HWSUPP\#\#}
  sucht so findet man die anzupassenden Stellen.}

\subsection{LED-Erweiterungen}
\subsubsection{LED-Typ}
  In der Datei \var{check/myopt.exp} wird die Liste der erlaubten LED-Typen die
  in \var{HWSUPP\_LED\_x} eingetragen werden können erweitert.

Beispiel:
\begin{verbatim}
+HWSUPP_LED_TYPE(OPT_MYOPT) = 'myopt' 
                            : ', myopt'
\end{verbatim}

\subsubsection{Parameterprüfung}
  In der Datei \var{check/myopt.ext} werden die Parameter die für den neuen
  LED-Typen in \var{HWSUPP\_LED\_x\_PARAM} eingetragen werden können geprüft.

Beispiel:
\begin{verbatim}
if (opt_hwsupp)
then
    depends on hwsupp version 4.0

    foreach i in hwsupp_led_n
    do
        set action=hwsupp_led_%[i]
        set param=hwsupp_led_%_param[i]
        if (action == "myopt")
        then
            if (!(param =~ "(RE:MYOPT_LED_PARAM)"))
            then
                error "When HWSUPP_LED_\${i}='myopt', ...
                       must be entered in HWSUPP_LED_\${i}_PARAM" 
            fi
        fi
    done
fi
\end{verbatim}

\subsubsection {LED schalten}
  Um eine LED zu schalten ist in einem eigenen Skript 
  (z.B. \var{/usr/bin/<opt>\_setled}) das Kommando 
  \var{/usr/bin/hwsupp\_setled <LED> <Status>/} aufzurufen.
  
  Die LED-Nummer kann aus \var{/var/run/hwsupp.conf>} ausgelesen werden.
  
  Als Status ist \var{off}, \var{on} oder \var{blink} zu übergeben.
  
 

Beispiel:
\begin{verbatim}
if [ -f /var/run/hwsupp.conf ]
then
    . /var/run/hwsupp.conf
    [ 0$hwsupp_led_n -eq 0 ] || for i in `seq 1 $hwsupp_led_n`
    do
        eval action=\$hwsupp_led_${i}
        eval param=\$hwsupp_led_${i}_param
        if [ "$action" = "<opt>" ]
        then
            if [ <myexpression> ]
            then
                /usr/bin/hwsupp_setled $i on
            else
                /usr/bin/hwsupp_setled $i off
            fi
        fi
    done
fi
\end{verbatim}

  Den aktuellen Zustand einer LED kann man mit 
  \var{/usr/bin/hwsupp\_getled <LED>/} abfragen.
  Es wird je nach Status \var{off}, \var{on} oder \var{blink} ausgegeben.
  
\subsection{Button-Erweiterungen}
\subsection{Button-Aktion}
  In der Datei \var{check/myopt.exp} wird die Liste der erlaubten Button-Typen
  die in \var{HWSUPP\_BUTTON\_x} eingetragen werden können erweitert.

Beispiel:
\begin{verbatim}
+HWSUPP_BUTTON_TYPE(OPT_MYOPT) = 'myopt' 
                               : ', myopt'
\end{verbatim}

\subsubsection{Parameterprüfung}
  In der Datei \var{check/myopt.ext} werden die Parameter, die für den neuen
  Button-Typen in \var{HWSUPP\_BUTTON\_x\_PARAM} eingetragen werden können, geprüft.

Beispiel:
\begin{verbatim}
if (opt_hwsupp)
then
    depends on hwsupp version 4.0

    foreach i in hwsupp_button_n
    do
        set action=hwsupp_buttonn_%[i]
        set param=hwsupp_button_%_param[i]
        if (action == "myopt")
        then
            add_to_opt "files/usr/bin/myopt_keyprog" "mode=555 flags=sh"
            if (!(param =~ "(RE:MYOPT_BUTTON_PARAM)"))
            then
                error "When HWSUPP_BUTTON_\${i}='myopt', ...
                       must be entered in  HWSUPP_BUTTON_\${i}_PARAM" 
            fi
        fi
    done
fi
\end{verbatim}

\subsubsection {Button-Funktion}
  Wenn eine Taste gedrückt wird, wird die Datei \var{/usr/bin/myopt\_keyprog}
  ausgeführt.
 
  Als Parameter wird er Inhalt von \var{HWSUPP\_BUTTON\_x\_PARAM} übergeben

Beispiel:
\begin{verbatim}
##TODO## example
\end{verbatim}
