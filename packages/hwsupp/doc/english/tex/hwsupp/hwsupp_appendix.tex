% Synchronized to r56820

\section{HWSUPP - Device dependant settings}

\marklabel{sec:leddevices}{
\subsection{Available LED devices}}
  Depending on the \var{HWSUPP\_TYPE} different LED devices are present.
  For hardware not listed here the PC keyboard LEDs are available using
  \smalljump{sec:leddevices_keyboard}{generic-pc}.

  Additional LED devices can be mounted on eg. WLAN adapters.
  The valid LED device names can be determined by executing \var{ls /sys/class/leds/},
  eg. via ssh on the router's console.


\subsubsection{sim}
LED simulation, will log to \var{syslog}:
\begin{itemize}
  \item \var{simu::1}
  \item ...
  \item \var{simu::8}
\end{itemize}

\marklabel{sec:leddevices_keyboard}{
\subsubsection{generic-pc}}
PC keyboard LEDs:
\begin{itemize}
  \item \var{inputX::capslock}
  \item \var{inputX::numlock}
  \item \var{inputX::scrolllock}
\end{itemize}

\subsubsection{generic-acpi}
PC keyboard LEDs, like \smalljump{sec:leddevices_keyboard}{generic-pc}

\subsubsection{generic-acpi-coretemp}
PC keyboard LEDs, like \smalljump{sec:leddevices_keyboard}{generic-pc}

\subsubsection{pcengines-alix}
\begin{itemize}
  \item \var{alix::1}
  \item \var{alix::2}
  \item \var{alix::3}
\end{itemize}

\subsubsection{pcengines-apu}
\begin{itemize}
  \item \var{apu::1}
  \item \var{apu::2}
  \item \var{apu::3}
\end{itemize}

\subsubsection{pcengines-apu2}
\begin{itemize}
  \item \var{apu::1}
  \item \var{apu::2}
  \item \var{apu::3}
\end{itemize}

\subsubsection{pcengines-wrap} 
\begin{itemize}
  \item \var{wrap::1}
  \item \var{wrap::2}
  \item \var{wrap::3}
\end{itemize}

\subsubsection{rpi}
\begin{itemize}
  \item \var{led0}
\end{itemize}

\subsubsection{soekris-net4801}
\begin{itemize}
  \item \var{net48xx::error}
\end{itemize}

\subsubsection{soekris-net5501}
\begin{itemize}
  \item \var{net5501::error}
\end{itemize}

\marklabel{sec:buttondevices}{
\subsection{Available Button Devices}}
  Depending on the \var{HWSUPP\_TYPE} different GPIO devices are predefined
  for buttons.

\subsubsection{generic-pc und generic-acpi}
\begin{itemize}
  \item \var{evdev:isa0060/serio0/input0}, any (available) button
\end{itemize}

Note that key \emph{combinations} are not possible. I.e. a router
shutdown via \textsc{Alt+F4} or else is not feasible.

\subsubsection{pcengines-alix}
\begin{itemize}
  \item \var{evdev:gpio-keys-polled/input0}, key 408 (Restart)
\end{itemize}

\subsubsection{pcengines-apu}
\begin{itemize}
  \item \var{evdev:gpio-keys-polled/input0}, key 408 (Restart)
\end{itemize}

\subsubsection{pcengines-apu2}
\begin{itemize}
  \item \var{evdev:gpio-keys-polled/input0}, key 408 (Restart)
\end{itemize}

\subsubsection{pcengines-wrap} 
\begin{itemize}
  \item \var{evdev:gpio-keys-polled/input0}, key 408 (Restart)
\end{itemize}

\subsubsection{soekris-net5501}
\begin{itemize}
  \item \var{evdev:gpio-keys-polled/input0}, key 408 (Restart)
  \newline The button is named 'Reset' on the Soekris case. 
  \newline Attention: The button must be enabled in BIOS.
\end{itemize}

\marklabel{sec:notes}{
\subsection{Hardware specific notes}}

\subsubsection{pcengines-alix}
A faulty driver for the lm90 temperatur sensor causes a loss of temperature 
monitoring.

As a workaraound the lm90 driver will be unloaded and reloaded again
automatically by a cron job. This requires the package easycron to be loaded
(\var{OPT\_EASYCRON='yes'}).

\marklabel{sec:examples}{
\section{HWSUPP - Configuration examples}}

\subsection{generic-pc}

\begin{verbatim}
OPT_HWSUPP='yes'
HWSUPP_TYPE='generic-pc'

HWSUPP_WATCHDOG='no'
HWSUPP_CPUFREQ='no'

HWSUPP_LED_N='3'
HWSUPP_LED_1='ready'
HWSUPP_LED_1_DEVICE='input1::numlock'
HWSUPP_LED_2='online'
HWSUPP_LED_2_DEVICE='input1::capslock'
HWSUPP_LED_3='wlan'
HWSUPP_LED_3_DEVICE='input1::scrolllock'
HWSUPP_LED_3_WLAN='wlan0'

HWSUPP_BUTTON_N='0'

\end{verbatim}

\subsection{pcengines-apu}

\begin{verbatim}
OPT_HWSUPP='yes'
HWSUPP_TYPE='pcengines-apu'

HWSUPP_WATCHDOG='yes'
HWSUPP_CPUFREQ='yes'
HWSUPP_CPUFREQ_GOVERNOR='ondemand'

HWSUPP_LED_N='3'
HWSUPP_LED_1='ready'
HWSUPP_LED_1_DEVICE='apu::1'
HWSUPP_LED_2='wlan'
HWSUPP_LED_2_DEVICE='apu::2'
HWSUPP_LED_2_WLAN='wlan0'
HWSUPP_LED_3='online'
HWSUPP_LED_3_DEVICE='apu::3'

HWSUPP_BUTTON_N='1'
HWSUPP_BUTTON_1='wlan'
HWSUPP_BUTTON_1_DEVICE='evdev:gpio-keys-polled/input0'
HWSUPP_BUTTON_1_DEVICE_KEY='408'
HWSUPP_BUTTON_1_PARAM='wlan0'
\end{verbatim}

\subsection{pcengines-apu with GPIOs}

\begin{verbatim}
OPT_HWSUPP='yes'
HWSUPP_TYPE='pcengines-apu'

HWSUPP_WATCHDOG='yes'
HWSUPP_CPUFREQ='yes'
HWSUPP_CPUFREQ_GOVERNOR='ondemand'

HWSUPP_LED_N='5'
HWSUPP_LED_1='ready'
HWSUPP_LED_1_DEVICE='apu::1'
HWSUPP_LED_2='wlan'
HWSUPP_LED_2_DEVICE='apu::2'
HWSUPP_LED_2_WLAN='wlan0'
HWSUPP_LED_3='online'
HWSUPP_LED_3_DEVICE='apu::3'
HWSUPP_LED_4='trigger'
HWSUPP_LED_4_PARAM='phy0rx'
HWSUPP_LED_4_DEVICE='gpio::237'
HWSUPP_LED_5='trigger'
HWSUPP_LED_5_PARAM='phy0tx'
HWSUPP_LED_5_DEVICE='gpio::245'

HWSUPP_BUTTON_N='2'
HWSUPP_BUTTON_1='wlan'
HWSUPP_BUTTON_1_DEVICE='evdev:gpio-keys-polled/input0'
HWSUPP_BUTTON_1_DEVICE_KEY='408'
HWSUPP_BUTTON_1_PARAM='wlan0'
HWSUPP_BUTTON_2='online'
HWSUPP_BUTTON_2_DEVICE='gpio::236' # for Linux kernel < 3.18
HWSUPP_BUTTON_2_DEVICE='gpio::492' # for Linux kernel >= 3.18 (x+256)
\end{verbatim}

\marklabel{sec:blink}{
\section{HWSUPP - Blink Sequences}}
  The following blink sequences are displayed during boot:
  
  {\def\led{\begin{math}\otimes\quad\quad\end{math}}
  \begin{tabular}{c l l l l l l l l l}
    \quad\\
    1. & \led & & & & \led & & & & ... \\ 
    2. & \led & \led & & & \led & \led & & & ... \\ 
    3. & \led & \led & \led & & \led & \led & \led & & ... \\ 
    4. & \led & \led & \led & \led & \led & \led & \led & \led & ... \\ 
    \quad 
  \end{tabular}}

  The first sequence is displayed while processing \var{rc002.*} to \var{rc250.*}
  \newline(1 * blink - pause),\newline
  for \var{rc250.*} to \var{rc500.*} the second (2 * blink - pause),\newline
  for \var{rc500.*} to \var{rc750.*} the third and \newline
  for \var{rc750.*} until the end of the boot process the forth sequence (coninuous blinking).
    
\marklabel{sec:hwsupp:evdev-keys}{
\section{HWSUPP - Keycodes}}

The following table presents the various key codes along with the key in the German, English 
and French layout. The abbreviation ``NP'' stands for ``numpad'' and denotes a key in the right 
portion of the keyboard. The entry ``---'' refers to a button that does not exist in the respective layout. Key codes which do not occur in all of the three keyboard layouts shown (this applies
i.e. for several keys on Japanese or Chinese keyboards) were completely omitted.

First is a table with the standard keys of the IBM AT keyboard with 101
or 102 keys.

\newcommand\twoheaduparrow{\mathrel{\rotatebox[origin=c]{90}{$\twoheadrightarrow$}}}
\newcommand\twoheaddownarrow{\mathrel{\rotatebox[origin=c]{270}{$\twoheadrightarrow$}}}

\begin{longtable}{|l|l|l|l|}
  \hline
  \textbf{Code} & \textbf{german} & \textbf{english} & \textbf{french} \\
  \endhead
  \hline
  1 & \texttt{Esc} & \texttt{Esc} & \texttt{Esc} \\ \hline
  2 & \texttt{1}   & \texttt{1}   & \texttt{\&} \\ \hline
  3 & \texttt{2}   & \texttt{2}   & \texttt{é} \\ \hline
  4 & \texttt{3}   & \texttt{3}   & \texttt{"} \\ \hline
  5 & \texttt{4}   & \texttt{4}   & \texttt{'} \\ \hline
  6 & \texttt{5}   & \texttt{5}   & \texttt{(} \\ \hline
  7 & \texttt{6}   & \texttt{6}   & \texttt{-} \\ \hline
  8 & \texttt{7}   & \texttt{7}   & \texttt{è} \\ \hline
  9 & \texttt{8}   & \texttt{8}   & \texttt{\_} \\ \hline
  10 & \texttt{9}   & \texttt{9}   & \texttt{ç} \\ \hline
  11 & \texttt{0}   & \texttt{0}   & \texttt{à} \\ \hline
  12 & \texttt{ß}   & \texttt{-}   & \texttt{)} \\ \hline
  13 & \texttt{?}   & \texttt{=}   & \texttt{=} \\ \hline
  14 & \multicolumn{3}{|l|}{Backspace $\longleftarrow$} \\ \hline
  15 & \multicolumn{3}{|l|}{Tabulator} \\ \hline
  16 & \texttt{Q} & \texttt{Q} & \texttt{A} \\ \hline
  17 & \texttt{W} & \texttt{W} & \texttt{Z} \\ \hline
  18 & \texttt{E} & \texttt{E} & \texttt{E} \\ \hline
  19 & \texttt{R} & \texttt{R} & \texttt{R} \\ \hline
  20 & \texttt{T} & \texttt{T} & \texttt{T} \\ \hline
  21 & \texttt{Z} & \texttt{Y} & \texttt{Y} \\ \hline
  22 & \texttt{U} & \texttt{U} & \texttt{U} \\ \hline
  23 & \texttt{I} & \texttt{I} & \texttt{I} \\ \hline
  24 & \texttt{O} & \texttt{O} & \texttt{O} \\ \hline
  25 & \texttt{P} & \texttt{P} & \texttt{P} \\ \hline
  26 & \texttt{Ü} & \texttt{[} & \texttt{\^} \\ \hline
  27 & \texttt{+} & \texttt{]} & \texttt{\$} \\ \hline
  28 & \multicolumn{3}{|l|}{Enter} \\ \hline
  29 & \multicolumn{3}{|l|}{left Ctrl key} \\ \hline
  30 & \texttt{A} & \texttt{A} & \texttt{Q} \\ \hline
  31 & \texttt{S} & \texttt{S} & \texttt{S} \\ \hline
  32 & \texttt{D} & \texttt{D} & \texttt{D} \\ \hline
  33 & \texttt{F} & \texttt{F} & \texttt{F} \\ \hline
  34 & \texttt{G} & \texttt{G} & \texttt{G} \\ \hline
  35 & \texttt{H} & \texttt{H} & \texttt{H} \\ \hline
  36 & \texttt{J} & \texttt{J} & \texttt{J} \\ \hline
  37 & \texttt{K} & \texttt{K} & \texttt{K} \\ \hline
  38 & \texttt{L} & \texttt{L} & \texttt{L} \\ \hline
  39 & \texttt{Ö} & \texttt{;} & \texttt{M} \\ \hline
  40 & \texttt{Ä} & \texttt{'} & \texttt{ù} \\ \hline
  41 & \texttt{\^} & \texttt{`} & $^2$ \\ \hline
  42 & \multicolumn{3}{|l|}{left Shift key} \\ \hline
  43 & \texttt{\#} & \texttt{\textbackslash} & \texttt{*} \\ \hline
  44 & \texttt{Z} & \texttt{Z} & \texttt{W} \\ \hline
  45 & \texttt{X} & \texttt{X} & \texttt{X} \\ \hline
  46 & \texttt{C} & \texttt{C} & \texttt{C} \\ \hline
  47 & \texttt{V} & \texttt{V} & \texttt{V} \\ \hline
  48 & \texttt{B} & \texttt{B} & \texttt{B} \\ \hline
  49 & \texttt{N} & \texttt{N} & \texttt{N} \\ \hline
  50 & \texttt{M} & \texttt{M} & \texttt{,} \\ \hline
  51 & \texttt{,} & \texttt{,} & \texttt{;} \\ \hline
  52 & \texttt{.} & \texttt{.} & \texttt{:} \\ \hline
  53 & \texttt{-} & \texttt{/} & \texttt{!} \\ \hline
  54 & \multicolumn{3}{|l|}{right Shift key} \\ \hline
  55 & \texttt{*} (NP) & \texttt{*} (NP) & \texttt{*} (NP) \\ \hline
  56 & \multicolumn{3}{|l|}{left Alt key} \\ \hline
  57 & \multicolumn{3}{|l|}{Space} \\ \hline
  58 & \multicolumn{3}{|l|}{Caps Lock} \\ \hline
  59 & \texttt{F1} & \texttt{F1} & \texttt{F1} \\ \hline
  60 & \texttt{F2} & \texttt{F2} & \texttt{F2} \\ \hline
  61 & \texttt{F3} & \texttt{F3} & \texttt{F3} \\ \hline
  62 & \texttt{F4} & \texttt{F4} & \texttt{F4} \\ \hline
  63 & \texttt{F5} & \texttt{F5} & \texttt{F5} \\ \hline
  64 & \texttt{F6} & \texttt{F6} & \texttt{F6} \\ \hline
  65 & \texttt{F7} & \texttt{F7} & \texttt{F7} \\ \hline
  66 & \texttt{F8} & \texttt{F8} & \texttt{F8} \\ \hline
  67 & \texttt{F9} & \texttt{F9} & \texttt{F9} \\ \hline
  68 & \texttt{F10} & \texttt{F10} & \texttt{F10} \\ \hline
  69 & \texttt{Num $\Downarrow$} & \texttt{Num Lock} & \texttt{Verr num} \\ \hline
  70 & \texttt{Rollen $\Downarrow$} & \texttt{Scroll Lock} & \texttt{Arrêt Défil} \\ \hline
  71 & \texttt{7} (NP) & \texttt{7} (NP) & \texttt{7} (NP) \\ \hline
  72 & \texttt{8} (NP) & \texttt{8} (NP) & \texttt{8} (NP) \\ \hline
  73 & \texttt{9} (NP) & \texttt{9} (NP) & \texttt{9} (NP) \\ \hline
  74 & \texttt{-} (NP) & \texttt{-} (NP) & \texttt{-} (NP) \\ \hline
  75 & \texttt{4} (NP) & \texttt{4} (NP) & \texttt{4} (NP) \\ \hline
  76 & \texttt{5} (NP) & \texttt{5} (NP) & \texttt{5} (NP) \\ \hline
  77 & \texttt{6} (NP) & \texttt{6} (NP) & \texttt{6} (NP) \\ \hline
  78 & \texttt{+} (NP) & \texttt{+} (NP) & \texttt{+} (NP) \\ \hline
  79 & \texttt{1} (NP) & \texttt{1} (NP) & \texttt{1} (NP) \\ \hline
  80 & \texttt{2} (NP) & \texttt{2} (NP) & \texttt{2} (NP) \\ \hline
  81 & \texttt{3} (NP) & \texttt{3} (NP) & \texttt{3} (NP) \\ \hline
  82 & \texttt{0} (NP) & \texttt{0} (NP) & \texttt{0} (NP) \\ \hline
  83 & \texttt{,} (NP) & \texttt{.} (NP) & \texttt{.} (NP) \\ \hline
  86 & \texttt{<} & --- & \texttt{<} \\ \hline
  87 & \texttt{F11} & \texttt{F11} & \texttt{F11} \\ \hline
  88 & \texttt{F12} & \texttt{F12} & \texttt{F12} \\ \hline
  96 & \multicolumn{3}{|l|}{Enter (NP)} \\ \hline
  97 & \multicolumn{3}{|l|}{right Ctrl key} \\ \hline
  98 & $\div$ (NP) & \texttt{/} (NP) & \texttt{/} (NP) \\ \hline
  99 & \texttt{Druck} & \texttt{Print Screen} & \texttt{Impr écran} \\ \hline
  100 & \multicolumn{3}{|l|}{right Alt key (AltGr)} \\ \hline
  102 & \texttt{Pos 1} & \texttt{Home} & $\nwarrow$ \\ \hline
  103 & $\uparrow$ & $\uparrow$ & $\uparrow$ \\ \hline
  104 & \texttt{Bild} $\uparrow$ & \texttt{Page Up} & $\twoheaduparrow$ \\ \hline
  105 & $\leftarrow$ & $\leftarrow$ & $\leftarrow$ \\ \hline
  106 & $\rightarrow$ & $\rightarrow$ & $\rightarrow$ \\ \hline
  107 & \texttt{Ende} & \texttt{End} & \texttt{Fin} \\ \hline
  108 & $\downarrow$ & $\downarrow$ & $\downarrow$ \\ \hline
  109 & \texttt{Bild} $\downarrow$ & \texttt{Page Down} & $\twoheaddownarrow$ \\ \hline
  110 & \texttt{Einfg} & \texttt{Insert} & \texttt{Inser} \\ \hline
  111 & \texttt{Entf} & \texttt{Delete} & \texttt{Suppr} \\ \hline
  119 & \texttt{Pause} & \texttt{Pause} & \texttt{Pause} \\ \hline
\end{longtable}

This is followed by a table with additional buttons. Not all
keyboards offer these keys. Some are typically generated only by
special hardware drivers, i.e. the \texttt{Restart} button which 
emulates the reset switch.

\begin{longtable}{|l|l|l|l|}
  \hline
  \textbf{Code} & \textbf{german} & \textbf{english} & \textbf{french} \\
  \endhead
  \hline
  116 & \texttt{Power} & \texttt{Power} & \texttt{Power} \\ \hline
  183 & \texttt{F13} & \texttt{F13} & \texttt{F13} \\ \hline
  184 & \texttt{F14} & \texttt{F14} & \texttt{F14} \\ \hline
  185 & \texttt{F15} & \texttt{F15} & \texttt{F15} \\ \hline
  186 & \texttt{F16} & \texttt{F16} & \texttt{F16} \\ \hline
  187 & \texttt{F17} & \texttt{F17} & \texttt{F17} \\ \hline
  188 & \texttt{F18} & \texttt{F18} & \texttt{F18} \\ \hline
  189 & \texttt{F19} & \texttt{F19} & \texttt{F19} \\ \hline
  190 & \texttt{F20} & \texttt{F20} & \texttt{F20} \\ \hline
  191 & \texttt{F21} & \texttt{F21} & \texttt{F21} \\ \hline
  192 & \texttt{F22} & \texttt{F22} & \texttt{F22} \\ \hline
  193 & \texttt{F23} & \texttt{F23} & \texttt{F23} \\ \hline
  194 & \texttt{F24} & \texttt{F24} & \texttt{F24} \\ \hline
  408 & \texttt{Restart} & \texttt{Restart} & \texttt{Restart} \\ \hline
\end{longtable}

\marklabel{sec:developer}{
\section{HWSUPP - Hints for package developers}}  This chapter describes the things a package developer has to do to add LED
  or button functionality to a package\footnotemark.

  \footnotetext{Search for \var{\#\#HWSUPP\#\#} in the WLAN package to find the
  places to adapt.}

\subsection{LED extensions}
\subsubsection{LED type}
  Within the file \var{check/myopt.exp} the list of LED types which can be
  entered in \var{HWSUPP\_LED\_x} is extended.

Example:
\begin{verbatim}
+HWSUPP_LED_TYPE(OPT_MYOPT) = 'myopt' 
                            : ', myopt'
\end{verbatim}

\subsubsection{Parameter check}
  Within \var{check/myopt.ext} the parameters which can be entered 
  in \var{HWSUPP\_LED\_x\_PARAM} will be checked.

Example:
\begin{verbatim}
if (opt_hwsupp)
then
    depends on hwsupp version 4.0

    foreach i in hwsupp_led_n
    do
        set action=hwsupp_led_%[i]
        set param=hwsupp_led_%_param[i]
        if (action == "myopt")
        then
            if (!(param =~ "(RE:MYOPT_LED_PARAM)"))
            then
                error "When HWSUPP_LED_\${i}='myopt', ...
                       must be entered in HWSUPP_LED_\${i}_PARAM" 
            fi
        fi
    done
fi
\end{verbatim}

\subsubsection {LED Display}
  The command \var{/usr/bin/hwsupp\_setled <LED> <status>/} has to be executed
  to set a LED in a package script (eg. \var{/usr/bin/<opt>\_setled}) 
  
  The LED number can be found in \var{/var/run/hwsupp.conf}.
  
  Status can be \var{off}, \var{on} or \var{blink}.\\

Example:
\begin{verbatim}
if [ -f /var/run/hwsupp.conf ]
then
    . /var/run/hwsupp.conf
    [ 0$hwsupp_led_n -eq 0 ] || for i in `seq 1 $hwsupp_led_n`
    do
        eval action=\$hwsupp_led_${i}
        eval param=\$hwsupp_led_${i}_param
        if [ "$action" = "<opt>" ]
        then
            if [ <myexpression> ]
            then
                /usr/bin/hwsupp_setled $i on
            else
                /usr/bin/hwsupp_setled $i off
            fi
        fi
    done
fi
\end{verbatim}

  The actual state of a LED can be queried with 
  \var{/usr/bin/hwsupp\_getled <LED>/}.
  The result will be \var{off}, \var{on} or \var{blink}.
  
\subsection{Button extensions}
\subsection{Button action}
  In \var{check/myopt.exp} the list of button types allowed in \var{HWSUPP\_LED\_x}
  can be extended.

Beispiel:
\begin{verbatim}
+HWSUPP_BUTTON_TYPE(OPT_MYOPT) = 'myopt' 
                               : ', myopt'
\end{verbatim}

\subsubsection{Parameter check}
  The parameters which can be entered in \var{HWSUPP\_BUTTON\_x\_PARAM}
  will be checked using \var{check/myopt.ext} .

Example:
\begin{verbatim}
if (opt_hwsupp)
then
    depends on hwsupp version 4.0

    foreach i in hwsupp_button_n
    do
        set action=hwsupp_buttonn_%[i]
        set param=hwsupp_button_%_param[i]
        if (action == "myopt")
        then
            add_to_opt "files/usr/bin/myopt_keyprog" "mode=555 flags=sh"
            if (!(param =~ "(RE:MYOPT_BUTTON_PARAM)"))
            then
                error "When HWSUPP_BUTTON_\${i}='myopt', ...
                       must be entered in  HWSUPP_BUTTON_\${i}_PARAM" 
            fi
        fi
    done
fi
\end{verbatim}

\subsubsection {Button function}
  When a button is pressed the script file \var{/usr/bin/myopt\_keyprog}
  will be executed.
 
  The content of \var{HWSUPP\_BUTTON\_x\_PARAM} is passed as a parameter.

Example:
\begin{verbatim}
##TODO## example
\end{verbatim}
