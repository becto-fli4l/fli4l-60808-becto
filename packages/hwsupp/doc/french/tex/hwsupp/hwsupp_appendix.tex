% Synchronized to r56820
\section{HWSUPP - Paramètres selon le périphérique}

\marklabel{sec:leddevices}{
\subsection{Périphérique disposant de LED}}
Selon le matériel voir la variable \var{HWSUPP\_TYPE} différent périphériques
disposent d'une ou plusieurs LED. Pour le matériel ne figurant pas ici, les LEDs du
clavier de l'ordinateur peuvent être utilisées voir la section
\smalljump{sec:leddevices_keyboard}{PC générique}.

Les périphériques supplémentaires disposant de LED peut être utilisé par exemple
pour le périphérique WLAN. Les noms valides des périphériques pour la LED peuvent
d'être trouvés avec la commande \var{ls /sys/class/leds/}. En utilisant par exemple
le ssh ou la console du routeur.

\subsubsection{sim}
 Pour gérérer une simulation de la LED, qui sera enregistré avec \var{syslog}~:
\begin{itemize}
  \item \var{simu::1}
  \item ...
  \item \var{simu::8}
\end{itemize}

\marklabel{sec:leddevices_keyboard}{
\subsubsection{PC générique}}
Les LEDs du clavier du PC~:
\begin{itemize}
  \item \var{inputX::capslock}
  \item \var{inputX::numlock}
  \item \var{inputX::scrolllock}
\end{itemize}

\subsubsection{Générique acpi}
Les LEDs du clavier du PC, comme dans \smalljump{sec:leddevices_keyboard}{PC générique}

\subsubsection{Générique acpi coretemp}
Les LEDs du clavier du PC, comme dans \smalljump{sec:leddevices_keyboard}{PC générique}

\subsubsection{PC engines alix}
\begin{itemize}
  \item \var{alix::1}
  \item \var{alix::2}
  \item \var{alix::3}
\end{itemize}

\subsubsection{PC engines apu}
\begin{itemize}
  \item \var{apu::1}
  \item \var{apu::2}
  \item \var{apu::3}
\end{itemize}

\subsubsection{PC engines apu2}
\begin{itemize}
  \item \var{apu::1}
  \item \var{apu::2}
  \item \var{apu::3}
\end{itemize}

\subsubsection{PC engines wrap}
\begin{itemize}
  \item \var{wrap::1}
  \item \var{wrap::2}
  \item \var{wrap::3}
\end{itemize}

\subsubsection{Raspberry Pi}
\begin{itemize}
  \item \var{led0}
\end{itemize}

\subsubsection{Soekris net4801}
\begin{itemize}
  \item \var{net48xx::error}
\end{itemize}

\subsubsection{Soekris net5501}
\begin{itemize}
  \item \var{net5501::error}
\end{itemize}

\marklabel{sec:buttondevices}{
\subsection{Périphérique disposant de bouton}}
  Selon le matériel de la variable \var{HWSUPP\_TYPE}, les périphériques GPIO
  suivants sont prédéfinis pour utiliser les boutons.

\subsubsection{PC générique et acpi générique}
\begin{itemize}
  \item \var{evdev:isa0060/serio0/input0}, n'importe quel touche du clavier (disponible)
\end{itemize}

Il convient de noter que les \emph{combinaisons} de touches ne sont pas possibles.
Donc, il n'est pas possible de descendre le routeur avec les touches \textsc{Alt+F4}
ou similaire.

\subsubsection{PC engines alix}
\begin{itemize}
  \item \var{evdev:gpio-keys-polled/input0}, touche 408 (Restart)
\end{itemize}

\subsubsection{PC engines apu}
\begin{itemize}
  \item \var{evdev:gpio-keys-polled/input0}, touche 408 (Restart)
\end{itemize}

\subsubsection{PC engines apu2}
\begin{itemize}
  \item \var{evdev:gpio-keys-polled/input0}, touche 408 (Restart)
\end{itemize}

\subsubsection{PC engines wrap}
\begin{itemize}
  \item \var{evdev:gpio-keys-polled/input0}, touche 408 (Restart)
\end{itemize}

\subsubsection{soekris net5501}
\begin{itemize}
  \item \var{evdev:gpio-keys-polled/input0}, touche 408 (Restart)
  \newline Le bouton est nommé 'Reset' pour le matériel Soekris.
  \newline Attention~: Le bouton doit être activé dans le BIOS.
\end{itemize}

\marklabel{sec:notes}{
\subsection{Note sur le matériel spécifique}}

\subsubsection{PC engines alix}
Un pilote défectueux du capteur de températur LM90, entraîne une perte de contrôle
de la température.

Pour contourner ce problème le pilote LM90 sera automatiquement déchargé et rechargé
en utilisant une tâche de cron. Il est nécessite que le paquetage easycron soit
installé avec la variable (\var{OPT\_EASYCRON='yes'}).

\marklabel{sec:examples}{
\section{HWSUPP - Exemple de configuration}}

\subsection{PC générique}

\begin{verbatim}
OPT_HWSUPP='yes'
HWSUPP_TYPE='generic-pc'

HWSUPP_WATCHDOG='no'
HWSUPP_CPUFREQ='no'

HWSUPP_LED_N='3'
HWSUPP_LED_1='ready'
HWSUPP_LED_1_DEVICE='input1::numlock'
HWSUPP_LED_2='online'
HWSUPP_LED_2_DEVICE='input1::capslock'
HWSUPP_LED_3='wlan'
HWSUPP_LED_3_DEVICE='input1::scrolllock'
HWSUPP_LED_3_WLAN='wlan0'

HWSUPP_BUTTON_N='0'
\end{verbatim}

\subsection{PC engines APU}

\begin{verbatim}
OPT_HWSUPP='yes'
HWSUPP_TYPE='pcengines-apu'

HWSUPP_WATCHDOG='yes'
HWSUPP_CPUFREQ='yes'
HWSUPP_CPUFREQ_GOVERNOR='ondemand'

HWSUPP_LED_N='3'
HWSUPP_LED_1='ready'
HWSUPP_LED_1_DEVICE='apu::1'
HWSUPP_LED_2='wlan'
HWSUPP_LED_2_DEVICE='apu::2'
HWSUPP_LED_2_WLAN='wlan0'
HWSUPP_LED_3='online'
HWSUPP_LED_3_DEVICE='apu::3'

HWSUPP_BUTTON_N='1'
HWSUPP_BUTTON_1='wlan'
HWSUPP_BUTTON_1_DEVICE='evdev:gpio-keys-polled/input0'
HWSUPP_BUTTON_1_DEVICE_KEY='408'
HWSUPP_BUTTON_1_PARAM='wlan0'
\end{verbatim}

\subsection{PC engines APU avec GPIO}

\begin{verbatim}
OPT_HWSUPP='yes'
HWSUPP_TYPE='pcengines-apu'

HWSUPP_WATCHDOG='yes'
HWSUPP_CPUFREQ='yes'
HWSUPP_CPUFREQ_GOVERNOR='ondemand'

HWSUPP_LED_N='5'
HWSUPP_LED_1='ready'
HWSUPP_LED_1_DEVICE='apu::1'
HWSUPP_LED_2='wlan'
HWSUPP_LED_2_DEVICE='apu::2'
HWSUPP_LED_2_WLAN='wlan0'
HWSUPP_LED_3='online'
HWSUPP_LED_3_DEVICE='apu::3'
HWSUPP_LED_4='trigger'
HWSUPP_LED_4_PARAM='phy0rx'
HWSUPP_LED_4_DEVICE='gpio::237'
HWSUPP_LED_5='trigger'
HWSUPP_LED_5_PARAM='phy0tx'
HWSUPP_LED_5_DEVICE='gpio::245'

HWSUPP_BUTTON_N='2'
HWSUPP_BUTTON_1='wlan'
HWSUPP_BUTTON_1_DEVICE='evdev:gpio-keys-polled/input0'
HWSUPP_BUTTON_1_DEVICE_KEY='408'
HWSUPP_BUTTON_1_PARAM='wlan0'
HWSUPP_BUTTON_2='online'
HWSUPP_BUTTON_2_DEVICE='gpio::236' # pour le noyau Linux < 3.18
HWSUPP_BUTTON_2_DEVICE='gpio::492' # pour le noyau Linux >= 3.18 (x+256)
\end{verbatim}

\marklabel{sec:blink}{
\section{HWSUPP - Séquence de clignotement de la LED}}
  Les séquences de clignotants suivantes seront affichés pendant le processus de boot~:

  {\def\led{\begin{math}\otimes\quad\quad\end{math}}
  \begin{tabular}{c l l l l l l l l l}
    \quad\\
    1. & \led & & & & \led & & & & ... \\
    2. & \led & \led & & & \led & \led & & & ... \\
    3. & \led & \led & \led & & \led & \led & \led & & ... \\
    4. & \led & \led & \led & \led & \led & \led & \led & \led & ... \\
    \quad 
  \end{tabular}}

  Lors de l'execution de \var{rc002.*} à \var{rc250.*} La première séquence s'affiche,\newline
  (1 * flash - pause),\newline
  de \var{rc250.*} à \var{rc500.*} la seconde séquence (2 * flash - pause),\newline
  de \var{rc500.*} à \var{rc750.*} la troisième séquence et\newline
  de \var{rc750.*} jusqu'à la fin du processus de boot la quatrième séquence
  (clignotement continu).

\marklabel{sec:hwsupp:evdev-keys}{
\section{HWSUPP - Code des touches du clavier}}

Le tableau suivant affiche les différents codes des touches du clavier, l'ensemble du
tableau correspond aux touches des claviers allemand, anglais et français. L'abréviation
"NB" signifie "numéro de bloc" et se réfère aux touches dans la partie droite du clavier.
Le signe "---" se réfère à une touche qui n'existe pas sur l'un des claviers respectif.
Les codes des touches, qui sont pas présents dans les trois configurations du clavier 
(cela s'applique, par exemple, pour les diverses touches du clavier japonais ou chinois)
ne seront pas affichés dans le tableau.

Le tableau fait référence aux touches standard du clavier AT avec 101 ou 102 touches de IBM.

\newcommand\twoheaduparrow{\mathrel{\rotatebox[origin=c]{90}{$\twoheadrightarrow$}}}
\newcommand\twoheaddownarrow{\mathrel{\rotatebox[origin=c]{270}{$\twoheadrightarrow$}}}

\begin{longtable}{|l|l|l|l|}
  \hline
  \textbf{Code} & \textbf{dt.} & \textbf{engl.} & \textbf{frz.} \\
  \endhead
  \hline
  1 & \texttt{Esc} & \texttt{Esc} & \texttt{Echap} \\ \hline
  2 & \texttt{1}   & \texttt{1}   & \texttt{\&} \\ \hline
  3 & \texttt{2}   & \texttt{2}   & \texttt{é} \\ \hline
  4 & \texttt{3}   & \texttt{3}   & \texttt{"} \\ \hline
  5 & \texttt{4}   & \texttt{4}   & \texttt{'} \\ \hline
  6 & \texttt{5}   & \texttt{5}   & \texttt{(} \\ \hline
  7 & \texttt{6}   & \texttt{6}   & \texttt{-} \\ \hline
  8 & \texttt{7}   & \texttt{7}   & \texttt{è} \\ \hline
  9 & \texttt{8}   & \texttt{8}   & \texttt{\_} \\ \hline
  10 & \texttt{9}   & \texttt{9}   & \texttt{ç} \\ \hline
  11 & \texttt{0}   & \texttt{0}   & \texttt{à} \\ \hline
  12 & \texttt{ß}   & \texttt{-}   & \texttt{)} \\ \hline
  13 & \texttt{?}   & \texttt{=}   & \texttt{=} \\ \hline
  14 & \multicolumn{3}{|l|}{Retour arrière $\longleftarrow$} \\ \hline
  15 & \multicolumn{3}{|l|}{Tabulation} \\ \hline
  16 & \texttt{Q} & \texttt{Q} & \texttt{A} \\ \hline
  17 & \texttt{W} & \texttt{W} & \texttt{Z} \\ \hline
  18 & \texttt{E} & \texttt{E} & \texttt{E} \\ \hline
  19 & \texttt{R} & \texttt{R} & \texttt{R} \\ \hline
  20 & \texttt{T} & \texttt{T} & \texttt{T} \\ \hline
  21 & \texttt{Z} & \texttt{Y} & \texttt{Y} \\ \hline
  22 & \texttt{U} & \texttt{U} & \texttt{U} \\ \hline
  23 & \texttt{I} & \texttt{I} & \texttt{I} \\ \hline
  24 & \texttt{O} & \texttt{O} & \texttt{O} \\ \hline
  25 & \texttt{P} & \texttt{P} & \texttt{P} \\ \hline
  26 & \texttt{Ü} & \texttt{[} & \texttt{\^} \\ \hline
  27 & \texttt{+} & \texttt{]} & \texttt{\$} \\ \hline
  28 & \multicolumn{3}{|l|}{Entrée} \\ \hline
  29 & \multicolumn{3}{|l|}{Touche Ctrl gauche} \\ \hline
  30 & \texttt{A} & \texttt{A} & \texttt{Q} \\ \hline
  31 & \texttt{S} & \texttt{S} & \texttt{S} \\ \hline
  32 & \texttt{D} & \texttt{D} & \texttt{D} \\ \hline
  33 & \texttt{F} & \texttt{F} & \texttt{F} \\ \hline
  34 & \texttt{G} & \texttt{G} & \texttt{G} \\ \hline
  35 & \texttt{H} & \texttt{H} & \texttt{H} \\ \hline
  36 & \texttt{J} & \texttt{J} & \texttt{J} \\ \hline
  37 & \texttt{K} & \texttt{K} & \texttt{K} \\ \hline
  38 & \texttt{L} & \texttt{L} & \texttt{L} \\ \hline
  39 & \texttt{Ö} & \texttt{;} & \texttt{M} \\ \hline
  40 & \texttt{Ä} & \texttt{'} & \texttt{ù} \\ \hline
  41 & \texttt{\^} & \texttt{`} & $^2$ \\ \hline
  42 & \multicolumn{3}{|l|}{Touche Maj gauche} \\ \hline
  43 & \texttt{\#} & \texttt{\textbackslash} & \texttt{*} \\ \hline
  44 & \texttt{Z} & \texttt{Z} & \texttt{W} \\ \hline
  45 & \texttt{X} & \texttt{X} & \texttt{X} \\ \hline
  46 & \texttt{C} & \texttt{C} & \texttt{C} \\ \hline
  47 & \texttt{V} & \texttt{V} & \texttt{V} \\ \hline
  48 & \texttt{B} & \texttt{B} & \texttt{B} \\ \hline
  49 & \texttt{N} & \texttt{N} & \texttt{N} \\ \hline
  50 & \texttt{M} & \texttt{M} & \texttt{,} \\ \hline
  51 & \texttt{,} & \texttt{,} & \texttt{;} \\ \hline
  52 & \texttt{.} & \texttt{.} & \texttt{:} \\ \hline
  53 & \texttt{-} & \texttt{/} & \texttt{!} \\ \hline
  54 & \multicolumn{3}{|l|}{Touche Maj droite} \\ \hline
  55 & \texttt{*} (NB) & \texttt{*} (NB) & \texttt{*} (NB) \\ \hline
  56 & \multicolumn{3}{|l|}{Touche Alt gauche} \\ \hline
  57 & \multicolumn{3}{|l|}{Barre d'espace} \\ \hline
  58 & \multicolumn{3}{|l|}{Verrouillage majuscule} \\ \hline
  59 & \texttt{F1} & \texttt{F1} & \texttt{F1} \\ \hline
  60 & \texttt{F2} & \texttt{F2} & \texttt{F2} \\ \hline
  61 & \texttt{F3} & \texttt{F3} & \texttt{F3} \\ \hline
  62 & \texttt{F4} & \texttt{F4} & \texttt{F4} \\ \hline
  63 & \texttt{F5} & \texttt{F5} & \texttt{F5} \\ \hline
  64 & \texttt{F6} & \texttt{F6} & \texttt{F6} \\ \hline
  65 & \texttt{F7} & \texttt{F7} & \texttt{F7} \\ \hline
  66 & \texttt{F8} & \texttt{F8} & \texttt{F8} \\ \hline
  67 & \texttt{F9} & \texttt{F9} & \texttt{F9} \\ \hline
  68 & \texttt{F10} & \texttt{F10} & \texttt{F10} \\ \hline
  69 & \texttt{Num $\Downarrow$} & \texttt{Num Lock} & \texttt{Verr num} \\ \hline
  70 & \texttt{Rollen $\Downarrow$} & \texttt{Scroll Lock} & \texttt{Arrêt défil} \\ \hline
  71 & \texttt{7} (NB) & \texttt{7} (NB) & \texttt{7} (NB) \\ \hline
  72 & \texttt{8} (NB) & \texttt{8} (NB) & \texttt{8} (NB) \\ \hline
  73 & \texttt{9} (NB) & \texttt{9} (NB) & \texttt{9} (NB) \\ \hline
  74 & \texttt{-} (NB) & \texttt{-} (NB) & \texttt{-} (NB) \\ \hline
  75 & \texttt{4} (NB) & \texttt{4} (NB) & \texttt{4} (NB) \\ \hline
  76 & \texttt{5} (NB) & \texttt{5} (NB) & \texttt{5} (NB) \\ \hline
  77 & \texttt{6} (NB) & \texttt{6} (NB) & \texttt{6} (NB) \\ \hline
  78 & \texttt{+} (NB) & \texttt{+} (NB) & \texttt{+} (NB) \\ \hline
  79 & \texttt{1} (NB) & \texttt{1} (NB) & \texttt{1} (NB) \\ \hline
  80 & \texttt{2} (NB) & \texttt{2} (NB) & \texttt{2} (NB) \\ \hline
  81 & \texttt{3} (NB) & \texttt{3} (NB) & \texttt{3} (NB) \\ \hline
  82 & \texttt{0} (NB) & \texttt{0} (NB) & \texttt{0} (NB) \\ \hline
  83 & \texttt{,} (NB) & \texttt{.} (NB) & \texttt{.} (NB) \\ \hline
  86 & \texttt{<} & --- & \texttt{<} \\ \hline
  87 & \texttt{F11} & \texttt{F11} & \texttt{F11} \\ \hline
  88 & \texttt{F12} & \texttt{F12} & \texttt{F12} \\ \hline
  96 & \multicolumn{3}{|l|}{Entrée (NB)} \\ \hline
  97 & \multicolumn{3}{|l|}{Touche Ctrl droite} \\ \hline
  98 & $\div$ (NB) & \texttt{/} (NB) & \texttt{/} (NB) \\ \hline
  99 & \texttt{Druck} & \texttt{Print Screen} & \texttt{Impr écran} \\ \hline
  100 & \multicolumn{3}{|l|}{Touche Alt droite (AltGr)} \\ \hline
  102 & \texttt{Pos 1} & \texttt{Home} & $\nwarrow$ \\ \hline
  103 & $\uparrow$ & $\uparrow$ & $\uparrow$ \\ \hline
  104 & \texttt{Bild} $\uparrow$ & \texttt{Page Up} & $\twoheaduparrow$ \\ \hline
  105 & $\leftarrow$ & $\leftarrow$ & $\leftarrow$ \\ \hline
  106 & $\rightarrow$ & $\rightarrow$ & $\rightarrow$ \\ \hline
  107 & \texttt{Ende} & \texttt{End} & \texttt{Fin} \\ \hline
  108 & $\downarrow$ & $\downarrow$ & $\downarrow$ \\ \hline
  109 & \texttt{Bild} $\downarrow$ & \texttt{Page Down} & $\twoheaddownarrow$ \\ \hline
  110 & \texttt{Einfg} & \texttt{Insert} & \texttt{Inser} \\ \hline
  111 & \texttt{Entf} & \texttt{Delete} & \texttt{Suppr} \\ \hline
  119 & \texttt{Pause} & \texttt{Pause} & \texttt{Pause} \\ \hline
\end{longtable}

Dans le tableau suivant vous avez les codes des touches supplémentaires qui ne sont pas
présent sur tous les claviers. Certaines de ces touches sont généralement générés
uniquement par un pilote de périphérique spécial, comme la touche \texttt{Restart}, une
pression sur cette touche réinitialisation le système.

\begin{longtable}{|l|l|l|l|}
  \hline
  \textbf{Code} & \textbf{dt.} & \textbf{engl.} & \textbf{frz.} \\
  \endhead
  \hline
  116 & \texttt{Power} & \texttt{Power} & \texttt{Power} \\ \hline
  183 & \texttt{F13} & \texttt{F13} & \texttt{F13} \\ \hline
  184 & \texttt{F14} & \texttt{F14} & \texttt{F14} \\ \hline
  185 & \texttt{F15} & \texttt{F15} & \texttt{F15} \\ \hline
  186 & \texttt{F16} & \texttt{F16} & \texttt{F16} \\ \hline
  187 & \texttt{F17} & \texttt{F17} & \texttt{F17} \\ \hline
  188 & \texttt{F18} & \texttt{F18} & \texttt{F18} \\ \hline
  189 & \texttt{F19} & \texttt{F19} & \texttt{F19} \\ \hline
  190 & \texttt{F20} & \texttt{F20} & \texttt{F20} \\ \hline
  191 & \texttt{F21} & \texttt{F21} & \texttt{F21} \\ \hline
  192 & \texttt{F22} & \texttt{F22} & \texttt{F22} \\ \hline
  193 & \texttt{F23} & \texttt{F23} & \texttt{F23} \\ \hline
  194 & \texttt{F24} & \texttt{F24} & \texttt{F24} \\ \hline
  408 & \texttt{Restart} & \texttt{Restart} & \texttt{Restart} \\ \hline
\end{longtable}

\marklabel{sec:developer}{
\section{HWSUPP - Conseil pour les développeurs de paquetage}}
  Dans ce chapitre nous vous proposont une description pour ajouter une LED ou un bouton
  pour les développeurs qui veulent créer un paquetage\footnotemark.

  \footnotetext{Vous recherchez un endroit approprié dans \var{\#\#HWSUPP\#\#}
  pour paramétrer le paquetage WLAN.}

\subsection{Extension pour LED}
\subsubsection{Type de LED}
  Dans le fichier \var{check/myopt.exp} vous configurez une liste de plusieurs types
  de LEDs que vous pouvez indiquer dans la variable \var{HWSUPP\_LED\_x}.

Exemple~:
\begin{verbatim}
+HWSUPP_LED_TYPE(OPT_MYOPT) = 'myopt' 
                            : ', myopt'
\end{verbatim}

\subsubsection{Contrôle de paramètre}
  Dans le fichier \var{check/myopt.ext} vous configurez les paramètres qui seront vérifiés,
  vous pouvez les indiquer dans la variable \var{HWSUPP\_LED\_x\_PARAM}.

Exemple~:
\begin{verbatim}
if (opt_hwsupp)
then
    depends on hwsupp version 4.0

    foreach i in hwsupp_led_n
    do
        set action=hwsupp_led_%[i]
        set param=hwsupp_led_%_param[i]
        if (action == "myopt")
        then
            if (!(param =~ "(RE:MYOPT_LED_PARAM)"))
            then
                error "When HWSUPP_LED_\${i}='myopt', ...
                       must be entered in HWSUPP_LED_\${i}_PARAM" 
            fi
        fi
    done
fi
\end{verbatim}

\subsubsection {affichage de la LED}
  Quand vous définissez une LED dans le script pour un paquetage (par ex. \var{/usr/bin/<opt>\_setled})
  la requête \var{/usr/bin/hwsupp\_setled <LED> <status>/} sera exécutée.

  Le nombre de LED peut être lu dans le fichier \var{/var/run/hwsupp.conf>}.

  L'état de la LED peut être \var{off}, \var{on} ou \var{blink}.

Exemple~:
\begin{verbatim}
if [ -f /var/run/hwsupp.conf ]
then
    . /var/run/hwsupp.conf
    [ 0$hwsupp_led_n -eq 0 ] || for i in `seq 1 $hwsupp_led_n`
    do
        eval action=\$hwsupp_led_${i}
        eval param=\$hwsupp_led_${i}_param
        if [ "$action" = "<opt>" ]
        then
            if [ <myexpression> ]
            then
                /usr/bin/hwsupp_setled $i on
            else
                /usr/bin/hwsupp_setled $i off
            fi
        fi
    done
fi
\end{verbatim}

  L'état actuel d'une LED peut être demandé avec la requête
  \var{/usr/bin/hwsupp\_getled <LED>/}. Le résultat sera
  \var{off}, \var{on} ou \var{blink}.

\subsection{Extension pour le bouton}
\subsection{Action du bouton}
  Dans le fichier \var{check/myopt.exp} vous configurez une liste de plusieurs types
  de boutons que vous pouvez indiquer dans la variable \var{HWSUPP\_BUTTON\_x}.

Exemple~:
\begin{verbatim}
+HWSUPP_BUTTON_TYPE(OPT_MYOPT) = 'myopt' 
                               : ', myopt'
\end{verbatim}

\subsubsection{Contrôle de paramètre}
  Dans le fichier \var{check/myopt.ext} vous configurez les paramètres qui seront vérifiés,
  vous pouvez les indiquer dans la variable \var{HWSUPP\_BUTTON\_x\_PARAM}.

Exemple~:
\begin{verbatim}
if (opt_hwsupp)
then
    depends on hwsupp version 4.0

    foreach i in hwsupp_button_n
    do
        set action=hwsupp_buttonn_%[i]
        set param=hwsupp_button_%_param[i]
        if (action == "myopt")
        then
            add_to_opt "files/usr/bin/myopt_keyprog" "mode=555 flags=sh"
            if (!(param =~ "(RE:MYOPT_BUTTON_PARAM)"))
            then
                error "When HWSUPP_BUTTON_\${i}='myopt', ...
                       must be entered in  HWSUPP_BUTTON_\${i}_PARAM" 
            fi
        fi
    done
fi
\end{verbatim}

\subsubsection {Fonction du bouton}
  Quand un bouton est pressé, le fichier script \var{/usr/bin/myopt\_keyprog}
  sera exécuté.

  La valeur de la variable \var{HWSUPP\_BUTTON\_x\_PARAM} est transmis.

Exemple~:
\begin{verbatim}
##TODO## exemple
\end{verbatim}

