% Synchronized to r0

% ##TRANSLATE## : FFL-1570: begin new package
 
\marklabel{sec:mailsendexample}{
\section{Examples}}
\subsection {\var{mailsend} with text as parameter}
\begin{verbatim}
mailsend -t "test@test.xy" -s "test" "this is a test"
\end{verbatim}

\subsection {\var{mailsend} with text by pipe}
\begin{verbatim}
echo "this is a test" | mailsend -t "test@test.xy" -s "test"
\end{verbatim}

\subsection {\var{mailsend} with text as file}
\begin{verbatim}
echo "this is a test" > /tmp/mail.txt
mailsend -t "test@test.xy" -s "test" -B "/tmp/mail.txt"
\end{verbatim}

\subsection {\var{mailsend} with variables}
\begin{verbatim}
_to=#test@test.xy'
_subject='test'
_body='this is a test'
mailsend -t "${_to}" -s "${_subject}" -B "${_body}"
\end{verbatim}

\subsection {\var{mailsend} with appendix}
\begin{verbatim}
mailsend -t "test@test.xy" -s "syslog" -a "/var/log/syslog" "syslog"
\end{verbatim}

\marklabel{sec:developer}{
\section{MAILSEND - Hints for package developers}}
    This chapter describes the things a package developer has to do
    to add MAILSEND functionality.
 
\subsection{Configuration}
    Within the file \var{check/myopt.txt} the list of configuration variables 
    is to extend by at least two variables:
\begin{verbatim}
MYOPT_MAIL_ACCOUNT      OPT_MYOPT       -       LABEL "default"
MYOPT_MAIL_TO           OPT_MYOPT       -       MAILADDR_LIST 
\end{verbatim}

    If necessaey the indexed variant has to be used:
\begin{verbatim}
MYOPT_%_MAIL_ACCOUNT    OPT_MYOPT   MYOPT_N     LABEL "default"
MYOPT_%_MAIL_TO         OPT_MYOPT   MYOPT_N     MAILADDR_LIST 
\end{verbatim}
    
    Further variables, for e.g. CC or BCC addresses and subject can also be added.
     
    The configuration file \var{config/myopt.txt} has to be extended by 
    the appropriate variables.
    
\begin{verbatim}
MYOPT_MAIL_ACCOUNT='default'
MYOPT_MAIL_TO='receiver@domain.xyz' 
\end{verbatim}
     
\subsection{Parameter check}
    Within the file  \var{check/myopt.ext} the installation of MAILSEND will 
    be checked:
  
\begin{verbatim}
if (my_opt)
then
    [...]
    if (opt_mailsend)
    then
        depends on mailsend version 4.0
        [...]
    else
        error "OPT_MAILSEND must be 'yes' when ...'"
    fi
fi
\end{verbatim}

    Also the mail account should be checked:
\begin{verbatim}
    if (myopt_mail_account != "default")
    then
        set found="no"
        foreach m in MAILSEND_N
        do
            if (mailsend_%_account[m] == myopt_mail_account)
            then
                set found="yes" 
            fi
        done
        if (found != "yes")
        then
            error "Account '${MYOPT_MAIL_ACCOUNT}' not found in MAILSEND_x_ACCOUNT"
        fi
    fi
\end{verbatim}

\subsection{Mail transmission}
    Within a script subject and text for a mail shall be generated
    and passed to \var{mailsend}:
    
\begin{verbatim}
. /var/run/myopt.conf
_subject='myopt mail subject'
_message='myopt mail message'
mailsend -A "${MYOPT_MAIL_ACCOUNT}" -t "${MYOPT_MAIL_TO}" \
    -s "${_subject}" "${_message}"
\end{verbatim}

\marklabel{sec:mailsendappendix}{
\section{Appendix}}
 
\subsection{Credits}

  The package MAILSEND is baes on the package MSMTP 
  written by Christph Fritsch {[\ref{opt_msmtp}]}.

  The mails are sent by msmtp {[\ref{msmtp}]}.
  
\subsection{References}

\newcounter{ref}
\begin{list}{\textbf{[\arabic{ref}]}}{\usecounter{ref}}
  
  \item \label{opt_msmtp}
  \altlink{http://extern.fli4l.de/fli4l_opt-db3/download.pl?file=2780-opt_msmtp-3.10.1-0.9.13_x86.zip}
  
  \item \label{msmtp}
  \altlink{http://msmtp.sourceforge.net/}

    \item \label{rfc5322}
    \altlink{https://tools.ietf.org/pdf/rfc5322}
\end{list}

% ##TRANSLATE## : FFL-1570: end
