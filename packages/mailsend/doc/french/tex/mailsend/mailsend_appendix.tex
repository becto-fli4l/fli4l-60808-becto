% Synchronized to r43684

\marklabel{sec:mailsendexample}{
\section{Examples}}
\subsection {\var{mailsend} avec texte et paramètre}
\begin{verbatim}
mailsend -t "test@test.xy" -s "test" "ceci est un test"
\end{verbatim}

\subsection {\var{mailsend} avec texte et pipe}
\begin{verbatim}
echo "ceci est un test" | mailsend -t "test@test.xy" -s "test"
\end{verbatim}

\subsection {\var{mailsend} avec fichier texte}
\begin{verbatim}
echo "ceci est un test" > /tmp/mail.txt
mailsend -t "test@test.xy" -s "test" -B "/tmp/mail.txt"
\end{verbatim}

\subsection {\var{mailsend} avec variables}
\begin{verbatim}
_to=#test@test.xy'
_subject='test'
_body='this is a test'
mailsend -t "${_to}" -s "${_subject}" -B "${_body}"
\end{verbatim}

\subsection {\var{mailsend} avec annexe}
\begin{verbatim}
mailsend -t "test@test.xy" -s "syslog" -a "/var/log/syslog" "syslog"
\end{verbatim}

\marklabel{sec:developer}{
\section{MAILSEND - Conseil pour les développeurs de paquetage}}
    Ce chapitre est pour les développeurs, qui veulent intéger
	le fonctionnement de MAILSEND à leur paquetage.

\subsection{Configuration}
    Vous pouvez étendre la liste des variables de configuration dans
	le fichier \var{check/myopt.txt} qui comporte déjà deux variables~:

\begin{verbatim}
MYOPT_MAIL_ACCOUNT      OPT_MYOPT       -       LABEL "default"
MYOPT_MAIL_TO           OPT_MYOPT       -       MAILADDR_LIST 
\end{verbatim}

    Si c'est nécessaire, vous pouvez utiliser les variantes indexées~:

\begin{verbatim}
MYOPT_%_MAIL_ACCOUNT    OPT_MYOPT   MYOPT_N     LABEL "default"
MYOPT_%_MAIL_TO         OPT_MYOPT   MYOPT_N     MAILADDR_LIST 
\end{verbatim}

    Vous pouvez ajouter d'autres variables, comme par exemple l'adresse CC ou BCC.

    Vous pouvez aussi étendre la liste des variables dans le fichier de configuration
	\var{config/myopt.txt} qui comporte déjà deux variables~:

\begin{verbatim}
MYOPT_MAIL_ACCOUNT='default'
MYOPT_MAIL_TO='receiver@domain.xyz' 
\end{verbatim}

\subsection{Vérification des paramètres}
	Le fichier \var{check/myopt.ext} vérifie si MAILSEND est activé pour
	l'installation~:

\begin{verbatim}
if (my_opt)
then
    [...]
    if (opt_mailsend)
    then
        depends on mailsend version 4.0
        [...]
    else
        error "OPT_MAILSEND must be 'yes' when ...'"
    fi
fi
\end{verbatim}

    De même, le compte de messagerie doit être vérifiée pour être validé~:

\begin{verbatim}
    if (myopt_mail_account != "default")
    then
        set found="no"
        foreach m in MAILSEND_N
        do
            if (mailsend_%_account[m] == myopt_mail_account)
            then
                set found="yes" 
            fi
        done
        if (found != "yes")
        then
            error "Account '${MYOPT_MAIL_ACCOUNT}' not found in MAILSEND_x_ACCOUNT"
        fi
    fi
\end{verbatim}

\subsection{Envoi d'un courriel}
	Dans ce script, vous indiquez l'objet, le corps du courriel et les paramètres
	pour l'envoi du courriel avec \var{mailsend}~:

\begin{verbatim}
. /var/run/myopt.conf
_subject='myopt mail subject'
_message='myopt mail message'
mailsend -A "${MYOPT_MAIL_ACCOUNT}" -t "${MYOPT_MAIL_TO}" \
    -s "${_subject}" "${_message}"
\end{verbatim}

\marklabel{sec:mailsendappendix}{
\section{Complément}}

\subsection{Merci}

  Le paquetage MAILSEND a été créé sur la base du paquetage MSMTP
  de Christph Fritsch {[\ref{opt_msmtp}]}.

  Le courriel est expédié avec msmtp {[\ref{msmtp}]}.

\subsection{Référence}

\newcounter{ref}
\begin{list}{\textbf{[\arabic{ref}]}}{\usecounter{ref}}

  \item \label{opt_msmtp}
  \altlink{http://extern.fli4l.de/fli4l_opt-db3/download.pl?file=2780-opt_msmtp-3.10.1-0.9.13_x86.zip}

  \item \label{msmtp}
  \altlink{http://msmtp.sourceforge.net/}

    \item \label{rfc5322}
    \altlink{https://tools.ietf.org/pdf/rfc5322}
\end{list}

