% Synchronized to r29817

\marklabel{sec:opt-easycron}
{
\section {EASYCRON - Time-based Job Scheduling}
}

This package was inititated by Stefan Manske
\email{fli4l@stephan.manske-net.de} and adapted to version 2.1 by 
the fli4l-team.


\subsection{Configuration}


       By using \var{OPT\_\-EASYCRON} it is possible to execute commands 
       at designated times by adding them to the config-file. 

       The following entries are supported:


\begin{description}
\config{OPT\_EASYCRON}{OPT\_EASYCRON}{OPTEASYCRON} \verb*?OPT_EASYCRON='yes'? activates the package

         Default setting: \verb*?OPT_EASYCRON='no'?


\config{EASYCRON\_MAIL}{EASYCRON\_MAIL}{EASYCRONMAIL}
         Avoid the sending of unwanted mails by crond. 

         Default setting: \verb*?EASYCRON_MAIL='no'?


\config{EASYCRON\_N}{EASYCRON\_N}{EASYCRONN}
         The number of commands to be executed by cron.


\config{EASYCRON\_x\_CUSTOM}{EASYCRON\_x\_CUSTOM}{EASYCRONxCUSTOM}
         Those familiar with the settings in crontab may define additional 
         settings for each entry (like MAILTO, PATH...). Entries have to be 
         separated by \var{$\backslash\backslash$}. You should be really 
         familiar with cron to use this option.

         Default setting: \verb*?EASYCRON_CUSTOM=''?


\config{EASYCRON\_x\_COMMAND}{EASYCRON\_x\_COMMAND}{EASYCRONxCOMMAND}
         Specify the command to be executed in \var{EASYCRON\_\-x\_\-COMMAND},
         for example:
\begin{example}
\begin{verbatim}
        EASYCRON_1_COMMAND='echo 1 '>' /dev/console'
\end{verbatim}
\end{example}

\config{EASYCRON\_x\_TIME}{EASYCRON\_x\_TIME}{EASYCRONxTIME}
        specifies execution times according to the cron syntax.

\subsection{Examples}

\begin{itemize}
\item The computer outputs ``Happy new year!'' to the console
\begin{example}
\begin{verbatim}
        EASYCRON_1_COMMAND = 'echo Happy new year! > /dev/console'
        EASYCRON_1_TIME    = '0 0 31 12 *'
\end{verbatim}
\end{example}



\item   xxx will be executed Monday to Friday from 7AM to 8PM Uhr every full hour.
\begin{example}
\begin{verbatim}
        EASYCRON_1_COMMAND = 'xxx'
        EASYCRON_1_TIME    = '0 7-20 0 * 1-5'
\end{verbatim}
\end{example}



\item  The router terminates the DSL internet connection each night at 03:40AM
       and reestablishes it after 5 seconds.
       Device names to be used: pppoe, ippp[1-9], ppp[1-9].
\begin{example}
\begin{verbatim}
        EASYCRON_1_COMMAND = 'fli4lctrl hangup pppoe; sleep 5; fli4lctrl dial pppoe'
        EASYCRON_1_TIME    = '40 3 * * *'
\end{verbatim}
\end{example}


\end{itemize}



       Further informations on the cron syntax can be found here (german)
       \begin{itemize}
       \item \altlink{http://www.pro-linux.de/artikel/2/146/der-batchdaemon-cron.html} 
       \item \altlink{http://de.linwiki.org/wiki/Linuxfibel_-_System-Administration_-_Zeit_und_Steuerung\#Die_Datei_crontab} 
       \item \altlink{http://web.archive.org/web/20021229004331/http://www.linux-magazin.de/Artikel/ausgabe/1998/08/Cron/cron.html}
       \item \altlink{http://web.archive.org/web/20070810063838/http://www.newbie-net.de/anleitung_cron.html}
       \end{itemize}





\subsection{Prerequisites}

\begin{itemize}
\item fli4l version $>$ 2.1.0    
\item for older versions please use opt\_easycron version from the OPT-Database
\end{itemize}



\subsection{Installation}

Unzip \var{OPT\_\-EASYCRON} to your fli4l directory like any other fli4l opt.


\end{description}
