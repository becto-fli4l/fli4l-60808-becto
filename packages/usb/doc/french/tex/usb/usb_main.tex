% Do not remove the next line
% Synchronized to r46491

\marklabel{sec:opt-usb }
{
\section {USB~- Support pour périphérique USB}
}
\begin{description}

\config{OPT\_USB}{OPT\_USB}{OPTUSB}

        La prise en charge des périphériques USB n'est pas activée automatiquement.
        Vous devez placer ici 'yes', pour que les périphériques-USB puissent être
        utilisés. Si vous avez choisi dans le fichier base.txt un Périphérique-USB,
        vous devez indiquer impérativement 'yes'. Autrement, le périphérique ne
        sera pas utilisable. si vous avez activez cette variable, les supports
        USB tels quel les clés, les disques durs externes, les claviers seront
        activés.

        Installation par défaut~: \var{OPT\_USB}='no'

\config{USB\_EXTRA\_DRIVER\_N}{USB\_EXTRA\_DRIVER\_N}{USBEXTRADRIVERN}

        Vous indiquez ici le nombre de périphérique spécifique à charger.

        Installation par défaut~: \var{USB\_EXTRA\_DRIVER\_N}='0'

\config{USB\_EXTRA\_DRIVER\_x}{USB\_EXTRA\_DRIVER\_x}{USBEXTRADRIVERx}

        Vous indiquez ici les périphériques spécifique qui seront chargés.

        Valeurs possibles en ce moment~:
        \begin{itemize}
	    \item printer~- Prise en charge des imprimantes-USB
	    \item belkin\_sa~- USB Belkin Serial converter
	    \item cyberjack~- REINER SCT cyberJack pinpad/e-com USB Chipcard Reader
	    \item digi\_acceleport~- Digi AccelePort USB-2/USB-4 Serial Converter
	    \item empeg~- USB Empeg Mark I/II
	    \item ftdi\_sio~- USB FTDI Serial Converter
	    \item io\_edgeport~- Edgeport USB Serial
	    \item io\_ti~- Edgeport USB Serial
	    \item ipaq~- USB PocketPC PDA
	    \item ir-usb~- USB IR Dongle
	    \item keyspan~- Keyspan USB to Serial Converter
	    \item keyspan\_pda~- USB Keyspan PDA Converter
	    \item kl5kusb105~- KLSI KL5KUSB105 chipset USB->Serial Converter
	    \item kobil\_sct~- KOBIL USB Smart Card Terminal (experimental)
	    \item mct\_u232~- Magic Control Technology USB-RS232 converter
	    \item omninet~- USB ZyXEL omni.net LCD PLUS
	    \item pl2303~- Prolific PL2303 USB to serial adaptor
	    \item visor~- USB HandSpring Visor / Palm OS
	    \item whiteheat~- USB ConnectTech WhiteHEAT
        \end{itemize}

        Installation par défaut~: \var{USB\_EXTRA\_DRIVER\_x}=''

\config{USB\_EXTRA\_DRIVER\_x\_PARAM}{USB\_EXTRA\_DRIVER\_x\_PARAM}{USBEXTRADRIVERxPARAM}

        Paramètre supplémentaire pour les pilotes. Normalement, vous ne devez
        rien enregistrer ici.

        Installation par défaut~: \var{USB\_EXTRA\_DRIVER\_x\_PARAM}=''

\config{USB\_MODEM\_WAITSECONDS}{USB\_MODEM\_WAITSECONDS}{USBMODEMWAITSECONDS}

        Installation par défaut~: \var{USB\_MODEM\_WAITSECONDS}='21'

        Malheureusement, le modem USB Speedtouch a besoin d'une
        "demi-éternité" avant que ceux-ci soit prêts. Normalement pour une
        installation standard il suffit de 21 secondes d'initialisation. Parfois
        on a de la chance et l'on peut diviser cette valeur par deux avec le
        modem USB Speedtouch, il sera disponible après 10 secondes,
        vous pouvez indiquer alors 10 secondes. Mais si vous n'avez pas de chance,
        vous devez augmenter cette valeur. Malheureusement, nous ne pouvons pas
        vous aider, il vous faut essayer pour trouver la bonne valeur.

\end{description}

\subsection{Problèmes avec les périphériques USB}

Si vous avez des problèmes avec certains appareils USB. Cela peut provenir de
différentes raisons, comme par exemple le pilote de périphérique ou le contrôleur
de gestion USB.

\subsection{Précautions d'utilisation}

Il faut faire attention à ce que le périphérique USB soit activé, dû côte
contrôleur-hardware. Par exemple avec la carte WRAP elle est livrée sans module
USB et peut être complétée par un module USB supplémentaire. C'est pour cette
raison que le périphérique USB est désactivé par défaut dans BIOS.

\subsection{Monter les périphériques USB}

Les appareils USB insérés sont reconnus certes, automatiquement, toutefois vous
devez les déclarer manuellement. Par exemple à l'insertion d'une clé-USB, elle
est reconnu comme un périphérique-SCSI. C'est pour cette raison que l'accès se
fait via le périphérique sd\verb=#= comme une super disquette ou via
sd\verb=#=$<$numéro de la partition$>$ pour la clé-USB avec une table de partitions.
Les clés-USB partitionnées sont traités comme des disques durs, c'est-à-dire
lors du branchement sur l'un des deux ports USB ils seront indiqués sda1 et sdb1.
En revanche pour les clés-USB ils seront indiqués sda ou sdb, donc sans
l'indication du numéro de partition.

Cela permet de monter la clé-USB avec la commande suivante~:

mount /dev/sda1 /mnt

Après /mnt la deuxième clé sera monté de la même façon~:

mount /dev/sdb1 /mnt

Au sujet de la seconde clé-USB. Les clés sont spécifiées dans l'ordre d'insertion,
donc la première clé-USB = sda, la deuxième clè-USB = sdb etc. on ne peut pas
défini les ports-USB par rapport au 'nom', puisque cela dépend de l'ordre
d'insertion des clés. Le démontage d'une clé-USB se fait par~:

umount /mnt

Vous devez absolument éviter d'introduire simultanément de plusieurs clés-USB,
dans le but de tous les monter. Je vous propose une solution, vous devez créer
des sous répertoires dans le répertoire /mnt, dans lesquelles, les clés pourront
alors être montée. Par exemple Cela peut être réalisé, comme indiqué ci-dessous~:

mkdir /mnt/usba
mkdir /mnt/usbb

Lorque l'on monte des clés-USB ces répertoires seront alors définit comme
indiqué ci-dessous~:

mount /dev/sda1 /mnt/usba
mount /dev/sdb1 /mnt/usbb

Ainsi, le contenu des clés-USB se trouve dans les sous répertoires /mnt/usba ou
/mnt/usbb. Pour le démontage des clés-USB on utilise la commande~:

umount /mnt/usba
umount /mnt/usbb

S'il existe plusieurs partitions sur la clé-USB, il faudra structurer en
conséquence les sous répertoires dans le répertoire /mnt.
