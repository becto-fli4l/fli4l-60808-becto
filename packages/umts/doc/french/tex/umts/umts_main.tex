% Do not remove the next line
% Synchronized to r43711


\marklabel{sec:opt-umts}
{
\section {UMTS~- Connexion UMTS via Internet}
}

Utiliser fli4l pour la connexion Interned via l'UMTS (Universal Mobile
Telecommunications System). D'autres paquerages optionnels sont nécessaires pour
cette installation.

\subsection{Configuration}

\begin{description}
\config{OPT\_UMTS}{OPT\_UMTS}{OPTUMTS}

  Configuration standard~: \var{OPT\_UMTS}='no'

  Indiquer 'yes' pour activer le paquetage.

\config{UMTS\_DEBUG}{UMTS\_DEBUG}{UMTSDEBUG}

  Configuration standard~: \var{UMTS\_DEBUG}='no'

  Pour afficher des informations de débogage supplémentaires avec pppd, vous
  devez définir la variable \var{UMTS\_DEBUG} sur 'oui'. Dans ce cas, les
  informations écrites sur pppd seront utilisables dans l'interface syslog.

  IMPORTANT~: Pour que les informations soit délivrées le démon syslogd doit
  être activé avec la variable \var{OPT\_SYSLOGD} (voir ci-dessus) elle doit
  être également paramétrée sur 'yes'.

\config{UMTS\_PIN}{UMTS\_PIN}{UMTSPIN}

  Configuration standard~: \var{UMTS\_PIN}='disabled'

  Pin pour carte-SIM

  Vous devez indiquer un numéro à 4 chiffres ou le mot 'disabled'

\config{UMTS\_DIALOUT}{UMTS\_DIALOUT}{UMTSDIALOUT}

  Configuration standard~: \var{UMTS\_DIALOUT}='*99***1\#'

  Propriétés de numérotation pour établir la connexion

\config{UMTS\_GPRS\_UMTS}{UMTS\_GPRS\_UMTS}{UMTSGPRSUMTS}

  Configuration standard~: \var{UMTS\_GPRS\_UMTS}='both'

  Dans cette variable vous indiquez le paramètre pour l'utilisation de la
  transmission, valeurs autorisées (both, GPRS, UMTS)

\config{UMTS\_APN}{UMTS\_APN}{UMTSAPN}

  Configuration standard~: \var{UMTS\_APN}='web.vodafone.de'

\config{UMTS\_USER}{UMTS\_USER}{UMTSUSER}

  Configuration standard~: \var{UMTS\_USER}='anonymer'

\config{UMTS\_PASSWD}{UMTS\_PASSWD}{UMTSPASSWD}

  Configuration standard~: \var{UMTS\_PASSWD}='surfer'

  Vous indiquez dans ces variables les données nécessaires pour la connexion.

  vous devez indiquer, l'ID de l'utilisateur et le mot de passe fournie par le
  fournisseur. la variable \var{UMTS\_USER} contient l'ID de l'utilisateur, la
  variable \var{UMTS\_PASSWD} contient le mot de passe.

  Pour certains opérateurs allemands, les fournisseurs d'accès réseaux sont les
  APNs (noeuds de connexion)

  \begin{table}
  \textbf{Accès de certains opérateurs de distributeur de réseau Allemand }

  \vspace{1ex}
  \begin{tabular}{llll}
  Fournisseur         &APN                   &Nom d'utilisateur &Mot de passe \\
  T-Mobile            &Internet.t-mobile     &libre     &mot de passe \\
  Vodafone            &Web.vodafone.de       &libre     &mot de passe \\
  E-Plus              &Internet.eplus.de     &eplus     &gprs \\
  O2 (Vertragskunden) &Internet              &libre     &mot de passe \\
  O2 (Prepaid-Kunden) &Pinternet.interkom.de &libre     &mot de passe \\
  Alice               &Internet.partner1     &libre     &mot de passe \\
  \end{tabular}
  \end{table}

  \begin{itemize}
  \item \altlink{http://www.teltarif.de/mobilfunk/internet/einrichtung.html}
  \end{itemize}

\config{UMTS\_NAME}{UMTS\_NAME}{UMTSNAME}

  Configuration standard~: \var{UMTS\_NAME}='UMTS'

  Vous indiquez dans ces variables, un nom pour le circuit~- max. 15 caractères.
  Celui-ci sera affiché dans le client-imon d'imonc. les Espaces (blancs) ne sont
  pas autorisés.

\config{UMTS\_HUP\_TIMEOUT}{UMTS\_HUP\_TIMEOUT}{UMTSHUPTIMEOUT}

  Configuration standard~: \var{UMTS\_TIMEOUT}='600'

  Vous indiquez dans cette variables, le temps en secondes, qui déterminera la
  fin de la connexion, si aucune communication ne circule sur la connexion UMTS.
  Si vous indiquez '0' vous avez aucun délai de fin de connexion.

\config{UMTS\_TIMES}{UMTS\_TIMES}{UMTSTIMES}

  Configuration standard~: \var{UMTS\_TIMES}='Mo-Su:00-24:0.0:Y'

  Déterminer les heures et jours d'activation pour ce circuit et combien cela
  coûte. Il est donc possible d'utiliser à des moments différents, différents
  circuits avec la route par défaut c'est du (Least Cost Routing). Dans ce cas,
  le démon imond contrôle la cession de route du circuit, si vous l'avez activé.

\config{UMTS\_CHARGEINT}{UMTS\_CHARGEINT}{UMTSCHARGEINT}

  Configuration standard~: \var{UMTS\_CHARGEINT}='60'

  Vous indiquez dans cette variables, l'interval de charge~: temps d'une unite
  de connexion en secondes. Elle est ensuite utilisée pour le calcul des coûts.

\config{UMTS\_USEPEERDNS}{UMTS\_USEPEERDNS}{UMTSUSEPEERDNS}

  Configuration standard~: \var{UMTS\_USEPEERDNS}='yes'

  Vous indiquez dans cette variables, le DNS qu'utilise le fournisseur.

\config{UMTS\_FILTER}{UMTS\_FILTER}{UMTSFILTER}

  Configuration standard~: \var{UMTS\_FILTER}='yes'

  fli4l raccroche automatiquement après le délai spécifié dans timeout, si aucune
  information ne circule sur l'interface ppp0. Malheureusement, l'interface
  évalue également le transfert de données, qui proviennent de l'extérieur, par
  exemple les tentatives de connexion par un des clients P2P tels que eDonkey.
  Comme on est en fait en permanence contacté par les autres, il peut arriver
  que fli4l ne termine jamais le contact d'UMTS.

  C'est là qu'intervient la variable \var{UMTS\_FILTER}. Vous devez placer la
  variable sur 'yes', ce filtre estime seulement la circulation générée pa
  votre propre machine,le trafic extérieur sera ignoré complètement. Puisque le
  trafic entrant "de l'extérieur" traverse habituellement le routeur et que les
  ordinateurs derrière celui-ci réagissent, en refusant par ex. des demandes de
  connexion, les paquets sortants sont aussi ignorés.

\config{UMTS\_ADAPTER}{UMTS\_ADAPTER}{UMTSADAPTER}

  (optionnelle)

  Ici vous indiquez s'il s'agit d'une carte PCMCIA, un adaptateur USB ou un câble
  USB pour Téléphone.

  Si la variable n'est pas paramétrée, seuls les fichiers nécessaires pour
  l'adaptateur USB sont copiés.

  Valeurs admissibles~: (PCMCIA, usbstick, usbphone)

\textbf{Toutes les variables suivantes sont facultatives et sont nécessaire que si la
  détection automatique a échouée.}

\config{UMTS\_IDVENDOR}{UMTS\_IDVENDOR}{UMTSIDVENDOR}

  (optionnelle) \var{UMTS\_IDVENDOR}='xxxx'

  L'ID (ou Identification) du fabricant, après avoir allumer l'adaptateur

\config{UMTS\_IDDEVICE}{UMTS\_IDDEVICE}{UMTSIDDEVICE}
  (optionnelle) \var{UMTS\_IDDEVICE}='xxxx'

  L'ID du produit, après avoir allumer l'adaptateur

  Les informations des deux paramètres suivants sont nécessaire seulement, si
  l'identification change après l'initialisation.

\config{UMTS\_IDVENDOR2}{UMTS\_IDVENDOR2}{UMTSIDVENDOR2}

  (optionnelle) \var{UMTS\_IDVENDOR2}='xxxx'

  L'ID du fabricant, après l'initialisation de l'adaptateur

\config{UMTS\_IDDEVICE2}{UMTS\_IDDEVICE2}{UMTSIDDEVICE2}

  (optionnelle) \var{UMTS\_IDDEVICE2}='xxxx'

  L'ID du produit, après l'initialisation de l'adaptateur

\config{UMTS\_DRV}{UMTS\_DRV}{UMTSDRV}

  (optionnelle) \var{UMTS\_DRV}='xxxx'

  Pilotes pour guider l'adaptateur lorsque 'usbserial' n'est pas spécifié

\config{UMTS\_SWITCH}{UMTS\_SWITCH}{UMTSSWITCH}

  (optionnelle) \var{UMTS\_SWITCH}='-v 0x0af0 -p 0x6971 -M 555...000 -s 10'

  Paramètres pour passer en mode USB pour l'initialisation du modem. 
  (Voir le site usb-modeswitch).

  Tous les modems sur ce site Web devaient être reconnus automatiquement, à
  quelques exceptions près.

  \begin{itemize}
  \item \altlink{http://www.draisberghof.de/usb_modeswitch/}
  \end{itemize}

\config{UMTS\_DEV}{UMTS\_DEV}{UMTSDEV}

	(Optionnelle)

  En cas de problèmes, l'interface de données pppd peut être indiquée ici. Pour
  les adaptateurs, ce sont la plupart du temps des suivants~:

  \begin{verbatim}
  ttyUSB0 pour usbstick
  ttyS2   pour pcmcia
  ttyACM0 pour usbphone
  \end{verbatim}

\config{UMTS\_CTRL}{UMTS\_CTRL}{UMTSCTRL}

  (optionnelle)

  Certaines cartes ont des interfaces multiples à travers duquel le modem est
  contrôlé. Les informations sur le statut seront disponibles et lues seulement
  en 'offline' (ou déconnexion). Lors de la fusion des options de l'interface
  UMTS Quad par ex.~: ttyUSB2

\end{description}

