% Do not remove the next line
% Synchronized to r29821

\section {UMTS}

\subsection{Matériel pris en charge}

\begin{verbatim}
 Les paquetages supplémentaires sont nécessaires pour le bon fonctionnement de
 UMTS. Pour les adaptateurs-USB le paquetage USB doit être activé avec
 OPT_USB ='yes'. Ce paquetage prend en charge le matériel UMTS suivant~:

Matériel               tester            paquetage supplémentaire

Adaptateur Novatel~:
Merlin U530             oui              PCMCIA, TOOLS (serial)
Merlin U630             non              PCMCIA, TOOLS (serial)
MC950D                  oui              USB

OPTION d'adaptateur~:
3G Datacard             non              PCMCIA, USB
GT 3G Quad              oui              PCMCIA, USB
GT Fusion               non              PCMCIA, USB
GT MAX HSUPA GX0301     oui              PCMCIA, USB
Pour ces quatre adaptateurs Cardbus vous devez indiquer PCMCIA_PCIC='yenta_socket'

Icon 225 (GIO225)       oui              USB

Adaptateur Huawei~:
E220, E230, E270        oui              USB
E510                    oui              USB
E800                    oui              USB
K3520                   oui              USB

Adaptateur ZTE~:
MF110                   oui              USB
MF190                   oui              USB
\end{verbatim}


\subsection{Si l'interface modem n'est pas activée}

 Il peut arriver, que certaines OPTIONS du Sticks UMTS ne soient pas activées
 pour l'interface du modem, ces options sont nécessaires pour le protocole pppd.

\begin{verbatim}
Exemple d'utilisation pour l'adaptateur GIO225

Contrôler avec la commande suivante~:
grep "" /sys/bus/usb/devices/*/tty/*/hsotype

La sortie sur la console devrait ressembler à ceci~:
/sys/bus/usb/devices/2-1:1.0/tty/ttyHS0/hsotype:Control
/sys/bus/usb/devices/2-1:1.0/tty/ttyHS1/hsotype:Application
/sys/bus/usb/devices/2-1:1.1/tty/ttyHS2/hsotype:Diagnostic

Dans cette distribution l'entrée modem "hsotype:Modem" existent pas.

Maintenant, vous pouvez vérifier la configuration de l'interface avec la
commande suivante~:
chat -e -t 1 '' "AT_OIFC?" OK >/dev/ttyHS0 </dev/ttyHS0

La sortie sur la console devrait ressembler à ceci~:
AT_OIFC?
_OIFC: 3,1,1,0

OK

ou à ceci~:
AT_OIFC?
_OIFC: 2,1,1,0

OK

Vous pouvez activer l'interface du modem avec la commande suivante~:
chat -e -t 1 '' "AT_OIFC=3,1,1,0" OK >/dev/ttyHS0 </dev/ttyHS0

Ensuite, retirer l'adaptateur et rebranchez le à nouveau, ensuite refaite
un contrôle~:
grep "" /sys/bus/usb/devices/*/tty/*/hsotype

Maintenant l'entrée modem existent.
/sys/bus/usb/devices/2-1:1.0/tty/ttyHS0/hsotype:Control
/sys/bus/usb/devices/2-1:1.0/tty/ttyHS1/hsotype:Application
/sys/bus/usb/devices/2-1:1.1/tty/ttyHS2/hsotype:Diagnostic
/sys/bus/usb/devices/2-1:1.2/tty/ttyHS3/hsotype:Modem
\end{verbatim}

