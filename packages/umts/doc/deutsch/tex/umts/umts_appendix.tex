% Last Update: $Id$
\section {UMTS}

\subsection{Unterstützte Hardware}

Dieses Paket unterstützt die im Folgenden gelistete UMTS-Hardware.
Für den Betrieb sind unter anderem auch weitere Pakete erforderlich.
Für USB-Adapter muss das USB-Paket via \verb+OPT_USB='yes'+ aktiviert werden.

\begin{verbatim}
Hardware:               getestet        zusätzliche Pakete

Novatel Adapter:
Merlin U530             ja              PCMCIA, TOOLS (serial)
Merlin U630             nein            PCMCIA, TOOLS (serial)

MC950D                  ja              USB


OPTION Adapter:
3G Datacard             nein            PCMCIA, USB
GT 3G Quad              ja              PCMCIA, USB
GT Fusion               nein            PCMCIA, USB
GT MAX HSUPA GX0301     ja              PCMCIA, USB
\end{verbatim}
(bei den vier Cardbusadaptern ist \verb+PCMCIA_PCIC='yenta_socket'+ zu setzen)

\begin{verbatim}
Icon 225 (GIO225)       ja              USB

Huawei Adapter:
E220, E230, E270        ja              USB
E510                    ja              USB
E800                    ja              USB
K3520                   ja              USB

ZTE Adapter:
MF110                   ja              USB
MF190                   ja              USB

Karten von verschiedenen Herstellern basierend auf den Chipsätzen:
GOBI 1000               ja              USB
GOBI 2000               ja              USB
Es werden zusätzlich die entsprechende Firmwaredateien im Verzeichnis
<config-dir>/lib/firmware/gobi/ benötigt. Wie man diese findet und bereitstellt
kann über eine Suchmaschine festgestellt werden: Begriffe gobi_loader, firmware
\end{verbatim}

\subsection{Modemschnittstelle nicht aktiviert}

Bei einigen UMTS-Sticks des Typs OPTION kann es vorkommen, dass die
Modemschnittstelle nicht aktiviert ist, welche aber für den PPP-Dämon benötigt
wird.

\subsection{Beispiel}

Es folgt ein Beispiel anhand eines GIO225 Adapters. Zuerst werden die
verfügbaren Schnittstellen geprüft:
\begin{verbatim}
grep "" /sys/bus/usb/devices/*/tty/*/hsotype
\end{verbatim}

Die Ausgabe sollte etwa so aussehen:
\begin{verbatim}
/sys/bus/usb/devices/2-1:1.0/tty/ttyHS0/hsotype:Control
/sys/bus/usb/devices/2-1:1.0/tty/ttyHS1/hsotype:Application
/sys/bus/usb/devices/2-1:1.1/tty/ttyHS2/hsotype:Diagnostic
\end{verbatim}

Hier fehlt die Ausgabe \verb+hsotype:Modem+.

Jetzt kann man mit dem folgenden Befehl die Interface-Konfiguration
abfragen:
\begin{verbatim}
chat -e -t 1 '' "AT_OIFC?" OK >/dev/ttyHS0 </dev/ttyHS0
\end{verbatim}

Die Ausgabe sollte folgendermassen aussehen:
\begin{verbatim}
AT_OIFC?
_OIFC: 3,1,1,0

OK
\end{verbatim}

Sollte dort Folgendes stehen:
\begin{verbatim}
AT_OIFC?
_OIFC: 2,1,1,0

OK
\end{verbatim}

kann man die Modemschnittstelle mit dem folgenden Befehl aktivieren:
\begin{verbatim}
chat -e -t 1 '' "AT_OIFC=3,1,1,0" OK >/dev/ttyHS0 </dev/ttyHS0
\end{verbatim}

Danach den Adapter noch einmal abziehen und neu anstecken.
Jetzt sollte mittels
\begin{verbatim}
grep "" /sys/bus/usb/devices/*/tty/*/hsotype
\end{verbatim}

auch ein Modemeintrag vorhanden sein:
\begin{verbatim}
/sys/bus/usb/devices/2-1:1.0/tty/ttyHS0/hsotype:Control
/sys/bus/usb/devices/2-1:1.0/tty/ttyHS1/hsotype:Application
/sys/bus/usb/devices/2-1:1.1/tty/ttyHS2/hsotype:Diagnostic
/sys/bus/usb/devices/2-1:1.2/tty/ttyHS3/hsotype:Modem
\end{verbatim}
