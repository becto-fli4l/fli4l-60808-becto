% Last Update: $Id$
\marklabel{sec:opt-wol}
{
\section {WOL - Wake On LAN}
}
OPT\_WOL erweitert fli4l um die Möglichkeit Rechner, die mit einer Wake on LAN
fähigen Netzwerkkarte ausgerüstet sind, über das Konsolenkommando 'wol.sh' oder
über das WebInterface vom Router zu booten.

Damit das klappt, muss die Netzwerkkarte normalerweise mit einem kleinen
dreiadrigen Kabel mit dem Mainboard verbunden sein, damit die Netzwerkkarte
auch bei abgeschaltetem Rechner vom ATX-Netzteil mit StandBy-Strom versorgt
werden kann.

\subsection {Konfiguration}

\begin{description}

\config{OPT\_WOL}{OPT\_WOL}{OPTWOL}

	Standard-Wert:  \var{OPT\_WOL}='no'

	Die Einstellung \var{'no'} deaktiviert das OPT\_WOL Paket vollständig.
	Es werden keine Änderungen am der fli4l Bootmedium bzw. dem Archiv
	\var{opt.img} vorgenommen.\\
	Die Einstellung \var{'yes'} aktiviert das OPT\_WOL Paket.

	Damit ein Client per WOL angeschaltet werden kann muss in der
	$<$config-dir$>$/dns\_dhcp.txt seine
	MAC-Adresse (HOST\_x\_MAC) angegeben sein.
	Alle Rechner für die keine MAC-Adresse angegeben ist werden automatisch von
	WOL ausgeschlossen.

\config{WOL\_LIST}{WOL\_LIST}{WOLLIST}

    Die Konfiguration erfolgt über black oder whitelisting.
    blacklisting bedeutet das alle clients auf dieser Liste von WOL ausgenommen
    sind, whitelisting bedeutet das für die clients auf der Liste WOL möglich ist.

	Standard-Wert:  \var{WOL\_LIST}='black'

	Gültige Werte:
	\begin{itemize}
		\item black - bedeutet das alle Clients auf dieser Liste nicht geweckt werden können
		\item white - bedeutet das alle Clients auf dieser Liste geweckt werden können
	\end{itemize}

\config{WOL\_LIST\_N}{WOL\_LIST\_N}{WOLLISTN}

	Standard-Wert:  \var{WOL\_LIST\_N}='0'

	In der Default Einstellung steht also kein Client auf der Blacklist, somit kann
	jeder Client per WOL angeschaltet werden.

\config{WOL\_LIST\_x}{WOL\_LIST\_x}{WOLLISTx}

	Standard-Wert:  \var{WOL\_LIST\_x}=''

	Gültige Werte:
	\begin{itemize}
		\item IP\_NET\_1	- Alle Clients die über \var{IP\_NET\_x} erreicht werden können (hier IP\_NET\_1)
		\item @client1		- Der Name eines Clients (\var{HOST\_x\_NAME}) hier 'client1'
		\item IP-Adresse	- Die IP eines Clients (\var{HOST\_x\_IP4} oder \var{HOST\_x\_IP6})
	\end{itemize}

Beispiel:
\begin{example}
\begin{verbatim}
      WOL_LIST='black'            # black oder white listing
      WOL_LIST_N='3'              # Anzahl Listeinträge
      WOL_LIST_1='IP_NET_1'       # Alle Clients im Netzwerk IP_NET_1
      WOL_LIST_2='@client1'       # Client mit dem Namen HOST_1_x
      WOL_LIST_3='192.168.6.3'    # Client mit der angegebenen IP
\end{verbatim}
\end{example}

\end{description}

\subsection{Wake On Lan beim Booten des Routers}
\begin{description}

\config{WOL\_BOOT}{WOL\_BOOT}{WOLBOOT}

Diese Einstellung sollte nur dann auf 'yes' gesetzt werden wenn Sie einen
Rechner in ihrem Netzwerk beim Starten des Routers mit Wake on LAN booten
wollen. Diese Konfiguration ist unabhängig von \var{WOL\_LIST}, d.h. hier können
Clients angegeben werden die in \var{WOL\_LIST} nicht aufgeführt sind.

\config{WOL\_BOOT\_N}{WOL\_BOOT\_N}{WOLBOOTN}

	Standard-Wert:  \var{WOL\_BOOT\_N}='0'

	In der Default Einstellung stehen also keine Clients auf der Liste, somit werden
	beim starten des routers keine Clients mit Wake on LAN gebootet.

\config{WOL\_BOOT\_x}{WOL\_BOOT\_x}{WOLBOOTx}

	Standard-Wert:  \var{WOL\_BOOT\_x}=''

	Gültige Werte:
	\begin{itemize}
		\item IP\_NET\_1   - Alle Clients die über \var{IP\_NET\_x} erreicht werden können (hier IP\_NET\_1)
		\item @client1     - Der Name eines Clients (\var{HOST\_x\_NAME}) hier 'client1'
		\item IP-Adresse   - Die IP eines Clients (\var{HOST\_x\_IP4} oder \var{HOST\_x\_IP6})
	\end{itemize}

Beispiel:
\begin{example}
\begin{verbatim}
      WOL_BOOT='yes'               # installiere WOL on Boot: yes or no
      WOL_BOOT_N='2'               # Anzahl der Rechner
      WOL_BOOT_1='@client1'        # erster Client
      WOL_BOOT_2='192.162.6.17     # zweiter Client
\end{verbatim}
\end{example}

\end{description}

\subsection{Benutzung}

Mit SSH oder direkt an der Konsole einloggen und einen Rechner wie folgt starten:
'wol.sh~$<$Rechnername$>$' oder 'wol.sh~$<$IP-Adresse$>$' oder 'wol.sh~$<$MAC-Adresse$>$'.

Nicht in der $<$config-dir$>$/wol.txt eingetragene Rechner können auch über
'etherwake~$<$MAC-Adresse$>$'
gestartet werden.

\subsection{Weboberfläche}

    Das Paket bringt außerdem eine Weboberfläche für den mini-httpd mit.
    Die Weboberfläche wird beim Setzen \var{OPT\_HTTPD='yes'} 
    automatisch mit aktiviert.
