% Do not remove the next line
% Synchronized to r38653

\marklabel{sec:hdinstall}{
  \section{HD~- Supporte les disques dur, CompactFlash, clé USB, ...}
  }

\subsection {OPT\_HDINSTALL~- Installation sur disque dur/CompactFlash}
\configlabel{OPT\_HDINSTALL}{OPTHDINSTALL}

    fli4l prend en charge divers supports d'installations (CD, HD, réseau, carte
    CompactFlash, ...) dans la version 4.0 les disquettes ne seront plus
    supportés, du au manque d'espace car la taille des fichiers fli4l augmente.\\
    
    Toutes les étapes nécessaires à l'installation d'un disque dur sont
    expliquées ci-dessous.\\
    
    La méthode habituelle pour une installation est d'utiliser un support de
    boot, vous pouvez aussi utiliser le boot par le réseau. La variable
    \var{OPT\_\-HDINSTALL} prépare le disque dur. Pour l'installation si vous
    utilisez un support de boot et un autre support et si le paramètre
    \verb*?BOOT_TYPE='hd'? est le même pour les deux, les fichiers d'installation
    seront transférés directement. Si une copie directe n'est pas possible,
    vous pouvez transférer les fichiers plus tard en utilisant le SCP ou Imonc.

    Une introduction sur les différentes variantes d'installation A ou B pour les
    disques durs se trouve \jump{VARIANTEN}{au début de la documentation fli4l}.
    Veuillez SVP lire absolument la documentation avant de commencer~!

    \subsubsection{Simple installation du HD en six étape}

\begin{enumerate}
    \item Créer un support de boot fli4l avec le paquetage base et en activant
    la variables \var{OPT\_\-HDINSTALL}. De plus, ce support de boot doit
    permettre une mise à jour à distance. Il faut donc activer la variable
    \var{OPT\_\-SSHD} ou activer la variable \var{OPT\_\-IMOND} avec \verb*?'yes'?. 
    Pour accéder au disque dur, si les pilotes l'installation par défaut ne
    suffit pas, vous pouvez installer des pilotes supplémentaires en activant
    la variable \var{OPT\_\-HDDRV}.
    \item Démarrer le routeur avec le support de boot.
    \item Lorsque le routeur est connecté, exécuter la commande "hdinstall.sh".
    \item Lorsque l'installation du disque est terminée, vous pourez copier les
    fichiers syslinux.cfg, Kernel, rootfs.img, opt.img et rc.cfg au moyens
    d’imonc ou du SCP sur le /boot du routeur. Il est recommandé de travailler
    avec deux répertoires fli4l, l'un pour la configuration et le second pour
    l'installation du hd. Pour la version HD vous allez définir la variable
    \verb*?BOOT_TYPE='hd'? et pour le support de boot il sera en fonction de
    son type.

    \achtung{Bien entendu, les fichiers systèmes pour l'installation de la
    version HD, doivent être transmis au routeur~!}
    \item Enlever le support de boot, descendre ou redémarrer le routeur à
    l'aide (des commandes halt/reboot/poweroff). Le routeur redémarre maintenant
    sur le disque dur.
    \item En cas de problème, lisez les sections suivantes.
\end{enumerate}

    \subsubsection{Explication en détail de l'installation du HD (avec exemples)}

    Tout d'abord, vous devez activer le fichier config/hd.txt dans le support
    de boot du routeur, la variable \var{OPT\_\-HDINSTALL} elle sert pour le
    script d'installation du HD et la variable \var{OPT\_\-HDDRV} (si les
    pilotes supplémentaires sont nécessaires) ces variables doivent
    être configurées correctements. Veuillez également lire soigneusement
    le paragraphe \var{OPT\_\-HDDRV}~!

    La variable \var{BOOT\_\-TYPE} dans base.txt sera sélectionné selon le
    support de configuration, enfin, vous pourrez effectuée la configuration.
    La Variable \var{MOUNT\_\-BOOT} dans base.txt doit être paramétrée sur \verb*?'rw'?,
    afin de permettre plus tard de charger si c'est nécessaire une nouvelles
    archives (*.img) par le réseau.

    Ensuite, vous démarrez le routeur à partir de la disquette. Sur la console de
    fli4l vous tapez "hdinstall.sh" le programme d'installation démarre. Après
    avoir répondu à quelques questions, le disque dur sera en cours de préparation
    pour le partitionner. A la fin de l'installation il s'affichera sur l'écran
    une invitation à copier les fichiers systèmes à distance, ces fichiers sont
    nécessaires au routeur pour booter sur le disque dur.

    \achtung{N'oublier en aucun cas de transférer les fichiers systèmes
    sur le disque dur, autrement le routeur ne démarrera pas. Après
    le transfert des fichiers, vous devez redémarrer le routeur, utiliser
    absolument les commandes reboot/halt/poweroff, pour redémarre le routeur,
    dans le cas contraire les modifications ne seront pas prises en compte les
    fichiers systèmes peuvent être perdus.}

    Le script d’installation du routeur peut être lancé directement sur la
    console d'un autre PC via le SSH. Dans tous les cas, vous devez au préalable
    entrer le mot de passe du routeur. Vous pouvez utiliser par exemple le
    freeware Putty comme client SSH pour les ordinateurs Windows.

\subsubsection{Configuration et installation du support de boot}

\begin{table}[htb]
  \begin{center}
    \begin{small}
    \begin{tabular}[h!]{lp{9cm}}
    \verb*?BOOT_TYPE? & Selon l'installation du support de boot\\
    \verb*?MOUNT_BOOT='rw'? & Nécessaire pour pouvoir copier plus tard, une
        nouvelle archive (*.img) par le réseau sur le disque du routeur\\
    \verb*?OPT_HDINSTALL='yes'? & Nécessaire pour l'installation du script et
        pour les outils, formatage, partitionnement du disque\\
    (\verb*?OPT_HDDRV='yes'?) & nécessaire uniquement si disque dur a besoin de
        pilotes spéciaux.\\
    \verb*?OPT_SSHD='yes'? & Nécessaire, aprés installation du disque, pour
        transférer les fichiers systèmes à distance sur le routeur. Pour cela,
        il faut soit SSHD soit Imond (\verb*?IMOND='yes'?), ou un autre programme,
        qui permet le transfert des fichiers.
    \end{tabular}
    \end{small}
    \caption{Exemple de configuration pour l'installation du support}
    \marklabel{tab:hd-setupdisk}{}
  \end{center}
\end{table}

    Ici la configuration du réseau doit être paramétré correctement pour que les
    fichiers soient transférés sur le disque dur par le réseau. Il est recommandé
    de ne pas activer le DNS\_DHCP, cela crée régulièrement des problèmes 
    (le serveur DHCP doit installer un fichier pour les baux sur le routeur).
    Pour une mise à jour à distance sur le routeur vous pouvez utiliser le SCP
    (il se trouve dans le paquetage SSHD) en activant la variable
    \verb*?OPT_SSHD='yes'?. Alternativement, vous pouvez transférer les fichiers
    via le logiciel Imond, mais une configuration DSL ou RNIS valide est
    nécessaire. Ne pas installer les paquetages qui ne sont pas absolument
    nécessaire, donc pas de DNS\_DHCP, SAMBA\_LPD, LCD, Portforwarding etc.\\ 

    Si à l'installation vous avez ce message d'erreur~:
\begin{example}
\begin{verbatim}
    *** ERROR: can't create new partition table, see docu ***
\end{verbatim}
\end{example}
       Abandonner, plusieurs sources d'erreur sont possible~:
        \begin{itemize}
        \item Le disque dur était en cours d'utilisation et a été interrompue 
        lors de l'installation. Redémarrer simplement et essayer à nouveau.
        \item Un pilote supplémentaire est peut-être nécessaires voir \verb*?OPT_HDDRV?
		\item Il y a des problèmes de matériel, s'il vous plaît lisez l'annexe de ce document.
        \end{itemize}
    Dans la dernière étape, vous pouvez maintenant produire la version définitive
    de fli4l, vous pouvez rajouter les fichiers de configuration et les paquetages
    supplémentaire souhaités.

\subsubsection{Exemple d'installation finalisé pour le type A ou type B~:}

Un exemple de chaque configuration est listé dans le tableau \ref{tab:hd-typ-a}.

\begin{table}[htb]
  \begin{center}
    \begin{small}
    \begin{tabular}{lp{9cm}}
    \verb*?BOOT_TYPE='hd'? & Nécessaire, si fli4l démarre sur le disque dur\\

    \verb*?MOUNT_BOOT='rw|ro|no'? & 
                        Nécessaire, pour copier les nouvelles archives sur le
                        disque par le réseau on a besoin du paramétrer 'rw'.\\

    \verb*?OPT_HDINSTALL='no'? & Nécessaire, après une installation réussie ce
                        programme ne sera plus utilisé.\\

    \verb*?OPT_MOUNT? & Nécessaire, si une partition de données a été créé sur
                         le disque dur.\\

    (\verb*?OPT_HDDRV='yes'?) & Nécessaire, uniquement si le disque dur doit
                        être utilisée avec un pilotes supplémentaires.\\
    \end{tabular}
    \end{small}
    \caption{Exemple d'installation pour le Type A ou B}
    \marklabel{tab:hd-typ-a}{}
  \end{center}
\end{table}

    La création d'une partition swap n'est pas utile sauf si le routeur 
    dispose de moins de 32 Mio de RAM et si l'installation ne s'exécute pas sur
    un périphérique flash~!

\subsection {OPT\_MOUNT~- Montage automatique du système de fichiers}
\configlabel{OPT\_MOUNT}{OPTMOUNT}

    La variable OPT\_MOUNT sert à monté une partition de donnée par ex. /data, en
    cas de besoin, la partition sera testée automatiquement pour vérifier les erreurs.
    Si éventuellement vous avez un lecteur de CD-ROM d'installé, il sera monté en
    /cdrom lorsque vous allez insérer un disque. Si vous avez activé la partition
    swap, vous n'avez pas besoin d'utiliser la variable OPT\_MOUNT~!

    \achtung{A l'installation la variable OPT\_MOUNT lit le fichier de
    configuration hd.cfg sur la partition de démarrage et monte les partitions
    prés enregistrées. Si \var{OPT\_\-MOUNT} est transféré via une mise à jour
    à distance sur un routeur déjà installé, le fichier hd.cfg doit être modifié
    manuellement.}

    \achtung{Avec un boot à partir du CD-ROM, la variable OPT\_MOUNT ne doit
    pas être utilisée. Le CD peut être monté en mettant la valeur MOUNT\_BOOT='ro'
    dans la variable.}

    Voici le fichier hd.cfg qui est sur la partition DOS, le routeur fonctionne
    avec le type B et une partition Swap, le contenu de celui-ci sera (par exemple)~:
    \begin{verbatim}
        hd_boot='sda1'
        hd_opt='sda2'
        hd_swap='sda3'
        hd_data='sda4'
        hd_boot_uuid='4A32-0C15'
        hd_opt_uuid='c1e2bfa4-3841-4d25-ae0d-f8e40a84534d'
        hd_swap_uuid='5f75874c-a82a-6294-c695-d301c3902844'
        hd_data_uuid='278a5d12-651b-41ad-a8e7-97ccbc00e38f'
    \end{verbatim}

    Les partitions qui n'existe pas dans ce fichier seront simplement ignorées,
    exemple d'installation sur un disque SCSI avec le type A et sans autres
    partition, le fichier hd.cfg contiendra~:
    \begin{verbatim}
        hd_boot='sda1'
        hd_boot_uuid='4863-65EF'
     \end{verbatim}


\subsection {OPT\_EXTMOUNT~- Montage manuel du fichier système}
\configlabel{OPT\_EXTMOUNT}{OPTEXTMOUNT}

    Avec la variable OPT\_EXTMOUNT vous pouvez monter le fichier système sur
    n'importe quelle partition et avec n'importe quel point de montage. Il est
    possible de monter manuellement le fichier système, pour disposer, par
    exemple d’un serveur rsync.

\begin{description}
\config{EXTMOUNT\_N}{EXTMOUNT\_N}{EXTMOUNTN}

    Vous indiquez dans cette variable le nombre d’extra partitions à monter.

\config{EXTMOUNT\_x\_VOLUMEID}{EXTMOUNT\_x\_VOLUMEID}{EXTMOUNTxVOLUMEID}

    Dans cette variable vous indiquez le nom ou l'UUID du
    volume à installer. Avec la commande 'blkid' vous pouvez avoir des
    informations sur les noms ou les UUID des volumes installés.

\config{EXTMOUNT\_x\_FILESYSTEM}{EXTMOUNT\_x\_FILESYSTEM}{EXTMOUNTxFILESYSTEM}

    Dans cette variable vous indiquez le nom du fichier système de la partition.
    Actuellement fli4l supporte les fichiers systèmes suivant~: isofs, fat,
    vfat, ext2, ext3 et ext4.//
    (La valeur par défaut est \verb*?EXTMOUNT_x_FILESYSTEM='auto'?, avec cette
    valeur fli4l tente de déterminer le fichier système automatiquement.)

\config{EXTMOUNT\_x\_MOUNTPOINT}{EXTMOUNT\_x\_MOUNTPOINT}{EXTMOUNTxMOUNTPOINT}

    Vous indiquez ici Le chemin d'accés (point de montage) dans lequel le
    dispositif sera monté pour les fichiers systèmes. Le chemin d'accès ne doit
    pas exister sur le support, il est automatiquement généré.

\config{EXTMOUNT\_x\_OPTIONS}{EXTMOUNT\_x\_OPTIONS}{EXTMOUNTxOPTIONS}

    Vous devez indiquer les options supplémenrtaires pour le montage, ils
    seront transmis lors du montage du disque.

\config{EXTMOUNT\_x\_HOTPLUG}{EXTMOUNT\_x\_HOTPLUG}{EXTMOUNTxHOTPLUG}

	Si vous indiquez 'yes' dans cette variable, il n'y aura pas erreur lors
	du boot si vous n'indiquez pas de partition de données. Dans ce cas, il est
	supposé, que la partition de données annexe sera manquante, si c'est nécessaire
	vous pourrez en intéger une plus tard (par exemple avec un SATA hot-plugging
	(ou branchement à chaud) ou avec une clé USB). L'activation de cette option
	nécessite obligatoirement l'activation de la variable \var{OPT\_AUTOMOUNT='yes'}.
	En outre, si vous souhaitez utiliser l'identification de partition de données avec
	l'identifiant universel unique (UUID) pour le système de fichiers, vous devez
	paramétrer la variable \var{EXTMOUNT\_x\_VOLUMEID}. Les autre IDs comme
	le nom de l'appareil ou la marque ne sont \emph{pas} supportées.

\end{description}

\begin{example}
\begin{verbatim}
       EXTMOUNT_1_VOLUMEID='sda2'        # device
       EXTMOUNT_1_FILESYSTEM='ext3'      # filesystem
       EXTMOUNT_1_MOUNTPOINT='/mnt/data' # mountpoint for device
       EXTMOUNT_1_OPTIONS=''             # extra mount options passed via mount -o
       EXTMOUNT_1_HOTPLUG='no'           # device must exist at boot time
\end{verbatim}
\end{example}


\subsection {OPT\_AUTOMOUNT -- Monter automatiquement une partition de données}
\configlabel{OPT\_AUTOMOUNT}{OPTAUTOMOUNT}

    Si vous activez cette variable \var{OPT\_AUTOMOUNT='yes'} vous pouvez monter
	automatiquement et dynamiquement les partitions de données pendant l'installation.
	Il existe deux configurations possibles. La première, en utilisant la variable
	\var{OPT\_EXTMOUNT} elle conserne que les partitions de données qui manquaient
	au moment du démarrage. la seconde est indépendante de \var{OPT\_EXTMOUNT} et
	conserne \emph{toutes} les partitions de données exploitables, au moment du
	démarrage ou tard plus. Pour cela vous devez utiliser la variable
	\var{AUTOMOUNT\_UNKNOWN}~:

\begin{description}
\config{AUTOMOUNT\_UNKNOWN}{AUTOMOUNT\_UNKNOWN}{AUTOMOUNTUNKNOWN}

    Si vous devez monter et contrôlez des partitions de données inconnues vous
	devez activer cette variable. Si vous indiquez \var{AUTOMOUNT\_UNKNOWN='no'} seul
	les partitions de données seront monté dynamiquemt au moment de l'installation, cela
	correspond à la variable \var{EXTMOUNT\_x}. N'oubliez pas d'activer la variable
	\var{EXTMOUNT\_x\_HOTPLUG='yes'}, ne ralez pas si la partition de données est manquante
	lors de l'amorçage avec \var{OPT\_EXTMOUNT}. Si vous indiquez \var{AUTOMOUNT\_UNKNOWN='yes'}
	les partitions de données inconnues seront rattachées. Cela ne fonctionne que si
	le système de fichiers sur la partition dispose d'un identifiant unique (UUID). Dans ce
	cas, la partition sera monté dans le répertoire \texttt{/media/<UUID>} (ce répertoire est
	créé si nécessaire.)

    Paramètre par défaut~: \var{AUTOMOUNT\_UNKNOWN='no'}

\config{AUTOMOUNT\_UNKNOWN\_OPTS}{AUTOMOUNT\_UNKNOWN\_OPTS}{AUTOMOUNTUNKNOWNOPTS}

    Dans cette variable vous indiquez les options de \texttt{montage} qui
	seront utilisés pour les partitions de données inconnus pendant l'installation.
	Si vous avez paramétré une partition de données avec \var{OPT\_EXTMOUNT} les
	informations seront enregistrées dans le fichier \texttt{/etc/fstab}, alors
	les options mentionnées ici ne seront \emph{pas} utilisées. Il faut plutôt
	indiquer les options dans la variable \var{EXTMOUNT\_x\_OPTIONS.}

    Paramètre par défaut~: \var{AUTOMOUNT\_UNKNOWN\_OPTS='ro'} (avec cette option
	les accès en écriture sur les partitions de données inconnues seront empêchés
	par défaut)

\end{description}

    Chaque partition de données est vérifiée avant de montage en utilisant un
	programme qui contrôle le système de fichier pour déceler les erreurs, le programme
	(\texttt{e2fsck} est utilisé pour les systèmes de fichier ext2/ext3/ext4 et
	\texttt{fsck.fat} pour le système de fichier (V)FAT. Si la vérification ou
	la correction automatique échoue, le système de fichier ne sera \emph{pas} installé
	pour éviter la corruption de données.
	
	Si le système de fichier a été suspendu sur le support et qu'il soit monté, le
	support se démontera plus tard via la commande \texttt{umount}. Naturellement
	vous ne pouvez pas écrire des données dessus (car le volume n'est plus là) et vous ne
	pourrez pas accéder à ce volume qui n'existe plus. La manière correcte pour supprimer
	un disque, est bien sûr, de démonter \emph{seulement} le système de fichier et
	\emph{ensuite} de retirer le disque. Car certains types de périphériques empêchent
	le retrait lorsque le système de fichier est installé (par exemple, cela fonctionne
	bien avec les lecteurs de CD), vous devez peut être contrôler l'ordre correct des actions
	pour l'installation un périphérique.

    Toutes les actions de la variable \var{OPT\_AUTOMOUNT} sont enregistrés dans le
	fichier journal \texttt{/var/log/automount.log}. Vous pouvez voir ci-dessous un exemple
	du fichier journal pour une installation. Dans ce fichier vous pouvez voir tout d'abord,
	les partions de données qui sont déjà disponibles au moment du démarrage avec
	(\texttt{ACTION=change})~:

\begin{tiny}
\begin{verbatim}
[2015-04-25 00:33:35] [INFO   ] ACTION=change SUBSYSTEM=block DEVNAME=vda1 DEVPATH=/devices/pci0000:00/0000:00:08.0/virtio4/block/vda/vda1 MDEV=vda1
[2015-04-25 00:33:35] [INFO   ] TYPE: vfat
[2015-04-25 00:33:35] [INFO   ] UUID: 442e-93ba
[2015-04-25 00:33:35] [INFO   ] mount point: /media/442e-93ba
[2015-04-25 00:33:35] [ERROR  ] /dev/vda1 already mounted on /boot, giving up
[2015-04-25 00:33:35] [INFO   ] ACTION=change SUBSYSTEM=block DEVNAME=vda2 DEVPATH=/devices/pci0000:00/0000:00:08.0/virtio4/block/vda/vda2 MDEV=vda2
[2015-04-25 00:33:35] [INFO   ] TYPE: ext3
[2015-04-25 00:33:35] [INFO   ] UUID: 77ab35b3-029e-42c9-93a0-d197c01e6e89
[2015-04-25 00:33:35] [INFO   ] mount point: /media/77ab35b3-029e-42c9-93a0-d197c01e6e89
[2015-04-25 00:33:35] [INFO   ] /dev/vda2: clean, 671/26208 files, 57544/104420 blocks
[2015-04-25 00:33:35] [NOTICE ] /dev/vda2 mounted on /media/77ab35b3-029e-42c9-93a0-d197c01e6e89
[2015-04-25 00:33:36] [INFO   ] ACTION=change SUBSYSTEM=block DEVNAME=vda3 DEVPATH=/devices/pci0000:00/0000:00:08.0/virtio4/block/vda/vda3 MDEV=vda3
[2015-04-25 00:33:36] [INFO   ] TYPE: ext3
[2015-04-25 00:33:35] [INFO   ] UUID: 1580b80c-92b1-4492-abfa-92a12a7d2027
[2015-04-25 00:33:35] [INFO   ] mount point: /media/1580b80c-92b1-4492-abfa-92a12a7d2027
[2015-04-25 00:33:35] [ERROR  ] /dev/vda3 already mounted on /data, giving up
[2015-04-25 00:33:35] [INFO   ] ACTION=change SUBSYSTEM=block DEVNAME=vdb1 DEVPATH=/devices/pci0000:00/0000:00:0a.0/virtio5/block/vdb/vdb1 MDEV=vdb1
[2015-04-25 00:33:35] [INFO   ] TYPE: ext3
[2015-04-25 00:33:35] [INFO   ] UUID: 4c1a03e1-3a0c-4835-88dc-a51879def464
[2015-04-25 00:33:35] [INFO   ] mount point: /mnt/extra
[2015-04-25 00:33:35] [ERROR  ] /dev/vdb1 already mounted on /mnt/extra, giving up
[2015-04-25 00:33:35] [INFO   ] ACTION=change SUBSYSTEM=block DEVNAME=vdc1 DEVPATH=/devices/pci0000:00/0000:00:1f.0/virtio6/block/vdc/vdc1 MDEV=vdc1
[2015-04-25 00:33:35] [INFO   ] TYPE: vfat
[2015-04-25 00:33:35] [INFO   ] UUID: ba6e-9ebd
[2015-04-25 00:33:35] [INFO   ] mount point: /media/ba6e-9ebd
[2015-04-25 00:33:35] [INFO   ] fsck.fat 3.0.26 (2014-03-07)
[2015-04-25 00:33:35] [INFO   ] /dev/vdc1: 0 files, 0/32672 clusters
[2015-04-25 00:33:35] [NOTICE ] /dev/vdc1 mounted on /media/ba6e-9ebd
\end{verbatim}
\end{tiny}

    Deux partitions de données ont été interrompues (\texttt{/dev/vda2} et
    \texttt{/dev/vdc1}), ces deux partition n'ont pas été configurées avec
	\var{OPT\_EXTMOUNT} et ont été suspendu dans le répertoire \texttt{/media}.
	Les trois autres partitions de données \texttt{/dev/vda1}, \texttt{/dev/vda3}
    et \texttt{/dev/vdb1} ont été configuré dans un autre script de démarrage et
	correspond aux partitions définie par l'utilisateur dans la variable \var{OPT\_AUTOMOUNT}.

    Maintenant les partitions \texttt{/dev/vdb1} et \texttt{/dev/vdc1} vont être
	démonté avec (\texttt{ACTION=remove}. Vous pouvez voir un avertissement pour
	\texttt{/dev/vdb1} il a été trouvé sur la partition une base de données et
	celle-ci n'a pas été supprimée, cela est sans danger et rappel que la partition
	à été montée au démarrage avec \var{OPT\_EXTMOUNT} et non avec \var{OPT\_AUTOMOUNT}...

\begin{tiny}
\begin{verbatim}
[2015-04-25 00:34:52] [INFO   ] ACTION=remove SUBSYSTEM=block DEVNAME=vdb1 DEVPATH=/devices/pci0000:00/0000:00:0a.0/virtio5/block/vdb/vdb1 MDEV=vdb1
[2015-04-25 00:34:52] [WARNING] /dev/vdb1 not found in volume database
[2015-04-25 00:34:52] [INFO   ] mount point: /mnt/extra
[2015-04-25 00:34:52] [NOTICE ] /dev/vdb1 unmounted from /mnt/extra
[2015-04-25 00:34:55] [INFO   ] ACTION=remove SUBSYSTEM=block DEVNAME=vdc1 DEVPATH=/devices/pci0000:00/0000:00:1f.0/virtio6/block/vdc/vdc1 MDEV=vdc1
[2015-04-25 00:34:55] [INFO   ] UUID: ba6e-9ebd
[2015-04-25 00:34:55] [INFO   ] mount point: /media/ba6e-9ebd
[2015-04-25 00:34:55] [NOTICE ] /dev/vdc1 unmounted from /media/ba6e-9ebd
\end{verbatim}
\end{tiny}

	... dans le sens inverse vous pouvez voir que les partition sont à nouveau
	installées avec (\texttt{ACTION=add})~:

\begin{tiny}
\begin{verbatim}
[2015-04-25 00:35:14] [INFO   ] ACTION=add SUBSYSTEM=block DEVNAME=vdb1 DEVPATH=/devices/pci0000:00/0000:00:0b.0/virtio5/block/vdb/vdb1 MDEV=vdb1
[2015-04-25 00:35:14] [INFO   ] TYPE: vfat
[2015-04-25 00:35:14] [INFO   ] UUID: ba6e-9ebd
[2015-04-25 00:35:14] [INFO   ] mount point: /media/ba6e-9ebd
[2015-04-25 00:35:15] [INFO   ] fsck.fat 3.0.26 (2014-03-07)
[2015-04-25 00:35:15] [INFO   ] /dev/vdb1: 0 files, 0/32672 clusters
[2015-04-25 00:35:15] [NOTICE ] /dev/vdb1 mounted on /media/ba6e-9ebd
[2015-04-25 00:35:18] [INFO   ] ACTION=add SUBSYSTEM=block DEVNAME=vdc1 DEVPATH=/devices/pci0000:00/0000:00:0c.0/virtio6/block/vdc/vdc1 MDEV=vdc1
[2015-04-25 00:35:18] [INFO   ] TYPE: ext3
[2015-04-25 00:35:18] [INFO   ] UUID: 4c1a03e1-3a0c-4835-88dc-a51879def464
[2015-04-25 00:35:18] [INFO   ] mount point: /mnt/extra
[2015-04-25 00:35:18] [INFO   ] /dev/vdc1: recovering journal
[2015-04-25 00:35:18] [INFO   ] /dev/vdc1: clean, 11/16384 files, 7477/65488 blocks
[2015-04-25 00:35:18] [NOTICE ] /dev/vdc1 mounted on /mnt/extra
\end{verbatim}
\end{tiny}

     Les messages d'erreur du système de fichier ext3 sur la partition \texttt{/dev/vdc1}
	 ont été enregistrés dans "recovering journal" lors du traîtement, ce n'est pas critique,
	 mais depuis aucune autre erreur n'a été trouvée.


\subsection {OPT\_HDSLEEP~- Régle pour l'arrêt automatique du disque dur}
\configlabel{OPT\_HDSLEEP}{OPTHDSLEEP}

    Un disque dur peut être arrêté automatiquement après une certaine
    période d'inactivité. De cette manière la disque utilise moins d'énergie
    et ne fait pratiquement plus aucun bruit. Si un accès au disque dur à lieu,
    il redémarre à nouveau automatiquement.

    \achtung{Certain disque dur ne tolèrent pas les redémarre fréquentes.
    C'est pour cela que nous ne devons pas régler un temps trop cours. Les
    vieux disques durs IDE n'offrent même pas cette fonction. Avec les supports
    Flash Media, ce paramètre n'est pas utile ni nécessaire.}

\begin{description}
\config{HDSLEEP\_TIMEOUT}{HDSLEEP\_TIMEOUT}{HDSLEEPTIMEOUT}

    On paramètre avec cette variable le temps d'inactivité, avant que le
    disque dur se met au repos. le disque dur s'éteindra automatiquement
    après le temps d'inactivité et redémarrera au prochain accès sur
    celui-ci. Le temps d'inactivité se paramètre en minute, de 1 minute à
    20 minutes, au-delà le réglage passe en intervalle de 30 minutes
    jusqu'à 5 heures. Donc si on paramètre 21 ou 25 minutes il sera arrondi
    à 30 minutes. Si le paramètre est trop élevé certains disques durs ne
    tiennent pas compte de cette valeur et se mettent au repos avant le
    temps indiqué. Faites plusieurs tests avec des valeurs différents,
    car cela dépend beaucoup du matériel respectif~!

\begin{example}
\begin{verbatim}
        HDTUNE_TIMEOUT='2'              # wait 2 minutes until power down
\end{verbatim}
\end{example}

\end{description}


\subsection {OPT\_RECOVER~- Option de secours}
\configlabel{OPT\_RECOVER}{OPTRECOVER}

    Cette variable est utilisé pour créer une option de secours (ou une
    restauration système), en cas de proplème. Si l'option est activée,
    vous pouvez utiliser la commande "mkrecover.sh" pour le transfère des
    données sur le routeur. La commande de secours peut être activé à
    partir de la console. Si le paquetage "HTTPD" est installé vous pourrez
    alors activer la restauration système dans le menu Recover.

    L'installation de secours sera disponible au prochain redémarrage de
    fli4l, dans le menu de Boot sélectionnez Recover sur la console.

\begin{example}
\begin{verbatim}
        OPT_RECOVER='yes'
\end{verbatim}
\end{example}


\subsection {OPT\_HDDRV~- Pilote pour contrôleur de disque dur}
\configlabel{OPT\_HDDRV}{OPTHDDRV}

    Si vous activez la variable \var{OPT\_\-HDDRV}='yes' vous pouvez installer
    les pilotes supplémentaires si nécessaires, pour les disque IDE et SATA.
    Normalement vous n'avez pas besoin d'installer de pilote supplémentaire,
    parce qu'ils sont déjà chargés depuis le paquetage base.

\begin{description}
\config{HDDRV\_N}{HDDRV\_N}{HDDRVN}
{
    Vous indiquez ici le nombre de pilote à charger.}

\config{HDDRV\_x}{HDDRV\_x}{HDDRVx}
{
    Vous indiquez dans la variable \var{HDDRV\_\-1} le pilote correspondant
    aux contrôleurs utilisés. Une liste de contrôleurs supportés par fli4l
    est incluent dans le fichier de configuration.
}

\config{HDDRV\_x\_OPTION}{HDDRV\_x\_OPTION}{HDDRVxOPTION}
{
    Vous indiquez dans la variable \var{HDDRV\_\-x\_\-OPTION} les options
    qui peut être nécessitent aux contrôleurs utilisés. Par exemple une
    adresse I/O. On peut laisser cette variable vide dans la plupart des cas.
}

    Vous pouvez voir dans \jump{sec:hd-errors}{l'annexe} un aperçu des erreurs
    qui se produisent le plus souvent sur les disques durs et les CompactFlash.


    Voici quelques exemples sur la façon de charger les pilotes HD dans le
    fichier de configuration.

    Exemple 1~: Accès au disque dur SCSI avec l'Adaptec 2940

\begin{example}
\begin{verbatim}
   OPT_HDDRV='yes'             # install Drivers for Harddisk: yes or no
   HDDRV_N='1'                 # number of HD drivers
   HDDRV_1='aic7xxx'           # various aic7xxx based Adaptec SCSI 
   HDDRV_1_OPTION=''           # no need for options yet
\end{verbatim}
\end{example}

    Exemple 2~: Activez l'accés-IDE pour ALIX de PC-Engines

\begin{example}
\begin{verbatim}
   OPT_HDDRV='yes'             # install Drivers for Harddisk: yes or no
   HDDRV_N='1'                 # number of HD drivers
   HDDRV_1='pata_amd'          # AMD PCI IDE/ATA driver (e.g. ALIX)
   HDDRV_1_OPTION=''           # no need for options yet
\end{verbatim}
\end{example}

\end{description}
