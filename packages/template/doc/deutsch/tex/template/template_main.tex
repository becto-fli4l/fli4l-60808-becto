% Last Update: $Id$
\section {Template - Ein Beispiel-Paket}

\subsection {Strukturierung des Dokumentes}

Die Strukturierung des Dokumentes erfolgt durch die folgenden
Kommandos:
\begin{itemize}
\item \verb*?\section{}?
\item \verb*?\subsection{}?
\item \verb*?\subsubsection{}?
\item \verb*?\paragraph{}?
\item \verb*?\subparagraph{}?
\end{itemize}

In den geschweiften Klammern steht dann der jeweilige Titel des
Abschnittes. Wie das konkret aussieht, kann man sich im
\jump{sec:templateanhang}{Anhang dieses Paketes} ansehen.

Zur Aufzählung von Dingen werden drei Umgebungen verwendet:

\begin{description}
\item [itemize] Mit itemize wird eine unnummerierte Aufzählung erzeugt,
  die wie folgt aussehen könnte:
  \begin{itemize}
  \item erster Punkt
  \item zweiter Punkt
    \begin{itemize}
    \item erster Unterpunkt
    \item zweiter Unterpunkt
    \end{itemize}
  \end{itemize}

Erzeugt wurde diese Aufzählung durch folgenden \LaTeX-Code:
\begin{verbatim}
  \begin{itemize}
  \item erster Punkt
  \item zweiter Punkt
    \begin{itemize}
    \item erster Unterpunkt
    \item zweiter Unterpunkt
    \end{itemize}
  \end{itemize}
\end{verbatim}

\item [enumerate] Mit enumerate wird eine nummerierte Aufzählung erzeugt,
  die wie folgt aussehen könnte:
  \begin{enumerate}
  \item erster Punkt
  \item zweiter Punkt
    \begin{enumerate}
    \item erster Unterpunkt
    \item zweiter Unterpunkt
    \end{enumerate}
  \end{enumerate}

Erzeugt wurde diese Aufzählung durch folgenden \LaTeX-Code:
\begin{verbatim}
  \begin{enumerate}
  \item erster Punkt
  \item zweiter Punkt
    \begin{enumerate}
    \item erster Unterpunkt
    \item zweiter Unterpunkt
    \end{enumerate}
  \end{enumerate}
\end{verbatim}

\item[description] Mit description wird eine unnummerierte Liste
  erzeugt, in der die einleitenden Worte fett gedruckt werden (siehe
  Quelltext dieser Dokumentation).
\end{description}


\marklabel{sec:fli4lenv}{
  \subsection{Spezielle Kommandos für die fli4l-Dokumentation}
}
In der fli4l-Dokumentation werden einige Kommandos und Umgebungen definiert,
die das Schreiben der Dokumentation vereinfachen und die für ein
einheitliches Aussehen sorgen sollen. Das sind im Wesentlichen die
Folgenden:

\begin{description}
\item [config]  Beschreibung einer Konfigurationsvariablen im
  folgenden Format:
  \begin{tabbing}
    aaaaaaaa\=\kill
    \>\verb*?\config{var var1 ...}{index}{label}{beschreibung}?.
  \end{tabbing}

  Die angeführten Variablen werden fett gedruckt der Beschreibung
  voran gestellt. Zusätzlich wird ein Eintrag im Index und ein Label
  erstellt, die dazu benutzt werden können, schnell zu dieser Variable
  zu gelangen. Aussehen könnte das wie folgt:

  \begin{description}
    \configlabel{FOO\_x}{FOOx}
    \config{FOO\_N  FOO\_x}{FOO\_N}{FOON}{Beschreibung der
      Konfiguration von \var{FOO\_N} und \var{FOO\_x}}
  \end{description}

  Erzeugt wird dies durch folgenden \LaTeX-Code:

\begin{verbatim}
    \begin{description}
      \config{FOO\_N  FOO\_x}{FOO\_N}{FOON}{Beschreibung der
        Konfiguration von \var{FOO_N} und \var{FOO_x}}
    \end{description}
\end{verbatim}

  Im Index erscheint das Stichwort \var{FOO\_N} und man kann mit Hilfe von
  \verb*?\jump{FOON}? eine Referenz auf diese Stelle erzeugen.

  Um Listen mit Unterschieden zwischen Versionen richtig generieren zu können,
  ist es nötig, dass das Label (hier: FOON) identisch ist mit dem Index
  (hier FOO\_N), jedoch ohne Unterstriche.

  Also: Index: \verb?FOO\_N => Label: FOON?.

  Die Umgebung \glqq{}description\grqq{} braucht man nur einmal auf und wieder
  zuzumachen, wenn man mehrere Variablen beschreibt.

\item [configlabel] Setzen eines Index-Eintrages und eines Labels für
  eine Variable. Im obigen Beispiel haben wir zwei Variablen
  beschrieben, aber nur für eine (\var{FOO\_N}) einen Index-Eintrag und ein
  Label erzeugt. Für \var{FOO\_x} muß man das separat machen, indem man
  folgende Zeile hinzufügt:

  \begin{tabbing}
    aaaaaaaa\=\kill
    \>\verb*?\configlabel{FOO\_x}{FOOx}?.
  \end{tabbing}

  Aussehen könnte das dann wie folgt:

\begin{verbatim}
    \begin{description}
      \configlabel{FOO\_x}{FOOx}
      \config{FOO\_N  FOO\_x}{FOO\_N}{FOON}{Beschreibung der
        Konfiguration von \var{FOO\_N} und \var{FOO\_x}}
    \end{description}
\end{verbatim}

\item [marklabel] Setzt ein Label, das man von anderen Stellen im
  Dokument aus referenzieren kann. So können wir z.B. die
  Abschnitte referenzieren, indem wir die sie wie folgt
  einleiten:
\begin{verbatim}
\marklabel{sec:fli4lenv}{
  \subsection{Spezielle Kommandos für die fli4l-Dokumentation}
}
\end{verbatim}

\item [jump] Mit \verb*?\jump{label}{text}? können wir einen Verweis auf
eine andere Stelle im Dokument generieren, die man z.B. im html/pdf
Format einfach anklicken kann. Wollen wir also zur Beschreibung der
Variable \var{FOO\_x} springen, schreiben wir:
\begin{verbatim}
(\jump{FOOx}{\var{FOO\_x}})
\end{verbatim}

Im Text sieht das dann so aus: (\jump{FOOx}{\var{FOO\_x}})

\item [smalljump] \glqq{}Kleiner jump\grqq{}, wie jump, nur dass die
Seitennummer nicht hinzugefügt wird (pdf/ps). Beispiel:
\begin{verbatim}
(\smalljump{FOOx}{\var{FOO\_x}})
\end{verbatim}

Im Text sieht das dann so aus: (\smalljump{FOOx}{\var{FOO\_x}})

\item [altlink] Mit \verb*?\altlink{url}? fügt man eine URL ins Dokument
  ein, z.B. könnte eine Referenz auf die fli4l-Webseite wie folgt aussehen:
  \altlink{http://www.fli4l.de}, generiert von folgendem Statement:

  \verb*?\altlink{http://www.fli4l.de}?\\
  Achtung: das veraltete Kommando \verb*?\link{url}? ist abgekündigt und
  sollte nicht mehr verwendet werden.

\item [achtung, wichtig] Mit \verb*?\achtung{text}? und
  \verb*?\wichtig{text}? können Dinge im Text hervor gehoben werden.

  Aus \verb*?\achtung{Beachten sie, dass ...}? wird:

  \achtung{Beachten sie, dass ...}

  und aus \verb*?\wichtig{Beachten sie, dass ...}? wird:

  \wichtig{Beachten sie, dass ...}

\item [email] Angabe einer E--Mail--Adresse in der Form
\verb*?\email{foo@bar.org}?, im Text sieht das dann so aus:

\email{foo@bar.org}

\item [var] Da Variablen einfach so in den Text eingestreut einfach
  häßlich aussehen, werden sie in \verb*?\var{FOO\_x}? geklammert und
  anders formatiert. Das sieht dann wie folgt aus: \var{FOO\_x} versus
  FOO\_x

\item [example] Beispiele haben die Tendenz, sehr viel Platz
  einzunehmen. Diese werden daher mit \verb*?\begin{example} ... \end{example}?
  eingeklammert und einheitlich in einem etwas kleineren Font gesetzt.
\end{description}

