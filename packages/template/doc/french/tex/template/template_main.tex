% Do not remove the next line
% Synchronized to r30359

\section {Template~- Modèle de documentation pour un paquetage}

\subsection {Structure de la documentation}

La structure de la documentation est réalisée par les éléments de commandes suivants~:
\begin{itemize}
\item \verb*?\section{}?
\item \verb*?\subsection{}?
\item \verb*?\subsubsection{}?
\item \verb*?\paragraph{}?
\item \verb*?\subparagraph{}?
\end{itemize}

Vous indiquez entre les accolades le titre de la section appropriée. Pour voir
un exemple concret allez dans la section
\jump{sec:templateanhang}{annexe du paquetage}.

Pour créer une liste d'objets trois environnements sont utilisés~:

\begin{description}
\item Avec [itemize] vous générez et détaillez une liste non numérotée, comme ceci~:
  \begin{itemize}
  \item Premier point
  \item Second point
    \begin{itemize}
    \item Premier sous-point
    \item Second sous-point
    \end{itemize}
  \end{itemize}

La liste ci-dessous est générée avec le code \LaTeX~:
\begin{verbatim}
  \begin{itemize}
  \item Premier point
  \item Second point
    \begin{itemize}
    \item Premier sous-point
    \item Second sous-point
    \end{itemize}
  \end{itemize}
\end{verbatim}

\item Avec [enumerate] vous générez et détaillez une liste numérotée, comme ceci~:
  \begin{enumerate}
  \item Premier point
  \item Second point
    \begin{enumerate}
    \item Premier sous-point
    \item Second sous-point
    \end{enumerate}
  \end{enumerate}

La liste ci-dessous est générée avec le code \LaTeX~:
\begin{verbatim}
  \begin{enumerate}
  \item Premier point
  \item Second point
    \begin{enumerate}
    \item Premier sous-point
    \item Second sous-point
    \end{enumerate}
  \end{enumerate}
\end{verbatim}

\item Avec [description] vous générez une liste non numérotée, donc 
  l'introduction les mots sont imprimés en caractères gras (voir la source
  de cette documentation).
\end{description}


\marklabel{sec:fli4lenv}{
  \subsection{Commande spéciale pour la documentation de fli4l}
}
Dans la documentation de fli4l les commandes et les environnements sont
prédéfinis cela simplifie l'écriture de la documentation et assure une apparence
uniforme. Les commandes essentielles sont les suivants~:

\begin{description}
\item Avec [config] vous faite une description de la variable de configuration
dans le format suivant~:
  \begin{tabbing}
    aaaaaaaa\=\kill
    \>\verb*?\config{var var1 ...}{index}{label}{description}?.
  \end{tabbing}

  Les variables sont indiqués en gras et précédée de la description. En outre
  si vous créez un Index et un Label, ils seront utilisés pour accéder rapidement
  à cette variable. Cela peut être représenté comme ceci~:

  \begin{description}
    \configlabel{FOO\_x}{FOOx}
    \config{FOO\_N  FOO\_x}{FOO\_N}{FOON}{Description de
    la configuration de \var{FOO\_N} et de \var{FOO\_x}}
  \end{description}

La description suivante est générée avec le code \LaTeX~:

\begin{verbatim}
    \begin{description}
      \config{FOO\_N  FOO\_x}{FOO\_N}{FOON}{Description de
        la configuration \var{FOO_N} et \var{FOO_x}}
    \end{description}
\end{verbatim}

  Vous indiquez dans l'Index le mot clé \var{FOO\_N} et vous pouvez utiliser
  \verb*?\jump{FOON}? pour générer un point de référence.

  Pour pouvoir générer correctement une liste pour différentes versions, il
  est nécessaire que le Label (ici~: FOON) soit identique à l'Index
  (ici FOO\_N), mais sans le trait de soulignement.

  Donc, Index~: \verb?FOO\_N => Label~: FOON?.

  Si vous devez mettre des guillemés \frqq{}description\frqq{} vous devez les
  placer en une seul fois, si vous décrivez plusieurs variables vous devez
  placer les guillemés pour chaque variable.

\item Avec [configlabel] vous pouvez ajouter un Index et un Label pour la
  variable. Dans l'exemple ci-dessus nous avons décrit deux variables pour
  (\var{FOO\_N}), ici nous allons définir seulement une variable pour l'Index
  et le Label. Pour \var{FOO\_x} on doit le paramétrer séparément, en ajoutant
  la ligne suivante~:

  \begin{tabbing}
    aaaaaaaa\=\kill
    \>\verb*?\configlabel{FOO\_x}{FOOx}?.
  \end{tabbing}

  Cela peut être représenté comme ceci~:

\begin{verbatim}
    \begin{description}
      \configlabel{FOO\_x}{FOOx}
      \config{FOO\_N FOO\_x}{FOO\_N}{FOON}{Description de
        la configuration \var{FOO\_N} et \var{FOO\_x}}
    \end{description}
\end{verbatim}

\item Avec [marklabel] vous pouvez définir un Label pour une référence
  ailleurs dans le document. Ainsi, vous pouvez par exemple référencer un
  paragraphe, vous pouvez introduire les paragraphes de la manière suivante~:

\begin{verbatim}
\marklabel{sec:fli4lenv}{
  \subsection{Commandes spéciales pour la documentation de fli4l}
}
\end{verbatim}

\item Avec [jump] \verb*?\jump{label}{text}? vous pouvez générer une référence
  qui sera dans un autre document, par exemple un document html/pdf, il suffira
  de cliquer sur ce lien. Donc, si vous voulez générer la variable \var{FOO\_x}
  pour un saut de document, vous devez écrire~:

\begin{verbatim}
(\jump{FOOx}{\var{FOO\_x}})
\end{verbatim}

Dans le texte le saut sera indiqué ainsi~: (\jump{FOOx}{\var{FOO\_x}})

\item Avec [smalljump] \frqq{}petit saut\frqq{}, la seule différence pour ce
  saut, les numéros de page ne sont pas ajoutés dans les documents (pdf/ps).
  Exemple~:

\begin{verbatim}
(\smalljump{FOOx}{\var{FOO\_x}})
\end{verbatim}

Dans le texte le saut sera indiqué ainsi~: (\smalljump{FOOx}{\var{FOO\_x}})

\item Avec [altlink] \verb*?\altlink{url}? vous pouvez ajouter une URL dans le
  texte par exemple vous pouvez référencer le site fli4l~:
  \altlink{http://www.fli4l.de}, vous devez gérérer le lien du site de la
  manière suivante~:

  \verb*?\altlink{http://www.fli4l.de}?\\
  Attention~: la commande \verb*?\link{url}? est obsolète et ne doit plus être
  utilisé.

\item Avec [achtung, wichtig] \verb*?\achtung{text}? et \verb*?\wichtig{text}? 
  vous pouvez mettre en évidence des éléments dans le texte.

  Pour \verb*?\achtung{S'il vous plaît noter que ...}? vous écrivez~:

  \achtung{S'il vous plaît noter que ...}

  Et pour \verb*?\wichtig{S'il vous plaît noter que ...}? vous écrivez~:

  \wichtig{S'il vous plaît noter que ...}

\item Avec [email] vous pouvez indiquez une Adresse--E--Mail sous la forme
  \verb*?\email{foo@bar.org}?, dans le texte cela ressemblera à ceci~:

\email{foo@bar.org}

\item Avec [var] une simple variable qui est parsemée dans le texte est tous
simplement laid, mais si elle est assemblée \verb*?\var{FOO\_x}? la forme est
différente. Cela ressemble à ceci~: \var{FOO\_x} par rapport à FOO\_x

\item Avec [example] les exemples ont tendances à prendre beaucoup d'espace.
  Vous pouvez utiliser \verb*?\begin{example} ... \end{example}? ainsi les
  exemples seront affichés uniformément et plus serré avec une police plus petit.
\end{description}

