% Synchronized to r30359

\section {Template - A Package Example}

\subsection {Structure Of The Document}

The structure of the document is generated by using the following commands:
\begin{itemize}
\item \verb*?\section{}?
\item \verb*?\subsection{}?
\item \verb*?\subsubsection{}?
\item \verb*?\paragraph{}?
\item \verb*?\subparagraph{}?
\end{itemize}

Between the curly brackets place the title of the current
section. What this looks like in concrete terms can be seen in
the \jump{sec:templateanhang}{Appendix of this package}.

Itemizing can be achieved in three different ways:

\begin{description}
\item [itemize] With itemize an unnumbered list is generated,
   that could look like this:
  \begin{itemize}
  \item first item
  \item second item
    \begin{itemize}
    \item first subitem
    \item second subitem
    \end{itemize}
  \end{itemize}

This item list was generated by the following \LaTeX-Code:
\begin{verbatim}
  \begin{itemize}
  \item first item
  \item second item
    \begin{itemize}
    \item first subitem
    \item second subitem
    \end{itemize}
  \end{itemize}
\end{verbatim}

\item [enumerate] With enumerate a numbererd list is generated,
  that could look like this:
  \begin{enumerate}
  \item first item
  \item second item
    \begin{enumerate}
    \item first subitem
    \item second subitem
    \end{enumerate}
  \end{enumerate}

This item list was generated by the following \LaTeX-Code:
\begin{verbatim}
  \begin{enumerate}
  \item first item
  \item second item
    \begin{enumerate}
    \item first subitem
    \item second subitem
    \end{enumerate}
  \end{enumerate}
\end{verbatim}

\item[description] With description an unnumbered list with
  introductory words in bold is generated (see the
  source of this documentation).
\end{description}


\marklabel{sec:fli4lenv}{
  \subsection{Special Commands For The fli4l Documentation}
}
In the fli4l documentation some commands and environments are defined,
that simplify the writing of documentation and ensure uniform appearance. 
These are basically the following:

\begin{description}
\item [config] Description of a config variable in the following format:
  \begin{tabbing}
    aaaaaaaa\=\kill
    \>\verb*?\config{var var1 ...}{index}{label}{description}?.
  \end{tabbing}

  The variables are listed in bold type and prefix the description.
  In addition, an entry in the index and a label is created that can
  be used to access this variable quickly. This might look as follows:

  \begin{description}
    \configlabel{FOO\_x}{FOOx}
    \config{FOO\_N  FOO\_x}{FOO\_N}{FOON}{Description Of Configuration
      For \var{FOO\_N} and \var{FOO\_x}}
  \end{description}

  This is generated by the following \LaTeX-Code:

\begin{verbatim}
    \begin{description}
      \config{FOO\_N  FOO\_x}{FOO\_N}{FOON}{Description Of Configuration
      For \var{FOO_N} and \var{FOO_x}}
    \end{description}
\end{verbatim}

  In the index the keyword \var{FOO\_N} appears and by the help of
  \verb*?\jump{FOON}? a reference to this part may be generated.

  In order to generate lists of differences between versions properly
  it is necessary that the label (in this case FOON) is identical with the index
  (here FOO\_N), but without underscores.

  Hence: Index: \verb?FOO\_N => Label: FOON?.

  When describing more variables the environment \glqq{}description\grqq{}
  only needs to be opened and closed once.

\item [configlabel]Insert an index entry and a label for
  a variable. In the example above we have described two variables,
  but only for (\var{FOO\_N}) an index entry and a label was generated. 
  For \var{FOO\_x} this has to be done separately, by adding the
  following line:

  \begin{tabbing}
    aaaaaaaa\=\kill
    \>\verb*?\configlabel{FOO\_x}{FOOx}?.
  \end{tabbing}

  This might look as follows:

\begin{verbatim}
    \begin{description}
      \configlabel{FOO\_x}{FOOx}
      \config{FOO\_N  FOO\_x}{FOO\_N}{FOON}{Description Of Configuration
      For \var{FOO\_N} und \var{FOO\_x}}
    \end{description}
\end{verbatim}

\item [marklabel] Sets a label that can be referenced from other
  locations in the document. In this way it is possible to reference
  sections by starting the sections as follows:
\begin{verbatim}
\marklabel{sec:fli4lenv}{
  \subsection{Special Commands For The fli4l Documentation}
}
\end{verbatim}

\item [jump] With \verb*?\jump{label}{text}? a clickable (i.e. in html/pdf format)
reference to another part of the document can be generated. If we want to jump to
the description for variable \var{FOO\_x} we write:
\begin{verbatim}
(\jump{FOOx}{\var{FOO\_x}})
\end{verbatim}

In the text this looks like this: (\jump{FOOx}{\var{FOO\_x}})

\item [smalljump] \glqq{}Small jump\grqq{}, like jump, but without page
number appended (pdf/ps). Example:
\begin{verbatim}
(\smalljump{FOOx}{\var{FOO\_x}})
\end{verbatim}

In the text this looks like this: (\smalljump{FOOx}{\var{FOO\_x}})

\item [altlink] With \verb*?\altlink{url}? an URL is inserted into the document
  i.e. a reference to fli4l's website might look like this:
  \altlink{http://www.fli4l.de}, generated by the statement:

  \verb*?\altlink{http://www.fli4l.de}?\\
  Attention: the old command \verb*?\link{url}? is deprecated and
  should not be used anymore.

\item [achtung, wichtig] With \verb*?\achtung{text}? and
  \verb*?\wichtig{text}? passages may be highlighted in the text.

  \verb*?\achtung{Please note that...}? will become:

  \achtung{Please note that...}

  and \verb*?\wichtig{Please note that...}? will become:

  \wichtig{Please note that...}

\item [email] Specifying an E--Mail--Address in the form
\verb*?\email{foo@bar.org}?, in the text looks like this:

\email{foo@bar.org}

\item [var] Since variables look ugly when just being inserted in the text, they
  might be bracketed with \verb*?\var{FOO\_x}? and get formatted different.
  This looks as follows: \var{FOO\_x} versus FOO\_x

\item [example] Examples tend to occupy a lot of space. Therefore they
  should be bracketed with \verb*?\begin{example} ... \end{example}?
  and hence will be uniformly set in a slightly smaller font.
\end{description}

