% Last Update: $Id$
\section {APCUPSD -- Daemon für APC USV}

\subsection{Einleitung}

  Dieses Paket stellt den APCUPSD  Daemon {[\ref{apcupsd}]} zur Überwachung von
  APC-USVs für den fli4l zur Verfügung.
  Alle Einstellungen sind direkt aus dem original Paket übernommen
  {[\ref{apcupsd_manual}]}.
 
\subsection {Konfiguration des Paketes APCUPSD}
     
  Die Konfiguration erfolgt, wie bei allen fli4l Paketen, durch Anpassung der \\
  Datei \var{Pfad/fli4l-\version/$<$config$>$/apcupsd.txt} 
  an die eigenen Anforderungen.

\begin{description}

\config {OPT\_APCUPSD}{OPT\_APCUPSD}{OPTAPCUPSD}

  Die Einstellung \var{'no'} deaktiviert das OPT\_APCUPSD vollständig. Es werden
  keine Änderungen am fli4l Archiv \var{rootfs.img} bzw. dem Archiv \var{opt.img}
  vorgenommen. Weiterhin überschreibt das OPT\_APCUPSD grundsätzlich keine anderen
  Teile der fli4l Installation.\\
  Um OPT\_APCUPSD zu aktivieren, ist die Variable \var{OPT\_APCUPSD} auf 
  \var{'yes'} zu setzen.

\end{description}

\subsubsection {Kommunikationseinstellungen}
\begin{description}

\config {APCUPSD\_UPSNAME}{APCUPSD\_UPSNAME}{APCUPSDUPSNAME}

  Setzt den Namen der USV für Logfles u.ä.
  Dies ist insbesondere dann nützlich wenn man mehrere USV's nutzt.
  Der Name wird nicht in das EEPROM der USV geschrieben.
  Der Name sollte maximal 8 Zeichen lang sein.


\config {APCUPSD\_UPSCABLE}{APCUPSD\_UPSCABLE}{APCUPSDUPSCABLE} 
 
  Setzt den Typ des Kabels mit der die USV mit dem fli4l verbunden ist.
  Mögliche Verbindungstypen sind:

   \var{'simple'}, \var{'smart'}, \var{'ether'} oder \var{'usb'}
  
  Oder es kann eines der folgenden spezifischen Kabel genutzt werden:

  \var{'940-0119A'}, \var{'940-0127A'}, \var{'940-0128A'}, \var{'940-0020B'},
  \var{'940-0020C'}, \var{'940-0023A'}, \var{'940-0024B'}, \var{'940-0024C'},
  \var{'940-1524C'}, \var{'940-0024G'}, \var{'940-0095A'}, \var{'940-0095B'},
  \var{'940-0095C'}or \var{'M-04-02-2000'}

\config {APCUPSD\_UPSTYPE} {APCUPSD\_UPSTYPE}{APCUPSDUPSTYPE}

  Setzt die Übertragungsart mit der die USV mit dem fli4l verbunden ist.
  Die Übertragungsart muss mit dem Gerätetyp korrespondieren der in 
  \smalljump{APCUPSDUPSDEVICE}{\var{APCUPSD\_UPSDEVICE}} gesetzt ist. 

\config {APCUPSD\_UPSDEVICE} {APCUPSD\_UPSDEVICE}{APCUPSDUPSDEVICE}

  Zusätzlich zum USV-Typ in \smalljump{APCUPSDUPSTYPE}{\var{APCUPSD\_UPSTYPE}} muss
  ein Device definiert werden, über welches mit der USV kommuniziert wird.
  In folgender Tabelle ist beschrieben für welcher USV-Typs welche Devices
  angegeben werden können. 
  
  Für USB-USV's sollte \var{APCUPSD\_UPSDEVICE} leer bleiben.
   
\begin{tabular}{p{20mm}p{120mm}}
  USV-Typ & Device \\ & Beschreibung
  \\\\ 
  
  \var{'apcsmart'} & \var{'/dev/tty*'} \\ &
  Serielles Device, für SmartUPS Modelle die ein serielles Kabel nutzen
  (kein USB).
  \\\\

  \var{'usb'} & \var{''} \\ &
  Die meisten neueren  USVen sind per USB angeschlossen. 
  Ein leeres \var{APCUPSD\_UPSDEVICE} erlaubt automatische Erkennung, welches
  die beste Wahl für die meisten Installationen ist.. \\\\

  \var{'net'} & \var{'hostname:port'} \\ & 
  Netzwerk-Verbindung zu einem Master-apcupsd über apcupsd's
  Netzwerk Information Server.
  Das wird genutzt wenn die USV, die den fli4l speist an einen anderen Computer 
  zur Überwachung angeschlossen ist.
  \\\\

%   snmp & hostname:port:vendor:community \\ & 
%   SNMP network link to an SNMP-enabled UPS device.
%   Hostname is the ip address or hostname of the UPS 
%   on the network. Vendor can be can be 'APC' or 
%   'APC\_NOTRAP'. 'APC\_NOTRAP' will disable SNMP trap 
%   catching; you usually want 'APC'. Port is usually 
%   161. Community is usually 'private'.
%   (disabled in build from apcupsd binary) \\
% 
%   netsnmp & :port:vendor:community \\ &
%   OBSOLETE \\ &
%   Same as SNMP above but requires use of the 
%   net-snmp library. Unless you have a specific need
%   for this old driver, you should use 'snmp' instead.
%   (disabled in build from apcupsd binary) \\
% 
%   dumb  & /dev/tty** \\ &
%   Old serial character device for use with simple-signaling UPSes. \\

  \var{'pcnet'} & \var{'ipaddr:username:passphrase[:port]'} \\ &
  
  PowerChute Network Shutdown Protokoll, welches alternativ zu SNMP mit der 
  AP9617 Smart-Slot-Card-Familie genutzt werden kann.  
  
  ipaddr ist die IP-Adresse der USV Management-Karte.
  
  username und passphrase sind die Login-Daten für die Management-Karte. 
  
  port ist der TCP-Port auf welchem auf Messages der USV gelauscht wird, 
  normalerweise ist die Portnummer 3052. Wenn dieser Parameter fehlt oder leer
  ist wird der Vorgabewert von 3052 genutzt.
  \\
\end{tabular}

\config {APCUPSD\_POLLTIME} {APCUPSD\_POLLTIME}{APCUPSDPOLLTIME}

  Intervall in Sekunden in dem apcupsd den USV-Status abfragt.
  
  Diese Einstellung betrifft sowohl direkt angeschlossenen USVen
  (\smalljump{APCUPSDUPSTYPE}{\var{APCUPSD\_UPSTYPE}} \var{'apcsmart'}, \var{'usb'}) 
  als auch Netzwerk USVen (\smalljump{APCUPSDUPSTYPE}{\var{APCUPSD\_UPSTYPE}} 
  \var{'net'}, \var{'pcnet'}).
  Eine Verkleinerung des Intervalls erhöht die Ansprechempfindlichkeit des apcupsd's
  auf Kosten einer höheren CPU-Last (Vorgabe \var{'60'}).
  Die Vorgabeeinstellung von 60 Sekunden ist in den meisten Situationen angebracht.

\end {description}

\subsection{Verzeichnis Einstellungen}
\begin {description}

\config {APCUPSD\_LOCKFILE} {APCUPSD\_LOCKFILE}{APCUPSDLOCKFILE}
  
  Pfad für die Device-Lock-Datei (Vorgabe \var{'/var/lock'}).
  

\config {APCUPSD\_SCRIPTDIR} {APCUPSD\_SCRIPTDIR}{APCUPSDSCRIPTDIR}
    
  Pfad zum Skript-Verzeichnis, in dem apccontrol und die Event-Skripte liegen
  (Vorgabe \var{'/etc'})


\config {APCUPSD\_PWRFAILDIR} {APCUPSD\_PWRFAILDIR}{APCUPSDPWRFAILDIR}

  Pfad zum Powerfail-Verzeichnis, in dem die Powerfail-Flag-Datei liegt. 
  
  Diese Datei wird erzeugt, wenn apcupsd einen System-Shutdown initiert und
  wird im Halt-Skript überprüft um ggf. die USV auszuschalten
  (Vorgabe \var{'/etc'}).


\config {APCUPSD\_NOLOGINDIR} {APCUPSD\_NOLOGINDIR}{APCUPSDNOLOGINDIR}

  Pfad zum NoLogin-Verzeichnis, in dem die NoLogin-Datei liegt.
  Die Existenz dieser Datei sperrt neue Logins 
  (Vorgabe \var{'/etc'}).

\end {description}

\subsection{Powerfail Einstellungen}
\begin {description}

\config {APCUPSD\_ONBATTERYDELAY} {APCUPSD\_ONBATTERYDELAY}{APCUPSDONBATTERYDELAY}
  
  Zeit in Sekunden zwischen dem Erkennen eines Spannungsausfalls und 
  der Reaktion mit einem OnBattery-Event (Vorgabe \var{'6'})

  Das Skript apccontrol wird mit dem Powerout Argument sofort nach erkennen 
  des Spannungsausfalls aufgerufem, aber das OnBattery Argument wird erst nach 
  \var{APCUPSD\_ONBATTERYDELAY} gesetzt.  

\config {APCUPSD\_BATTERYLEVEL} {APCUPSD\_BATTERYLEVEL}{APCUPSDBATTERYLEVEL}
 
  Wenn während eines Spannungsausfalls die Batterie-Restladung in Prozent
  (wie von der USV gemessen) auf oder unter \var{APCUPSD\_BATTERYLEVEL} fällt, 
  wird apcupsd einen System-Shutdown einleiten
  (Vorgabe \var{'5'}).

\config {APCUPSD\_MINUTES} {APCUPSD\_MINUTES}{APCUPSDMINUTES}
 
  Wenn während eines Spannungsausfalls die Restlaufzeit in Minuten
  (wie von der USV berechnet) auf oder unter  \var{APCUPSD\_MINUTES} fällt, 
  wird apcupsd einen System-Shutdown einleiten
  (Vorgabe \var{'3'}).

\config {APCUPSD\_TIMEOUT} {APCUPSD\_TIMEOUT}{APCUPSDTIMEOUT}

  Wenn während eines Spannungsausfalls die USV für \var{APCUPSD\_BATTERYLEVEL}
  oder länger im Batteriebtrieb läuft, 
  wird apcupsd einen System-Shutdown einleiten
  (Vorgabe \var{'0'}).
  Ein Wert von \var{'0'} schaltet diesen Timer aus.
 
  Anmerkung, when Sie eine SmartUPS haben werden , werden Sie wahrscheinlich 
  diesen Timer ausschalten wollen.
  In dieem Fall läuft die USV im Batteriebetrieb, solange bis entweder 
  die Batterie-Restladung auf oder unter \var{APCUPSD\_BATTERYLEVEL}
  oder die  Restlaufzeit auf oder unter \var{APCUPSD\_MINUTES} fällt.
  Für Testzwecke bewirkt ein Setzen dieses Werts auf \var{'60'} einen
  schnellen System-Shutdown wenn Sie den Netzstecker ziehen.   
 
  Anmerkung: \var{APCUPSD\_BATTERYLEVEL}, \var{APCUPSD\_MINUTES} und 
  \var{APCUPSD\_TIMEOUT} spielen zusammen, so daß das erste Auftreten einen
  System-Shutdown einleitet.


\config {APCUPSD\_ANNOY} {APCUPSD\_ANNOY}{APCUPSDANNOY}
  
  Zeit in Sekunden vor dem System-Shutdown zu dem die  User zum Ausloggen 
  aufgefordet werden ( Vorgabe \var{'300'}). 
  \var{'0'} schaltet die Aufforderung aus. 


\config {APCUPSD\_ANNOYDELAY} {APCUPSD\_ANNOYDELAY}{APCUPSDANNOYDELAY}
 
  Zeit in Sekunden nach dem Spannungsausfall bis die User zum Ausloggen 
  aufgefordet werden (Vorgabe \var{'60'}).


\config {APCUPSD\_NOLOGON} {APCUPSD\_NOLOGON}{APCUPSDNOLOGON}
 
  Die Bedingung die bestimmt ob sich User während eines Spannungsausfall einloggen
  können.
  \var{APCUPSD\_NOLOGON} kann einen der Werte 
  \var{'disable'}, \var{'timeout'}, \var{'percent'}, \var{'minutes'} or
  \var{'always'} annehmen (Vorgabe \var{'disable'}).


\config {APCUPSD\_KILLDELAY} {APCUPSD\_KILLDELAY}{APCUPSDKILLDELAY}
 
  Wenn dieser Wert ungleich Null ist, wird apcupsd nach der Shutdown-Anforderung 
  weiterlaufen und wird nach der eingetragenen Zeit in Sekunden versuchen die
  Spannung auszuschalten.
  
  Diese Einstellung ist für die Systeme auf denen apcupsd nach einem Shutdown
  nicht wieder die Kontrolle erlangen kann
  (Vorgabe \var{'0'}).
  \var{'0'} zum Auschalten.

\end {description}

\subsection{Network server settings}

\begin {description}

\config {APCUPSD\_NETSERVER} {APCUPSD\_NETSERVER}{APCUPSDNETSERVER}
 
  Der Wert \var{`yes'} aktiviert, \var{'no'} deaktiviert den Netzwerk 
  Information Server. Wenn aktiviert wird ein Netzwerk-Information-Server 
  Prozess wird gestartet, der Status- und Ereigniss-Daten über das Netzwerk verteilt.
  (Vorgabe \var{'no'}).


\config {APCUPSD\_NISIP} {APCUPSD\_NISIP}{APCUPSDNISIP}
 
  IP-Addresse auf der der NIS-Server auf eingehende Verbindungen lauscht.
  Das ist nützlich falls der fli4l über mehrere Netzwerkverbindungen 
  verfügt. Vorgabe ist \var{'0.0.0.0'} was alle eingenden Verbindungen erlaubt.
  Alternativ, können Sie eine beliebige IP-Addresse des fli4l eintragen und der 
  NIS wird nur auf diesem Interface auf eingehende Verbindungen lauschen.
  Tragen Sie die loopback Addresse (\var{'127.0.0.1'}) ein ,um nur lokale 
  Verbindungen zu erlauben


\config {APCUPSD\_NISPORT} {APCUPSD\_NISPORT}{APCUPSDNISPORT}
 
  TCP-Port zum Senden der Status- und Ereigniss-Daten über das Netzwerk.
  Nur genutzt wenn \smalljump{APCUPSDNETSERVER}{\var{APCUPSD\_NETSERVER}}
  \var{'on'} ist.
  Wenn Sie den Port ändern müssen auch die Porteinstellungen im cgi-Verzeichnis 
  geändert werden und die cgi-Programme neu erstellt werden.
  Vorgabe ist \var{'3551'} wie bei der IANA registriert.


\config {APCUPSD\_EVENTSFILE} {APCUPSD\_EVENTSFILE}{APCUPSDEVENTSFILE}
 
  Wenn die letzten Ereignisse über den Netzwerk-Information-Server abgefragt 
  werden sollen, muss eine Ereignissdatei erzeugt werden.
  (Vorgabe \var{'/var/log/apcupsd.events'})


\config {APCUPSD\_EVENTSFILEMAX} {APCUPSD\_EVENTSFILEMAX}{APCUPSDEVENTSFILEMAX}

  Im Defaultfall darf die Größe des  
  \smalljump{APCUPSDEVENTSFILE}{\var{APCUPSD\_EVENTSFILE}} 
  10 kilobytes nicht überschreiten.  
  Wenn die Datei diese Größe überschreitet, werden ältere Ereignisse vom Anfang 
  Datei gelöscht (first in first out). 
  Der Parameter \var{APCUPSD\_EVENTSFILEMAX} kann auf einen anderen kilobyte  
  Wert oder auf Null gesetzt werden um 
  \smalljump{APCUPSDEVENTSFILE}{\var{APCUPSD\_EVENTSFILE}} ohne Grenze wachsen 
  zu lassen.

\end {description}

\subsection{configuration for share the UPS}

\begin {description}

\config {APCUPSD\_UPSCLASS} {APCUPSD\_UPSCLASS}{APCUPSDUPSCLASS}

   Normalerweise \var{'standalone'} ausser die USV wird mit einer 
   APC ShareUPS-Karte geteilt.
   \var{APCUPSD\_UPSCLASS} kann die Werte 
   \var{'standalone'}, \var{'shareslave'} oder \var{'sharemaster'} annehmen
   (Vorgabe \var{'standalone'}).


\config {APCUPSD\_UPSMODE} {APCUPSD\_UPSMODE}{APCUPSUPSMODE}

   Normalerweise \var{'disable'} ausser die USV wird mit einer 
   APC ShareUPS-Karte geteilt.
   \var{APCUPSD\_UPSMODE} kann die Werte \var{'disable'} or \var{'share'} 
   annehmen (Vorgabe \var{'disable'}).

\end {description}

\subsection{Logging system}

\begin {description}

\config {APCUPSD\_STATTIME} {APCUPSD\_STATTIME}{APCUPSSTATTIME}
 
  Zeitintervall in Sekunden in dem die Status-Datei geschrieben wird 
  (Vorgabe \var{'0'}). \var{'0'} schaltet das Schreiben aus.


\config {APCUPSD\_STATFILE} {APCUPSD\_STATFILE}{APCUPSSTATFILE}
 
 Pfad zur Status-Datei (Vorgabe \var{'/var/log/apcupsd.status'})
 (Wird nur geschrieben wenn \smalljump{APCUPSDSTATTIME}{\var{APCUPSD\_STATFILE}} 
 nicht Null ist).


\config {APCUPSD\_LOGSTATS} {APCUPSD\_LOGSTATS}{APCUPSLOGSTATS}

  \var{'on'} aktiviert, \var{'off'} deaktiviert das Loggen des Status.
  
  Anmerkung! Dies erzeugt viel Output, falls eingeschaltet, so stellen Sie
  sicher, das die Datei die in syslog.conf für LOG\_NOTICE 
  eine 'named pipe' ist (Vorgabe \var{'off'}).
  Sie wollen normalerweise diesen Wert nicht auf \var{'on'} setzen.


\config {APCUPSD\_DATATIME} {APCUPSD\_DATATIME}{APCUPSDATATIME}
 
   Zeitinterval in Sekunden in dem die Daten in die Log-Datei geschrieben werden
   (Vorgabe \var{'0'}). \var{'0'} schaltet das Loggen aus..
   


\config {APCUPSD\_FACILITY} {APCUPSD\_FACILITY}{APCUPSFACILITY}
 
  Definiert den Protokoll-Bereich (Klasse) zum Loggen ins syslog. 
  (Vorgabe \var{'daemon'}).
  Sinnvoll um die Daten die vom apcupsd geloggt werden von anderen Programmen 
  zu unterscheiden.

\end {description}

\subsection{Ereignis-Mails}

\begin {description}

\config {OPT\_APCUPSD\_EVENTMAIL} {OPT\_APCUPSD\_EVENTMAIL}{OPTAPCUPSDEVENTMAIL}

  Wenn auf \var{'yes'} gesetzt werden Ereignis-Mails über das Konto 
  \smalljump{APCUPSDEVENTMAILACCOUNT}{\var{APCUPSD\_EVENTMAIL\_ACCOUNT}} an die Adresse 
  \smalljump{APCUPSDEVENTMAILTO}{\var{APCUPSD\_EVENTMAIL\_TO}} gesendet.
  Das MAILSEND Paket muss mit \var{OPT\_MAILSEND='yes'} aktiviert sein.
  
  (Vorgabe \var{'no'}). 
  
\config {APCUPSD\_EVENTMAIL\_ACCOUNT} {APCUPSD\_EVENTMAIL\_ACCOUNT}{APCUPSDEVENTMAILACCOUNT}

  Optionaler MAILSEND Kontoname über den Ereignis-Mails verschickt werden.
  Wenn der Kontoname nicht angegeben wird, so wird das Konto \var{'default'}
  verwendet.

\config {APCUPSD\_EVENTMAIL\_TO} {APCUPSD\_EVENTMAIL\_TO}{APCUPSDEVENTMAILTO}

  Die eMail Adresse an die Ereignis-Mails verschickt werden.
  Es können eine oder mehrere, durch Komma getrennte eMail Adressen angegeben werden 

\end {description}

% \subsection{Im moment nicht gesetzte Variablen von apcupsd}
% 
%  ========== Configuration statements used in updating the UPS EPROM =========
% 
% 
%  These statements are used only by apctest when choosing "Set EEPROM with conf
%  file values" from the EEPROM menu. THESE STATEMENTS HAVE NO EFFECT ON APCUPSD.
% 
%  UPS name, max 8 characters 
% UPSNAME UPS\_IDEN
% 
%  Battery date - 8 characters
% BATTDATE mm/dd/yy
% 
%  Sensitivity to line voltage quality (H cause faster transfer to batteries)  
%  SENSITIVITY H M L        (default = H)
% SENSITIVITY H
% 
%  UPS delay after power return (seconds)
%  WAKEUP 000 060 180 300   (default = 0)
% WAKEUP 60
% 
%  UPS Grace period after request to power off (seconds)
%  SLEEP 020 180 300 600    (default = 20)
% SLEEP 180
% 
%  Low line voltage causing transfer to batteries
%  The permitted values depend on your model as defined by last letter 
%   of FIRMWARE or APCMODEL. Some representative values are:
%     D 106 103 100 097
%     M 177 172 168 182
%     A 092 090 088 086
%     I 208 204 200 196     (default = 0 => not valid)
% LOTRANSFER  208
% 
%  High line voltage causing transfer to batteries
%  The permitted values depend on your model as defined by last letter 
%   of FIRMWARE or APCMODEL. Some representative values are:
%     D 127 130 133 136
%     M 229 234 239 224
%     A 108 110 112 114
%     I 253 257 261 265     (default = 0 => not valid)
% HITRANSFER 253
% 
%  Battery charge needed to restore power
%  RETURNCHARGE 00 15 50 90 (default = 15)
% RETURNCHARGE 15
% 
%  Alarm delay 
%  0 = zero delay after pwr fail, T = power fail + 30 sec, L = low battery, N = never
%  BEEPSTATE 0 T L N        (default = 0)
% BEEPSTATE T
% 
%  Low battery warning delay in minutes
%  LOWBATT 02 05 07 10      (default = 02)
% LOWBATT 2
% 
%  UPS Output voltage when running on batteries
%  The permitted values depend on your model as defined by last letter 
%   of FIRMWARE or APCMODEL. Some representative values are:
%     D 115
%     M 208
%     A 100
%     I 230 240 220 225     (default = 0 => not valid)
% OUTPUTVOLTS 230
% 
%  Self test interval in hours 336=2 weeks, 168=1 week, ON=at power on
%  SELFTEST 336 168 ON OFF  (default = 336)
% SELFTEST 336
