% Last Update: $Id$

\section{Allgemeine Skript-Erstellung auf fli4l}


Hier folgt jetzt \emph{keine} allgemeine Einführung in Shell-Skripte, das
kann jeder im Internet selber nachlesen, es wird nur auf die spezielle
Gegebenheiten bei fli4l eingegangen. Informationen dazu gibt es in den diversen
Unix-/Linux-Hilfeseiten. Folgende Links können als Einstiegspunkte zu
diesem Thema dienen:
\begin{itemize}
\item Einführung in Shell-Skripte:
  \begin{itemize}
  \item \altlink{http://cip.physik.uni-freiburg.de/main/howtos/sh.php}
  \end{itemize}
 \item
   Hilfeseiten online:
   \begin{itemize}
   \item \altlink{http://linux.die.net/}
   \item \altlink{http://heapsort.de/man2web}
   \item \altlink{http://man.he.net/}
   \item \altlink{http://www.linuxcommand.org/superman_pages.php}
   \end{itemize}
\end{itemize}

\subsection{Aufbau}

    In der Unix-Welt ist es nötig, ein Skript mit dem Namen des Interpreters
    zu beginnen, daher steht in der ersten Zeile:
\begin{example}
\begin{verbatim}
      #!/bin/sh
\end{verbatim}
\end{example}

    Damit man später leichter erkennen kann, was ein Skript macht und wer es
    geschrieben hat, sollte jetzt ein kurzer Header folgen, in etwa so:

\begin{example}
\begin{verbatim}
      #--------------------------------------------------------------------
      # /etc/rc.d/rc500.dummy - start my cool dummy server
      #
      # Creation:     19.07.2001  Toller Hecht <toller-hecht@example.net>
      # Last Update:  11.11.2001  Süße Maus <suesse-maus@example.net>
      #--------------------------------------------------------------------
\end{verbatim}
\end{example}

    Nun kann das eigentliche Skript folgen...


\marklabel{dev:sec:config-variables}{
\subsection{Umgang mit Konfigurationsvariablen}
}


    Pakete werden über die Datei \texttt{config/<PACKAGE>.txt}
    konfiguriert. Die darin enthaltenen und
    \jump{subsec:dev:var-check}{aktiven Variablen} werden beim Erzeugen
    des Boot-Mediums in die Datei \texttt{rc.cfg} übernommen. Beim Booten des
    Routers wird diese Datei eingelesen, bevor irgend ein rc-Skript
    (Skripte unter \texttt{/etc/rc.d/}) gestartet wird. Diese Skripte können
    dadurch auf alle Konfigurationsvariablen einfach durch
    \var{\$<Variablenname>} zugreifen.

    Benötigt man Werte von Konfigurationsvariablen auch noch nach dem
    Booten, dann kann man sie aus der \texttt{/etc/rc.cfg} extrahieren, in
    welche während des Bootens die Konfiguration des Boot-Mediums geschrieben
    wurde. Möchte man beispielsweise den Wert der Variable \texttt{OPT\_DNS}
    in einem Skript auslesen, so kann man dies folgendermaßen tun:

\begin{example}
\begin{verbatim}
    eval $(grep "^OPT_DNS=" /etc/rc.cfg)
\end{verbatim}
\end{example}

    Das funktioniert auch mit mehreren Variablen effizient (d.\,h.\ mit nur
    einem Aufruf des \texttt{grep}-Programms):

\begin{example}
\begin{verbatim}
    eval $(grep "^\(HOSTNAME\|DOMAIN_NAME\|OPT_DNS\|DNS_LISTEN_N\)=" /etc/rc.cfg)
\end{verbatim}
\end{example}

\marklabel{dev:sec:persistent-data}{
\subsection{Persistente Speicherung von Daten}
}

Gelegentlich benötigt ein Paket die Möglichkeit, Daten persistent abzulegen,
die also einen Neustart des Routers überleben. Dazu existiert die Funktion
\texttt{map2persistent}, die von einem Skript in \texttt{/etc/rc.d/}
aufgerufen werden kann. Sie erwartet eine Variable, die einen Pfad enthält,
und ein Unterverzeichnis. Die Idee ist, dass die Variable entweder einen
tatsächlichen Pfad beschreibt~-- dann wird dieser Pfad auch genommen, denn der
Nutzer hat ihn so gewünscht, oder die Zeichenkette "`auto"'~-- dann wird
unterhalb eines Verzeichnisses auf einem persistenten Medium ein entsprechendes
Unterverzeichnis gemäß dem zweiten Parameter erzeugt. Die Funktion liefert
das Resultat in eben der Variable zurück, deren Name im ersten Parameter
übergeben wurde.

Ein Beispiel soll dies verdeutlichen. Sei \var{VBOX\_SPOOLPATH} eine Variable,
die einen Pfad oder die Zeichenkette "`auto"' enthält. Dann führt der Aufruf

\begin{example}
\begin{verbatim}
    begin_script VBOX "Configuring vbox ..."
    [...]
    map2persistent VBOX_SPOOLPATH /spool
    [...]
    end_script
\end{verbatim}
\end{example}

dazu, dass die Variable \var{VBOX\_SPOOLPATH} entweder gar nicht verändert
wird (falls sie einen Pfad enthält), oder dass sie durch den Pfad
\texttt{/var/lib/persistent/vbox/spool} ersetzt wird (falls sie die Zeichenkette
"`auto"' enthält). Dabei verweist\footnote{mit Hilfe eines so genannten
"`bind"'-Mounts} \texttt{/var/lib/persistent} auf ein
Verzeichnis auf einem beschreibbaren und nicht flüchtigen Speichermedium, und
\var{<SCRIPT>} stellt das aufrufende Skript in Kleinbuchstaben dar (dieser Name
wird vom ersten Argument des
\jump{subsec:dev:bug-searching}{\texttt{begin\_script}-Aufrufs} abgeleitet).
Falls kein geeignetes Medium existieren sollte (was durchaus sein kann),
ist \texttt{/var/lib/persistent} ein Verzeichnis in der RAM-Disk.

Zu beachten ist, dass der Pfad, der von \texttt{map2persistent} zurückgegeben
wird, \emph{nicht} automatisch erzeugt wird~-- das muss der Aufrufer selbst
erledigen (etwa durch einen Aufruf von \texttt{mkdir -p <Pfad>}).

In der Datei \texttt{/var/run/persistent.conf} kann nachgeschaut werden, ob
die persistente Speicherung von Daten möglich ist. Beispiel:

\begin{example}
\begin{verbatim}
    . /var/run/persistent.conf
    case $SAVETYPE in
    persistent)
        echo "Persistente Speicherung möglich!"
        ;;
    transient)
        echo "Persistente Speicherung NICHT möglich!"
        ;;
    esac
\end{verbatim}
\end{example}

\marklabel{subsec:dev:bug-searching}{
\subsection{Fehlersuche}
}

    Bei Start-Skripten ist es oft sinnvoll, diese bei Bedarf im Debug-Modus
    der Shell laufen zu lassen, um festzustellen, wo "`der Wurm drin ist"'.
    Dazu wird am Anfang und am Ende folgendes eingefügt:

\begin{example}
\begin{verbatim}
      begin_script <OPT-Name> "start message"
      <script code>
      end_script
\end{verbatim}
\end{example}

Im normalen Betrieb erscheint jetzt beim Start des Skriptes der
angegebene Text und am Ende der gleiche Text mit einem vorangestellten
"`finished"'.

Will man die Skripte debuggen, muss man zwei Dinge tun:

\begin{enumerate}

\item Man muss \jump{DEBUGSTARTUP}{\var{DEBUG\_\-STARTUP}} auf "`yes"'
  setzen.
\item Man muss das Debuggen für dieses OPT aktivieren. Das tut man in
  der Regel durch den Eintrag
\begin{example}
\begin{verbatim}
      <OPT-Name>_DO_DEBUG='yes'
\end{verbatim}
\end{example}
in der Konfigurationsdatei.\footnote{Manchmal werden mehrere
  Start-Skripte verwendet, die dann auch verschiedene Namen für ihre
  Debug-Variablen haben. Hier hilft ein kurzer Blick in die Skripte.}
    Jetzt wird während der Ausführung am Bildschirm genau dargestellt, was
    passiert.
\end{enumerate}


\subsubsection{Weitere beim Debuggen hilfreiche Variablen}

\begin{description}

  \config{DEBUG\_ENABLE\_CORE}{DEBUG\_ENABLE\_CORE}{DEBUGENABLECORE}

  Diese Variable gestattet das Erzeugen von "`Core-Dumps"' (Speicherauszügen).
  Stürzt ein Programm aufgrund eines Fehlers ab, wird ein Abbild des aktuellen
  Zustandes im Dateisystem abgelegt, der hinterher zur Analyse des
  Problems verwendet werden kann. Die Core-Dumps werden unter
  \texttt{/var/log/dumps/} abgelegt.

  \config{DEBUG\_IP}{DEBUG\_IP}{DEBUGIP}

  Wird diese Variable gesetzt, werden alle Aufrufe des Programms \texttt{ip}
  protokolliert.

  \config{DEBUG\_IPUP}{DEBUG\_IPUP}{DEBUGIPUP}

  Wird diese Variable auf "`yes"' gesetzt, werden während der
  Ausführung der \texttt{ip-up}/\texttt{ip-down}-Skripte die ausgeführten
  Anweisungen mitgezeichnet und im System-Protokoll gespeichert.

  \config{LOG\_BOOT\_SEQ}{LOG\_BOOT\_SEQ}{LOGBOOTSEQ}

  Wird diese Variable auf "`yes"' gesetzt, protokolliert der \texttt{bootlogd}
  während des Bootens alle auf der Konsole getätigten Ausgaben in der Datei
  \texttt{/var/tmp/boot.log}. Diese Variable hat standardmäßig den Wert "`yes"'.

  \config{DEBUG\_KEEP\_BOOTLOGD}{DEBUG\_KEEP\_BOOTLOGD}{DEBUGKEEPBOOTLOGD}

  Normalerweise wird der \texttt{bootlogd} am Ende des Bootvorganges
  beendet. Ein Setzen dieser Variable unterbindet das und erlaubt ein
  Protokollieren der Konsolenausgaben über den Bootvorgang hinaus.

  \config{DEBUG\_MDEV}{DEBUG\_MDEV}{DEBUGMDEV}

  Ein Setzen dieser Variable generiert ein Protokoll des \texttt{mdev}-Daemons,
  der für das Anlegen der Geräte-Dateien unter \texttt{/dev} zuständig ist.

\end{description}
\subsection{Hinweise}
\begin{itemize}
\item  Es ist \emph{immer} besser, geschweifte Klammern "`\{\ldots\}"' an Stelle von runden
      Klammern "`(\ldots)"' zu benutzen. Allerdings muss dabei darauf geachtet werden,
      dass nach der öffnenden Klammer ein Leerzeichen oder eine neue Zeile vor
      dem nächsten Befehl kommt und vor der schließenden Klammer ein
      Semikolon oder auch eine neue Zeile kommt. Beispielsweise ist

\begin{example}
\begin{verbatim}
        { echo "cpu"; echo "quit"; } | ...
\end{verbatim}
\end{example}

      \noindent gleichbedeutend mit:

\begin{example}
\begin{verbatim}
        {
                echo "cpu"
                echo "quit"
        } | ...
\end{verbatim}
\end{example}


      \item Ein Skript kann mit "`exit"' vorzeitig beendet
        werden. Dies ist aber bei den Start-Skripten (\texttt{opt/etc/boot.d/...},
        \texttt{opt/etc/rc.d/...}), den Stopp-Skripten (\texttt{opt/etc/rc0.d/...}) und den
        \texttt{ip-up}/\texttt{ip-down}-Skripten (\texttt{opt/etc/ppp/...}) geradezu tödlich, da
        auch nachfolgende Skripte nicht mehr ausgeführt werden. Im
        Zweifelsfall immer weglassen.


      \item KISS~-- Keep it small and simple. Du willst Perl als
        Skript-Sprache verwenden? Dir reichen die
        Skripting-Möglichkeiten von fli4l nicht?  Überdenke deine
        Einstellung! Ist dein OPT wirklich nötig? fli4l ist immer noch
        "`nur"' ein Router, ein Router sollte eigentlich keine
        Serverdienste anbieten.


      \item Die Fehlermeldung "`: not found"' heißt meistens, dass das
        Skript noch im DOS-Format vorliegt. Weitere Fehlerquelle: Das
        Skript ist nicht ausführbar. In beiden Fällen sollte die
\texttt{opt/<PACKAGE>.txt}-Datei daraufhin geprüft werden, ob sie die
korrekten Optionen (in Bezug auf "`mode"', "`gid"', "`uid"' und Flags) enthält. Wenn das Skript erst
bem Booten erzeugt wird, sollte ein "`chmod +x <Skriptname>"' ausgeführt
werden.

      \item Für temporäre Dateien sollte der Pfad \texttt{/tmp} genutzt werden.
        Es ist aber unbedingt darauf zu achten, dass hier nur wenig
        Platz ist, weil dies in der RootFS-RAM-Disk liegt! Wenn mehr
        Platz benötigt wird, muss man sich eine eigene RAM-Disk
        erstellen und mounten. Details dazu verrät der Abschnitt "`RAM-Disks"'
        in dieser Dokumentation.

    \item Damit temporäre Dateien eindeutige Namen erhalten, sollte man
      grundsätzlich die aktuelle Prozess-ID, die in der Shell-Variable "`\$"'
      gespeichert ist, an den Dateiname anhängen.
      \texttt{/tmp/<OPT-Name>.\$\$} stellt somit einen guten Dateinamen dar,
      \texttt{/tmp/<OPT-Name>} eher weniger, wobei \texttt{<OPT-Name>} natürlich
      nicht so stehen bleiben soll, sondern geeignet ersetzt werden muss.

\end{itemize}
