% Last Update: $Id$
\marklabel{sec:domainkonfiguration}{\section{Domain-Konfiguration}}

  Windows-PCs im LAN haben eine unangenehme
  Eigenschaft: Sobald ein Nameserver benötigt wird und man diesen
  deshalb im Windows einstellt, fragen diese Windows-Rechner den
  angegebenen Nameserver in regelmäßigen Abständen ab
  ~-- auch wenn man gar nicht daran arbeitet!
  Würde man also auf dem Windows-PC einen DNS-Server im Internet angeben,
  wird's teuer\ldots

  Der Trick ist nun folgender: Wenn im LAN nicht bereits ein
  DNS-Server vorhanden ist, kann man den DNS-Server im fli4l-Router
  verwenden.

  Es wird DNSMASQ als DNS-Server eingesetzt.

  Wenn man jedoch mit der DNS-Konfiguration beginnt, sollte man sich
  zunächst Gedanken über den Domain-Namen und die Namen der PCs im
  Netz machen. Der verwendete Domain-Name wird nicht im Internet
  sichtbar. Deshalb kann man sich hier prinzipiell beliebige
  Domain-Namen ausdenken.

  Außerdem sollte man jedem Windows-Rechner einen Namen verpassen.
  Diese Namen müssen dem fli4l-Router bekannt sein.

  \begin{description}

    \config{DOMAIN\_NAME}{DOMAIN\_NAME}{DOMAINNAME}
    
    Standard-Einstellung: \var{DOMAIN\_NAME='lan.fli4l'}
    
    {Hier kann sich jeder austoben, da die lokal
      verwendete Domain nicht im Internet sichtbar wird.
      Sie sollten lediglich vermeiden, einen Namen zu benutzen, den es im
      Internet geben könnte (z.B. irgendwas.de), da Sie sonst nicht
      auf diese Domain werden zugreifen können.}


    \config{DNS\_FORWARDERS}{DNS\_FORWARDERS}{DNSFORWARDERS}
    
    Standard-Einstellung: \var{DNS\_FORWARDERS=''}
    
    {Hier ist die DNS-Server-Adresse des Internet-Providers anzugeben,
      wenn fli4l als Router in das Internet verwendet wird. Der
      fli4l-Router gibt dann sämtliche DNS-Anfragen, die er nicht
      selbst beantworten kann, an diese Adresse weiter.

      Möchte man mehrere DNS-Forwarder angeben, trennt man die
      IP-Adressen durch Leerzeichen.

      Sind mehrere DNS-Server konfiguriert werden diese in der angegebenen 
      Reihenfolge für DNS-Anfragen genutzt, somit wird der zweite angegebene
      Server nur genutzt, wenn der erste keine Antwort liefert usw.

      Es ist auch möglich, optional Port-Nummern an die IP-Adressen durch
      Doppelpunkt getrennt anzugeben. Allerdings muss dann \jump{OPTDNS}{\var{OPT\_DNS='yes'}} sein (Paket
\jump{sec:dnsdhcp}{dns\_dhcp})
      und es darf nirgends die Option \var{*\_USEPEERDNS} benutzt werden.

      Achtung: Auch wenn
        \begin{itemize}
        \item \jump{PPPOEUSEPEERDNS}{\var{PPPOE\_USEPEERDNS}},
        \item \jump{ISDNCIRCxUSEPEERDNS}{\var{ISDN\_CIRC\_x\_USEPEERDNS}}
          oder
        \item \jump{DHCPCLIENTxUSEPEERDNS}{\var{DHCP\_CLIENT\_x\_USEPEERDNS}}
        \end{itemize}

        gesetzt (='yes') ist, ist hier die Eintragung eines Servers
        nötig, da sonst direkt nach dem Start keine Namensauflösung
        möglich ist.

      Ausnahme: fli4l als Router in einem lokalen Netz \emph{ohne}
      Anschluss an das Internet oder (Firmen-)Netze mit weiteren
      DNS-Servern. In diesem Fall ist 127.0.0.1 anzugeben, um das
      Weiterleiten zu unterbinden.}

    \config{HOSTNAME\_IP}{HOSTNAME\_IP}{HOSTNAMEIP} (optional)

    {Hiermit kann optional festgelegt werden, an welches Netz \var{IP\_NET\_x}
    der \var{HOSTNAME} gebunden wird.}

    \config{HOSTNAME\_ALIAS\_N}{HOSTNAME\_ALIAS\_N}{HOSTNAMEALIASN} (optional)

    {Anzahl der zusätzlichen Alias-Hostnamen für den Router.}

    \config{HOSTNAME\_ALIAS\_x}{HOSTNAME\_ALIAS\_x}{HOSTNAMEALIASx} (optional)

    {Zusätzlicher Alias-Name für den Router.}

  \end{description}
