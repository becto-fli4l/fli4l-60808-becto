% Last Update: $Id$
\section{Netzwerk-Konfiguration (IPv6)}

\subsection{Einleitung}

Zusätzlich zu der im letzten Abschnitt vorgestellten Konfiguration von
IPv4-Netzwerken ist es auch möglich, den fli4l-Router in vielerlei Hinsicht
IPv6-tauglich zu machen. Dazu gehören Angaben über die IPv6-Adresse des
Routers, die verwalteten IPv6-(Sub-)Netze sowie vordefinierte IPv6-Routen und
Firewall-Regeln bzgl. IPv6-Paketen. IPv6 ist der Nachfolger des
Internet-Protokolls IPv4. Hauptsächlich wurde es entwickelt, um die relativ
kleine Menge von eindeutigen Internet-Adressen zu vergrößern: Während IPv4
ungefähr \(2^{32}\) Adressen unterstützt,\footnote{nur ungefähr, weil einige
Adressen speziellen Zwecken dienen, etwa für Broad- und Multicasting} sind es
bei IPv6 bereits \(2^{128}\) Adressen. Dadurch kann mit IPv6 jedem
kommunizierenden Host eine eindeutige Adresse zugeordnet werden, und man ist
nicht mehr auf Techniken wie NAT, PAT, Masquerading etc. angewiesen.

Neben diesem Aspekt spielten bei der Entwicklung des IPv6-Protokolls auch
Themen wie Selbstkonfiguration und Sicherheit eine Rolle. Dies wird in späteren
Abschnitten aufgegriffen.

Das größte Problem bei IPv6 ist dessen Verbreitung: Momentan wird IPv6~--
verglichen mit IPv4~-- nur sehr spärlich verwendet. Das liegt daran, dass IPv6
und IPv4 technisch nicht miteinander kompatibel sind und somit alle Software-
und Hardware-Komponenten, die an der Paket-Weiterleitung im Internet beteiligt
sind, für IPv6 nachgerüstet werden müssen. Auch bestimmte Dienste wie DNS
(Domain Name System) müssen für IPv6 entsprechend aufgebohrt werden.

Hier tut sich also ein Teufelskreis auf: Die geringe Verbreitung von IPv6 bei
Dienstanbietern im Internet führt zu Desinteresse seitens der
Router-Hersteller, ihre Geräte mit IPv6-Funktionalität auszustatten, was
wiederum dazu führt, dass Dienstanbieter die Umstellung auf IPv6 scheuen, weil
sie fürchten, dass sich der Aufwand nicht lohnt. Erst langsam wendet sich das
Blatt zugunsten von IPv6, nicht zuletzt unter dem immer stärkeren Druck der
knappen Adressvorräte.\footnote{Inzwischen sind die letzten IPv4-Adressblöcke
von der IANA vergeben worden.}

\subsection{Adressformat}

Eine IPv6-Adresse besteht aus acht Zwei-Byte-Werten, die hexadezimal notiert
werden:

\emph{Beispiel 1:} \verb*?2001:db8:900:551:0:0:0:2?

\emph{Beispiel 2:} \verb*?0:0:0:0:0:0:0:1? (IPv6-Loopback-Adresse)

Um die Adressen etwas übersichtlicher zu gestalten, werden aufeinander folgende
Nullen zusammengelegt, indem sie entfernt werden und lediglich zwei unmittelbar
aufeinander folgende Doppelpunkte verbleiben. Die obigen Adressen können also
auch so geschrieben werden:

\emph{Beispiel 1 (kompakt):} \verb*?2001:db8:900:551::2?

\emph{Beispiel 2 (kompakt):} \verb*?::1?

Eine solche Kürzung ist aber nur höchstens einmal erlaubt, um Mehrdeutigkeiten
zu vermeiden. Die Adresse \verb*?2001:0:0:1:2:0:0:3? kann also entweder zu
\verb*?2001::1:2:0:0:3? oder zu \verb*?2001:0:0:1:2::3? verkürzt werden, nicht
aber zu \verb*?2001::1:2::3?, da jetzt unklar wäre, wie die vier Nullen jeweils
auf die zusammengezogenen Bereiche verteilt werden sollen.

Eine weitere Mehrdeutigkeit existiert, wenn eine IPv6-Adresse mit einem Port
(TCP oder UDP) kombiniert werden soll: In diesem Fall darf man den Port nicht
unmittelbar mit Doppelpunkt und Wert anschließen, weil der Doppelpunkt bereits
innerhalb der Adresse verwendet wird und somit in manchen Fällen unklar wäre,
ob die Port-Angabe nicht vielleicht doch eine Adress-Komponente darstellt.
Deshalb muss in solchen Fällen die IPv6-Adresse in eckigen Klammern angegeben
werden. Dies ist auch die Syntax, wie sie in URLs gefordert wird (etwa wenn
im Web-Browser eine numerische IPv6-Adresse verwendet werden soll).

\emph{Beispiel 3:} \verb*?[2001:db8:900:551::2]:1234?

Ohne die Verwendung von Klammern entsteht die Adresse
\verb*?2001:db8:900:551::2:1234?, die unverkürzt der Adresse
\verb*?2001:db8:900:551:0:0:2:1234? entspricht und somit keine Port-Angabe
besitzt.

Wenn Sie keinen Internet-Zugang mit nativer IPv6-Anbindung nutzen können,
benötigen Sie einen IPv6-Tunnel zu einem IPv6-Anbieter. Dies wird durch das
Paket ``ipv6'' ermöglicht. Bitte lesen Sie die Dokumentation des ``ipv6''-Pakets
für Details.

\subsection{Konfiguration}

\subsubsection{Allgemeine Einstellungen}

Die allgemeinen Einstellungen beinhalten zum einen die Aktivierung der
IPv6-Unterstützung und zum anderen die optionale Vergabe einer IPv6-Adresse an
den Router.

\begin{description}
\config{OPT\_IPV6}{OPT\_IPV6}{OPTIPV6}{
Aktiviert die IPv6-Unterstützung.

Standard-Einstellung:
}
\verb*?OPT_IPV6='no'?

\config{HOSTNAME\_IP6}{HOSTNAME\_IP6}{HOSTNAMEIP6}{(optional)
Diese Variable stellt explizit die IPv6-Adresse des Routers ein. Falls die
Variable nicht gesetzt wird, wird die IPv6-Adresse auf die Adresse des ersten
konfigurierten IPv6-Subnetzes gesetzt (\var{IPV6\_NET\_x}, siehe unten).

Beispiel:
}
\verb*?HOSTNAME_IP6='IPV6_NET_1_IPADDR'?
\end{description}

\subsubsection{Subnetz-Konfiguration}

In diesem Abschnitt wird die Konfiguration von einem oder mehreren
IPv6-Subnetzen vorgestellt. Ein IPv6-Subnetz ist ein IPv6-Adressraum, der über
ein so genanntes Präfix spezifiziert wird und an eine bestimmte
Netzwerk-Schnittstelle gebunden ist. Weitere Einstellungen betreffen das
Veröffentlichen des Präfixes und des DNS-Dienstes innerhalb des Subnetzes
sowie einen optionalen Router-Namen innerhalb dieses Subnetzes.

\begin{description}
\config{IPV6\_NET\_N}{IPV6\_NET\_N}{IPV6NETN}{
Diese Variable enthält die Anzahl der verwendeten IPv6-Subnetze. Mindestens ein
IPv6-Subnetz sollte definiert werden, um IPv6 im lokalen Netz nutzen zu können.

Standard-Einstellung:
}
\verb*?IPV6_NET_N='0'?

\config{IPV6\_NET\_x}{IPV6\_NET\_x}{IPV6NETx}{
Diese Variable enthält für ein bestimmtes IPv6-Subnetz die IPv6-Adresse des
Routers sowie die Größe der Netzmaske in CIDR-Notation. Soll das Subnetz
öffentlich geroutet werden, so stammt es in der Regel vom Internet- bzw.
Tunnel-Anbieter.

\wichtig{Soll in dem Subnetz die zustandslose Selbstkonfiguration (siehe
den Abschnitt zu \var{IPV6\_NET\_x\_ADVERTISE} weiter unten) aktiviert werden,
dann muss die Länge des Sub\-netz-Präfixes 64 Bit betragen!}

Ist das Subnetz an einen Tunnel angeschlossen oder wird DHCPv6 zum
Ermitteln eines Präfixes verwendet, dann darf hier nur der Teil der
Router-Adresse angegeben werden, der \emph{nicht} zum dem Tunnel zugeordneten
bzw. vom DHCPv6-Server erhaltenen Subnetz-Präfixes  gehört, da jenes
Präfix und diese Adresse miteinander kombiniert werden! Siehe die Dokumentation
der ``ipv6''- (Tunnel) bzw. ``dns\_dhcp''-Pakete (DHCPv6) für Details.

Beispiele:
}

\begin{example}
\begin{verbatim}
IPV6_NET_1='2001:db8:1743:42::1/64'       # statisch: komplette Adresse
IPV6_NET_2='{he}+0:0:0:42::1/64'          # HE-Tunnel: partielle Adresse
IPV6_NET_3='{dhcpv6}+0:0:0:43::1/64'      # DHCPv6: partielle Adresse
\end{verbatim}
\end{example}

\config{IPV6\_NET\_x\_DEV}{IPV6\_NET\_x\_DEV}{IPV6NETxDEV}{
Diese Variable enthält für ein bestimmtes IPv6-Subnetz den Namen der
Netzwerk-Schnittstelle, an welche das IPv6-Netz gebunden wird. Dies
kollidiert \emph{nicht} mit den in der Basis-Konfiguration (\texttt{base.txt})
vergebenen Netzwerk-Schnittstellen, da einer Netzwerk-Schnittstelle sowohl
IPv4- als auch IPv6-Adressen zugeordnet werden dürfen.

Beispiel:
}
\verb*?IPV6_NET_1_DEV='eth0'?

\config{IPV6\_NET\_x\_ADVERTISE}{IPV6\_NET\_x\_ADVERTISE}{IPV6NETxADVERTISE}{
Diese Variable legt fest, ob das eingestellte Subnetz-Präfix im LAN mittels
``Router Advertisements'' verbreitet wird. Dies wird für die so genannte
``stateless autoconfiguration'' (zustandslose Selbstkonfiguration) verwendet
und ist nicht mit DHCPv6 zu verwechseln. Mögliche Werte sind ``yes'' und
``no''.

Es ist empfehlenswert, diese Einstellung zu aktivieren, es sei denn, alle
Adressen im Netz werden statisch vergeben oder ein anderer Router ist bereits
dafür zuständig, das Subnetz-Präfix anzukündigen.

\wichtig{Die automatische Verteilung des Subnetzes funktioniert nur, wenn das
Subnetz ein /64-Netz ist, d.h. wenn die Länge des Subnetz-Präfixes 64 Bit
beträgt! Der Grund hierfür ist, dass die anderen Hosts im Netzwerk ihre
IPv6-Adresse aus dem Präfix und ihrer Host-MAC-Adresse berechnen und dies nicht
funktioniert, wenn der Host-Anteil nicht 64 Bit beträgt. Wenn die
Selbstkonfiguration fehlschlägt, sollte also geprüft werden, ob das
Subnetz-Präfix nicht vielleicht falsch angegeben worden ist (z.B. als /48).}

Standard-Einstellung:
}
\verb*?IPV6_NET_1_ADVERTISE='yes'?

\config{IPV6\_NET\_x\_ADVERTISE\_DNS}{IPV6\_NET\_x\_ADVERTISE\_DNS}{IPV6NETxADVERTISEDNS}{
Diese Variable legt fest, ob auch der lokale DNS-Dienst im IPv6-Subnetz
mittels ``Router Advertisements'' (RDNSS-Option) angekündigt werden soll. Dies
funktioniert aber nur, wenn die IPv6-Funktionalität des DNS-Dienstes mit Hilfe
der Variable \var{DNS\_SUPPORT\_IPV6}='yes' aktiviert ist. Mögliche Werte sind
``yes'' und ``no''.

Standard-Einstellung:
}
\verb*?IPV6_NET_1_ADVERTISE_DNS='no'?

\config{IPV6\_NET\_x\_NAME}{IPV6\_NET\_x\_NAME}{IPV6NETxNAME}{(optional)
Diese Variable legt einen Interface-spezifischen Hostnamen für den Router in
diesem IPv6-Subnetz fest.

Beispiel:
}
\verb*?IPV6_NET_1_NAME='fli4l-subnet1'?

\end{description}
