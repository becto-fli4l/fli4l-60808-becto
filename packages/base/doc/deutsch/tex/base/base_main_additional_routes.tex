% Last Update: $Id$
\section{Zusätzliche Routen (IPv4)}

\begin{description}

  \config{IP\_ROUTE\_N}{IP\_ROUTE\_N}{IPROUTEN}

    Standard-Einstellung: \var{IP\_\-ROUTE\_\-N}='0'

    {Anzahl von zusätzlichen Netzwerkrouten. Zusätzliche Netzwerkrouten
    sind zum Beispiel dann erforderlich, wenn sich im LAN weitere
    Router befinden, über die andere Netzwerke erreichbar sein sollen.

    Im Normalfall ist die Angabe von weiteren Netzwerkrouten nicht
    erforderlich.
    }

  \config{IP\_ROUTE\_x}{IP\_ROUTE\_x}{IPROUTEx}

  {Die zusätzlichen Routen \var{IP\_\-ROUTE\_\-1}, \var{IP\_\-ROUTE\_\-2}, \ldots haben folgenden
    Aufbau:

    \begin{example}
    \begin{verbatim}
        network/netmaskbits gateway
    \end{verbatim}
    \end{example}

    Dabei ist \texttt{network} die Netzwerkadresse, \texttt{/netmaskbits} die Netzmaske
    in \jump{tab:cidr}{CIDR} Schreibweise und \texttt{gateway} die Adresse des Rechners,
    der die Verbindung zu dem Netzwerk herstellt. Der Gateway-Rechner muss
    natürlich im gleichen Netzwerk, wie der fli4l-Router liegen!
    Ist z.B. das Netzwerk 192.168.7.0 mit der Netzwerkmaske
    255.255.255.0 über das Gateway 192.168.6.99 erreichbar, dann lautet der
    Eintrag:

    \begin{example}
    \begin{verbatim}
        IP_ROUTE_N='1'
        IP_ROUTE_1='192.168.7.0/24 192.168.6.99'
    \end{verbatim}
    \end{example}}

    Wenn der fli4l-Router nicht als Internet-Router eingesetzt wird
    sondern nur als reiner Ethernetrouter kann man in einem IP\_ROUTE\_x
    Eintrag eine Default-Route angeben. Dazu wird analog zu der
    network/netmaskbits Schreibweise 0.0.0.0/0 eingetragen, so
    wie im Beispiel zu sehen ist.

    \begin{example}
    \begin{verbatim}
        IP_ROUTE_N='3'
        IP_ROUTE_1='192.168.1.0/24 192.168.6.1'
        IP_ROUTE_2='10.73.0.0/16 192.168.6.1'
        IP_ROUTE_3='0.0.0.0/0 192.168.6.99'
    \end{verbatim}
    \end{example}

    Intern werden die hierüber spezifizierten Routen in Circuits (siehe
    \jump{sect:route-circuits}{Circuits vom Typ ``route''}) umgewandelt. Dabei
    werden alle Routen mit demselben Gateway demselben Circuit zugeordnet.
  \end{description}
