% Last Update: $Id$
\marklabel{beispielbase}{\section{Beispiel-Datei}}\index{Beispiel-Datei(base.txt)}\index{base.txt}

Die Beispiel-Datei \verb+base.txt+ im Verzeichnis \verb+config/+ hat
folgenden Inhalt:

\begin{example}
\verbatimfile{\basedir/config/base.txt}
\end{example}

\medskip

Zu beachten ist, dass diese Datei im DOS-Format gespeichert ist. Das
heißt, sie enthält jeweils am Zeilenende ein zusätzliches
Carriage-Return (CR). Da die meisten Unix-Editoren damit keine
Probleme bekommen wurde dieses Format gewählt, denn
umgekehrt hat der Windows-Editor bei fehlendem CR am Zeilenende keine
Chance!

Sollte es wider Erwarten unter Unix/Linux doch Probleme mit dem
Lieblingseditor geben, kann die Datei vor dem Editieren mit einem
Befehl in das Unix-Format konvertiert werden:

\begin{example}
\begin{verbatim}
        sh unix/dtou config/base.txt
\end{verbatim}
\end{example}

Für die Erstellung des Boot-Mediums ist es völlig unerheblich, ob die
Datei CRs am Zeilenende enthält oder nicht. Sie werden beim Schreiben
auf das Boot-Medium einschließlich der Kommentare komplett ignoriert.

Jetzt aber zum Inhalt \ldots
