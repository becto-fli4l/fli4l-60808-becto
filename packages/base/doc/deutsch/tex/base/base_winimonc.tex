% Last Update: $Id$
  \marklabel{sec:winimonc}{
    \section{Windows-Client imonc.exe}}

  \subsection{Einleitung}

  Das Gespann imond auf dem Router und imonc auf dem Client beherrschen
  zwei Benutzermodi: den User- und den Adminmodus. Im Adminmodus sind alle
  Steuerelemente aktiviert. Im Usermodus steuern die Variablen 
  \jump{IMONDENABLE}{\var{IMOND\_ENABLE}}, \jump{IMONDDIAL}{\var{IMOND\_DIAL}}, 
  \jump{IMONDROUTE}{\var{IMOND\_ROUTE}} und \jump{IMONDREBOOT}{\var{IMOND\_REBOOT}} ob die
  jeweiligen Funktionen im Usermodus zur Verfügung stehen. Sind alle diese 
  Variablen auf `no' gesetzt, bedeutet dies für die Überblick-Seite, dass alle 
  Buttons bis auf den Exit- und den Admin-Mode-Button deaktiviert sind. Die 
  Entscheidung, ob der User- oder Admin-Modus benutzt wird, wird anhand des 
  übermittelten Passwortes getroffen. Über den Button Admin-Mode, der sich in 
  der Statusleiste befindet, kann jederzeit unter Eingabe des Admin-Passwortes 
  vom User- zum Admin-Modus gewechselt werden. Um wieder zurück zu wechseln, 
  muss imonc beendet und neu gestartet werden.

  Sobald imonc gestartet ist, wird ein zusätzliches Tray-Icon angezeigt, welches 
  den Verbindgungsstatus der vorhandenen Kanäle anzeigt.

  Die Farben bedeuten:
  \begin{description}
    \item[Rot]: Offline
    \item[Gelb]: Es wird gerade eine Verbindung aufgebaut
    \item[Hellgrün]: Online und Traffic auf dem Kanal
    \item[Dunkelgrün]: Online und so gut wie kein Traffic auf dem Kanal
  \end{description}
  
  \noindent Ein etwas vom Windows-Standard abweichendes Verhalten zeigt imonc, wenn der 
  Minimieren-Button in der Titelleiste angeclickt wird. Daraufhin minimiert sich 
  imonc in den Systemtray und es bleibt nur noch das Tray-Icon neben der Uhr 
  übrig. Ein Doppelklick mit der linken Maustaste auf das Tray-Symbol holt das 
  imonc-Fenster wieder in den Vordergrund. Mit der rechten Maustaste besteht 
  auch die Möglichkeit über das Kontextmenü, die wichtigsten imonc-Kommandos 
  direkt auszuwählen, ohne imonc wieder auf den Bildschirm zu holen.

  Viele Eigenschaften (darunter auch alle Spaltenbreiten der StringGrids) 
  speichert imonc in der Registry, damit imonc so an die eigenen Bedürfnisse 
  angepasst werden kann. Imonc speichert die Informationen in dem 
  Registry-Schlüssel HKCU{\textbackslash}Software{\textbackslash}fli4l.

  Bestehen trotz sorgfältigen Lesens der Dokumentation noch Probleme in Bezug 
  auf imonc oder auch des Routers selber, die man z.B. in der Newsgroup posten 
  möchte, ist es sinnvoll, auf der Über-Seite des imonc den Punkt SystemInfo 
  auszuwählen und dort den Punkt Support Infos. Daraufhin wird das 
  Router-Passwort abgefragt (nicht das imond-Passwort!). Imonc erstellt dann 
  eine Datei fli4lsup.txt, welche alle wichtigen Informationen bezüglich des 
  Routers und imonc beinhaltet. Diese Datei kann auf explizite Nachfrage in die 
  Newsgroup gepostet werden, so dass deutlich bessere Chancen auf rasche Hilfe
  bestehen.

  Nähere Details betreffend der Entwicklung des Windows-Clients imonc findet man 
  auf der Homepage vom Windows ImonC-Seiten \altlink{http://www.imonc.de/}. Hier 
  kann man sehen, welche neuen Features und Bug-Fixes in der nächsten Version 
  von imonc enthalten sein werden. Ausserdem gibt es dort den neusten imonc, 
  wenn dieser nicht schon in der fli4l-Distribution enthalten ist.

  \subsection{Startparameter}

  ImonC benötigt den Namen oder die IP-Adresse des fli4l-Routers. Standardmäßig 
  versucht das Programm, eine Verbindung mit dem Rechner ``fli4l'' herzustellen. 
  Wenn dieser im DNS korrekt eingetragen ist, sollte es also direkt 
  funktionieren. Ansonsten kann man in der Verknüpfung folgende Parameter 
  übergeben:

  \begin{itemize}
    \item /Server:IP oder Hostname des Routers (Kurzform: /S:IP oder Hostname)
    \item /Password:Passwort (Kurzform: /P:Password)
    \item /log Die Logging-Option zum Protokollieren der Kommunikation zwischen 
      imonc und imond. Ist diese Option eingeschaltet, wird beim Beenden von 
      imonc eine Datei imonc.log geschrieben. Diese Datei beinhaltet die gesamte 
      Kommunikation zwischen Router und Client und wird darum sehr groß. Deshalb 
      sollte dieser Startparameter nur gesetzt werden, wenn Probleme bestehen.
    \item /iport:Portnummer Die Portnummer auf die imond lauscht. Default: 5000
    \item /tport:Portnummer Port auf dem telmond lauscht. Default: 5001
    \item /rc:''Command'' Das hier angegebene Kommando wird ohne weitere 
      Überprüfung an den Router übertragen und anschliessend imonc beendet. 
      Sollen mehrere Kommandos gleichzeitig ausgeführt werden, müssen diese 
      durch Semikolons getrennt werden. Damit es funktioniert, muss ein 
      gesetztes imond-Passwort mit übergeben werden, da keine Abfrage des
      Passwortes erfolgt. Die möglichen Kommandos sind beim imond dokumentiert,
      siehe Kapitel 8.1. Zusätzlich zu den dort aufgeführten Befehlen gibt es
      noch den Befehl timesync. Dieser bewirkt, dass die Uhrzeit des Clients
      mit der des Routers synchronisiert wird. Der Befehl dialtimesync wird
      nicht mehr unterstützt da er sich als \glqq{}dial; timesync\grqq{} schreiben lässt.
    \item /d:''fli4l-Directory'' Hiermit kann das fli4l-Directory per 
      Startparameter übergeben werden. Interessant wenn man mit mehreren
      fli4l-Versionen herumspielt
    \item /wait Wenn der Hostname nicht aufgelöst werden kann, beendet sich 
      imonc nicht mehr~-- erneuter Verbindungsaufbau durch Doppelclick auf das 
      TrayIcon
    \item /nostartcheck Dieser schaltet die Überprüfung ab, ob imonc bereits 
      läuft. Nur sinnvoll, wenn mehrere, unterschiedliche fli4l-Router in einem 
      Netz überwacht werden sollen. Bei weiteren Instanzen werden die 
      eingebauten Syslog- und \mbox{E-Mail}-Funktionalitäten deaktiviert.
  \end{itemize}

  Usage (einzutragen in der Verknüpfung):

\begin{example}
\begin{verbatim}
X:\...imonc.exe [/Server:Host] [/Password:Passwort] [/iport:Portnummer]
            [/log] [/tport:Portnummer] [/rc:"Command"]
\end{verbatim}
\end{example}

  Beispiel mit IP-Adresse:

\begin{example}
\begin{verbatim}
        C:\wintools\imonc /Server:192.168.6.4
\end{verbatim}
\end{example}

  oder mit Namen und Passwort:

\begin{example}
\begin{verbatim}
        C:\wintools\imonc /S:fli4l /P:geheim
\end{verbatim}
\end{example}

  oder mit Namen, Passwort und Routerkommando:

\begin{example}
\begin{verbatim}
        C:\wintools\imonc /S:fli4l /P:geheim /rc:"dialmode manual"
\end{verbatim}
\end{example}

  \subsection{Seite Überblick}

  Der Windows-Client fragt einige imond-Informationen über die bestehenden 
  Verbindungen ab und bereitet sie im Anzeigefenster auf. Neben generellen 
  Statusinformationen wie Uptime des Router oder auch der Uhrzeit sowohl lokal 
  wie auch vom Router selber, werden für jede bestehende Verbindung die 
  folgenden Informationen angezeigt:
  
  \begin{tabular}{lp{9cm}}
    Status             &Verbindungsaufbau/Online/Offline\\
    Name               &Telefonnummer des Gegners oder Circuit-Name\\
    Richtung           &Zeigt an, ob es sich um eine eingehende oder ausgehende
    Verbindung handelt\\
    IP                 &Die IP, die man zugewiesen bekommen hat\\
    IBytes             &Empfangene Bytes\\
    OBytes             &Gesendete Bytes\\
    Online-Zeit        &Aktuelle Online-Zeit\\
    Zeit               &Summe aller Online-Zeiten\\
    KZeit              &Summe Online-Zeiten unter Berücksichtigung des Zeittaktes\\
    Kosten             &Berechnete Kosten\\
  \end{tabular}

  \medskip

  Die Daten werden standardmäßig alle 2 Sekunden aktualisiert. Im Kontextmenü
  dieser Übersicht besteht die Möglichkeit für jeden vorhandenen Kanal, mit dem 
  der Router gerade online ist, sowohl die zugewiesene IP in die Zwischenablage 
  zu kopieren, als auch den Kanal gezielt auflegen zu können. Letzteres ist für 
  den Fall interessant, dass mehrere unterschiedliche Verbindungen bestehen, 
  z.B. eine um im Internet zu surfen und eine andere zur Firma, und gezielt eine 
  dieser Verbindungen getrennt werden soll.

  Ist zusätzlich auf dem fli4l-Router der telmond-Prozess aktiv, kann imonc 
  zusätzlich Informationen über eingehende Telefonanrufe (nämlich anrufende und 
  angerufene MSN) anzeigen. Der letzte eingegangene Telefonanruf wird oberhalb 
  der Buttons angezeigt.   Ein Protokoll der eingegangenen Telefonanrufe erhält 
  man durch Anzeige der Seite Anrufe.

  Mit den sechs Buttons im imonc können folgende Kommandos angewählt werden:

  \begin{tabular}{clp{9cm}}
    Button & Beschriftung & Funktion \\
    1& Verbinden/Trennen  & Wählen/Einhängen\\
    2& Add link/Rem link  & Kanäle bündeln: ja/nein~-- dieses Feature steht nur
                            im Admin-Mode zur Verfügung\\
    3& Reboot             & fli4l neu booten!\\
    4& PowerOff           & fli4l sauber runterfahren und anschliessend
                            ausschalten\\
    5& Halt               & fli4l sauber runterfahren, um ihn anschliessend
                            sicher ausschalten zu können\\
    6& Beenden            & Client beenden\\
  \end{tabular}

  \medskip

  \noindent Die ersten fünf Kommandos können in der Konfigurationsdatei des fli4l-Routers
  config/base.txt für den User-Modus einzeln ein- und ausgeschaltet werden. Im 
  Admin-Modus sind immer alle aktiviert.
  Die Auswahl Dialmode steuert das Wahlverhalten des Routers:

  \begin{tabular}{lp{9cm}}
    Auto    & Der Router baut automatisch eine Verbindung auf dem entsprechenden
              Circuit auf, wenn eine Anfrage aus dem lokalen Netz eintrifft.\\
    Manuell & Der Benutzer muss selber die Verbindung aufbauen.\\
    Aus     & Es ist weder manuell noch automatisch möglich, eine
              Verbindung aufzubauen. Der Dial-Button ist dann deaktiviert.\\
  \end{tabular}

  \medskip

  \noindent Bleibt noch anzumerken, dass fli4l standardmäßig selbständig rauswählt, wenn
  man mit seinem Rechner in's Internet will. Man muss also eigentlich nie den 
  Verbinden-Button drücken \ldots

  Es besteht auch die Möglichkeit, den Default-Route-Circuit manuell zu 
  wechseln, also das automatische LCR-Routing ein- und auszuschalten. Dafür ist 
  in der Windows-Version von imonc die Auswahlliste ``Default Route'' 
  vorgesehen. Ausserdem kann man die Hangup-TimeOut-Zeit jetzt auch über imonc 
  direkt konfigurieren. Dazu dient der Button Config neben der Default Route. 
  Dort werden alle konfigurierten Circuits des Routers angezeigt. Der Wert in 
  der Spalte Hup-timeout kann für ISDN-Circuits direkt im StringGrid editiert 
  werden (funktioniert bis dato noch nicht für DSL).

  Einen Überblick über das LCR-Routing findet man auf der Seite Admin/TimeTable. 
  Dort sieht man, welchen Circuit imond zu welcher Zeit automatisch auswählt.


  \subsection{Config-Dialog}

  Der Konfigurationsbereich ist über den Button Config in der Statuszeile 
  erreichbar. Das aufgehende Fenster ist dann in die folgenden Bereiche 
  unterteilt:

  \begin{itemize}
  \item Der Bereich Allgemein:
    \begin{itemize}
    \item Aktualisierungsintervall: Hier wird eingestellt, wie oft die Seite
      Überblick aktualisiert werden soll.
    \item Zeit beim Programmstart synchronisieren: Übernimmt beim Starten des
      Client die Zeit und das Datum des Routers als lokale Zeit. Diese Funktion 
      kann auch manuell mit dem Button Synchronisieren auf der Überblicks-Seite 
      aufgerufen werden.
    \item Minimiert starten: Startet das Programm direkt minimiert, man sieht 
      nur das Icon neben der Uhr.
    \item Zusammen mit Windows starten: Hier kann man angeben, ob der Client 
      direkt beim Starten von Windows mit gestartet werden soll. In dem Feld 
      Parameter kann man die nötigen Start-Parameter angeben.
    \item News von fli4l.de abholen: Sollen die News, die auf der fli4l-Homepage 
      in der News-Sektion angezeigt werden, auch vom imonc geholt und angezeigt 
      werden? Die Schlagzeilen werden dann in der Statusbar angezeigt. Ausserdem 
      wird dann eine neue Seite News angezeigt, in der die kompletten Meldungen 
      angezeigt werden.
    \item Logdatei für Verbindungen: Den Dateinamen, den man hier angeben kann, 
      wird dazu benutzt, die Verbindungs-Liste unter diesem Namen lokal auf dem 
      Rechner abzuspeichern.
    \item TimeOut für Router zum antworten: Wie lange soll auf eine Antwort der 
      Routers gewartet werden, bevor angenommen wird, dass die Verbindung nicht 
      mehr besteht.
    \item Sprache: Hier kann die Sprache des imoncs ausgewählt werden.
    \item Router Befehle bestätigen: Ist dieses Feature aktiviert, müssen alle 
      Router"=beeinflussenden Kommandos, wie zum Beispiel Reboot, Hangup \ldots
      generell bestätigt werden.
    \item Auflegen auch bei Traffic: Soll kein Hinweis erfolgen, wenn die 
      Verbindung beendet wird und noch Traffic auf der Leitung ist.
    \item Automatisch Verbindung zum Router aufbauen: Soll, wenn die Verbindung 
      zum Router unterbrochen wird (z.B. durch einen Neustart des Routers),
      automatisch probiert werden, die Verbindung wieder herzustellen.
    \item Fenster in System Tray minimieren: Soll imonc beim Clicken auf den 
      Beenden-Button in der Titelleiste sich in den System-Tray neben der Uhr
      minimieren anstatt zu beenden.
    \end{itemize}

  \item Der Unterbereich Proxy:
    Hier kann ein Proxy für die http-Anfragen des imoncs definiert werden. 
    Dieser wird dann zur Zeit für das Holen der News benutzt.
    \begin{itemize}
    \item Proxy-Unterstützung für Http-Anfragen aktivieren: Soll ein Proxy 
      benutzt werden
          \begin{itemize}
            \item Adresse: Die Adresse des Proxy-Servers
            \item Port: Die Portnummer des Proxy-Server (default: 8080)
          \end{itemize}
    \end{itemize}
    
  \item Der Unterbereich TrayIcon:
  	Hier können die Farben des TrayIcons neben der Uhr an die eigene Bedürfnisse
  	angepasst werden. Weiterhin kann ausgewählt werden, dass der aktuelle
  	Dialmode als farblicher Hintergrund des TrayIcons dargestellt wird.

  \item Der Bereich Anrufe: Die Position des Call Notification-Fensters wird in 
    der Registry gespeichert, so dass man sich das Fenster an die Position 
    schieben kann, wo man es haben möchte. Es erscheint anschliessend immer 
    wieder an dieser Stelle.
    \begin{itemize}
      \item Aktualisierung: Hier kann ausgewählt werden, wie imonc über neue
        Anrufe informiert wird. Es gibt drei verschiedene Möglichkeiten. Diese
        erste besteht darin, regelmäßig den telmond-Dienst auf dem Router 
        abzufragen. Eine weitere Möglichkeit besteht in der Auswertung der 
        Syslog-Meldungen. Diese Variante ist der ersten vorzuziehen~-- dazu muss
        natürlich der Syslog-Client des imonc aktiviert sein. Wird imonc mit 
        einem routenden eisfair eingesetzt, bietet sich noch die Möglichkeit das
        Capi2Text-Paket zur Anrufsignalisierung zu benutzen.
      \item Führende Null wegen Telefonanlage löschen: Telefonanlage setzen 
        manchmal eine zusätzliche Null vor die Rufnummer des Anrufer. Diese kann 
        mit dieser Option unterdrückt werden.
      \item Eigene Vorwahl: Hier kann die eigene Vorwahl hinterlegt werden. Wann 
        dann ein Anruf mit gleichen Vorwahl eintrifft, wird die gesendete 
        Vorwahl ausgeblendet.
      \item Telefonbuch: Hier kann die Datei angegeben werden, in der das lokale 
        Telefonbuch zur Auflösung von Telefonnummer gespeichert wird. Existiert 
        die Datei nicht, wird sie vom Programm angelegt.
      \item Logdatei: Der Dateinamen, den man hier angeben kann, wird dazu 
        benutzt, die Calls-Liste unter diesem Namen lokal auf dem Rechner zu 
        speichern. Dieser Menüpunkt ist nur sichtbar, wenn die Config-Variable 
        \var{TELMOND\_\-LOG} auf `yes' gesetzt ist, dieses gilt auch für die 
        eigentliche Anruf-Liste.
      \item Externes Suchprogramm benutzen: In diesem Bereich kann ein Programm
        angegeben werden, dass aufgerufen wird, wenn eine Telefonnummer mittels 
        des lokalen Telefonbuches nicht aufgelöst werden kann. Nähere Infos 
        sollten den entsprechenden Programmen beiliegen. Bis jetzt gibt es eine 
        Anbindung an die Telefonbuch-CD KlickTel sowie von Marcel Wappler eine 
        Anbindung an die Palm-Datenbank.
    \end{itemize}

  \item Der Unterbereich Call Notification: 
    Hier kann das bestimmt werden, ob ein Hinweis auf eingehende Telefonanrufe
    angezeigt werden soll und wie dieser sich optisch präsentiert.
    \begin{itemize}
      \item Call Notification aktivieren: Bestimmt, ob Anrufe signalisiert
        werden sollen.
      \item Call Notification anzeigen: Soll bei eingehenden Anrufen ein
        Hinweisfenster mit den Infos: angerufene MSN, Rufnummer des Anrufers und 
        Datum/Uhrzeit erscheinen? Dafür ist es nötig, dass in der Datei 
        config/isdn.txt die Variable \var{OPT\_\-TELMOND} auf `yes' gesetzt 
        wird.
        \begin{itemize}
          \item Unterdrücken, wenn keine Nummer übertragen wurde: Soll Die 
            Call Notification nicht angezeigt werden, wenn keine Rufnummer 
            übertragen wurde.
          \item Anzeigendauer: Diese Angabe beeinflußt die Dauer, wie lange das 
            Call Noti\-fication-Fenster geöffnet bleiben soll. Die Angabe von 
            ``0'' an dieser Stelle bewirkt, dass das Fenster nicht automatisch 
            geschlossen wird.
          \item Fontsize: Hiermit wird die Schriftgröße bestimmt. Dieses hat 
            einen Einfluss auf die Größe des Fenster, da die notwendige Größe 
            des Fenster anhand der tatsächlichen Größe der Mitteilung berechnet 
            wird.
          \item Farbe: Hiermit kann die Schriftfarbe ausgewählt werden. Ich 
            selber benutzte die Farbe rot, damit ich es auch direkt wahrnehme.
      \end{itemize}
    \end{itemize}
    

  \item Der Unterbereich Phonebook: Die Seite Phonebook beinhaltet das
    Telefonbuch, welches zur Rufnummerauflösung der anrufenden Nummer
    als auch der eigenen MSN benutzt wird. Die Seite wird auch
    angezeigt, wenn die Konfigurationsvariable \var{TELMOND\_\-LOG} auf `no'
    gesetzt ist, da die Rufnummerauflösung auch für die Anzeige des
    letzten Anrufes auf der Summary-Seite benutzt wird. Alternativ
    kann statt dem Telefonbuch auf dem Router auch eine lokale Datei
    ausgewählt werden.

    Der Aufbau der Eintrag sieht wie folgt aus:

\begin{example}
\begin{verbatim}
  # Format:
  # Telefonnummer=anzuzeigender Name[, Wavefilename]
  # 0241123456789=Testuser
  00=unbekannt
  508402=Fax
  0241606*=Elsa AG Aachen
\end{verbatim}
\end{example}

    Dabei sind die ersten drei Zeilen Kommentare. Die vierte Zeile
    bewirkt, dass, wenn keine Rufnummer übermittelt wird,
    ``unbekannt'' angezeigt wird. In der fünften Zeile wird der MSN
    508402 der Name ``Fax'' zugeordnet. Ansonsten ist das Format immer
    Telefonnummer=Name, der stattdessen angezeigt werden soll. In der
    sechsten Zeile ist noch die Möglichkeit demonstriert, eine
    Sammelrufnummer zu definieren. Damit wird erreicht, dass für alle
    Nebenstellen von 0241606 der Name angezeigt wird. Zu beachten
    hierbei ist, dass der erste Eintrag im Telefonbuch, welcher auf
    den Anruf passt, genommen wird. Optional kann auch noch ein
    Wave-Datei angegeben werden, die abgespielt wird, wenn ein
    Telefonanruf von dieser Rufnummer eingeht.

    Ab der Version 1.5.2 besteht jetzt auch die Möglichkeit auf der
    Seite Names das lokale Telefonbuch mit dem auf dem Router
    abgespeicherten (in /etc/phonebook) zu synchronisieren und
    umgekehrt. Dabei werden nicht nur einfach die Dateien ersetzt,
    sondern es werden die noch fehlende Einträge hinzugefügt. Gibt es
    eine Telefonnummer in beiden Telefonbüchern mit unterschiedlichen
    Namen, wird nachgefragt, welcher Eintrag genommen werden soll. Für
    die Synchronisierung des Telefonbuches auf dem Router ist noch
    anzumerken, dass dieses nur in der Ramdisk verändert wird, d.h.
    dass nach einem Reboot sämtliche Änderungen verloren gehen.

  \item Der Bereich Sound: Die Wave-Dateien, die hier angegeben werden, werden 
    abgespielt, wenn das jeweilige Ereignis eingetreten ist.
    \begin{itemize}
      \item \mbox{E-Mail}: Wenn der \mbox{E-Mail}-Checker auf einem angegebenen POP3-Server neue 
        \mbox{E-Mails} vorfindet, wird die angegebene Wave-Datei abgespielt.
      \item \mbox{E-Mail}-Error: Wenn ein Fehler beim Abrufen der \mbox{E-Mails} auftritt, wird 
        diese Wave-Datei abgespielt.
      \item Verbindung verloren: Wenn die Verbindung zum Router nicht mehr
        vorhanden ist (z.B. wenn der Router von einem anderen Client gerade neu 
        gebootet wird), wird diese Wave-Datei abgespielt. Wenn die Option 
        ``Automatic Reconnect to router'' nicht aktiviert ist, erscheint 
        ausserdem eine MessageBox, die nachfragt, ob versucht werden soll, eine 
        neue Verbindung zum Router aufzubauen.
      \item Verbindungsmeldung: Wenn der Router eine Verbindung zum Internet
        aufgebaut hat, wird diese Wave-Datei abgespielt.
      \item Verbingsabbau: Wenn der Router die Verbindung zum Internet wieder 
        abgebaut hat, wird diese Wave-Datei abgespielt.
      \item Anrufmeldung: Wenn die Call Notification aktiviert ist und ein neuer 
        Anruf eingeht, wird die angegebene Wave-Datei abgespielt.
      \item Fax Notification: Die hier angegebene Wave-Datei wird nach dem 
        Empfang neuer Faxe abgespielt.
    \end{itemize}

  \item Der Bereich \mbox{E-Mails}
    \begin{itemize}
      \item Accounts: Dieser Bereich dient dazu, die vorhandenen
        POP3-Accounts zu konfigurieren.
      \item \mbox{E-Mail}-Check aktivieren: Soll der \mbox{E-Mail}-Checker automatisch nach neuen
        \mbox{E-Mails} suchen
        \begin{itemize}
          \item Check jede x Min: Hiermit wird angegeben, wie oft der 
            \mbox{E-Mail}-Checker automatisch nach neuen \mbox{E-Mails} suchen soll. Achtung: ein 
            zu kurzes Intervall kann dazu führen, dass der Router komplett 
            online bleibt! Dies ist der Fall, wenn das Intervall kleiner ist als 
            der Hangup-Timeout des verwendeten Circuits.
          \item TimeOut x Sec: Wie lange soll auf einen einen POP3-Server 
            gewartet werden, bis er antwortet. Der Wert ``0'' bedeutet, dass 
            kein TimeOut gesetzt wird.
          \item Auch wenn Router offline: Hiermit wird erreicht, dass der Router 
            sich selbstständig einwählt, um nach \mbox{E-Mails} zu sehen. Nachdem alle 
            POP3-Konton nach \mbox{E-Mails} überprüft worden sind, wird die Verbindung 
            wieder getrennt. Um dieses Feature nutzen zu können, muss Dialmode 
            auf `auto' stehen. Achtung: Dadurch entstehen zusätzliche Kosten, 
            wenn nicht gerade eine Flatrate benutzt wird!
          \item Zu benutzender Circuit: Hiermit wird angegeben, welcher Circuit
            zur Einwahl beim \mbox{E-Mail}-Checken benutzt werden soll.
          \item Anschliessend online bleiben: Soll direkt nach dem \mbox{E-Mail}-Check 
            direkt die Verbindung getrennt werden oder eine 
            Verbindungstrennung durch das Hangup-timeout realisiert werden.
          \item \mbox{E-Mail}-Header laden: Sollen auch die \mbox{E-Mail}-Header geladen oder 
            nur die Anzahl der vorhandenen \mbox{E-Mails} abgefragt werden? Das Laden 
            der \mbox{E-Mail}-Header ist Voraussetzung, wenn man \mbox{E-Mails} direkt auf dem
            Server löschen möchte.
         \item Notify only new \mbox{E-Mails}: Sollen nur neue \mbox{E-Mails} akustisch und mit 
           dem Tray-Icon gemeldet werden
         \item \mbox{E-Mail}-Client starten: Soll der angegebene \mbox{E-Mail}-Client 
           automatisch gestartet werden, wenn neue \mbox{E-Mails} vorhanden sind.
         \item \mbox{E-Mail}-Client: Hier wird der zu startende \mbox{E-Mail}-Client angegeben.
         \item Param: Hier kann man zusätzliche Parameter angeben, die beim 
           Start des \mbox{E-Mail}-Clients übergeben werden sollen. Wenn Outlook als 
           \mbox{E-Mail}-Client benutzt wird (nicht Outlook Express), sollte /recyle als 
           Parameter eingetragen werden, damit eine bereits geöffnete Instanz 
           von Outlook beim Eintreffen von neuen \mbox{E-Mails} benutzt wird.
      \end{itemize}
    \end{itemize}

  \item Der Bereich Admin
    \begin{itemize}
      \item root-Passwort: Hier sollte das Router-Passwort (in config/base.txt 
        unter \verb+PASSWORD+ eingetragen) eingetragen werden, damit z.B. das 
        Portforwarding lokal bearbeitet und wieder auf dem Router hinterlegt 
        werden kann.
      \item Dateien auf dem Router, die angezeigt werden sollen: Alle hier 
        angegebenen Dateien, die sich auf dem Router befinden, können einfach 
        per Maus-Click auf der Seite Admin/Dateien angezeigt werden. Somit kann 
        man sich auf einfache Weise die Logfiles des Routers direkt im imonc 
        anzeigen lassen.
      \item Konfigdateien bearbeiten: Hier kann ausgewählt werden, ob die
        Konfigdateien alle im Editor geöffnet werden sollen (dies kann, wenn 
        TXT-Dateien noch mit einem einfachen Editor verknüpft sind, dazu führen, 
        dass sehr viele Instanzen des Editors geöffnet werden). Alternativ kann 
        auch einfach nur das Verzeichnis geöffnet werden, so dass die 
        Möglichkeit besteht, nur die Dateien auszuwählen, die bearbeitet werden 
        sollen.
      \item DynEisfaiLog: Wenn ein Account bei DynEisfair vorhanden ist, kann 
        man hier seine Zugangsdaten eintragen und sich dann ein Log der 
        Aktualisierung des Dienstes auf der Seite Admin/DynEisfairLog anschauen.     
    \end{itemize}

  \item Der Bereich LaunchList dient dazu, die Launchliste zu konfigurieren. 
    Diese wird nach einem erfolgreichen Connect ausgeführt, wenn die Option 
    ``Activate Launchlist'' aktiviert ist.
    \begin{itemize}
      \item Programme: Alle hier eingetragenen Programme werden automatisch 
        gestartet, sobald der Router eine Verbindung aufgebaut hat und die
        Launchliste aktiviert ist.
      \item LaunchList aktivieren: Soll die Launchliste beim erfolgreichen
        Verbindungsaufbau ausgeführt werden?
    \end{itemize}

  \item Der Bereich Traffic dient dazu, dass TrafficInfo-Fenster den eigenen 
    Bedürfnissen anzupassen. Von einem User habe ich den Hinweis bekommen, dass 
    es mit älteren DirectX-Versionen offenbar Darstellungsfehler gibt.
    \begin{itemize}
      \item Separates Traffic-Info-Fenster anzeigen: Soll eine grafische 
        Kanalauslastung in einem separaten Fenster angezeigt werden? In dem 
        Kontextmenü des Fensters kann man festlegen, ob das Fenster das 
        Attribut StayOnTop bekommen soll. Dieses bewirkt, dass sich das Fenster 
        immer über allen anderen Fenstern plaziert. Auch dieser Wert wird in der 
        Registry abgespeichert und steht somit auch nach einem erneuten 
        Programmstart wieder zur Verfügung.
      \item Titelleiste anzeigen: Soll die Titelleiste des Traffic-Info-Fensters 
        angezeigt werden? In der Titelzeile wird angezeigt, mit welchem Circuit 
        der Router gerade online ist.
        \begin{itemize}
          \item CPU-Auslastung in Titelleiste: Soll auch die CPU-Auslastung in
            der Titelzeile angezeigt werden?
          \item Online-Zeit in Titelleiste: Soll die Onlinezeit des Kanals auch 
            in der Titelzeile angezeigt werden?
        \end{itemize}
      \item Semi-transparentes Fenster: Soll das Fenster transparent dargestellt 
        werden? Diese Funktion steht nur unter Windows 2000 und Windows XP zur 
        Verfügung.
      \item Farben: Hier werden die Farben für das TrafficInfo-Fenster 
        definiert. Zu Berücksichtigen ist dabei, dass der DSL-Kanal und der 
        erste ISDN-Kanal die selben Farbwerte zugewiesen bekommen.
      \item Limits: Hier können die max. Übertragungswerte für DSL eingestellt 
        werden~-- Upload und Download.
    \end{itemize}

  \item Der Bereich Syslog dient dazu, die Anzeige der Syslog-Meldungen zu 
    konfigurieren.
    \begin{itemize}
      \item Syslog-Client aktivieren: Sollen Syslog-Meldungen im imonc angezeigt 
        werden? Diese Option sollte ausgeschaltet sein, wenn ein externer 
        Syslog-Client, wie zum Beispiel Kiwi's Syslog Client, benutzt wird.
      \item Alle Meldungen ab Stufe anzeigen: Ab welcher Prioritätsstufe sollen 
        die Syslog-Meldungen angezeigt werden? Es ist sinnvoll am Anfang mit der 
        Stufe Debug anzufangen, um damit festzustellen, welche Meldungen einen 
        interessieren. Anschliessend kann hier dann die entsprechende Stufe 
        eingetragen werden.
      \item Syslog-Meldungen in einer Datei speichern: Sollen die angezeigten
        Syslog-Meldungen in einer Datei gespeichert werden? In der Groupbox 
        können dann die Meldungen ausgewählt werden, die in der Datei geloggt 
        werden sollen. Für den Dateinamen sind folgende Platzhalter eingefügt 
        worden:
        \begin{description}
          \item[\%y]~-- wird durch das aktuelle Jahr ersetzt
          \item[\%m]~-- wird durch den aktuellen Monat ersetzt
          \item[\%d]~-- wird durch den aktuellen Tag ersetzt
        \end{description}
      \item Portnamen anzeigen: Sollen statt den Portnummern deren Bedeutungen
        angezeigt werden?
      \item Firewall-Meldungen auch im User-Modus anzeigen: Hiermit wird
        festgelegt, dass Firewall-Meldungen auch im User-Modus angezeigt werden
        sollen.
    \end{itemize}

  \item Der Bereich Fax dient dazu, die Faxanzeige vom imonc zu konfigurieren. 
    Damit dieser Punkt angezeigt wird, muss mgetty bzw. faxrcv auf dem Router 
    installiert sein (zu finden als OPT-Pakete auf der fli4l-Homepage).
    \begin{itemize}
      \item Logdatei für Faxe: Den Dateinamen, den man hier angeben kann, wird 
        dazu benutzt, die Fax-Liste unter diesem Namen lokal auf dem Rechner 
        abzuspeichern.
      \item Lokales Verzeichnis: Um die Faxe anzeigen zu können, müssen sie
        lokal gespeichert werden. Dieses kann hier eingestellt werden.
      \item Aktualisierung: Es gibt zwei verschiedene Möglichkeiten, wie imonc
        mitbekommt, dass ein neues Fax eingegangen ist. Entweder wertet imonc
        die entsprechenden Syslogmeldungen aus (dazu muss natürlich der 
        Syslog-Client im imonc aktiviert sein) oder er schaut regelmäßig selber
        in der Logdatei nach. Die erste Variante ist zu bevorzugen. Falls die
        zweite Variante genutzt wird, kann man noch angeben, wieoft die
        Faxübersichtsseite aktualisiert werden soll. Dabei ist zu beachten, dass 
        dieser Wert keine Angabe in Sekunden ist, sondern noch mit der Angabe 
        von Allgemein/Aktualisierungsintervall multipliziert wird. 
    \end{itemize}

  \item Der Bereich Grids dient dazu die Grids (Tabellen) im imonc an die 
    eigenen Bedürfnisse anzupassen. Einerseits kann für jedes Grid angegeben 
    werden, welche Spalten angezeigt werden sollen, andererseits gibt es die 
    Möglichkeit für die Grids im Bereich Anrufe, Verbindungen und Faxe 
    anzugeben, von wann ab die Infos angezeigt werden sollen.
  \end{itemize}

  \subsection{Seite Anrufe}

  Die Seite Calls wird nur angezeigt, wenn die Konfigurationsvariable
  \var{TELMOND\_\-LOG} auf `yes' eingestellt ist, denn sonst wird kein
  Anruf-Log geführt. Auf dieser Seite werden alle abgespeicherten
  Telefonanrufe angezeigt, die eingegangen sind, während der Router
  eingeschaltet war. Dabei kann umgeschaltet werden zwischen der
  Ansicht der lokal gespeicherten Anrufe oder nur der auf dem Router
  gespeicherten Anrufe. Wird bei der Anzeige der auf dem Router
  gespeicherten Anrufe der Zurücksetzen-Button gedrückt, wird das Logfile auf
  dem Router gelöscht.

  In der Anruf-Übersicht kann mit der rechten Maustaste auf der
  Rufnummer oder der eigenen MSN diese ins Telefonbuch übernommen
  werden, um der Rufnummer bzw. MSN dort einen Namen zuzuweisen, der
  dann stattdessen angezeigt wird.


  \subsection{Seite Verbindungen}

  Neu ist ab der Version 1.4 die Anzeige der vom Router aufgebauten
  Verbindungen zum Internet. Diese befindet auf der Seite Connections.
  Somit hat man einen guten Überblick, wie sich der Router bei der
  automatischen Einwahl ins Internet verhält. Damit diese Seite
  angezeigt werden kann, muss in der Datei config/base.txt die
  Variable \var{IMOND\_\-LOG} auf `YES' gesetzt werden.

  Genauso wie bei der Anruf-Übersicht kann auch hier zwischen den
  lokal gespeicherten und auf dem Router gespeicherten Verbindungen
  umgeschaltet werden.  In der Ansicht der auf dem Router
  gespeicherten Verbindungen bewirkt ein Drücken des Zurücksetzen-Buttons,
  dass das Logfile auf dem Router gelöscht wird.

  Angezeigt werden pro Verbindung
  \begin{itemize}
  \item Provider
  \item Startdatum und -zeit
  \item Enddatum und -zeit
  \item Onlinezeit
  \item Abrechnungszeit
  \item entstandene Kosten
  \item empfangene Zeichen
  \item gesendete Zeichen
  \end{itemize}

  \subsection{Seite Fax}

  Damit die Seite Faxe angezeigt wird, muss auf dem Router entweder das
  \var{OPT\_\-MGETTY} von Michael Heimbach oder \var{OPT\_\-MGETTY} von Felix Eckhofer
  installiert werden. Diese gibt es auf der fli4l-Homepage unter
  OPT-Pakete. Auf dieser Seite werden dann alle eingegangenen Faxe
  aufgelistet. Das Kontextmenü der Übersicht bietet mehrere Möglichkeiten,
  diese stehen allerdings nur im Admin-Modus zur Verfügung:

  \begin{itemize}
  \item Es kann ein Fax angezeigt werden. Dazu muss unter
    Admin/Remoteupdate der Pfad für das fli4l-Verzeichnis korrekt
    gesetzt werden, da die Faxe auf dem Router in gepackter Form
    vorliegen und somit gzip aus dem fli4l-Paket benötigt wird.
    Alternativ kann gzip.exe und win32gnu.dll auch ins
    imonc-Verzeichnis kopiert werden. Kann gzip.exe nicht an einer der
    beiden Stellen gefunden werden, wird stattdessen der Webserver des
    Routers probiert zu öffnen (direkt mit dem Aufruf des richtigen
    CGIs).
  \item Ein einzelnes Fax kann gelöscht werden. Dabei wird das Fax
    sowohl lokal als auch auf dem Router gelöscht (sowohl die
    eigentliche Faxdatei, als auch der Eintrag in den Logdateien).
  \item Sämtliche auf dem Router befindlichen Faxe löschen. Damit
    werden die Faxe und die Logdatei auf dem Router gelöscht. Die Faxe
    werden nicht aus der lokalen Logdatei gelöscht.
  \end{itemize}
  Genauso wie bei der Anruf-Übersicht kann auch hier zwischen den
  lokal gespeicherten und auf dem Router gespeicherten Faxen
  umgeschaltet werden.


  \subsection{Seite E-Mail}

  Diese Seite wird nur angezeigt, wenn im Config-Dialog mindestens ein
  aktiviertes POP3-\mbox{E-Mail}-Konto eingerichtet worden ist.

  Die Seite \mbox{E-Mail} dürfte sich eigentlich selber erklären. Hiermit wird
  der mittlerweile eingebaute \mbox{E-Mail}-Checker beobachtet. Ist die Option
  ``Check even if the router is offline'' nicht aktiviert, überprüft
  der \mbox{E-Mail}-Checker alle \mbox{E-Mail}-Konten nach \mbox{E-Mails}, sobald der Router
  online ist und anschließend im eingestellt Intervall. Ist die
  genannte Option aktiviert, überprüft der \mbox{E-Mail}-Checker im
  eingestellten Intervall. Ist der Router gerade online, wird die
  bestehende Verbindung benutzt. Ist er nicht online, wird eine
  Verbindung selbständig mit dem ausgewählten Circuit hergestellt,
  die, sobald alle \mbox{E-Mail}-Konten abgearbeitet sind, wieder getrennt
  wird. Damit man diese Option nutzen kann, muss Dialmode auf ``auto''
  stehen.

  Sind \mbox{E-Mails} auf dem POP3-Server vorhanden, wird entweder automatisch
  der eingestellte \mbox{E-Mail}-Client gestartet oder ein neues Symbol im
  Tray neben der Uhr angezeigt, welches als Hint die Anzahl der \mbox{E-Mails}
  auf jedem Server liefert. Ein Doppelclick startet dann den
  eingestellten \mbox{E-Mail}-Client. Ist ein Fehler bei einem der
  \mbox{E-Mail}-Konten aufgetreten, erscheint einerseits ein Hinweis in der
  \mbox{E-Mail}-History, andererseits wird das \mbox{E-Mail}-TrayIcon angezeigt,
  welches dadurch gekennzeichnet ist, dass die obere rechte Ecke rot
  gefärbt ist.

  In der \mbox{E-Mail}-Übersicht kann man mit dem Kontextmenü Mails direkt auf
  dem Server löschen, ohne sie vorher komplett downloaden zu müssen.
  Dies geschieht, indem mit der rechten Maustaste das Kontextmenü
  angezeigt wird. Dabei sollte eine Zelle der entsprechenden Zeile
  markiert sein, wo die zu löschende \mbox{E-Mail} eingetragen ist. Im
  Kontextmenü wählt man den einzige Punkt Delete MailMessage aus.


  \subsection{Admin}

  Dieser Abschnitt steht nur zur Verfügung, wenn sich imonc im
  Admin-Modus befindet.

  Der erste Punkt lieferte eine Übersicht über die verwendeten
  Circuits~-- sprich Internetprovider~-- die der Router automatisch per
  LCR auswählt. Ein Doppelclick auf einen Provider in der
  Providerübersicht zeigt an, für welche Zeiträume der Circuit in
  config/base.txt definiert worden ist.

  Der zweite Punkt dort ist die Möglichkeit ein Fernupdate auf dem
  Router einzuspielen. Dabei kann ausgewählt werden, welche fünf
  Programmpakete (Kernel, Rootfilesystem, Opt-Datei, rc.cfg und syslinux.cfg) auf 
  den Router kopiert werden sollen. Damit man das Update einspielen kann,
  muss man zuerst mal das fli4l-Verzeichnis angeben, damit imonc
  weiss, woher es die nötigen Dateien nehmen soll. Ausserdem muss
  angegeben werden, in welchem Unterverzeichnis die Konfigurationsdateien liegen 
  (standardmäßig config), um die Opt-Datei, rc.cfg und syslinux.cfg 
  jeweils neu zu erzeugen. Es ist ratsam, einen Reboot nach dem Einspielen des
  Updates durchführen, damit die Änderungen auch direkt wirksam
  werden.  Wird während des Updates nach einem Passwort nachgefragt,
  ist das Passwort gemeint, welches in config/base.txt unter PASSWORD
  eingetragen ist.

  Um die Beschränkung des Port-Forwarding zu umgehen, dass ein Port nur
  an genau einen Client-Rechner gebunden ist, besteht jetzt die Möglichkeit,
  die Konfiguration auf dem Router zu editieren. Damit die Änderungen aktiv
  werden, muss die Verbindung neu hergestellt werden. Da die Datei nur in
  der Ramdisk ersetzt wird, bleiben die Änderungen nur bis zum nächsten
  Neustart des Routers erhalten. Um die Änderungen dauerhaft zu speichern,
  muss ein neues Opt-File auf dem Router installiert werden mit einer
  geeignet angepassten base.txt aus dem Config-Verzeichnis.

  Der vierte Punkt auf der Admin-Seite~-- Dateien~-- dient dazu,
  Konfigurations- und Logdateien des Routers einfach per Maus-Click
  anzuzeigen. Die Auswahlliste wird über den Punkt Config/Admin und
  dort ``files on router to view'' konfiguriert.  Anschliessend kann
  einfach über die ComboBox auf dieser Seite ausgewählt werden, welche
  Datei angezeigt werden soll.

  Der fünfte Punkt ist die Seite DynEisfairLog, sie erscheint nur wenn im 
  Config-Dialog unter Admin die Zugangsdaten des DynEisfair-Accounts eingetragen
  worden sind. Ist dies geschehen, wird auf dieser Seite Log des Dienstes 
  angezeigt.
  
  Als letzten Punkt gibt es noch die Seite Hosts. Hier werden alle in der Datei
  /etc/hosts eingetragenen Rechner angezeigt. Weiterhin wird probiert jeden 
  dieser dort eingetragenen Rechner anzupingen und das Ergebnis davon wird 
  ebenfalls angezeigt. Somit kann man schnell rausbekommen, welche dieser 
  Rechner eingeschaltet sind. 

  \subsection{Seiten Fehler, Syslog und Firewall}

  Die Seiten Fehler, Syslog und Firewall werden nur angezeigt, wenn in
  den entsprechenden Logs Einträge vorhanden sind. Die Einträge der
  Seiten Syslog und Firewall werden nur angezeigt, wenn man im
  Admin-Modus ist.

  Auf der Seite Fehler werden sämtliche imonc/imond-spezifischen Fehler
  festgehalten. Wenn Probleme bestehen, kann unter Umständen ein Blick
  auf diese Seite die Ursache der Probleme anzeigen.

  Auf der Seite Syslog werden die ankommende Syslog-Meldungen
  angezeigt, bis auf die Meldungen der Firewall. Diese werden auf der
  eigenen Seite Firewall dargestellt. Damit dies funktioniert, muss
  die Variable \var{OPT\_\-SYSLOGD} in der Konfigurationsdatei config/base.txt
  auf `yes' gesetzt werden. Ausserdem muss die Variable \var{SYSLOGD\_\-DEST}
  auf die IP des Clients gesetzt werden (genau:
  \var{SYSLOGD\_\-DEST}='@100.100.100.100~-- wobei die IP natürlich an die IP
  des Clients angepasst werden muss).  Angezeigt wird neben der
  eigentlichen Syslog-Meldung auch Datum, Uhrzeit, IP des Senders und die
  Prioritätsstufe.

  Damit die Firewall-Meldungen bei den ganzen Syslog-Meldungen nicht
  untergehen, werden diese auf der separaten Seite Firewall angezeigt.
  Damit die Firewall- Meldungen angezeigt werden können, muss
  zusätzlich in der Datei config/base.txt die Konfigurationsvariable
  \var{OPT\_\-KLOGD} auf `yes' gesetzt werden.

  \subsection{Seite News}

  Auf dieser Seite werden, vorausgesetzt die Option ist im Config-Bereich des
  Imonc aktiviert, die News, welche auf der fli4l-Homepage angezeigt werden,
  auch direkt im Imonc angezeigt werden. Dazu wird mittels des http-Protokolls 
  die URL http://www.fli4l.de/german/news.xml abgerufen. Neben den News werden
  mittlerweile auch die fünf aktuellsten Opt-Pakete angezeigt. Dazu wird 
  die URL http://www.fli4l.de/german/imonc\_opt\_show.php abgefragt. Außerdem
  wird in der Statusleiste vom Imonc die Überschriften der News alternierend
  angezeigt.
