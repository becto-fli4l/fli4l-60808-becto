% Synchronized to r38663

\section{Démarrer, arrêter, se connecter et se déconnecter avec fli4l}

\subsection{Concept de Boot}

fli4l 2.0 peut être installé sur différent média, disque-dur ou
carte-Compact-Flash(TM), il est aussi possible de l’installer sur un support
Zip ou sur un CD-ROM amorçable. En outre, l’installation d’une version sur un
disque-dur n'est pas fondamentalement différente que sur une disquette.
\footnote{ À l'origine fli4l pouvait s'exécuter à partir d'une disquette. Depuis
fli4l est devenue trop volumineux, la disquette ne peut plus est utilisée.}

Ces exigences ont été réalisé grâce à l’archive \texttt{opt.img}, jusqu'ici il était
installé sur un disque-RAM maintenant il peut être placé sur d’autre média.
Il peut s'agir d'une partition sur un disque dur ou une carte-CF. Pour ce second
volume, le répertoire \texttt{/opt} sera monté et les programmes auront des liens symboliques
et seront intégrés dans le rootfs. La structure apparaissant dans le système de fichier
RootFS correspond au répertoire \texttt{opt} de la distribution fli4l à une exception prète~--
le préfixe des \texttt{fichiers} seront omis. Le fichier \texttt{opt/etc/rc} se trouve
directement dans \texttt{/etc/rc} et pour le fichier \texttt{opt/bin/busybox} il est
dans \texttt{/bin/busybox}. Ces fichiers sont seulement des liens dans le média monté en
lecture seule, on peut les ignorer, tant que les fichiers ne sont pas modifiés. Si vous
voulez les modifier, il faut d'abord rendre ces fichiers accessibles en écriture
avec la commande \texttt{mk\_writable} (voir ci-dessous).


\subsection{Scripts de démarrage et d'arrêt}

Les Scripts qui sont exécutés lors du boot système, sont dans les répertoires
\texttt{opt/etc/boot.d} et \texttt{opt/etc/rc.d} ils sont également exécutés
dans l'ordre.

\wichtig{Quand ces scripts sont exécutés aucun processus particulier
n'est produit, ils ne peuvent pas être arrêtés avec la commande \og{}exit\fg{}.
Cette commande conduirait à une rupture du processus de Boot~!}


\subsubsection{Scripts de démarrage dans \texttt{opt/etc/boot.d/}}

    Les scripts de ce répertoire sont exécutés les premiers. Ils ont pour mission,
	de monter le périphérique de boot, le fichier de configuration \texttt{rc.cfg}
	se trouve sur le support de boot et décompresse l'archive \texttt{opt.img}. Selon
	le \jump{BOOTTYPE}{type de Boot} il est plus ou moins complexe et font les choses
	suivantes~:

\begin{itemize}
\item Charge les pilotes du matériels (optionnel)
\item Monte le volume de boot (optionnel)
\item Le fichier de configuration \texttt{rc.cfg} est lu depuis le volume de boot
  et sera écrit dans \texttt{/etc/rc.cfg}
\item Monte le volume Opt (optionnel)
\item Extrait les archives Opt (optionnel)
\end{itemize}

Lors de la construction des scripts, vous avez la chance d'en apprendre davantage sur
la configuration de fli4l, le fichier de configuration est également intégré dans
l'archive rootfs, dans ce fichier \texttt{/etc/rc.cfg} vous trouverez les variables
de configuration qui seront analysées puis les scripts de démarrage seront exécutés
depuis le répertoire \texttt{opt/etc/boot.d/}. Après le montage du volume de boot,
le fichier \texttt{/etc/rc.cfg} est remplacé par le fichier de configuration dans le
volume de boot, de sorte que les scripts de démarrage dans \texttt{opt/etc/rc.d/} soit
disponible pour l'actuel configutation du volume de boot (voir ci-dessous).
\footnote{Normalement, ces deux fichiers sont identiques. Un changement n'est possible
que si le fichier de configuration sur le volume de démarrage a été modifié manuellement,
par exemple pour modifier une configuration qui sera utilisé plus tard, sans avoir à
reconstruire l'archive fli4l.}


\subsubsection{Scripts de démarrage dans \texttt{opt/etc/rc.d/}}

    C'est les commandes qui sont exécutées à chaque démarrage du routeur, elles peuvent
	être stockés dans le répertoire \texttt{opt/etc/rc.d/}. Les conventions suivantes
	s'appliquent~:

    \begin{enumerate}
    \item Il faut classer les noms de script comme ceci~:

\begin{example}
\begin{verbatim}
    rc<nombre à trois chiffres>.<nom de l'OPT>
\end{verbatim}
\end{example}

    Les scripts sont démarrés dans l'ordre croissant des numéros.
    Si plusieurs scripts ont le même numéro attribué, c’est le caractère
    alphabétique après le point qui détermine l'ordre. L’installation des
    paquetages s’effectue les uns après les autres, ils sont définis par un numéro.

          Voici une estimation approximative, des numéros pouvant être utilisé
          pour une installation~:

          \begin{table}[htbp]
          \centering
          \begin{tabular}{ll}
                  \hline
                  Numéro        &       Fonction \\
                  \hline
                  \hline
                  000-099       &       Système de base (hardware, fuseau horaire, système de fichiers) \\
                  100-199       &       Module Kernel (drivers) \\
                  200-299       &       Connexions externe (PPPoE, ISDN4Linux, PPtP) \\
                  300-399       &       Réseau (routage, interface, filtrage de paquet) \\
                  400-499       &       Serveur (DHCP, HTTPD, Proxy, etc.) \\
                  500-900       &       Tout le reste \\
                  900-997       &       Tout ce qui peut générer un dial-up \\
                  998-999       &       Réservé (ne pas utiliser!) \\
                  \hline
          \end{tabular}
          \end{table}

    \item Vous devez \emph{placer} dans ces scripts, toutes les fonctions
    nécessaires pour changer le RootFS. (par ex. pour la création d'un
    répertoire \texttt{/var/log/lpd}).

    \item Vous ne \emph{devez} pas effectuer d'écriture dans les fichiers
    scripts qui font partie de l'archive-opt, car ces fichiers sont en
    lecture seule sur le média. Pour modifier de tel fichier, il faut, au
    préalable rendre accessible ce fichier en écriture via \texttt{mk\_writable}
	(voir ci-dessous). En exécutant cette fonction, le fichier si nécessaire,
	sera copié et sera accessible en écriture sur le RootFS. Si le fichier est
	déjà en écriture, rien ne se passera lors de l’exécution de la fonction \texttt{mk\_writable}.

    \wichtig{\texttt{mk\_writable} doit être utilisé sur les fichiers qui sont dans le
    rootfs et non pas par le biais du dossier \texttt{/opt}. Si vous voulez modifier
    \texttt{/usr/local/bin/foo} vous exécutez \texttt{mk\_writable /usr/local/bin/foo}.}

    \item Ces scripts ont besoin de vérifiés, avant d'exécuter les commandes
    réelles, si l'OPT correspondant est actif. Cela se fait habituellement par
	une simple distinction de cas~:

\begin{example}
\begin{verbatim}
    if [ "$OPT_<OPT-Name>" = "yes" ]
    then
        ...
        # ici OPT start!
        ...
    fi
\end{verbatim}
\end{example}

    \item Pour pouvoir déboguer plus facile, vous devez inséter dans le script
      les fonctions \texttt{begin\_script} et \texttt{end\_script}~:

\begin{example}
\begin{verbatim}
    if [ "$OPT_<OPT-Name>" = "yes" ]
    then
        begin_script FOO "configuring foo ..."
        ...
        end_script
    fi
\end{verbatim}
\end{example}

    Pour juste déboguer de script au démarrage, vous devez simplement activer
    la variable \verb+FOO_DO_DEBUG='yes'+.

  \item Toutes les variables de configuration sont directement disponible par les scripts.
	Des explications pour accéder à d'autres scripts à partir des variables de configuration
	peuvent être trouvés dans la section \jump{dev:sec:config-variables}{\og{}Gestion
    des variables de configuration\fg{}}

  \item Le dossier \texttt{/opt} ne doit pas être utilisé comme stocker des données OPTs.
	Si de l'espace supplémentaire est nécessaire, l'utilisateur doit de définir un chemin
	approprié en utilisant la variable de configuration. Selon le type de données à stocker
	(données persistante ou transitoire) vous devez utiliser différentes affectations par défaut.
	Le chemin \texttt{/var/run/} est logique pour les données transitoires, tandis que pour
	les données persistant, il est conseillé d'utiliser la fonction
	\jump{dev:sec:persistent-data}{map2persistent} combiné avec une variable de configuration
	appropriée.
    \end{enumerate}


\subsubsection{Scripts d'arrêt dans \texttt{opt/etc/rc0.d/}}

Chaque ordinateur a besoin d’un temps pour s’arrêter ou redémarrer. Il se
pourrait bien que vous pouvez avoir à effectuer des opérations avant que
l'ordinateur s'éteigne ou redémarre. Les commandes officielles sont
\og{}halt\fg{} et \og{}reboot\fg{}. Ces commandes sont également placées
dans IMONC ou dans le Web-GUI si vous l’avez installé, un clique sur
le bouton suffira pour arrêter ou redémarrer le routeur.

Tous les scripts d'arrêt se trouvent dans le répertoire \texttt{opt/etc/rc0.d/}.
Le nom du fichier est analogue à celui du script de démarrage. Ils sont également exécutées
dans ordre \emph{croissant} des numéros.


\subsection{Fonctions auxiliaires}

Dans \texttt{/etc/boot.d/base-helper} différentes fonctions sont mises à disposition,
elles peuvent être utilisées pour les scripts de démarrage. Cela se applique pour
certaines choses comme pour un support de débogage, pour le chargement des modules
du Kernel ou pour la sortie des messages.

Les différentes fonctions sont les suivantes et brièvement décrites.


\subsubsection{Contrôle des scripts}

\begin{description}

\item[\texttt{begin\_script <Symbol> <Message>}~:] envoie un message et active
le débogueur de script en utilisant \texttt{set -x}, si \var{<Symbol>\_DO\_DEBUG}
est sur \og{}yes\fg{}.

\item[\texttt{end\_script}~:] envoie un message final et fournit l'état de débogage
si vous avez activé le \texttt{begin\_script}. Pour chaque activation du \texttt{begin\_script}
un \texttt{end\_script} sera associé et activé (et vice versa).

\end{description}


\subsubsection{Chargement des modules du Kernel}

\begin{description}

\item[\texttt{do\_modprobe [-q] <Module> <Paramètre>*}:]
Charge le module du Kernel, y compris ses paramètres, en résolvant en même temp
les dépendances du module. Le paramètre \og{}-q\fg{} empêche qu'un message erreur
soit mis. La fonction retourne en cas de succès une valeur nulle, dans le cas
d'une erreur une valeur non nulle. Cela permet décrire un code pour gérer les échecs
lors du chargement les modules du Kernel~:

\begin{example}
\begin{verbatim}
    if do_modprobe -q acpi-cpufreq
    then
        # pas de contrôle de fréquence du CPU via ACPI
        log_error "le contrôle de fréquence du CPU via ACPI n'est pas disponible."
        # [...]
    else
        log_info "le contrôle de fréquence du CPU via ACPI est activé."
        # [...]
    fi
\end{verbatim}
\end{example}

\item[\texttt{do\_modrobe\_if\_exists [-q] <Chemin> <Module> <Paramètre>*}~:]\mbox{}\\
vérifie d'abord si le module \texttt{/lib/modules/<version du kernel>/<Chemin>/<Module>}
existe et ensuite exécute \texttt{do\_modprobe}.

\wichtig{Le module doit exister précisément par ce nom, aucun alias ne peuvent être utilisé.
Lorsque vous utilisez un alias \texttt{do\_modprobe} sera exécuté immédiatement.}

\end{description}


\subsubsection{Messages et gestion des erreurs}

\begin{description}

\item[\texttt{log\_info <Message>}~:] envoi un message sur la console
et dans \texttt{/bootmsg.txt}. Si aucun message n'est enregistré dans ce paramètre
le fichier \texttt{log\_info} sera lu depuis l'entrée par défaut. La fonction renvoie
toujours en retour une valeur nulle.

\item[\texttt{log\_warn <Message>}~:] envoi un message d'avertissement sur la console
et dans \texttt{/bootmsg.txt}, la chaîne \texttt{WARN:} sera utilisée comme préfixe.
Si aucun message n'est enregistré dans ce paramètre le fichier \texttt{log\_warn}
sera lu depuis l'entrée par défaut. La fonction renvoie toujours en retour une valeur null.

\item[\texttt{log\_error <Message>}~:] envoi un message d'erreur sur la console
et dans \texttt{/bootmsg.txt}, la chaîne \texttt{ERR:} sera utilisée comme préfixe.
Si aucun message n'est enregistré dans ce paramètre le fichier \texttt{log\_error}
sera lu depuis l'entrée par défaut. La fonction renvoie toujours en retour une valeur null.

\item[\texttt{set\_error <Message>}~:] envoi un message d'erreur qui sera définit dans
  une variable d'erreur interne, ensuite elle pourra être examiné via \texttt{is\_error}.

\item[\texttt{is\_error}~:] réinitialise la variable d'erreur interne et renvoie true,
si elle a été précédemment défini par \texttt{set\_error}.

\end{description}


\subsubsection{Fonctions réseau}

\begin{description}

\item[\texttt{translate\_ip\_net <Valeur> <Nom de Variable> [<Variable de Résultat>]}~:]\mbox{}\\
remplace les références symboliques par des paramètres. Actuellement les traductions
suivantes ont lieu~:
\begin{description}
\item[*.*.*.*, \texttt{none}, \texttt{default}, \texttt{pppoe}] ne sont pas remplacé
\item[\texttt{any}] sera remplacé par \texttt{0.0.0.0/0}
\item[\texttt{dynamic}] sera remplacée par une adresse IP du routeur, à travers laquel
il existe une connexion Internet.
\item[\var{IP\_NET\_x}] sera remplacé par le réseau se trouvant dans la configuration.
\item[\var{IP\_NET\_x\_IPADDR}] sera remplacé par l'adresse IP se trouvant dans
la configuration.
\item[\var{IP\_ROUTE\_x}] sera remplacé par le réseau routé se trouvant dans la configuration
\item[\texttt{@<Hostname>}] sera remplacé par l'hôte ou par l'adresse IP se trouvant dans
la configuration.
\end{description}

Le résultat de la traduction est mémorisée dans la variable dont le nom est transmis dans
le troisième paramètre, si ce paramètre est manquant, le résultat sera stocké dans la variable 
\var{res}. Le nom de la variable qui est transmis dans le second paramètre est utilisé
uniquement pour les messages d'erreur si la traduction échoue, cela permet d'appeler la source
de la valeur à traduire. Si une erreur se produit, un message est alors exécuté.

\begin{example}
\begin{verbatim}
    Unable to translate value '<Valeur>' contained in <Nom de Variable>.
\end{verbatim}
\end{example}

La valeur nulle est retourée si la traduction a réussie et une valeur non nulle sera retounée
si une erreur s'est produite.

\end{description}


\subsubsection{Divers}

\begin{description}

\item[\texttt{mk\_writable <Fichier>}~:] pour vous assurer que le fichier transmis
sera accessible en écriture. Seulement si le fichier est en lecture seule dans
le fichier-système et juste monté par un lien symbolique, une copie locale
sera alors crée pour avoir le fichier en écrit.

\item[\texttt{list\_unique <Liste>}~:] supprime les doublons dans la liste transmise.
Le résultat est écrit dans la sortie standard.

\end{description}

\subsection{Règles mdev}

Il est possible d'établir des règles mdev supplémentaires qui exécutent des
actions spécifiques pour l'apparition ou la disparition de certains dispositifs
dans les paquetages OPT. Par exemple la variable \var{OPT\_AUTOMOUNT} dans le
paquetage hd, utilise une telle règle pour monter automatiquement un nouveau
support de stockage émergentes. Si vous souhaitez intégrer une règle mdev
supplémentaire, vous pouvez utiliser le script sous la forme~:

\begin{verbatim}
    /etc/mdev.d/mdev<Nummer>.<Name>
\end{verbatim}

Installation du rootFS, le nombre doit être similaire au début et à la fin du script
et constitué de trois chiffres, le nom peut être choisi arbitrairement. Dans ce
scénario, toutes les sorties sont intégrés dans le fichier standard
\texttt{/etc/mdev.conf}. Exemple avec \var{OPT\_AUTOMOUNT} mentionné ci-dessus~:

\begin{small}
\begin{verbatim}
#!/bin/sh
#----------------------------------------------------------------------------
# /etc/mdev.d/mdev500.automount - mdev HD automounting rules     __FLI4LVER__
#
#
# Last Update:  $Id$
#----------------------------------------------------------------------------

cat <<"EOF"
#
# mdev500.automount
#

-SUBSYSTEM=block;DEVTYPE=partition;.+       0:0 660 */lib/mdev/automount

EOF
\end{verbatim}
\end{small}

L'en-tête de la syntaxe des règles du fichier \texttt{/etc/mdev.conf}
et la documentation mdev, se trouve sur la page Web
\altlink{http://git.busybox.net/busybox/plain/docs/mdev.txt} tout y est référencées.
Si le script invoque une règle (comme dans l'exemple ci-dessus \texttt{/lib/mdev/automount}),
il déclenchera un accès à toutes les variables "événements" du kernel et en particulier~:

\begin{itemize}
\item \var{ACTION} (Typiquement \texttt{add} ou \texttt{remove}, plus rarement \texttt{change})
\item \var{DEVPATH} (le chemin sysfs pour le composant concerné)
\item \var{SUBSYSTEM} (le sous-système du kernel affecté, voir ci-dessous)
\item \var{DEVNAME} (le fichier de périphérique affecté dans \texttt{/dev}, si manquant,
					le périphérique ne sera pas créé ou supprimé, mais par exemple un module)
\item \var{MDEV} (est fixé par mdev le nom du fichier du périphérique sera finalement généré)
\end{itemize}

Exemple de sous-systèmes du Kernel~:

\begin{description}
\item[block] Bloque les périphériques (carte mémoire) comme \texttt{sda} pour le (premier
			 disque dur), \texttt{sr0} pour le (premier lecteur de CD) ou \texttt{ram1}
			 pour le (deuxième disque RAM)
\item[input] Périphériques d'entrée de clavier, de la souris, etc., comme le fichier
			 \texttt{input/event0} qui correspond à un périphérique, il doit être défini
			 avec sysfs c'est un système de fichiers virtuel.
\item[mem]   Périphériques pour accéder aux ports de la mémoire et du matériel, comme
			 \texttt{mem} et \texttt{ports}, il peut également comprendre les pseudo-périphériques
			 comme \texttt{zero} (délivre en permanence le caractère ASCII par la valeur null)
			 et \texttt{null} (pas de retours, tout sera absorbé)
\item[sound] Divers Périphériques pour les sorties audio, appellation hétérogène
\item[tty]   Périphériques pour l'accès aux consoles physiques ou virtuels, comme \texttt{tty1}
			 (première console virtuelle) ou \texttt{ttyS0} (première console série)
\end{description}

Un exemple pour les deux premiers ports série~:

\begin{scriptsize}
\begin{verbatim}
mdev[42]: 30.050644 add@/devices/pnp0/00:04/tty/ttyS0
mdev[42]:   ACTION=add
mdev[42]:   DEVPATH=/devices/pnp0/00:04/tty/ttyS0
mdev[42]:   SUBSYSTEM=tty
mdev[42]:   MAJOR=4
mdev[42]:   MINOR=64
mdev[42]:   DEVNAME=ttyS0
mdev[42]:   SEQNUM=613

mdev[42]: 30.051477 add@/devices/platform/serial8250/tty/ttyS1
mdev[42]:   ACTION=add
mdev[42]:   DEVPATH=/devices/platform/serial8250/tty/ttyS1
mdev[42]:   SUBSYSTEM=tty
mdev[42]:   MAJOR=4
mdev[42]:   MINOR=65
mdev[42]:   DEVNAME=ttyS1
mdev[42]:   SEQNUM=614
\end{verbatim}
\end{scriptsize}

Un exemple pour connecter un clavier MF II~:

\begin{scriptsize}
\begin{verbatim}
mdev[41]: 4.030653 add@/devices/platform/i8042/serio0/input/input0
mdev[41]:   ACTION=add
mdev[41]:   DEVPATH=/devices/platform/i8042/serio0/input/input0
mdev[41]:   SUBSYSTEM=input
mdev[41]:   PRODUCT=11/1/1/ab41
mdev[41]:   NAME="AT Translated Set 2 keyboard"
mdev[41]:   PHYS="isa0060/serio0/input0"
mdev[41]:   PROP=0
mdev[41]:   EV=120013
mdev[41]:   KEY=4 2000000 3803078 f800d001 feffffdf ffefffff ffffffff fffffffe
mdev[41]:   MSC=10
mdev[41]:   LED=7
mdev[41]:   MODALIAS=input:b0011v0001p0001eAB41-e0,1,4,11,14,k71,72,73,74,75,76,77,79,
7A,7B,7C,7D,7E,7F,80,8C,8E,8F,9B,9C,9D,9E,9F,A3,A4,A5,A6,AC,AD,B7,B8,B9,D9,E2,ram4,l0,
1,2,sfw
mdev[41]:   SEQNUM=604
\end{verbatim}
\end{scriptsize}

Un exemple pour charger un module kernel USB (\texttt{uhci\_hcd})~:

\begin{scriptsize}
\begin{verbatim}
mdev[41]: 6.537506 add@/module/uhci_hcd
mdev[41]:   ACTION=add
mdev[41]:   DEVPATH=/module/uhci_hcd
mdev[41]:   SUBSYSTEM=module
mdev[41]:   SEQNUM=633
\end{verbatim}
\end{scriptsize}

Un exemple pour connecter un disque dur~:

\begin{scriptsize}
\begin{verbatim}
mdev[41]: 7.267527 add@/devices/pci0000:00/0000:00:07.1/ata1/host0/target0:0:0/0:0:0:0/block/sda
mdev[41]:   ACTION=add
mdev[41]:   DEVPATH=/devices/pci0000:00/0000:00:07.1/ata1/host0/target0:0:0/0:0:0:0/block/sda
mdev[41]:   SUBSYSTEM=block
mdev[41]:   MAJOR=8
mdev[41]:   MINOR=0
mdev[41]:   DEVNAME=sda
mdev[41]:   DEVTYPE=disk
mdev[41]:   SEQNUM=688
\end{verbatim}
\end{scriptsize}

Ci-dessus est un disque dur ATA/IDE (ata1), le nom du périphérique \texttt{sda} sera utilisé
pour le traitement.

Un exemple pour un périphérique de blocage à distance (affecte le marquage d'un fichier d'image
de VM-fli4l et a été résolu via \texttt{virsh detach-device})~:

\begin{scriptsize}
\begin{verbatim}
mdev[42]: 52.600646 remove@/devices/pci0000:00/0000:00:0a.0/virtio5/block/vdb/vdb1
mdev[42]:   ACTION=remove
mdev[42]:   DEVPATH=/devices/pci0000:00/0000:00:0a.0/virtio5/block/vdb/vdb1
mdev[42]:   SUBSYSTEM=block
mdev[42]:   MAJOR=254
mdev[42]:   MINOR=17
mdev[42]:   DEVNAME=vdb1
mdev[42]:   DEVTYPE=partition
mdev[42]:   SEQNUM=776

mdev[42]: 52.644642 remove@/devices/virtual/bdi/254:16
mdev[42]:   ACTION=remove
mdev[42]:   DEVPATH=/devices/virtual/bdi/254:16
mdev[42]:   SUBSYSTEM=bdi
mdev[42]:   SEQNUM=777

mdev[42]: 52.644718 remove@/devices/pci0000:00/0000:00:0a.0/virtio5/block/vdb
mdev[42]:   ACTION=remove
mdev[42]:   DEVPATH=/devices/pci0000:00/0000:00:0a.0/virtio5/block/vdb
mdev[42]:   SUBSYSTEM=block
mdev[42]:   MAJOR=254
mdev[42]:   MINOR=16
mdev[42]:   DEVNAME=vdb
mdev[42]:   DEVTYPE=disk
mdev[42]:   SEQNUM=778

mdev[42]: 52.644777 remove@/devices/pci0000:00/0000:00:0a.0/virtio5
mdev[42]:   ACTION=remove
mdev[42]:   DEVPATH=/devices/pci0000:00/0000:00:0a.0/virtio5
mdev[42]:   SUBSYSTEM=virtio
mdev[42]:   MODALIAS=virtio:d00000002v00001AF4
mdev[42]:   SEQNUM=779

mdev[42]: 52.644973 remove@/devices/pci0000:00/0000:00:0a.0
mdev[42]:   ACTION=remove
mdev[42]:   DEVPATH=/devices/pci0000:00/0000:00:0a.0
mdev[42]:   SUBSYSTEM=pci
mdev[42]:   PCI_CLASS=10000
mdev[42]:   PCI_ID=1AF4:1001
mdev[42]:   PCI_SUBSYS_ID=1AF4:0002
mdev[42]:   PCI_SLOT_NAME=0000:00:0a.0
mdev[42]:   MODALIAS=pci:v00001AF4d00001001sv00001AF4sd00000002bc01sc00i00
mdev[42]:   SEQNUM=780
\end{verbatim}
\end{scriptsize}

Comme vous pouvez le voir, plusiers extration du sous-système du kernel ont été impliquès
(avec \texttt{block}, \texttt{bdi}, \texttt{virtio} et \texttt{pci}).

\subsection{Périphériques ttyI}

Au sujet les périphériques ttyI, vous pouvez utiliser (\texttt{/dev/ttyI0} \ldots \texttt{/dev/ttyI15}),
pour la création d’un \og{}émulateur de modem\fg{} avec plusieurs cartes ISDN (ou RNIS),
il existe un compteur pour les conflits entre les différents OPTs et les utilisations
de ces périphériques, c'est un dispositif à éviter. Ces émulateur seront créés lors
du démarrage du routeur, dans le fichier \texttt{/var/run/next\_ttyI}, avec l'utilisation
un compteur. Dans l'exemple de script suivant, la valeur est interrogé et peut être augmentée
de un, il sera exporté de nouveau dans le prochain OPT.

\begin{example}
\begin{verbatim}
    ttydev_error=
    ttydev=$(cat /var/run/next_ttyI)
    if [ $ttydev -le 16 ]
    then                                    # ttyI device available? yes
        ttydev=$((ttydev + 1))              # ttyI device + 1
        echo $ttydev >/var/run/next_ttyI    # save it
    else                                    # ttyI device available? no
        log_error "No ttyI device for <Nom de votre OPT> available!"
        ttydev_error=true                   # set error for later use
    fi

    if [ -z "$ttydev_error" ]               # start OPT only if next tty device
    then                                    # was available to minimize error
        ...                                 # messages and minimize the
                                            # risk of uncomplete boot
    fi
\end{verbatim}
\end{example}

\subsection{Scripts de connexion et de déconnexion par modem}


\subsubsection{Généralités}

Après avoir établi ou coupé une connexion par modem, les scripts \texttt{/etc/ppp/} sont
traitées. Voici les actions qui sont nécessaires pour activer ou déactiver une connexion,
qui seront stocker dans l'OPT. Le schéma des noms de fichier est le suivant~:

\begin{table}[htbp]
\centering
\begin{tabular}{l}
    \texttt{ip-up<numéro à trois chiffres>.<nom d'OPT>}\\
    \texttt{ip-down<numéro à trois chiffres>.<nom d'OPT>}\\
\end{tabular}
\end{table}

Le script \texttt{ip-up} est exécuté après la \emph{connexion} et le script
\texttt{ip-down} est exécuté après la \emph{déconnexion}

\wichtig{Dans le script \texttt{ip-down} aucune intervention ne doit être réalisé,
autrement cela conduirait à une nouvelle connexion Internet, avec uniquement un accès
à l’état online permanent, pour les utilisateurs qui non pas de forfait illimité,
cela peut couter très cher.}

\wichtig{Car pour ce script, il n’y a aucun processus qui est généré, en plus, ce
script ne peut \emph{pas} être arrêté avec la commande \og{}exit\fg{}!}

\textbf{Remarque~:} le scripts \texttt{ip-up} peut être examiné lors qu'il est exécuté,
pour cela le fichier \texttt{rc400} sera vérifier avec la variable \var{ip\_up\_events}.
Si c'est réglé sur "yes", une connexion par modem existe et le script \texttt{ip-up} sera
exécuté. Si c'est réglé sur "no", une connexion par modem n'existe pas et le script
\texttt{ip-up} ne sera pas exécuté. Il y a une exception pour cette règle~: Si un
routeur Ethernet pur n'est pas configuré pour des connexions commutées, mais configuré
pour une route par défaut (\texttt{0.0.0.0/0}), le script \texttt{ip-up} sera exécuté qu'une
fois, exactement à la fin du processus de boot. (De même le script \texttt{ip-down} sera
exécuté qu'une fois avant l'arrêt du routeur).


\subsubsection{Les variables}

Grâce au concept d'appel spécial les scripts \texttt{ip-up} et \texttt{ip-down} sont exécutés
et les variables suivantes sont utilisées~:

\begin{table}[htbp]
\centering
\begin{tabular}{lp{10cm}}
    \var{real\_interface}    & L'interface actuelle, par ex.
                               \texttt{ppp0}, \texttt{ippp0}, \ldots\\
    \var{interface}          & Interface-IMOND, avec \texttt{pppoe},
                               \texttt{ippp0}, \ldots\\
    \var{tty}                & Terminal de connexion, peut-être vide~!\\
    \var{speed}              & Vitesse de connexion, par ex. avec ISDN 64000\\
    \var{local}              & Adresse-IP spécifique\\
    \var{remote}             & Adresse-IP d'ordinatuer auquel vous êtes connecté\\
    \var{is\_default\_route} & Indique si l'actuel \texttt{ip-up}/\texttt{ip-down}
								est utilisé pour l'interface de la route par défaut
								peut-être \og{}yes\fg{} ou \og{}no\fg{})\\
\end{tabular}
\end{table}


\subsubsection{Route par défaut}

Depuis la version 2.1.0, les scripts \texttt{ip-up} et \texttt{ip-down} sont exécutés
non seulement pour l'interface sur laquelle la route par défaut est configurée, mais
aussi pour toutes les connexions qui ont besoin les scripts \texttt{ip-up} et \texttt{ip-down}.
Pour émuler des comportements anciens, vous devez inclure les éléments à déclencher
dans les scripts \texttt{ip-up} et \texttt{ip-down} la requête suivante doit être insérée~:

\begin{example}
\begin{verbatim}
    # is a default-route-interface going up?
    if [ "$is_default_route" = "yes" ]
    then
        # Les actions à déclencher
    fi
\end{verbatim}
\end{example}

Naturellement, les nouveaux comportements doivent être utilisés, pour des actions spécifiques.

