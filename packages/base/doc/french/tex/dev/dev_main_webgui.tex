% Synchronized to r36388

\section{Création d'un CGI pour le paquetage \emph{httpd}}

\subsection{Informations générales sur le serveur Web}

Le serveur web, qui est utilisé dans fli4l est un \texttt{mini\_httpd} de ACME Labs.
Les fichiers sources peuvent être téléchargés à partir du site
\altlink{http://www.acme.com/software/mini_httpd/}. Cependant, quelques modifications
ont été apportées pour fli4l. Les ajustements sont dans le paquetage \emph{src} et dans
le répertoire \texttt{src/fbr/buildroot/package/mini\_httpd}.


\subsection{Nom du script}

Le nom du script doit être significatif, de sorte qu'il soit plus facile à distinguer
par rapport au autres scripts, il ne doit pas avoir de collision de noms avec
les différents OPTs.

Les sauts de ligne dans les scripts exécuté sous DOS, seront convertis en sauts de
ligne UNIX, vous devez aussi modifier le fichier \texttt{opt/<PACKAGE>.txt}, voir
Table~\jump{table:options}{\ref{table:options}}.


\subsection{Configuration du menu}

Pour faire une nouvelle entrée dans le menu, vous devez paramétrer le fichier
\texttt{/etc/httpd/menu}. Ce mécanisme permet de faire des modifications du menu
pendant le fonctionnement de l'OPT. Pour cela on utilise le script \texttt{/etc/httpd/menu},
il vérifie le format du fichier pour que celui-ci soit toujours consistant.
Pour ajouter un nouvel élément dans le menu, vous devez utiliser la commande de
la manière suivante~:

\begin{example}
\begin{verbatim}
    httpd-menu.sh add [-p <priority>] <link> <name> [section] [realm]
\end{verbatim}
\end{example}

Si vous indiquez un nom dans \texttt{<name>} il sera inclus dans la section,
que vous avez indiquez dans \texttt{[section]}. Si vous omettez la section, le nom
sera par défaut inclus dans la section du "paquetage-OPT". Vous indiquez dans
\texttt{<link>} la cible du nouveau lien. Dans \texttt{<priority>} vous spécifiez
la priorité de l'élément du menu dans sa section. S'il n'est pas spécifié, 500 sera
utilisé pour la priorité par défaut. La priorité doit être un nombre à trois chiffres.
Le lien qui se situ le plus haut dans la section a une priorité la plus faible, si
vous voulez un lien le plus bas, vous devez par exemple choisir une priorité de 900.
De même, la priorité des données sont triées avec la cible du lien. Dans \texttt{[realm]}
vous indiquez le domaine pour que l'utilisateur connecté puisse avoir les autorisations
nécessaires pour \var{voir} l'élément du menu qui sera affiché. S'il \texttt{[realm]} n'est
pas spécifié, l'élément du menu sera toujours affiché, voir aussi la section
\jump{sec:rights}{\og{}Droits des utilisateurs\fg{}}.

Exemple~:

\begin{example}
\begin{verbatim}
    httpd-menu.sh add "Nouveau-fichier.cgi" "Cliquez ici" "Outils" "outils"
\end{verbatim}
\end{example}

Cet exemple produit dans la section "Outils" un lien avec le titre "Cliquez ici",
la destination du lien sera "Nouveau-fichier.cgi", la section sera créé si elle n'est
pas définie.

Vous pouvez également supprimer dans le script un enregistrement de lien du menu~:

\begin{example}
\begin{verbatim}
    httpd-menu.sh rem <link>
\end{verbatim}
\end{example}

Avec cette commande le lien \texttt{<link>} qui a été enregistré sera supprimé.

\wichtig{Si plusieurs enregistrements se réfèrent au même fichier dans le menu, toutes ces
enregistrements seront supprimés dans le menu.}

Puisque des sections peuvent aussi avoir des priorités, celles-ci peuvent être créées
manuellement. Si une section est créée automatiquement lorsque vous ajoutez une entré,
il recevera automatiquement la priorité 500. Voici la syntaxe pour la création d'une section~:

\begin{example}
\begin{verbatim}
    httpd-menu.sh addsec <priority> <name>
\end{verbatim}
\end{example}

Ici aussi \texttt{<priority>} doit contenir un nombre à trois chiffres.

Pour apprendre à configurer judicieusement les priorités, il est
intéressant de regarder le fichier \texttt{/etc/httpd/menu} pendant l'exécution
de fli4l les priorités sont dans la deuxième colonne.

Pour être complet, voici un bref commentaire sur le format du fichier menu.
Si la commande \texttt{httpd-menu.sh} ne suffit pas, vous pouvez sauter ce paragraphe.
Le fichier \texttt{/etc/httpd/menu} a la structure suivante~: il est divisé en quatre
colonnes. La première colonne il y a une lettre de code, pour diffèrencier les
intitulés et les enregistrements. Dans la deuxième colonne il y a les priorités triées.
La troisième colonne contient en intitulé un trait d'union, cela veut dire qu'il n'y a
aucune signification pour le titre et de la configuration de la cible du lien. Dans
le reste de la ligne se trouve le texte qui apparaîtra plus tard dans le menu.

Avec la lettre d'identification \og{}t\fg{}, une nouvelle section dans le menu sera installée.
Pour configurer une entrée normale dans le menu vous utilisez la lettre d'identification \og{}e\fg{}.
Un exemple~:

\begin{example}
\begin{verbatim}
    t 300 - Mon grand OPT
    e 200 monopt1.cgi lien pour grand
    e 500 monopt1.cgi?plus=oui lien plus pour grand
\end{verbatim}
\end{example}

Lors de l'édition de ce fichier vous devez vous assurer que le script 
\texttt{httpd-menu.sh} soit toujours trié avant de sauvegarde le fichier. Que les
différentes sections soit triées et que les sections enregistrées sont triées
par section l'algorithme de tri exacte peut être repris par \texttt{httpd-menu.sh}~--
Cependant, il est possible d'étendre ce script avec de nouvelles fonctionnalités,
pour que tous les modifications du menu se produisent à un emplacement central.


\subsection{Construction d'un script CGI}

\subsubsection{L'en-tête}

Tous les scripts du serveur Web sont de simples scripts shell (les interpréteurs
tels que Perl, PHP, etc, sont beaucoup trop volumineux pour fli4l). Le script doit
obligatoirement commencer par l'en-tête avec (la référence de l'interprèteur, le nom,
une indiquation sur le scénario, l'auteur, la licence).


\subsubsection{Script auxiliaire \texttt{cgi-helper}}

Tout de suite après l'en-tête vous devez placer le script auxiliaire
\texttt{cgi-helper} avec la commande suivante~:

\begin{example}
\begin{verbatim}
    . /srv/www/include/cgi-helper
\end{verbatim}
\end{example}

L'espace entre le point et le slash (ou la barre oblique) est important~!

Ce script auxiliaire permet diverses fonctionnalités pour simplifier la création
des CGIs, il est essentiel à fli4l. De plus, avec l'intégration de \texttt{cgi-helper}
il effectue également des tâches standard, telles que l'analyse syntaxique
des variables, qui ont été associées à des formulaires ou des transitions URL
ou de l'affichage de la langue pour le texte ou des fichiers CSS.

Le tableau~\ref{tab:dev:cgi-helper} donne un aperçu des fonctionnalitées des scripts
\texttt{cgi-helper}.

\begin{table}[htbp]
  \centering
  \caption{Les fonctions pour le script \texttt{cgi-helper}}
  \label{tab:dev:cgi-helper}
  \begin{small}
    \begin{tabular}{|l|p{0.8\textwidth}|}
      \hline
      Nom                  & Fonction \\
      \hline
      \texttt{check\_rights}      & Vérification des droits des utilisateurs \\
      \texttt{http\_header}       & Édition d’une en-tête HTTP standard ou d’une
          en-tête spécifique, par ex., la manière de télécharger des fichiers \\
      \texttt{show\_html\_header} & Édition complètes d'une en-tête de page (y compris
          les en-têtes HTTP et les en-têtes du Menu) \\
      \texttt{show\_html\_footer} & Édition de la fin d'une page HTML \\
      \texttt{show\_tab\_header}  & Édition d’une fenêtre de contenu pour un tableau \\
      \texttt{show\_tab\_footer}  & Édition de la fin de contenu pour un tableau \\
      \texttt{show\_error}        & Édition d'une fenêtre pour les messages d'erreur (couleur: rouge) \\
      \texttt{show\_warn}         & Édition d'une fenêtre pour les messages d'alerte (couleur: jaune) \\
      \texttt{show\_info}         & Édition d'une fenêtre pour les messages d'informations/réussites (couleur: vert) \\
      \hline
    \end{tabular}
  \end{small}
\end{table}


\subsubsection{Contenu d'un script CGI}

Afin d'assurer un aspect homogène et surtout pour la compatibilité avec les
futures versions-fli4l, il est fortement recommandé d'utiliser les fonctions
du script auxiliaire \texttt{cgi-helper}, même si nous pouvons théoriquement générer
toutes les sorties dans un même CGI.

Un simple script-CGI peut ressembler à ceci~:

\begin{example}
\begin{verbatim}
    #!/bin/sh
    # --------------------
    # Header (c) Autor Datum
    # --------------------
    # get main helper functions
    . /srv/www/include/cgi-helper

    show_html_header "Mon premier CGI"
    echo '   <h2>Bienvenue</h2>'
    echo '   <h3>Ceci est un exemple de script-CGI</h3>'
    show_html_footer
\end{verbatim}
\end{example}


\subsubsection{Fonction \texttt{show\_html\_header}}

La valeur qui est indiqué dans la fonction \texttt{show\_html\_header}, est utilisé
pour le titre. Il sera généré automatiquement dans le menu, il inclue aussi
automatiquement le script CSS et le fichier langue. Il faut pour cela que les
fichiers scripts ont le même nom et qu’ils soient dans les répertoires
/srv/www/css et /srv/www/lang (bien sûr, avec une extension différente).
Exemple~:

\begin{example}
\begin{verbatim}
    /srv/www/admin/OpenVPN.cgi
    /srv/www/css/OpenVPN.css
    /srv/www/lang/OpenVPN.de
\end{verbatim}
\end{example}

L'utilisation du fichier-langue ainsi que le fichier-CSS sont facultatif.
Intégré toujours \texttt{css/nom.css} et \texttt{lang/nom.<lang>}, \texttt{<lang>} correspond à la langue
choisie.

Vous pouvez placer à côté du titre de la fonction \texttt{show\_html\_header} d'autres
paramètres. Une exécution de la fonction avec tous les paramètres pourrait
ressembler à ceci~:

\begin{example}
\begin{verbatim}
show_html_header "Titre" "refresh=$time;url=$url;cssfile=$cssfile;showmenu=no"
\end{verbatim}
\end{example}

Tous les paramètres supplémentaires, comme le montre l'exemple ci-dessus,
doivent avoir des guillemets et être séparés par un point-virgule. Autrement
le contrôle de syntaxe ne sera \emph{pas} réussi~! Il est nécessaire de respecter la
syntaxe des paramètres.

Voici un bref aperçu des fonctions de ces paramètres~:

\begin{itemize}
 \item \texttt{refresh=}\emph{time}: Temps en secondes dans lequel la page sera rechargée par le navigateur.
 \item \texttt{url=}\emph{url}: L'URL est rechargé après un rafraîchissement.
 \item \texttt{cssfile=}\emph{cssfile}: Nom du fichier CSS, si celui-ci est différent du CGI.
 \item \texttt{showmenu=no}: L'affichage du menu et l'en-tête peut être supprimée.
\end{itemize}

D'autres conseils pour le contenu du CGI~:

\begin{itemize}
 \item Prenez votre temps~:-)
 \item Écrivez proprement le HTML (SelfHTML\footnote{voir
    \altlink{http://de.selfhtml.org/}} est un bon point de départ)
 \item Renoncer au bric-à-brac très moderne, (le JavaScript est OK, si cela ne crée pas
    de problème avec les utilisateurs, le tout fonctionnent également sans JavaScript)
\end{itemize}


\subsubsection{Fonction \texttt{show\_html\_footer}}

La fonction \texttt{show\_html\_footer} ferme le bloc dans le script CGI, qui a été lancé par
la fonction \texttt{show\_html\_header}.


\subsubsection{Fonction \texttt{show\_tab\_header}}

Dans le CGI, pour que le contenu de vos résultats soit bien rangé et affiché
sur la page Web, avec l'aide du \texttt{cgi-helper} vous pouvez utiliser la fonction
show\_tab\_header. Il peut générer des titres dans des onglets cliquables dans
un 'Tableau', ainsi votre page peut être divisé en plusieurs zones logiquement séparées.

Les paramètres dans la fonction \texttt{show\_tab\_header} sont toujours écrits deux par
deux. Le premier paramètre correspond au titre de l’onglet et le deuxième
correspond au lien. Si vous indiquez 'no' comme deuxième paramètre, le titre
sera seulement placé dans l'onglet (en couleur) et le lien ne sera pas cliquable.

Dans l'exemple suivant, nous allons créer une 'fenêtre' dans laquelle nous
allons mettre le titre 'Une grande fenêtre’. Dans la fenêtre sera écrit 'foo bar'~:

\begin{example}
\begin{verbatim}
    show_tab_header "Une grande fenêtre" "no"
    echo "foo"
    echo "bar"
    show_tab_footer
\end{verbatim}
\end{example}

Dans l'exemple suivant, deux onglets cliquables sont générés, la variable
\var{action} dans le script va transmettre des valeurs différentes.

\begin{example}
\begin{verbatim}
    show_tab_header "1. onglet" "$myname?action=machdies" \
                    "2. onglet" "$myname?action=machjenes"
    echo "foo"
    echo "bar"
    show_tab_footer
\end{verbatim}
\end{example}

Maintenant le contenu du script peut exploiter la variable \var{FORM\_action} (voir
ci-dessous pour l’exploitation de la variable) en fonction des différents
paramètres. l'onglet apparaîtra et l’on pourra le sélectionner en cliquant
dessus, si vous ne voulez pas que cet onglet soit sélectionnable, vous devez
placer dans la fonction 'no', pour ce passé du lien comme indiqué plus haut.
Cela sera plus facile, si vous utilisez l'exemple suivant~:

\begin{example}
\begin{verbatim}
    _opt_machdies="1. onglet"
    _opt_machjenes="2. onglet"
    show_tab_header "$_opt_machdies" "$myname?action=opt_machdies" \
                    "$_opt_machjenes" "$myname?action=opt_machjenes"
    case $FORM_action in
        opt_machdies) echo "foo" ;;
        opt_machjenes) echo "bar" ;;
    esac
    show_tab_footer
\end{verbatim}
\end{example}

Si vous utilisez une variable pour sélectionner un titre, ce nom correspond à
l’action de la variable et commencera par un tiret (\texttt{\_}), ainsi l'onglet
correspond à ce nom pourra être sélectionné.


\subsubsection{Fonction \texttt{show\_tab\_footer}}

La fonction \texttt{show\_tab\_footer} ferme le bloc dans le script CGI, qui a été lancé par
la fonction \texttt{show\_tab\_header}.


\subsubsection{Supporte le multi-language}

Le script \texttt{cgi-helper} contient également une fonction pour faire des scripts CGI en
utilisant d'autres langues. Pour ce faire, vous devez 'juste' utiliser les variables
commençant par un tiret bas (\texttt{\_}) pour traduire le texte et de définir ces variables
dans la langue de votre choix.

Exemple~:

lang/opt.de contient~:

\begin{example}
\begin{verbatim}
    _opt_machdies="Eine Ausgabe"
\end{verbatim}
\end{example}

lang/opt.en contient~:

\begin{example}
\begin{verbatim}
    _opt_machdies="An Output"
\end{verbatim}
\end{example}

admin/opt.cgi contient~:

\begin{example}
\begin{verbatim}
    ...
    echo $_opt_machdies
    ...
\end{verbatim}
\end{example}


\subsubsection{Utilisation de formulaires}

Afin de traiter des formulaires, il faut savoir certaines choses. L’utilisateur
doit choisir la méthode de formulaire \var{GET} ou \var{POST}, les paramètres peuvent
être trouvées après le montage du script \texttt{cgi-helper} (qui à son tour appelle le
de programme auxiliaire proccgi) avec la nouvelle variable \var{FORM\_<Paramètre>}.
Si vous avez entré un nom, une "adresse" dans le champ de formulaire et si vous indiquez
dans le script CGI \var{\$FORM\_adresse}, cette adresse sera accessible.

Pour plus d'informations sur le programme \texttt{proccgi} vous pouvez les trouver ici
\altlink{http://www.fpx.de/fp/Software/ProcCGI.html}.


\marklabel{sec:rights}{
    \subsubsection{Droits des utilisateurs~: Fonction \texttt{check\_rights}}
}

Afin de vérifier si l'utilisateur a les droits suffisants pour utiliser un script CGI,
la fonction \texttt{check\_rights}  est appelée au début du script CGI comme ceci~:

\begin{example}
\begin{verbatim}
    check_rights <Section> <Action>
\end{verbatim}
\end{example}

Le script CGI ne sera exécuté que si l'utilisateur est enregistré.
\begin{itemize}
\item A tous les droits (\verb+HTTPD_RIGHTS_x='all'+), ou
\item tous les droits dans un domaine spécifique
    (\verb+HTTPD_RIGHTS_x='<Domaine>:all'+), ou
\item a le droit d'effectuer une action spécifique dans un domaine spécifique \\
    (\verb+HTTPD_RIGHTS_x='<Domaine>:<Action>'+).
\end{itemize}


\subsubsection{Fonction \texttt{show\_error}}

Cette fonction renvoie un message d'erreur dans une fenêtre rouge. Il a besoin de
deux paramètres~: un titre et un message. Exemple~:

\begin{example}
\begin{verbatim}
    show_error "Error: No key" "No key was specified!"
\end{verbatim}
\end{example}


\subsubsection{Fonction \texttt{show\_warn}}

Cette fonction renvoie un message d'avertissement dans une fenêtre jaune. Il a besoin de
deux paramètres~: un titre et un message. Exemple~:

\begin{example}
\begin{verbatim}
    show_info "Warnung" "Actuellement, il n'y pas de connexion!"
\end{verbatim}
\end{example}


\subsubsection{Fonction \texttt{show\_info}}

Cette fonction renvoie un message de réussite ou d'information dans une fenêtre verte.
Il a besoin de deux paramètres~: un titre et un message. Exemple~:

\begin{example}
\begin{verbatim}
    show_info "Info" "Action a été exécutée avec succès!"
\end{verbatim}
\end{example}


\subsubsection{Script auxiliaire \texttt{cgi-helper-ip4}}

Immédiatement après le script \texttt{cgi-helper}, vous devez ensuite intégrer
le script auxiliaire \texttt{cgi-helper-ip4} avec la commande suivante~:

\begin{example}
\begin{verbatim}
. /srv/www/include/cgi-helper-ip4
\end{verbatim}
\end{example}

L'espace entre le point et le slash (ou la barre oblique) est important~!

Ce script fournit des fonctions pour vous aidez à effectuer des tests d'adresse IPv4.


\subsubsection{Fonction \texttt{ip4\_isvalidaddr}}

Cette fonction vérifie si une adresse IPv4 qui a été enregistrée est valide.
Exemple~:

\begin{example}
\begin{verbatim}
    if ip4_isvalidaddr ${FORM_inputip}
    then
        ...
    fi
\end{verbatim}
\end{example}


\subsubsection{Fonction \texttt{ipv4\_normalize}}

Cette fonction supprime les zéros inutile dans l'adresse IPv4 enregistrée.
Exemple~:

\begin{example}
\begin{verbatim}
    ip4_normalize ${FORM_inputip}
    IP=$res
    if [ -n "$IP" ]
    then
        ...
    fi
\end{verbatim}
\end{example}


\subsubsection{Fonction \texttt{ipv4\_isindhcprange}}

Cette fonction vérifie si l'adresse IPv4 enregistrée est dans la plage d'adresse
de début et de fin. Exemple~:

\begin{example}
\begin{verbatim}
    if ip4_isindhcprange $FORM_inputip $ip_start $ip_end
    then
        ...
    fi
\end{verbatim}
\end{example}


\subsection{Divers}

Des choses et d'autres (et oui, c'est aussi important!)~:

\begin{itemize}
 \item Le \texttt{mini\_httpd} ne protège pas les sous-répertoires par mot de passe.
      Vous devez avoir un répertoire \texttt{.htaccess} pour chaque utilisateur
      ou créer un lien vers un autre fichier \texttt{.htaccess}.
 \item KISS - Keep it simple, stupid~!
 \item Ces informations peuvent être changées à tout moment et sans préavis~!
\end{itemize}


\subsection{Dépannage}

Pour faciliter le débogage d'un script CGI, vous pouvez activer le mode débogage avant
l'intégration du script \texttt{cgi-helper}. Pour cela, la variable \var{set\_debug} doit
être paramétrée sur "yes". Cela conduit à la création du fichier \texttt{debug.log}, que
vous pourrez télécharger sur l'URL \texttt{http://<fli4l-Host>/admin/debug.log}. Cela inclut
tous les appels vers ce script CGI. La variable \texttt{set\_debug} n'est pas globale, et
doit être renouvelée dans tous les scripts CGI en question. Exemple~:

\begin{example}
\begin{verbatim}
    set_debug="yes"
    . /srv/www/include/cgi-helper
\end{verbatim}
\end{example}

En outre, cURL\footnote{voir \altlink{http://de.wikipedia.org/wiki/CURL}} est
idéal pour le dépannage, en particulier lorsque les en-têtes HTTP ne sont pas
assemblé correctement ou que le navigateur affiche une page blanche. Et aussi,
le comportement de mise cache des navigateurs modernes est un véritable problème.

Exemple~: Avec l'exécution suivant, en-tête HTTP ("`\emph{d}ump"', \texttt{-D})
l'affichage normale du CGI \texttt{admin/mon.cgi} sera publié. Vous pouvez utiliser
le nom d'utilisateur ("`\emph{u}ser"', \texttt{-u}) "`admin"'.

\begin{example}
\begin{verbatim}
    curl -D - http://fli4l/admin/mon.cgi -u admin
\end{verbatim}
\end{example}
