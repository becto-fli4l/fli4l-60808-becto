% Do not remove the next line
% Synchronized to r30771

\section{Type de circuit spécial du paquetage base}

\marklabel{sect:route-circuits}{\subsection{Type de circuit "route"}}

Le type de circuit "route" est utilisé pour gérer une \emph{route} préconfiguré.
Pour qu'une machine soit en mesure d'en contacter une autre, il faut que soit mis
en place un mécanisme qui décrit comment aller de l'une vers l'autre. Le fait de se
connecter et de se déconnecter correspond à la structure de la route. Un unique circuit
route peut router simultanément plusieurs réseaux IPv4 et IPv6 (si le support IPv6
est activé). Dans ce contexte, un circuit route est plus souple que la configuration
d'une route via la variable \var{IP\_ROUTE\_x}, car vous devez toujours spécifier
seulement un réseau par route.

En plus des variables du circuit générale paragraphe
\jump{sec:circuits:general}{"Circuit général"}, les informations suivantes sont
également nécessaires~:

\begin{description}

\config{CIRC\_x\_ROUTE\_DEV}{CIRC\_x\_ROUTE\_DEV}{CIRCxROUTEDEV}

Vous indiquez dans cette variable, l'interface réseau, à travers laquel passera
la route. Cette option est facultative, si au moins une gateway (ou passerelle)
est paramétrée (voir ci-dessous) et que l'interface réseau peut être dérivé. Vous
pouvez utiliser pour le nom de l'interface ("eth0") une référence de l'interface
("IP\_NET\_1\_DEV", "IPV6\_NET\_1\_DEV"), un circuit ("ppp0"), ou un mot clé
("Internet-V4"), le routage dynamique, pour le circuit de l'interface est contrôlé de
nature plutôt ésotérique et doit être utilisé seulement lorsque vous voyez clairement
les effets dynamiques se produire.

Si cette variable est vide et que les deux adresses de la gateway (IPv4 et IPv6) sont
définies (voir ci-dessous), l'interface doit être quand même attribué. En effet, une interface
réseau doit être exactement affecté pour chacun des circuits (et pas plus).

Exemple 1~: \verb+CIRC_x_ROUTE_DEV='IP_NET_2_DEV'+

Exemple 2~: \verb+CIRC_x_ROUTE_DEV='pppoe-v6'+

\config{CIRC\_x\_ROUTE\_GATEWAY\_IPV4}{CIRC\_x\_ROUTE\_GATEWAY\_IPV4}{CIRCxROUTEGATEWAYIPV4}

Dans cette variable vous indiquez l'adresse du routeur pour le prochain bond (communément
appelé "gateway" ou "passerelle") pour acheminer le réseau IPv4. Cette variable peut être vide
si vous avez paramétrer la variable \var{CIRC\_x\_ROUTE\_DEV}, dans ce cas, une route sans
passerelle est généré. L'adresse de passerelle peut être combinée avec un circuit dynamique pour
la route, le circuit de la gateway est contrôlé de nature plutôt ésotérique et doit être utilisé
seulement lorsque vous voyez clairement les effets dynamiques se produire.

Exemple 1~: \verb+CIRC_x_ROUTE_GATEWAY_IPV4='192.168.1.254'+

Exemple 2 (ésotérique)~: \verb+CIRC_x_ROUTE_GATEWAY_IPV4='{dhcp}0.0.0.253'+

Dans le deuxième exemple, l'adresse dynamique de la gateway est assignée via le réseau DHCP (par
exemple 192.168.12.0/24) et la partie attribuée à l'hôte est (dans ce cas 192.168.12.253). Ceci est
utile uniquement si vous savez quel préfixe du réseau est attribué au routeur fli4l par le serveur
DHCP, le routeur spécifique peut trouvé l'hôte à l'adresse 253. Habituellement, le routeur qui
transmet via le DHCP devrait être capable de gérer toutes les routes, donc vous pouvez noter
que pour router un tel réseau vous pouvez utiliser directement un circuit DHCP et ne pas utiliser
un circuit route.

\config{CIRC\_x\_ROUTE\_GATEWAY\_IPV6}{CIRC\_x\_ROUTE\_GATEWAY\_IPV6}{CIRCxROUTEGATEWAYIPV6}

Dans cette variable vous indiquez l'adresse du routeur pour le prochain bond (communément
appelé "gateway" ou "passerelle") pour acheminer le réseau IPv6. tout est dit dans la variable
\var{CIRC\_x\_ROUTE\_GATEWAY\_IPV4}, c'est la même chose ici. En outre, cette variable peut être
utilisé que si le support IPv6 est activé.

Exemple 1~: \verb+CIRC_x_ROUTE_GATEWAY_IPV6='2001:db8:42::1'+

Exemple 2 (ésotérique)~: \verb+CIRC_x_ROUTE_GATEWAY_IPV6='{dhcpv6}::0:0:0:254'+

\end{description}
