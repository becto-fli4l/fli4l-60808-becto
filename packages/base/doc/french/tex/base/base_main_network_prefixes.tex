% Last Update: $Id$
% Synchronized to r50375

\section{Configuration du préfixe réseau}

\begin{description}
  \config{OPT\_NET\_PREFIX}{OPT\_NET\_PREFIX}{OPTNETPREFIX}{
  Avec cette variable vous activez la prise en charge personnalisée du préfixe réseau.

  Techniquement un préfixe réseau n'est rien de plus qu'une adresse réseau,
  il désigne généralement un réseau qui doit être subdivisé davantage.
  Ceci est particulièrement utile si le routeur fli4l ne gère pas seul le réseau,
  mais laisse la gestion des sous-réseaux à un autre routeur. Par cette définition
  et aussi par cette désignation le réseau sera disponible et distribué en totalité,
  il est possible d'utiliser l'adresse réseau en plusieurs endroits sans avoir à
  réécrire le préfixe à chaque fois.

  Vous trouverez ci-dessous des exemples concrets sur la définition du préfixe
  réseau et pour les différents types de préfixes réseau.

  Paramètre par défaut~:
  }
  \verb*?OPT_NET_PREFIX='yes'?

  \config{NET\_PREFIX\_x}{NET\_PREFIX\_x}{NETPREFIXx}{

  Ce tableau définit les différents préfixes du réseau. Les composants
  sont expliqués individuellement ci-dessous.
  }

  \config{NET\_PREFIX\_x\_NAME}{NET\_PREFIX\_x\_NAME}{NETPREFIXxNAME}{
  Nom du préfixe réseau.

  Vous indiquez dans cette variable le nom du préfixe. Ce nom peut ensuite être
  utilisé dans les informations de l'adresse réseau et pour l'utilisation le préfixe.
  Le nom est utilisé comme un nom de circuit, c'est-à-dire qu'il doit être écrit
  entre les accolades.
  }

  \config{NET\_PREFIX\_x\_TYPE}{NET\_PREFIX\_x\_TYPE}{NETPREFIXxTYPE}{
  Type de préfixe de réseau.

  Vous indiquez dans cette variable le type de préfixe. Les types supportés
  sont indiqués dans le tableau~\ref{base:net:prefix:types}.

  \begin{center}
      \begin{longtable}{|l|p{0.7\textwidth}|}
          \hline
          \multicolumn{1}{|l}{\textbf{Type}} &
          \multicolumn{1}{|l|}{\textbf{Signification}} \\
          \hline
          \endhead
          \hline
          \endfoot
          \endlastfoot
          static        & Le préfixe réseau est spécifié directement
						  en tant qu'adresse fixe.
                          \\
          generated-ula & Le préfixe de réseau est généré par fli4l en tant
						  que ULA\footnote{"Unique Local Address"} selon RFC 4193.
						  \footnote{\altlink{https://tools.ietf.org/html/rfc4193}}
						  Si fli4l a accès à un stockage persistant, le préfixe est
						  généré qu'une seule fois, il reste donc intact même après
						  un redémarrage du routeur.
                          \\
          \hline
          \caption{Types de préfixes réseau}\label{base:net:prefix:types}
      \end{longtable}
  \end{center}
  }
\end{description}

\subsection{Préfixe réseau de type "stable"}
Pour les préfixes réseau "static", les paramètres suivants sont disponibles~:

\begin{description}
  \configlabel{NET\_PREFIX\_x\_STATIC\_IPV6}{NETPREFIXxSTATICIPV6}
  \config{NET\_PREFIX\_x\_STATIC\_IPV4 NET\_PREFIX\_x\_STATIC\_IPV6}{NET\_PREFIX\_x\_STATIC\_IPV4}{NETPREFIXxSTATICIPV4}{
  Adresse(n) du préfixe réseau.

  Vous indiquez dans ces variables les paramètres de l'adresse IPv4 et/ou IPv6
  du préfixe réseau que vous allez utiliser.

  Exemple~:
  }
  \begin{example}
  \begin{verbatim}
    NET {
      PREFIX {
        [] {
          NAME='site'
          TYPE='static'
          STATIC {
            IPV4='10.1.0.0/16'
            IPV6='fdce:1c35:301f::/48'
          }
        }
      }
    }
  \end{verbatim}
  \end{example}

\end{description}

\subsection{Préfixe réseau de type "generated-ula"}
Pour les préfixes réseau "generated-ula", les paramètres suivants sont disponibles~:

\begin{description}
  \config{NET\_PREFIX\_x\_ULA\_DEV}{NET\_PREFIX\_x\_ULA\_DEV}{NETPREFIXxULADEV}{
  Interface Ethernet.

  Vous indiquez dans cette variable l'interface Ethernet, ainsi l'adresse MAC sera
  utilisée pour générer l'ULA.

  Exemple~:
  }
  \begin{example}
  \begin{verbatim}
    NET {
      PREFIX {
        [] {
          NAME='site'
          TYPE='generated-ula'
          ULA {
            DEV='eth0'
          }
        }
      }
    }
  \end{verbatim}
  \end{example}

\end{description}
