% Do not remove the next line
% Synchronized to r29817

\section{Configuration du clavier}

\begin{description}
  \config{KEYBOARD\_LOCALE}{KEYBOARD\_LOCALE}{KEYBOARDLOCALE}

  Configuration par défaut~: \var{KEYBOARD\_LOCALE='auto'}

  Lorsque l'on travail directement sur le routeur fli4l avec un clavier
  de son pays c'est une aide non négligable. Avec \var{KEYBOARD\_LOCALE='auto'}
  le clavier est réglé par rapport à la variable \var{LOCALE} et correspond au
  pays. Si aucun paramètre \var{''} est indiqué à l'installation du routeur fli4l,
  le clavier standard présent dans le Kernel est alors utilisé. On peut
  aussi indiquer directement le nom du pilote dans keyboard\_locale, par ex. on
  indique \var{'fr-latin1'}, au démarrage du Buildprocess (ou processus de
  construction), il examine le répertoire opt/etc s'il trouve le fichier
  fr-latin1.map il charge le fichier-.map pour le code clavier demandé.

  \config{OPT\_MAKEKBL}{OPT\_MAKEKBL}{OPTMAKEKBL}

  Configuration par défaut~: \var{OPT\_MAKEKBL='no'}

  Si vous voulez créer un fichier pour un code clavier spécifique, procéder
  comme indiqué si dessous~:

\begin{itemize}
  \item \var{OPT\_MAKEKBL} mettrez ici 'yes'.
  \item On appel le programme 'makekbl.sh'. Vous devez utilisez de préférence
    une connexion ssh, car le changements de disposition du clavier et qui peut
    être gênant.
  \item Exécuter les instructions.
  \item Le nouveau fichier $<$locale$>$.map est dans le répertoire /tmp.
  \item La création du fichier avec le routeur est maintenant achevée.
  \item Copier maintenant le nouveau code clavier généré dans votre fli4l= dans
    le répertoire opt/etc/$<$locale$>$.map. Vous pouvez utiliser le nouveau code
    clavier créé \var{KEYBOARD\_LOCALE}='$<$locale$>$' dans le prochaine processus
    de construction.
  \item N'oubliez pas de remettre la variable \var{OPT\_MAKEKBL} sur 'no'.
\end{itemize}
\end{description}