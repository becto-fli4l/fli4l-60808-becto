% Do not remove the next line
% Synchronized to r39620

\section{Réglage personnel dans opt/etc/inittab}

  On peut lancés au démarrage du système des programmes supplémentaires,
  ou ajouter des commandes supplémentaires à partir de la console ou changer
  les commandes standard dans le fichier de configuration inittab. Voici
  une description~:

  \begin{example}
  \begin{verbatim}
    device:runlevel:action:command
  \end{verbatim}
  \end{example}

  \emph{device} est le périphérique, sur lequel le programme doit faire
  ses Entrées/Sorties. Pour les terminaux normaux tty1 tty4 ou pour les terminaux
  serie ttyS0 ttySn avec $n <$ le numéro du ports serie.

  \emph{action} décrit l'action à exécuter comme par exemple \emph{askfirst}
  ou \emph{respawn}. askfirst fonctionne comme respawn à la différence prêt
  qu'il demande à l'utilisateur d'appuyer sur une touche  avant l'exécution
  d'un programme. respawn permet d'exécuter automatiquement un programme à la fin
  de l'initialisation.

  \emph{command} est le programme qui doit être exécuté. On doit spécifier le
  chemin d'accès complet.

  Voici la documentation de Busybox \altlink{http://www.busybox.net} le site contient
  une description exacte du format inittab.

  Cela pourrait ressembler à ce qui suit~:

  \begin{example}
  \begin{verbatim}
::sysinit:/etc/rc
::respawn:cttyhack /usr/local/bin/mini-login
::ctrlaltdel:/sbin/reboot
::shutdown:/etc/rc0
::restart:/sbin/init
  \end{verbatim}
  \end{example}

  On pourrait par exemple rajouter ceux-ci

  \begin{example}
  \begin{verbatim}
tty2::askfirst:cttyhack /usr/local/bin/mini-login
  \end{verbatim}
  \end{example}

  Pour obtenir un deuxième login sur le terminal numéro deux. Il suffit
  simplement de rechercher le fichier opt/etc/inittab puis de copier la
  $<$ligne de config$>$ ci-dessus dans le fichier/etc/inittab avec un
  éditeur de texte.

