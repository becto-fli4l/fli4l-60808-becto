% Do not remove the next line
% Synchronized to r38340

\section{Fichier log pour la séquence de Boot et du chargement des modules}

    fli4l écrit l'ensemble du processus de boot (ou démarrage) dans le fichier
    (\emph{/var/tmp/boot.log}). Ce fichier, peut être vu à la fin du processus de
    boot sur la console ou sur l'interface-Web dans menu correspondant.

    Il est parfois utile en cas de problème, de générer des traces détaillées de
    la séquence de boot, pour ensuite examiner le processus de boot plus en
    détail. On utilise pour cela la variable \var{DEBUG\_STARTUP}.
    Dans certaines situations les développeurs on besoin d'autre paramètres pour
    les aider à résoudre des erreurs, ces paramètres supplémentaires sont
    documentés dans cette section.

\begin{description}
  \config{DEBUG\_STARTUP}{DEBUG\_STARTUP}{DEBUGSTARTUP}

  Configuration par défaut~: \var{DEBUG\_STARTUP='no'}

   Si la valeur est sur 'yes', chaque commande exécutée est écrite sur l'écran
   de contrôle pendant le boot. Comme un changement dans le fichier syslinux.cfg
   est nécessaire pour l'activation de cette fonctionnalité, c'est aussi valable
   pour la variable \var{SER\_CONSOLE}. Vous pouvez adapter le fichier
   syslinux.cfg manuellement en ajoutant \verb+fli4ldebug=yes+. Toutefois 
   \var{DEBUG\_STARTUP} doit être placé malgré tout sur 'yes'.

  \config{DEBUG\_MODULES}{DEBUG\_MODULES}{DEBUGMODULES}

  Configuration par défaut~: \var{DEBUG\_MODULES='no'}

   Certains modules du Kernel sont chargés automatiquement, sans pouvoir
   les détecter à l'avance. Si vous activez la variable \var{DEBUG\_MODULES='yes'}
   vous pouvez voir entièrement la séquence de chargement de ces modules,
   qu'ils soient chargés par un script ou émis par le Kernel.

  \config{DEBUG\_ENABLE\_CORE}{DEBUG\_ENABLE\_CORE}{DEBUGENABLECORE}

  Configuration par défaut~: \var{DEBUG\_ENABLE\_CORE='no'}

  Si vous activez cette variable, tout accident causé sur le routeur créera un
  soi-disant fichier-"core", C'est une image mémoire du processus qui est
  enregistrée juste avant le crash. Ce fichier se trouve sur le routeur dans
  \texttt{/var/log/dumps}. Ce fichier peut ensuite être utilisé pour trouver
  plus facilement le bug du programme. Pour plus de détails, reportez-vous dans
  la section \jump{sec:debugging}{"programme de débogage sur fli4l"} dans la
  documentation du paquetage SRC.

  \config{DEBUG\_MDEV}{DEBUG\_MDEV}{DEBUGMDEV}

  Configuration par défaut~: \var{DEBUG\_MDEV='no'}

  Si la variable \var{DEBUG\_MDEV='yes'} est activée, toutes les actions qui
  sont en rapport avec les Démons-\texttt{mdev}, sur l'ajout ou la suppression
  de n\oe{}ud de périphériques dans \texttt{/dev} ou encore au chargement d’un
  firmware, seront consignées dans le fichier \texttt{/dev/mdev.log}.

  \config{DEBUG\_IPTABLES}{DEBUG\_IPTABLES}{DEBUGIPTABLES}

  Configuration par défaut~: \var{DEBUG\_IPTABLES='no'}

  Si la variable \var{DEBUG\_IPTABLES='yes'}, est activée, tous les
  appels-\texttt{iptables} y compris les valeurs de retour seront consignés
  dans le fichier \texttt{/var/log/iptables.log}.

  \config{DEBUG\_IP}{DEBUG\_IP}{DEBUGIP}

    Configuration par défaut~: \var{DEBUG\_IP='no'}

    Si vous activez la variable \var{DEBUG\_IP='yes'} tous les requêtes vers
    le programme \texttt{/sbin/ip} seront consignés dans le fichier
    \texttt{/var/log/wrapper.log}.
\end{description}
