% Do not remove the next line
% Synchronized to r29848

  \section{Configuration de Imond}

  \begin{description}
    \config{OPT\_IMOND}{OPT\_IMOND}{OPTIMOND}

    Configuration par défaut~: \var{OPT\_\-IMOND='no'}

    {En mettant \var{OPT\_\-IMOND} sur 'yes' vous décidez, d'activer
    le serveur imond. Imond est le centre de surveillance et le contrôle du
    moindres coût des connexions pour le routeur-fli4l. Par conséquent, un
    chapitre supplémentaire lui est consacré \jump{IMONDSCHNITTSTELLE}{Description d'imond}.

      Important~: le dispositif LC-Routing de fli4l peut uniquement
      être utilisé par imond. Les commutations basées sur le temps de
      connexion n'est pas possible, sans utiliser imond~!

      Avec le routage ISDN (RNIS en france) et DSL il est nécessaire
      d'employer la version 1.5 d'imond. Pour l'activer, vous paramétrez
      \var{OPT\_\-IMOND}='yes'.

      Si fli4l est uniquement utilisé comme routeur avec deux cartes réseaux,
      vous devez paramétrer la variable sur \var{OPT\_\-IMOND}='no'.}

    \config{IMOND\_PORT}{IMOND\_PORT}{IMONDPORT}

    {Dans cette variable on indique le port TCP/IP d'écoute pour les
    connexions, La valeur par défaut '5000' il doit être modifié uniquement
    par nécessité.}

    \config{IMOND\_PASS}{IMOND\_PASS}{IMONDPASS}

      Configuration par défaut~: \var{IMOND\_\-PASS=''}

    {Dans cette variable on peut placer un mot de passe spécial utilisateur
    pour imond. Si un client se connecte sur le port 5000, imond demandera
    le mot de passe avant qu'il réponde à n'importe quelle commande.
    Excepté~: Les commandes "quit", "help" et "pass" si \var{IMOND\_\-PASS}
    est vide, aucun mot de passe est demandé.

      Le client peut exécuter les instructions déterminées, comme les commandes
      Dial, Enable, Reboot, la commutation de la route par défaut peut déjà
      exécuté ou pour entrer les mots de passe administrateur nécessaire,
      la commuter la route par defaut, vous pouvez paramétrer les variables

      \begin{itemize}
      \item \jump{IMONDENABLE}{IMOND\_ENABLE},
      \item \jump{IMONDDIAL}{IMOND\_DIAL},
      \item \jump{IMONDROUTE}{IMOND\_ROUTE} et
      \item \jump{IMONDREBOOT}{IMOND\_REBOOT}
      \end{itemize}

      voir plus bas.
    }

    \config{IMOND\_ADMIN\_PASS}{IMOND\_ADMIN\_PASS}{IMONDADMINPASS}

    Configuration par défaut~: \var{IMOND\_ADMIN\_\-PASS=''}

    {En utilisant le mot de passe admin, le client reçoit toutes les
    droits et peut donc utiliser toutes les commandes du serveur imond,
    plus exactement les variables \var{IMOND\_\-ENABLE}, \var{IMOND\_\-DIAL}
    etc indépendamment. Si vous laissez \var{IMOND\_\-ADMIN\_\-PASS} vide il
    n'est pas possible d'utiliser ces commandes, le mot de passe utilisateur
    est nécessaire pour avoir tous des droits~!
    }

    \config{IMOND\_LED}{IMOND\_LED}{IMONDLED}

    {Maintenant imond peut indiquer le statut Online/Offline par une LED.
    Elle est branché sur un port COM de la manière suivante~:

      Connecteur 25-broches~:

\begin{example}
\begin{verbatim}
        20 DTR  -------- 1kOhm -----led->| ---------- 7 GND
\end{verbatim}
\end{example}

      Connecteur 9-broches~:

\begin{example}
\begin{verbatim}
         4 DTR  -------- 1kOhm -----led->| ---------- 5 GND
\end{verbatim}
\end{example}

      S'il y a une connexion établie avec ISDN (RNIS) ou la DSL, la LED
      est allumée, autrement elle est éteinte. Si vous voulez inverser
      l'éclairage de la LED, vous devez inverser la polarité de celle-ci.
      Dans le cas où la luminosité de la LED serait trop faible, vous pouvez
      réduire la résistance à 470 ohms.

      Il est possible de relier deux LED de couleurs différente. La deuxième
      LED peut être reliée en utilisant une deuxième résistance entre DTR
      et GND, et inverser la polarité par rapport à la première. Une des deux
      LED sera allumée selon le statut. Vous pourrez employer une LED-DUO
      (deux couleurs, trois connecteurs).

      Actuellement, le connecteur RTS de l'interface série comporte un
      comme DTR. Vous pourriez relier une autre LED a ce connecteur (RTS),
      pour le statut d'affichage Online/Offline. Cela pourrait changer dans
      les versions futures de fli4l.

      Dans la variable \var{IMOND\_\-LED} un port COM doit être indiqué,
      c-à-d. 'com1', 'com2', 'com3' ou 'com4'. Si aucune LED n'est branchée,
      vous devriez laisser la variable vide.}

    \config{IMOND\_BEEP}{IMOND\_BEEP}{IMONDBEEP}

    {Si la variable est placée sur \var{IMOND\_\-BEEP}='yes', imond produit deux
    signaux sonores par le PC = haut-parleur, selon l'état de Offline à Online 
    et vice versa. Dans le premier cas un ton grave, dans le second cas un ton
    aigu. Lors de la modification de l'état offline le son sera d'abord le plus
    élevé, il émettra ensuite un son plus faible.}

    \config{IMOND\_LOG}{IMOND\_LOG}{IMONDLOG}

      Configuration par défaut~: \var{IMOND\_\-LOG='no'}

    {Si la variable est placée sur \var{IMOND\_\-LOG}='yes', les connexions
    sont enregistrées dans \verb+/var/log/imond.log+. Ce fichier peut être recopié
    pour être analyse sur un ordinateur du réseau local (LAN), en utilisant par
    exemple le programme SCP. Si vous voulez utiliser SCP vous devrez installer
    le paquetage sshd et le configurer, de sorte que le programme SCP soit
    également disponible.

      La description des formats du fichier journal est décrit dans le tableau \ref{tab:imondlog}.

      \begin{table}[htbp]
        \small
        \centering
        \caption{Format du fichier Log d'Imond}\label{tab:imondlog}
        \begin{tabular}{lp{12cm}}
          \hline
          Les entrées & Signification \\
          \hline
          Circuit & Nom du Circuit dans lequel les notations d'événement
          sont produites \\
          Heure connexion & Date et heure où la connexion a été établie \\
          Heure déconnexion & Date et heure où la déconnexion a été terminée \\
          Temps en ligne & Durée de connexion \\
          Temps restant & Temps de connexion restante offre du FAI/mois
          (dépend du réglage des unités) \\
          Coût (prix) & Prix du temps de connexion du FAI \\
          Bande passante & Bande passante utilisée, les valeurs sont
          représentée séparément une pour les entrées l'autre pour les sorties,
          et seront additionnées dans la bande passante =\newline
          \emph{4Gio~$*<$premier~chiffre$>+<$deuxième~chiffre$>$} \\
          Device & Périphérique utilisé pour la communication \\
          Relevé d'unités de compte & Unité consultées par le fournisseur
          d'accés pour le relevé de compte (données à configurer) \\
          Prix de l'unité & Prix de l'unité par connexion (des données à configurer) \\
          \hline
        \end{tabular}
      \end{table}

      Les frais sont donnés en Euro. Il est important pour cette fonction
      que la variable \jump{ISDNCIRCxTIMES}{\var{ISDN\_CIRC\_x\_TIMES}} du circuit
      soient correctement réglée.
    }

    \config{IMOND\_LOGDIR}{IMOND\_LOGDIR}{IMONDLOGDIR}

    {Si vous activez l'enregistrement les connexions, vous pouvez avec la variable
    \var{IMOND\_\-LOGDIR} enregistrer un répertoire alternatif au lieu d'utiliser
    /var/log, vous pouvez indiquez par exemple '/boot'. Le fichier journal
    imond.log sera ensuite créé sur le média de démarrage. De plus le média doit
    être en \flqq{}lecture/écriture\frqq{}. Par défaut la variable est sur 'auto'
    l'emplacement est déterminé automatiquement. Selon la configuration du \var{FLI4L\_UUID}
    le chemin déterminé se trouvera alors sous /boot/persistent/base ou un autre
    chemin. Si le chemin /boot n'est pas en lecture/écriture, le répertoire
    persistent avec \var{FLI4L\_UUID} ne sera pas actif et le fichier log sera
    enregistré dans le répertoire /var/run.}

    \configlabel{IMOND\_DIAL}{IMONDDIAL}
    \configlabel{IMOND\_ROUTE}{IMONDROUTE}
    \configlabel{IMOND\_REBOOT}{IMONDREBOOT}
    \config{IMOND\_ENABLE  IMOND\_DIAL  IMOND\_ROUTE  IMOND\_REBOOT}{IMOND\_ENABLE}{IMONDENABLE}

    {Dans ces variables, vous pouvez activer plusieurs commandes
     qui seront utilisées par le clients imonc et envoyées au serveur imond,
     elles seront exécutées en mode utilisateur.

      Avec ces commandes vous pouvez, par l'intermédiaire du serveur imond
      allumer/éteindre l'interface ISDN, composer/raccrocher, ajouter
      une nouvelle route par défaut et redémarrer le routeur.

      Configuration par défaut~:

\begin{example}
\begin{verbatim}
        IMOND_ENABLE='yes'
        IMOND_DIAL='yes'
        IMOND_ROUTE='yes'
        IMOND_REBOOT='yes'
\end{verbatim}
\end{example}

      Des dispositifs additionnels pour l'interface imond sont décrits
      \jump{IMONDSCHNITTSTELLE}{dans ce chapitre} pour le client et le serveur.}
  \end{description}

