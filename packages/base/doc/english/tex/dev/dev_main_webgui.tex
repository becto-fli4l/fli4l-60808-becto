% Synchronized to r36388

\section{CGI-Creation for Package \emph{httpd}}

\subsection{General information about the web server}

The web server used in fli4l is \texttt{mini\_httpd} by ACME
Labs. The sources can be found at
\altlink{http://www.acme.com/software/mini_httpd/}.
However, a few changes in the current version were made for fli4l.
The modifications are located in the \emph{src} package in the directory
\texttt{src/fbr/buildroot/package/mini\_httpd}.

\subsection{Script Names}

The script names should be self-explanatory in order to be
easy to distinguish from other scripts and even similar names
have to be avoided to differ from other OPTs.

To make scripts executable and convert DOS line breaks to UNIX ones
a corresponding entry has to be created in \texttt{opt/<PACKAGE>.txt}, see
Table~\jump{table:options}{\ref{table:options}}.

\subsection{Menu Entries}

To create an entry in the menu you have to enter it in the file
\texttt{/etc/httpd/menu}. This mechanism enables OPTs to change the menu during
runtime. This should only be done using the script \texttt{/etc/httpd/menu}
because this will check for valid file formatting. New menu items
are inserted as follows:

\begin{example}
\begin{verbatim}
    httpd-menu.sh add [-p <priority>] <link> <name> [section] [realm]
\end{verbatim}
\end{example}

Thus, an entry with the name \texttt{<name>} is inserted to the \texttt{[section]}.
If \texttt{[section]} is omitted, it will be inserted in the section ``OPT-Packages''
as default. \texttt{<link>} specifies the target of the new link. \texttt{<priority>}
specifies the priority of a menu item in its section. If not set, the default priority
used is 500. The priority should be a three digit number. The lower the
priority, the higher the link is placed in the section. If an entry should be
placed as far down as possible the priority to choose is e.g. 900. Entries with
the same priority are sorted by the target of the link.
In \texttt{[realm]} the range is specified for which a logged-in user must have
\var{view} rights so the item is displayed for him. If \texttt{[realm]} is not
specified, the menu item is always displayed.
For this, see also the section \jump{sec:rights}{``User access rights''}.

Example:

\begin{example}
\begin{verbatim}
    httpd-menu.sh add "newfile.cgi" "Click here" "Tools" "tools"
\end{verbatim}
\end{example}

This example creates a link named ``Click here'' with the target ``newfile.cgi''
in the section ``Tools'' which will be created if not present.

The script may also delete entries from a menu:

\begin{example}
\begin{verbatim}
    httpd-menu.sh rem <link>
\end{verbatim}
\end{example}

By executing this the entry containing the link \texttt{<link>} will be
deleted.

\wichtig{If several entries have the same link target file they
will all be removed from the menu.}

Since sections can have priorities they can also be created manually. If a section
was created automatically when adding a menu entry it defaults to priority 500.
The syntax for creating sections is as follows:

\begin{example}
\begin{verbatim}
    httpd-menu.sh addsec <priority> <name>
\end{verbatim}
\end{example}

\texttt{<priority>} should only be a three digit number.

In order to create meaningful priorities it is worthwhile to
have a look at the file\\ \texttt{/etc/httpd/menu} of fli4l during runtime,
priorities are placed in the second column.\\

A short description of the file format of the file menu follows for completeness.
Those satisfied with the function of \texttt{httpd-menu.sh} may skip this
section. The file \texttt{/etc/httpd/menu} is divided into four columns.
The first column is a letter identifying the line as a heading
or a menu entry. The second column is the sort priority. The third column
contains the target of the link for entries and for headlines a hyphen,
as this field has no meaning for headings. The rest of the line is the text
that will appear in the menu.

Headings use the letter ``t'', a new menu section will be started then. Normal
menu items use the letters ``e''.
An example:

\begin{example}
\begin{verbatim}
    t 300 - My beautiful OPT
    e 200 myopt1.cgi Do something beautiful
    e 500 myopt1.cgi?more=yes Do something even more beautiful
\end{verbatim}
\end{example}

When editing this file you have to be aware that the script \texttt{httpd-menu.sh} 
always stores the file sorted. The individual
sections are sorted and the entries in this section are sorted too.
The sorting algorithm can be stolen from \texttt{httpd-menu.sh}, however,
it would be better to expand the script itself with possible new
functions, so that all menu-editing takes place at a central location.

\subsection{Construction of a CGI script}

\subsubsection{The headers}
All web server scripts are simple shell scripts (interpreter as e.g.
Perl, PHP, etc. are much too big in filesize for fli4l). You should start with the
mandatory script header (reference to the interpreter, name, what does the script,
author, license).

\subsubsection{Helper Script \texttt{cgi-helper}}
After the header you should include the helper script \texttt{cgi-helper}
with the following call:

\begin{example}
\begin{verbatim}
    . /srv/www/include/cgi-helper
\end{verbatim}
\end{example}

A space between the dot and the slash is important!

This script provides several helper functions that should greatly simplify the
creation of CGIs for fli4l. With the integration some standard tasks are performed,
such as the parsing of variables that were passed via forms or via the URL or loading
of language and CSS files.

Here is a small function overview:

\begin{table}[htbp]
  \centering
  \caption{Functions of the \texttt{cgi-helper} script}
  \label{tab:dev:cgi-helper}
  \begin{small}
    \begin{tabular}{|l|p{0.8\textwidth}|}
      \hline
      Name                         & Function         \\
      \hline
      \texttt{check\_rights}      & Check for user access rights \\
      \texttt{http\_header}       & Creation of a standard HTTP header or a special header, e.g. for file download\\
      \texttt{show\_html\_header} & Creation of a complete page header (inc. HTTP header, headline and menu)\\
      \texttt{show\_html\_footer} & Creation of a footer for the HTML page\\
      \texttt{show\_tab\_header}  & Creation of a content window with tabs\\
      \texttt{show\_tab\_footer}  & Creation of a footer for the content window\\
      \texttt{show\_error}        & Creation of a box for error messages (background color: red)\\
      \texttt{show\_warn}         & Creation of a box for warning messages (background color: yellow)\\
      \texttt{show\_info}         & Creation of a box for information/ success messages (background color: green)\\
      \hline
    \end{tabular}
  \end{small}
\end{table}

\subsubsection{Contents of a CGI script}

To ensure consistency in design and especially the compatibility
with future versions of fli4l it is highly recommended to use the
functions of the \texttt{cgi-helper} script even if theoretically everything
in a CGI could be generated from scratch.

A simple CGI script might look like this:

\begin{example}
\begin{verbatim}
    #!/bin/sh
    # --------------------
    # Header (c) Author Date
    # --------------------
    # get main helper functions
    . /srv/www/include/cgi-helper

    show_html_header "My first CGI"
    echo '   <h2>Welcome</h2>'
    echo '   <h3>This is a CGI script example</h3>'
    show_html_footer

\end{verbatim}
\end{example}

\subsubsection{The Function \texttt{show\_html\_header}}

The \texttt{show\_html\_header} function expects a string as a parameter.
This string represents the title of the generated page. It automatically
generates the menu and includes associated CSS and language files
as long as they can be found in the directories \texttt{/srv/www/css} resp.\
\texttt{/srv/www/lang} and have the same name (but of course a different
extension) as the script. An example:

\begin{example}
\begin{verbatim}
    /srv/www/admin/OpenVPN.cgi
    /srv/www/css/OpenVPN.css
    /srv/www/lang/OpenVPN.de
\end{verbatim}
\end{example}

Both the use of language files and CSS files is optional. The files \texttt{css/main.css} and
\texttt{lang/main.<lang>} (where \texttt{<lang>} refers to the chosen language) are always included.

Additional parameters can be passed to the function \texttt{show\_html\_header}. A call with
all possible parameters might look like this:

\begin{example}
\begin{verbatim}
    show_html_header "Title" "refresh=$time;url=$url;cssfile=$cssfile;showmenu=no"
\end{verbatim}
\end{example}

Any additional parameters must, as shown in the example, be enclosed with quotation marks
and separated by a semicolon. The syntax will \emph{not} be checked! So it is necessary to pay
close attention to the exact parameter syntax.

Here is a brief overview of the function of the parameters:
\begin{itemize}
 \item \texttt{refresh=}\emph{time}: Time in seconds in which the page should be reloaded by the browser.
 \item \texttt{url=}\emph{url}: The URL which is reloaded on a refresh.
 \item \texttt{cssfile=}\emph{cssfile}: Name of a CSS file if it differs from the name of the CGI.
 \item \texttt{showmenu=no}: By using this the display of the menu and the header can be suppressed.
\end{itemize}

Other Content Guidelines:

\begin{itemize}
 \item Don't write novels, use short desriptions :-)
 \item Use clean HTML (SelfHTML\footnote{see
    \altlink{http://de.selfhtml.org/}} is a good starting point)
 \item Omit the bells and whistles (JavaScript is OK, if it
does not interfere and support the user, everything also has to
work without JavaScript)
\end{itemize}

\subsubsection{The Function \texttt{show\_html\_footer}}

The function \texttt{show\_html\_footer} closes the block from the CGI script which was
openend by the function \texttt{show\_html\_header}.

\subsubsection{The Function \texttt{show\_tab\_header}}

For good looking content of your generated webpage generated by the CGI
you may use the \texttt{cgi-helper} function \texttt{show\_tab\_header}.
It creates clickable ``Tabs'' in order to present your page divided into
multiple logically separated areas.

Parameters are always passed in pairs to the show\_tab\_header function.
The first value reflects the title of a tab, the second reflects the link.
If the string ``no'' is passed as a link only the title will be created
and the tab is not clickable (and blue).

In the following example a ``window'' with the title ``A great window''
is generated. In the window is ``foo bar'':

\begin{example}
\begin{verbatim}
    show_tab_header "A great window" "no"
    echo "foo"
    echo "bar"
    show_tab_footer
\end{verbatim}
\end{example}

In this example, two clickable selection tabs are generated that pass
the variable \var{action} to the script, each with a different value.

\begin{example}
\begin{verbatim}
    show_tab_header "1st selection tab" "$myname?action=dothis" \
                    "2nd selection tab" "$myname?action=dothat"
    echo "foo"
    echo "bar"
    show_tab_footer
\end{verbatim}
\end{example}

Now the script can change the content of the variable \var{FORM\_action} (see variable
evaluation below) and provide different content depending on the selection.
For the clicked tab to appear selected and not clickable anymore, a ``no''
would have to be passed to the function instead of the link. But there is an
easier way, if you hold to the convention in the following example:

\begin{example}
\begin{verbatim}
    _opt_dothis="1st selection tab"
    _opt_dothat="2nd selection tab"
    show_tab_header "$_opt_dothis" "$myname?action=opt_dothis" \
                    "$_opt_dothat" "$myname?action=opt_dothat"
    case $FORM_action in
        opt_dothis) echo "foo" ;;
        opt_dothat) echo "bar" ;;
    esac
    show_tab_footer
\end{verbatim}
\end{example}

Hence, if a variable whose name equals the content of the variable \var{action} with a
leading underscore (\texttt{\_}) is passed as the title then the tab will be displayed selected.

\subsubsection{The Function \texttt{show\_tab\_footer}}

The function \texttt{show\_tab\_footer} closes the block in the CGI script that was
opened by the function \texttt{show\_tab\_header}.

\subsubsection{Multi-Language Capabilities}

The helper script \texttt{cgi-helper} furthermore contains functions to create multi-langual CGI scripts.
You only have to use variables with a leading underscore (\texttt{\_}) for all text output. This variables have to
be defined in the respective language files.

Example:

Let \texttt{lang/opt.de} contain:

\begin{example}
\begin{verbatim}
    _opt_dothis="Eine Ausgabe"
\end{verbatim}
\end{example}

Let \texttt{lang/opt.en} contain:

\begin{example}
\begin{verbatim}
    _opt_dothis="An Output"
\end{verbatim}
\end{example}

Let \texttt{admin/opt.cgi} contain:

\begin{example}
\begin{verbatim}
    ...
    echo $_opt_dothis
    ...
\end{verbatim}
\end{example}


\subsubsection{Form Evaluation}

To process forms you have to know a few more things. Regardless of using the
form's \var{GET} or \var{POST} methods, after including the \texttt{cgi-helper} script (which in turn calls
the utility proccgi) the parameters can be accessed by variables named \var{FORM\_<Parameter>}. 
If i.e. the form field had the name ``input'' the CGI script can access
its content in the shell variable \var{\$FORM\_input}.

Further informations on the program \texttt{proccgi} can be found under
\altlink{http://www.fpx.de/fp/Software/ProcCGI.html}.

\marklabel{sec:rights}{
    \subsubsection{User access rights: The Function \texttt{check\_rights}}
}

At the beginning of a CGI scripts the \texttt{check\_rights} function
has to be called in order to check whether a user has sufficient rights to use
the script. Do this like here:

\begin{example}
\begin{verbatim}
    check_rights <Section> <Action>
\end{verbatim}
\end{example}

The CGI script will only be executed if the user, who is logged in at the moment
\begin{itemize}
\item has all rights (\verb+HTTPD_RIGHTS_x='all'+), or
\item has all rights for the current area
    (\verb+HTTPD_RIGHTS_x='<Bereich>:all'+), or
\item has the right to execute the function in the current area\\
    (\verb+HTTPD_RIGHTS_x='<Bereich>:<Aktion>'+).
\end{itemize}

\subsubsection{The Function \texttt{show\_error}}

This funtion displays an error message in a red box. It expects two
parameters: a title and a message. Example:

\begin{example}
\begin{verbatim}
    show_error "Error: No key" "No key was specified!"
\end{verbatim}
\end{example}

\subsubsection{The Function \texttt{show\_warn}}

This funtion displays a warning message in a yellow box. It expects two
parameters: a title and a message. Example:

\begin{example}
\begin{verbatim}
    show_info "Warning" "No connection at the moment!"
\end{verbatim}
\end{example}

\subsubsection{The Function \texttt{show\_info}}

This funtion displays an information or success message in a green box. It expects two
parameters: a title and a message. Example:

\begin{example}
\begin{verbatim}
    show_info "Info" "Action successfully executed!"
\end{verbatim}
\end{example}

\subsubsection{The Helper Script \texttt{cgi-helper-ip4}}

Right after \texttt{cgi-helper} the helper script \texttt{cgi-helper-ip4}
may be included by writing the following line:

\begin{example}
\begin{verbatim}
. /srv/www/include/cgi-helper-ip4
\end{verbatim}
\end{example}

A space between the dot and the slash is important!

The script provides helper funtions for checking IPv4 addresses.

\subsubsection{The Function \texttt{ip4\_isvalidaddr}}

This function checks if a valid IPv4 address was passed. Example:

\begin{example}
\begin{verbatim}
    if ip4_isvalidaddr ${FORM_inputip}
    then
        ...
    fi
\end{verbatim}
\end{example}

\subsubsection{The Function \texttt{ipv4\_normalize}}

This function removes leading zeros from the passed IPv4 address. Example:

\begin{example}
\begin{verbatim}
    ip4_normalize ${FORM_inputip}
    IP=$res
    if [ -n "$IP" ]
    then
        ...
    fi
\end{verbatim}
\end{example}
\subsubsection{The Function \texttt{ipv4\_isindhcprange}}

This function checks whether the passed IPv4 address is ranged between the
passed start and end addresses. Example:

\begin{example}
\begin{verbatim}
    if ip4_isindhcprange $FORM_inputip $ip_start $ip_end
    then
        ...
    fi
\end{verbatim}
\end{example}

\subsection{Miscellaneous}

This and that (yes, also important!):

\begin{itemize}
 \item \texttt{mini\_httpd} does not protect subdirectories with a password. Each directory
       must contain a \texttt{.htaccess} file or a link to another \texttt{.htaccess} file.
 \item KISS - Keep it simple, stupid!
 \item This information may change at any time without prior notice!
\end{itemize}

\subsection{Debugging}

To ease debugging of a CGI script you may activate the debugging mode
by sourcing the \texttt{cgi-helper} script. Set the variable \var{set\_debug}
to ``yes'' in order to do so. This will create a file \texttt{debug.log}
which may be loaded down with the URL \texttt{http://<fli4l-Host>/admin/debug.log}.
It contains all calls of the CGI script. The variable \texttt{set\_debug} is not a
global one, it has to be set anew for each CGI in question. Example:

\begin{example}
\begin{verbatim}
    set_debug="yes"
    . /srv/www/include/cgi-helper
\end{verbatim}
\end{example}

Furthermore, cURL\footnote{see \altlink{http://de.wikipedia.org/wiki/CURL}} is
ideal for troubleshooting, especially if the HTTP headers
are not assembled correctly or the browser displays only blank pages.
Also, the caching behavior of modern Web browser is obstructive when troubleshooting.

Example: Get a dump of the HTTP-Header with ("`\emph{d}ump"',
\texttt{-D}) and the normal output of the CGI \texttt{admin/my.cgi}.
The ``\emph{u}ser'' (\texttt{-u}) name here shall be ``admin''.

\begin{example}
\begin{verbatim}
    curl -D - http://fli4l/admin/my.cgi -u admin
\end{verbatim}
\end{example}
