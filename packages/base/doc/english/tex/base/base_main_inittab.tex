% Synchronized to r39620

\section{Usage of a customized /etc/inittab}
  It is possible to let the ``init'' process start additional programs
  on additional consoles or to change the default commands. An inittab entry
  is structured as follows:

  \begin{example}
  \begin{verbatim}
    device:runlevel:action:command
  \end{verbatim}
  \end{example}

  The \emph{device} denotes the terminal used for program input/output.
  Possible devices are terminals tty1-tty4 or serial terminals
  ttyS0-ttySn with $n <$ the number of available serial ports.

  The possible \emph{action}s are typically \emph{askfirst} or \emph{respawn}.
  Using askfirst lets ``init'' wait for a keypress before running that
  command. The respawn action causes the command to be automatically restarted
  whenever it terminates.

  \emph{command} specifies the program to execute. You have to use a fully
  qualified path.

  The documentation of the Busybox toolkit at \altlink{http://www.busybox.net}
  contains a detailed description of the inittab format.

  The normal inittab file is as follows:

  \begin{example}
  \begin{verbatim}
    ::sysinit:/etc/rc
    ::respawn:cttyhack /usr/local/bin/mini-login
    ::ctrlaltdel:/sbin/reboot
    ::shutdown:/etc/rc0
    ::restart:/sbin/init
  \end{verbatim}
  \end{example}

  You could e.g. extend it by the entry

  \begin{example}
  \begin{verbatim}
    tty2::askfirst:cttyhack /usr/local/bin/mini-login
  \end{verbatim}
  \end{example}

  in order to get a second login process on the second terminal. To achieve
  this, simply copy the file opt/etc/inittab to $<$config directory$>$/etc/inittab
  and edit the copy accordingly.
