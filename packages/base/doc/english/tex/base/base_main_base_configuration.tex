% Synchronized to r49626

\chapter{Base configuration}

Since fli4l 2.0 the distribution is designed to be modular and consists of
multiple packages which have to be downloaded separately. The package
\texttt{fli4l-\version.tar.gz} contains only the base software for a pure ethernet
router. For DSL, ISDN, and other software you will have to download
further packages and extract them into the directory \texttt{fli4l-\version/}.
In order to allow free choice of the fli4l's linux system kernel, the
kernel has been removed from the base package and was put into an own package.
This implies that at least both the base and kernel packages are required.
Table \ref{tab:zusatzpakete} gives an overview of additional software
packages.

\begin{table}[ht!]
 \caption{Overview of additional packages}\marklabel{tab:zusatzpakete}{}
  \begin{center}
    \begin{tabular}{ll}
      \textbf{Archive to download}    &    \textbf{Package} \\
      \hline
      \texttt{fli4l-\version}         &    BASE, required!\\
      \verb*zkernel_4_19z             &    Linux kernel, required!\\
      \texttt{fli4l-\version-doc}     &    Complete documentation\\
      \verb*zadvanced_networkingz     &    Extended network configuration\\
      \verb*zcertz                    &    Certificate management\\
      \verb*zchronyz                  &    Time server/client\\
      \verb*zdhcp_clientz             &    Miscellaneous DHCP clients\\
      \verb*zdns_dhcpz                &    DNS und DHCP servers\\
      \verb*zdslz                     &    DSL router (PPPoE, PPTP)\\
      \verb*zdyndnsz                  &    Support for DYNDNS services\\
      \verb*zeasycronz                &    Time planning service\\
      \verb*zhdz                      &    Needed for hard disk installation\\
      \verb*zhttpdz                   &    Minimalistic Web server\\
      \verb*zhwsuppz                  &    Hardware support\\
      \verb*zimonc_windowsz           &    Windows imonc client\\
      \verb*zimonc_unixz              &    GTK/Unix imonc client\\
      \verb*zipv6z                    &    Internet Protocol Version 6\\
      \verb*zisdnz                    &    ISDN router\\
      \verb*zopenvpnz                 &    VPN support\\
      \verb*zpcmciaz                  &    Support for PCMCIA (PC cards)\\
      \verb*zpppz                     &    PPP connection over serial port\\
      \verb*zproxyz                   &    Proxy server\\
      \verb*zqosz                     &    Quality of Service\\
      \verb*zsshdz                    &    SSH server\\
      \verb*ztoolsz                   &    Miscellaneous Linux tools\\
      \verb*zumtsz                    &    Connection to the Internet via UMTS\\
      \verb*zusbz                     &    USB support\\
      \verb*zwlanz                    &    Support for WLAN cards
    \end{tabular}
  \end{center}
\end{table}

The files necessary for configuring the fli4l router are placed in the
directory \texttt{config/}. They are described later in this chapter.

These files can be edited with a \emph{simple} text editor, or alternatively
with an editor specially designed for fli4l. Miscellaneous editors can be found
under

\par

\altlink{http://www.fli4l.de/en/download/additional-packages/addons/}.

If you need to adapt/extend the fli4l system in addition to the possible
settings described below, you will need a working Linux system in order to
adjust the RootFS. The file \verb+src/README+ will provide you with more
information.

\newpage
