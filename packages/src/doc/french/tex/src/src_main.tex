% Do not remove the next line
% Synchronized to r51513
marklabel{buildroot}{
\section{SRC - Le Buildroot fli4l}
}

Ce chapitre est intéressant, uniquement pour les développeurs qui veulent
compiler des programmes binaires ou le Kernel Linux pour fli4l. Si vous
utilisez fli4l seulement comme routeur et si vous ne voulez pas donner au
routeur fli4l des opt-packages (ou paquetage optionnel) ou créer vos propre
programmes binaires, vous pouvez ignorer complètement ce chapitre.

En général, un système Linux est nécessaire pour la création de progiciels pour
fli4l. La création sur d'autres systèmes d'exploitation (Microsoft Windows,
OS X, FreeBSD etc.) ne sera \emph{pas} support.

Les exigences du système Linux pour le développement de fli4l sont les suivantes~:
\begin{itemize}
\item GNU gcc et g++ dans la version 2.95 ou suppérieur
\item GNU gcc-multilib (en fonction de votre système)
\item GNU binutils (y compris Binder ainsi que d'autres programmes sont nécessaires)
\item GNU make dans la Version 3.81 ou suppérieur
\item GNU bash
\item libncurses5-dev pour \texttt{fbr-make *-menuconfig} (en fonction de votre système)
\item Les programmes sed, awk, which, flex, bison et patch
\item Les programmes makeinfo (paquet texinfo) et msgfmt (paquet gettext)
\item Les programmes tar, cpio, gzip, bzip2 et unzip
\item Les programmes wget, rsync, svn et git
\item Les programmes perl et python
\end{itemize}

Dans la documentation ci-dessous les caractères en \textbf{gras} indiquent
une commande, le caractère \enter représente la touche Entrée de votre
clavier et ferme la commande.

\subsection{Vue d'ensemble des répertoires sources}

Dans le répertoire \texttt{src}, vous trouverez les sous-répertoires suivants~:

\begin{longtable}{|l|p{10cm}|}
    \hline
    \multicolumn{1}{|l}{\textbf{Répertoire}} &
    \multicolumn{1}{|l|}{\textbf{Contenu}} \\
    \hline
    \endhead
    \hline
    \endfoot
    \endlastfoot
\texttt{fbr} &
    Ce répertoire contient le système de construction personnalisable qui
    est basé sur le buildroot uClibc (avec actuellement la version 0.9.33.2).
    Le FBR signifie "fli4l Buildroot". Il est ainsi possible de recompiler tous
    les programmes fli4l (le Kernel, les bibliothèques et les applications). \\
\hline
\texttt{fli4l} &
    Ce répertoire contient les sources spécifiques à fli4l, triées par paquets.
    Toutes les sources incluses dans les sous-répertoires ont été écrites
    spécifiquement pour l'utilisation de fli4l ou du moins fortement
    personnalisées. \\
\hline
\texttt{cross} &
    Ce répertoire contient les scripts nécessaires pour la compilation croisée,
    ils permettent de créer et de compiler le mkfli4l avec les différentes
    plates-formes. \\
\hline
\end{longtable}

\subsection{Compiler un programme pour fli4l}

Dans le répertoire "fbr" vous avez le script \texttt{fbr-make} qui contrôle
la compilation de tous les programmes du paquet base du routeur fli4l. Ce script
se charge de télécharger et de compiler tous les fichiers binaires requis pour
fli4l. En règle générale, les fichiers script définient sont placés dans le
répertoire \texttt{\~{}/.fbr} s'il n'existe pas encore, il sera créé.
(Ce répertoire peut être modifié à l'aide de la variable d'environnement
\texttt{FBR\_BASEDIR}, voir ci-dessous.)

\emph{Précision~:} la quantité d'espace nécessaire pour le processus de compilation
est (actuellement environ de 900~Mio pour les archives téléchargées et à peu
près de 30~Gio avec les résultats intermédiaires en réalisant une compilation).
Assurez-vous que dans le répertoire \texttt{\~{}/.fbr} vous avez suffisamment
d'espace~! (Sinon, vous pouvez également utiliser l'option \var{FBR\_TIDY}, voir
ci-dessous).

La structure du répertoire \texttt{\~{}/.fbr} est la suivante~:

\begin{longtable}{|l|p{10cm}|}
    \hline
    \multicolumn{1}{|l}{\textbf{Répertoire}} &
    \multicolumn{1}{|l|}{\textbf{Contenu}} \\
    \hline
    \endhead
    \hline
    \endfoot
    \endlastfoot
\texttt{fbr-<branch>-<arch>} &
    Dans ce répertoire, le buildroot uClibc sera décompacté. \texttt{<branch>}
    est la branche de développement (par exemple \texttt{trunk}), à partir du
    quelle le FBR provient. A l'origine le FBR est un paquet \texttt{src} décompacté,
    qui était utilisé pour \texttt{personnalisé le fbr}. \texttt{<arch>} est
    l'architecture de processeur (par exemple \texttt{x86} ou \texttt{x86\_64}).
    En plus de ce répertoire. \\
\hline
\texttt{dl}                            &
    Les archives téléchargées sont stockées ici. \\
\hline
\texttt{own}                       &
    Les paquets FBR peuvent être stockées ici, ils seront également compilés. \\
\hline
\end{longtable}

Ci-dessous le répertoire Buildroot
\texttt{\~{}/.fbr/fbr-<branch>-<arch>/buildroot} les répertoires suivants
sont intéressants~:

\begin{longtable}{|l|p{10cm}|}
    \hline
    \multicolumn{1}{|l}{\textbf{Répertoire}} &
    \multicolumn{1}{|l|}{\textbf{Contenu}} \\
    \hline
    \endhead
    \hline
    \endfoot
    \endlastfoot
\texttt{output/sandbox} &
    Dans ce répertoire il y aura un sous-répertoire pour chaque paquet FBR,
    celui-ci contiendra les fichiers du paquet FBR après le processus de compilation.
    Dans le répertoire \texttt{output/sandbox/<paquet>/target} seront placé les
    fichiers qui sont prévus pour le routeur fli4l. Dans le répertoire
    \texttt{output/sandbox/<paquet>/staging} seront placé les fichiers qui sont
    nécessaires pour convertir \emph{d'autre} paquet FBR qui ont besoin du
    paquet FBR principale. \\
\hline
\texttt{output/target} &
    Dans ce répertoire, \emph{tous} les programmes stockés seront complilés pour
    le routeur fli4l. Cette liste reflète la structure des répertoires sur le
    routeur fli4l. Avec l'aide de la commande \texttt{chroot} vous pouvez changer
    ce répertoire et tester les programmes compilés \footnote{cette fonction est
    soumis à certaines conditions, se référer au paragraphe \jump{sec:src:test}
    {"tester un programme compilé"}.} \\
\hline
\end{longtable}

\subsubsection{Paramètres généraux}

Pour l'utilisation de la commande \texttt{fbr-make} vous pouvez affecter
plusieurs variables d'environnement~:

\begin{longtable}{|l|p{10cm}|}
    \hline
    \multicolumn{1}{|l}{\textbf{Variable}} &
    \multicolumn{1}{|l|}{\textbf{Description}} \\
    \hline
    \endhead
    \hline
    \endfoot
    \endlastfoot
\texttt{FBR} &
    Indique explicitement le chemin d'accès au FBR. L'utilisation du chemin par
    défaut est \texttt{\~{}/.fbr/fbr-<branch>-<arch>} (voir ci-dessus). \\
\hline
\texttt{FBR\_BASEDIR} &
    Indique explicitement le chemin d'accès au FBR. Par défaut, le chemin \texttt{\~{}/.fbr}
	est utilisé (voir ci-dessus). Cette variable est ignorée si la variable d'environnement
	\texttt{FBR} est définie. \\
\hline
\texttt{FBR\_DLDIR} &
    Indique le répertoire qui contient les archives sources. L'utilisation du
    chemin par défaut est \texttt{\$\{FBR\}/../dl} (par exemple \texttt{\~{}/.fbr/dl}). \\
\hline
\texttt{FBR\_BRANCH} &
    Spécifie explicitement le nom de la branche sous laquelle les paquets sont construits
	en dessous de \texttt{\~{}/.fbr} (voir plus haut). Cette variable est ignorée si
	la variable d'environnement \texttt{FBR} est définie. \\
\hline
\texttt{FBR\_CATEGORY} &
    Spécifie explicitement le nom de la catégorie sous laquelle les paquets sont construits
	en dessous de \texttt{\~{}/.fbr} (voir plus haut). Cette variable est ignorée si
	la variable d'environnement \texttt{FBR} est définie. \\
\hline
\texttt{FBR\_OWNDIR} &
    Indique le répertoire qui contient les paquets spécifiques. L'utilisation du
    chemin par défaut est \texttt{\$\{FBR\}/../own} (par exemple \texttt{\~{}/.fbr/own}). \\
\hline
\texttt{FBR\_TIDY} &
    Si cette variable contient la valeur "y", les résultats intermédiaires qui
    apparaissent pendant la compilation des paquets FBR seront effacés directement
    après l'installation du répertoire \texttt{output/target}. Cela permet
    d'économiser beaucoup d'espace, en faite cette valeur est toujours recommandé,
    si vous ne vous sentez pas l'envie de vérifier les répertoires \texttt{output/build/...}
    après la construction des paquetages. Si cette variable contient la valeur "k",
	c'est seulement les résultats intermédiaires dans les différents répertoires du Kernel
	de Linux qui seront effacés, cela permet d'économiser relativement beaucoup d'espace
	sans perdre les fonctionnalités. Tous les autres dépacement (ou si la variable est
	manquante) sont effectuées et tous les résultats intermédiaires seront conservés. \\
\hline
\texttt{FBR\_ARCH} &
    Cette variable spécifie l'architecture du processeur pour lequel le FBR
    (ou les paquets FBR individuels) doit être construit. Si elle est absente
    le \texttt{x86} sera utilisé pour la construction. Voir ci-dessous les
    architectures supportées. \\
\hline
\end{longtable}

Le FBR prend actuellement en charge les architectures suivantes~:

\begin{longtable}{|l|p{10cm}|}
    \hline
    \multicolumn{1}{|l}{\textbf{Architecture}} &
    \multicolumn{1}{|l|}{\textbf{Description}} \\
    \hline
    \endhead
    \hline
    \endfoot
    \endlastfoot
\texttt{x86} &
    Intel Architecture x86 (32-Bit), aussi connu sous le nom IA-32. \\
\hline
\texttt{x86\_64} &
    AMD Architecture x86-64 (64-Bit), appelé aussi Intel 64 ou EM64T. \\
\hline
\end{longtable}

\subsubsection{Compiler tous les paquets FBR}

Lorsque vous exécutez la commande \texttt{fbr-make} avec l'argument \texttt{world},
pour que toutes les archives sources soit téléchargés et compilées, l'ordinateur
devra être utilisé pendant plusieurs heures et ceci en fonction de l'ordinateur
et le type de connexion Internet.\footnote{Le téléchargement des archives source
sera bien entendu effectuée qu'une seule fois, tant que vous ne mettez pas à
jour le FBR, si vous le mettez à jour vous aurez besoin de nouvelles versions de
paquets avec d'autres archives source.}

\subsubsection{Compiler avec Toolchain}

Lorsque vous exécutez la commande \texttt{fbr-make} avec l'argument \texttt{toolchain},
tous les paquets FBR sont téléchargés et convertis, ils sont nécessaires pour
construire les fichiers binaires réels pour fli4l (c-à-d construire Binder,
la bibliothèque uClibc etc.). Normalement, cette commande n'est pas nécessaire
car tous les paquets FBR dépendante du toolchain, les programmes toolchain
téléchargés et sont de toute façon compilés.

\subsubsection{Compiler un unique paquet FBR}

Si vous voulez compiler seulement un certain paquet FBR (pour auto-développé un
programme OPT), vous pouvez indiquer le nom du paquet FBR ou le nom de plusieurs
paquets FBR avec la commande \texttt{fbr-make}, (vous pouvez indiquer
\texttt{fbr-make openvpn} pour télécharger et compiler le programme openVPN).
Toutes les fichiers nécessaires qui dépende du programme seront également
téléchargés et compilés.

\subsubsection{Recompiler un unique paquet FBR }

Si vous voulez recompiler un certain paquet FBR (pour une raison quelconque),
vous devez d'abord supprimer les informations du processus de compilation dans
le FBR avant de recommancer. Vous pouvez utiliser la commande \texttt{fbr-make <paquet>-clean}
(par ex. \texttt{fbr-make openvpn-clean}). Les informations de tous les paquets
FBR qui dépendent du paquet FBR spécifié seront également réinitialisées, de sorte
qu'ils pourront également être recompilés la prochaine fois avec \texttt{fbr-make world}.

\subsubsection{Recompiler tous les paquets FBR}

Si vous souhaitez recompiler complétement le FBR (par exemple parce que vous
voulez l'utiliser un programme de référence pour développer votre nouveau système
haut de gamme ;-). Vous pouvez tous supprimer en utilisant la commande
\texttt{fbr-make clean} après avoir été invité à confirmé la commande, tous
les artefacts qui ont été générés au cours de la dernière construction FBR seront
supprimés.\footnote{Tout les répertoires \texttt{\~{}/.fbr/fbr-<branch>-<arch>/buildroot/output}
seront supprimés.} Cela est également utile pour libérer de l'espace disque.

\marklabel{sec:src:test}{\subsection{Tester d'un programme compilé}}

Si un programme a été compilé avec \texttt{fbr-make}, il peut également être
testé sur l'ordinateur de développement. Un tel test ne fonctionne que lorsque
l'architecture du processeur de l'ordinateur de développement et l'architecture
du processeur pour fli4l pour lequelle les programmes ont être compilés,
correspondent. (Par exemple, il n'est pas possible d'exécuter les programmes
fli4l \texttt{x86\_64} sur un système d'exploitation \texttt{x86}). Si la
condition est remplie, vous pouvez faire un teste,

\begin{example}
\begin{Verbatim}[commandchars=\\\{\}]
\textbf{chroot ~/.fbr/fbr-<branch>-<arch>/buildroot/output/target /bin/sh}\enter
\end{Verbatim}
\end{example}

\noindent allez dans le répertoire cible fli4l et essayez directement avec le/les
programme(s) compilé(s). S'il vous plaît faites attention, vous avez besoin pour
utiliser \texttt{chroot} des droits administrateur et en fonction de vos
préférences et de la configuration du système vous devez utiliser le service
\texttt{sudo} ou \texttt{su} c'est une exigence~! Vous devez aussi, avoir
compilé dans le paquet FBR \texttt{busybox} (via \texttt{fbr-make busybox}) de
sorte de pouvoir disposer d'un environnement shell dans le répertoire \texttt{chroot}.
Voici un petit exemple~:

\begin{example}
\begin{Verbatim}[commandchars=\\\{\}]
$ \textbf{sudo chroot ~/.fbr/fbr-trunk-x86/buildroot/output/target /bin/sh}\enter
Passwort:\textbf{(Ihr Passwort)}\enter


BusyBox v1.22.1 (fli4l) built-in shell (ash)
Enter 'help' for a list of built-in commands.

# \textbf{ls}\enter
THIS_IS_NOT_YOUR_ROOT_FILESYSTEM  mnt
bin                               opt
dev                               proc
etc                               root
home                              run
img                               sbin
include                           share
lib                               sys
lib32                             tmp
libexec                           usr
man                               var
media                             windows
# \textbf{bc --version}\enter
bc 1.06
Copyright 1991-1994, 1997, 1998, 2000 Free Software Foundation, Inc.
# \textbf{echo "42 - 23" | bc}\enter
19
#
\end{Verbatim}
\end{example}

\marklabel{sec:debugging}{\subsection{Déboguer un programme compilé}}

Si vous avez un problème sur un programme fli4l, en d'autres termes~: s'il se
bloque, vous avez la possibilité d'analyser l'état du programme plus tard, juste
avant l'accident (aussi nommé "débogage post-mortem"). Pour cela il est d'abord
nécessaire de configurer le paquet \texttt{base} et d'activer \var{DEBUG\_ENABLE\_CORE='yes'}.
Lors du crash un vidage de la mémoire est généré dans \texttt{/var/log/dumps/core.<PID>},
le "PID" est le numéro de processus accidenté, on peut analyser de la manière
suivante l'état du programme sur un ordinateur Linux avec un FBR complètement
compilé. Dans l'exemple suivant, le programme à analyser est \texttt{/usr/sbin/collectd},
avec le SIGBUS accepté. Le vidage de la mémoire est placé dans le fichier
\texttt{/tmp/core.collectd}.

\begin{example}
\begin{Verbatim}[commandchars=\\\%!]
fli4l@eisler:~$ \textbf%.fbr/fbr-trunk-86/buildroot/output/host/usr/bin/i586-linux-gdb!\enter
GNU gdb (GDB) 7.5.1
Copyright (C) 2012 Free Software Foundation, Inc.
License GPLv3+: GNU GPL version 3 or later <http://gnu.org/licenses/gpl.html>
This is free software: you are free to change and redistribute it.
There is NO WARRANTY, to the extent permitted by law.  Type "show copying"
and "show warranty" for details.
This GDB was configured as "--host=x86_64-unknown-linux-gnu --target=i586-buildr
oot-linux-uclibc".
For bug reporting instructions, please see:
<http://www.gnu.org/software/gdb/bugs/>.
(gdb) \textbf%set sysroot /project/fli4l/.fbr/fbr-trunk-86/buildroot/output/target!\enter
(gdb) \textbf%set debug-file-directory /project/fli4l/.fbr/fbr-trunk-86/buildroot/output/de!
\textbf%t/debug!\enter
(gdb) \textbf%file /project/fli4l/.fbr/fbr-trunk-86/buildroot/output/target/usr/sbin/collec!
\textbf%llectd!\enter
Reading symbols from /project/fli4l/.fbr/fbr-trunk-86/buildroot/output/target/usr/
sbin/collectd...Reading symbols from /project/fli4l/.fbr/fbr-trunk-86/buildroot/outp
ut/debug/.build-id/8b/28ab573be4a2302e1117964edede2e54ebbdbf.debug...done.
done.
(gdb) \textbf%core /tmp/core.collectd!\enter
[New LWP 2250]
[New LWP 2252]
[New LWP 2259]
[New LWP 2257]
[New LWP 2255]
[New LWP 2232]
[New LWP 2235]
[New LWP 2238]
[New LWP 2242]
[New LWP 2244]
[New LWP 2245]
[New LWP 2231]
[New LWP 2243]
[New LWP 2251]
[New LWP 2248]
[New LWP 2239]
[New LWP 2229]
[New LWP 2249]
[New LWP 2230]
[New LWP 2247]
[New LWP 2233]
[New LWP 2256]
[New LWP 2236]
[New LWP 2246]
[New LWP 2240]
[New LWP 2241]
[New LWP 2237]
[New LWP 2234]
[New LWP 2253]
[New LWP 2254]
[New LWP 2258]
[New LWP 2260]
Failed to read a valid object file image from memory.
Core was generated by `collectd -f'.
Program terminated with signal 7, Bus error.
#0  0xb7705f5d in memcpy ()
   from /project/fli4l/.fbr/fbr-trunk-86/buildroot/output/target/lib/libc.so.0
(gdb) \textbf%backtrace!\enter
#0  0xb7705f5d in memcpy ()
   from /project/fli4l/.fbr/fbr-trunk-86/buildroot/output/target/lib/libc.so.0
#1  0xb768a251 in rrd_write (rrd_file=rrd_file@entry=0x808e930, buf=0x808e268,
    count=count@entry=112) at rrd_open.c:716
#2  0xb76834f3 in rrd_create_fn (
    file_name=file_name@entry=0x808d2f8 "/data/rrdtool/db/vm-fli4l-1/cpu-0/cpu-i
nterrupt.rrd.async", rrd=rrd@entry=0xacff2f4c) at rrd_create.c:727
#3  0xb7683d7b in rrd_create_r (
    filename=filename@entry=0x808d2f8 "/data/rrdtool/db/vm-fli4l-1/cpu-0/cpu-int
errupt.rrd.async", pdp_step=pdp_step@entry=10, last_up=last_up@entry=1386052459,
    argc=argc@entry=16, argv=argv@entry=0x808cf18) at rrd_create.c:580
#4  0xb76b77fd in srrd_create (
    filename=0xacff33f0 "/data/rrdtool/db/vm-fli4l-1/cpu-0/cpu-interrupt.rrd.asy
nc",
    pdp_step=10, last_up=1386052459, argc=16, argv=0x808cf18) at utils_rrdcreate
.c:377
#5  0xb76b78cb in srrd_create_thread (targs=targs@entry=0x808bab8)
    at utils_rrdcreate.c:559
#6  0xb76b7a8f in srrd_create_thread (targs=0x808bab8) at utils_rrdcreate.c:491
#7  0xb7763430 in ?? ()
   from /project/fli4l/.fbr/fbr-trunk-86/buildroot/output/target/lib/libpthread.so.
0
#8  0xb775e672 in clone ()
   from /project/fli4l/.fbr/fbr-trunk-86/buildroot/output/target/lib/libpthread.so.
0
(gdb) \textbf%frame 1!\enter
#1  0xb768a251 in rrd_write (rrd_file=rrd_file@entry=0x808e930, buf=0x808e268,
    count=count@entry=112) at rrd_open.c:716
716         memcpy(rrd_simple_file->file_start + rrd_file->pos, buf, count);
(gdb) \textbf%print (char*) buf!\enter
$1 = 0x808e268 "RRD"
(gdb) \textbf%print rrd_simple_file->file_start!\enter
value has been optimized out
(gdb) \textbf%list!\enter
711         if((rrd_file->pos + count) > old_size)
712         {
713             rrd_set_error("attempting to write beyond end of file");
714             return -1;
715         }
716         memcpy(rrd_simple_file->file_start + rrd_file->pos, buf, count);
717         rrd_file->pos += count;
718         return count;       /* mimmic write() semantics */
719     #else
720         ssize_t   _sz = write(rrd_simple_file->fd, buf, count);
(gdb) \textbf%list 700!\enter
695      * rrd_file->pos of rrd_simple_file->fd.
696      * Returns the number of bytes written or <0 on error.  */
697
698     ssize_t rrd_write(
699         rrd_file_t *rrd_file,
700         const void *buf,
701         size_t count)
702     {
703         rrd_simple_file_t *rrd_simple_file = (rrd_simple_file_t *)rrd_file->
pvt;
704     #ifdef HAVE_MMAP
(gdb) \textbf%print *(rrd_simple_file_t *)rrd_file->pvt!\enter
$2 = {fd = 9, file_start = 0xa67d0000 <Address 0xa67d0000 out of bounds>,
  mm_prot = 3, mm_flags = 1}
\end{Verbatim}
\end{example}

Vous pouvez voir ci-dessus, il faut un peu "fouiller" que dans le répertoire
d'objet \texttt{rrd\_simple\_file\_t} le pointeur est invalide ("Address ... out
of bounds") dans cette suite de débogage, il est clair que l'échec de \texttt{posix\_fallocate}
est la cause de l'interruption du programme.

Ce qui est important ici, c'est que \emph{tous} les chemins indiquer sont
pleinement qualifié (\texttt{/project/...}) et qu'il n'y a aucun "raccourcis"
utilisé (par exemple \texttt{\~{}/...}). Si cela n'est pas respecté, il
peut arriver que les informations de débogage \texttt{gdb} ne trouve pas
l'application et/ou n'utilise pas les bibliothèques. Les informations de débogage
ne sont pas directement incluses dans le programme testé, mais stocké dans le
répertoire \texttt{\~{}/.fbr/fbr-<branch>-<arch>/buildroot/output/debug/} sur fichier séparé.

\subsection{Information sur le FBR}

\subsubsection{Accéder à l'aide}

Que peut faire \texttt{fbr-make} pour vous, vous pouvez peut-être utiliser
la commande \texttt{fbr-make help}.

\subsubsection{Affichage des informations sur le programme}

Vous pouvez voir tous les paquets FBR disponibles et leurs versions, en
utilisant la commande \texttt{fbr-make show-versions}~:

\begin{example}
\begin{Verbatim}[commandchars=\\\{\}]
$ \textbf{fbr-make show-versions}\enter
Configured packages

acpid 2.0.20
actctrl 3.25+dfsg1
add-days undefined
[...]
\end{Verbatim}
\end{example}

\subsubsection{Affichage des associations de bibliothèques}

Avec la commande \texttt{fbr-make links-against <soname>} et en indiquant le nom
de la bibliothèque à la place de \texttt{soname}, vous pourrez voir tous les
fichiers dans le répertoire \texttt{\~{}/.fbr/fbr-<branch>-<arch>/buildroot/output/target}
qui sont associés à cette bibliothèque. Par exemple pour identifier tous les
programmes et bibliothèques qui utilisent \texttt{libm} (bibliothèque de fonction
mathématique), vous indiquez la commande \texttt{fbr-make links-against libm.so.0}
car le nom de la bibliothèque \texttt{libm.so.0} est libm-bibliothèque.
Un autre exemple~:

\begin{example}
\begin{Verbatim}[commandchars=\\\{\}]
$ \textbf{fbr-make links-against librrd_th.so.4}\enter
Executing plugin links-against
Files linking against librrd_th.so.4
collectd usr/lib/collectd/rrdcached.so
collectd usr/lib/collectd/rrdtool.so
rrdtool usr/bin/rrdcached
\end{Verbatim}
\end{example}

Dans la première colonne le nom du paquet est indiqué et dans la seconde le chemin
(relatif) du fichier qui est associé à la bibliothèque.

Pour trouver le nom d'une bibliothèque, vous pouvez utiliser
\texttt{readelf}, comme ceci~:

\begin{example}
\begin{Verbatim}[commandchars=\\\{\}]
$ \textbf{readelf -d ~/.fbr/fbr-trunk-x86/buildroot/output/target/lib/libm-0.9.33.2.so |}\enter
> \textbf{grep SONAME}\enter
 0x0000000e (SONAME)                     Library soname: [libm.so.0]
\end{Verbatim}
\end{example}

\subsubsection{Affichage des changements de version}

(Uniquement) intéressant pour les développeurs de l'équipe fli4l avec un accès
en écriture au répertoire fli4l-SVN-Repository, utilisez la commande
\texttt{fbr-make version-changes}. Vous pourrez voir la liste de tous les paquets
dont la version FBR a été modifié localement, la version différente de la copie
de travail et la version du référentiel. Ainsi, le développeur peut voir un
aperçu des paquets FBR mis à jour, avant d'écrire les changements dans le repo.
Voici une entrée possible~:

\begin{example}
\begin{Verbatim}[commandchars=\\\{\}]
$ \textbf{fbr-make version-changes}\enter
Executing plugin version-changes
Package version changes
KAMAILIO: 4.0.5 --> 4.1.1
\end{Verbatim}
\end{example}

Ici vous pouvez voir immédiatement que le paquet FBR \texttt{kamailio} avait
la version 4.0.5 et a été mise à jour avec la version 4.1.1.

\subsection{Modification de la configuration du FBR}

\subsubsection{Reconfiguration des FBRs}

D'une part, à l'aide de la commande \texttt{fbr-make buildroot-menuconfig}, il
est possible de choisir les paquets FBR à compiler. Ceci est utile si vous voulez
compiler d'autres paquets FBR pour fli4l et qui ne sont pas activés par défaut,
mais qui sont intégrés dans le buildroot uClibc, ou si vous souhaitez activer
vos propres paquets FBR. D'autre part, des propriétés globales du FBRs peuvent
être modifiées, telles que la version du compilateur GCC utilisée. Pour couronnée
de succès la configuration du menu, la nouvelle configuration doit être sauvegardée
dans le répertoire \texttt{src/fbr/buildroot/.config}.

\achtung{S'il vous plaît noter, que des modifications dans la configuration du
toolchain ne sont officiellement \emph{pas} prisent en charge, car les fichiers
binaires auront une forte probabilité d'incompatibilités avec la distribution
officielle de fli4l. Donc, si vous avez besoin de fichiers binaires pour votre
propre OPT et que vous souhaitez publier cette OPT, vous ne devez pas modifier
un paramètre de chaîne de compilation~!}

\subsubsection{Reconfiguration de la bibliothèque uClibc}

Si vous utilisez de commande \texttt{fbr-make uclibc-menuconfig} vous pouvez
modifier la fonctionnalité de la bibliothèque uClibc. Pour couronnée de succès
la configuration du menu, la nouvelle configuration doit être enregistrée dans
le répertoire \texttt{fbr-make uclibc-menuconfig}.

\achtung{Il faut prendre également en compte comme dans le dernier paragraphe~:
une modifications sera hautement probable qu'elle ne soit pas compatible avec la
distribution officielle de fli4l et ne sera donc pas pris en charge~!}

\subsubsection{Reconfiguration de Busybox}

A l'aide de la commande \texttt{fbr-make busybox-menuconfig}, vous pouvez régler
le fonctionnement de Busybox. Pour couronnée de succès la configuration du menu,
la nouvelle configuration doit être enregistrée dans le répertoire
\texttt{src/fbr/buildroot/package/busybox/busybox-<Version>.config}.

\achtung{Ici aussi, un changement n'est probablement pas compatible avec la
distribution officielle de fli4l et n'est donc pas pris en charge~! Tout au plus
le fait de compléter le Busybox autour de nouvelles applets est sans problèmes,
tant que vous utilisez le Busybox ainsi modifié seulement sur votre routeur fli4l,
(il ne permet pas l'utilisation d'un tel Busybox modifié, pour vos propres OPTs).}

\subsubsection{Reconfiguration du paquet Kernel-Linux}

A l'aide de la commande \texttt{fbr-make linux-menuconfig} ou
\texttt{fbr-make linux-<version>-menuconfig} la configuration de tous les paquets
Kernel ou d'un paquet Kernel spécifique est permis. Pour couronnée de succès
la configuration du menu, la nouvelle configuration doit être enregistrée dans
le répertoire \texttt{src/fbr/buildroot/linux/linux-<version>/dot-config-<arch>}.
Ceci est valable seulement pour un Kernel standard. Pour une variante du paquet
Kernel, au lieu que le fichier soit indiqué \texttt{diff}, il sera indiqué
\texttt{src/fbr/buildroot/linux/linux-<version>/linux-<version>\_<variante>/dot-config-<arch>.diff}.

\achtung{Il faut prendre également en compte comme dans le dernier paragraphe~:
un changement n'est probablement pas compatible avec la distribution officielle
de fli4l et n'est donc pas pris en charge~! Tout au plus le fait de compléter le
Kernel Linux autour de nouveaux modules c'est sans problèmes (il ne permet pas
l'utilisation d'un tel Kernel modifié, pour vos propres OPTs).}

\subsection{Mise à jour des FBRs}

Pour chacune des commandes décrites ci-dessus, un examen des FBRs est effectué
pour une évolution de la mise à jour. En cas de divergence entre les sources,
après avoir utilé le \texttt{fbr-make} pour (décompacter le paquet \texttt{src}
ou pour la copie de travail SVN) et le FBR dans \texttt{\~{}/.fbr/fbr-<branch>-<arch>/buildroot},
les différentes sources trouvées seront mise à jour. Les nouveaux paquets FBR
ainsi que les anciens seront intégrés, les paquets FBR existants seront effacés.
Les configurations sont comparées~: les paquets FBR dont la configuration a été
récemment modifiée ainsi que tous les paquets FBR qui en découle seront construits.
Cela garantit que le FBR sur votre ordinateur est toujours conforme pour le
développement de fli4l (excepté vos propres paquets FBR qui sont dans \texttt{\~{}/.fbr/own/}).
\textbf{Cela signifie aussi que des modifications de la partie officielle et de
la configuration du Buildroot seront perdues lors de la prochaine de la mise à jour~!}
C'est pourquoi, une reconfiguration des FBRs (voir ci-dessus) n'est pasrecommandé,
du moins pas si à la place des paquets \texttt{src} vous travaillez avec une copie
de travail \texttt{src}. (Lors de la mise à jour d'une copie de travail SVN,
vos changements de configuration locaux et les modifications apportées dans le
SVN-Repository sont fusionnées, si bien que le problème de la perte de la
configuration ne se produira pas ici). En revanche, vous pouvez reconfigurer
vos propres paquets FBR facilement, sans provoquer une mise à jour avec la perte
de données.

\subsection{Intégrer vos propres programmes dans le FBR}

La compilation des paquets FBR individuels est contrôlé par un petit fichier
make. Si vous voulez développer vos propres paquets FBR, et si vous utilisez
le fichier make une description de la configuration est dans \texttt{\~{}/.fbr/own/<paquet>/}.
Pour comprendre comment ceux-ci sont construits et comment écrire son propre fichier
Make, une documentation décrit en détail le uClibc-Buildroots dans
\altlink{http://buildroot.uclibc.org/downloads/manual/manual.html\#adding-packages}.
