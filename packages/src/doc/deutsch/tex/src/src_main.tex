% Last Update: $Id$
\marklabel{buildroot}{
\section{SRC - Das fli4l-Buildroot}
}

Dieses Kapitel ist hauptsächlich für Entwickler interessant, die Binärprogramme
oder Linux-Kernel für den fli4l übersetzen wollen. Wenn Sie fli4l nur als Router
einsetzen und keine Pakete für den fli4l anbieten wollen, die eigene
Binärprogramme benötigen, können Sie dieses Kapitel komplett überspringen.

Generell ist für die Übersetzung von Programmpaketen für den fli4l ein
Linux-System erforderlich. Eine Übersetzung unter anderen Betriebssystemen
(Microsoft Windows, OS X, FreeBSD etc.) wird \emph{nicht} unterstützt.

Die Anforderungen an ein Linux-System zur fli4l-Entwicklung sind wie folgt:
\begin{itemize}
\item GNU gcc und g++ in der Version 2.95 oder neuer
\item GNU gcc-multilib (je nach Hostsystem nötig)
\item GNU binutils (enthält den Binder sowie andere, notwendige Programme)
\item GNU make in der Version 3.81 oder neuer
\item GNU bash
\item libncurses5-dev für \texttt{fbr-make *-menuconfig} (je nach Hostsystem nötig)
\item die Programme sed, awk, which, flex, bison und patch
\item die Programme makeinfo (Paket texinfo) und msgfmt (Paket gettext)
\item die Programme tar, cpio, gzip, bzip2 und unzip
\item die Programme wget, rsync, svn und git
\item die Programme perl und python
\end{itemize}

Im Folgenden repräsentieren \textbf{fett} gedruckte Zeichen zu tätigende
Eingaben, das \enter-Zeichen steht für die Eingabetaste Ihrer Tastatur und
schließt einzugebende Befehle ab.

\subsection{Eine Übersicht über die Quellen}

Im \texttt{src}-Verzeichnis finden Sie folgende Unterverzeichnisse:

\begin{longtable}{|l|p{10cm}|}
    \hline
    \multicolumn{1}{|l}{\textbf{Verzeichnis}} &
    \multicolumn{1}{|l|}{\textbf{Inhalt}} \\
    \hline
    \endhead
    \hline
    \endfoot
    \endlastfoot
\texttt{fbr} &
    In diesem Verzeichnis befindet sich ein angepasstes Buildsystem, das auf
    dem Buildroot zur uClibc (aktuell in der Version 0.9.33.2) basiert.
    FBR steht hierbei für ``fli4l-Buildroot''. Damit ist es möglich, alle auf
    dem fli4l verwendeten Programme (Kernel, Anwendungen und Bibliotheken) neu
    zu übersetzen. \\
\hline
\texttt{fli4l} &
    Dieses Verzeichnis enthält die fli4l-spezifischen Quellen, nach Paketen
    geordnet. Alle Quellen, die in diesem Unterverzeichnis enthalten sind,
    wurden entweder speziell für die Verwendung mit fli4l geschrieben oder
    zumindest stark angepasst. \\
\hline
\texttt{cross} &
    In diesem Verzeichnis befinden sich Skripte, mit denen die Cross-Compiler
    erstellt werden können, die für das Übersetzen von mkfli4l für diverse
    Plattformen benötigt werden. \\
\hline
\end{longtable}

\subsection{Übersetzen eines Programms für den fli4l}

Im Unterverzeichnis ``fbr'' finden Sie das Skript \texttt{fbr-make}, das die
Übersetzung aller Programme aus den Basispaketen für den fli4l-Router
steuert. Dieses Skript kümmert sich um das Herunterladen und Übersetzen aller
für den fli4l benötigten Binärdateien. Generell legt das Skript Dateien in dem
Verzeichnis \texttt{\~{}/.fbr} ab; existiert dieses noch nicht, wird es
angelegt. (Dieses Verzeichnis kann mit Hilfe der Umgebungsvariable
\texttt{FBR\_BASEDIR} verändert werden, siehe unten.)

\emph{Hinweis:} Während des Übersetzungsvorgangs wird viel Platz benötigt
(momentan etwa 900~MiB für die heruntergeladenen Archive und knapp 30~GiB
für die Zwischenergebnisse und die resultierenden Kompilate). Stellen Sie somit
sicher, dass Sie unterhalb von \texttt{\~{}/.fbr} über genug Platz verfügen!
(Alternativ können Sie auch die \var{FBR\_TIDY}-Option nutzen, siehe unten.)

Die Verzeichnisstruktur unterhalb von \texttt{\~{}/.fbr} ist wie folgt:

\begin{longtable}{|l|p{10cm}|}
    \hline
    \multicolumn{1}{|l}{\textbf{Verzeichnis}} &
    \multicolumn{1}{|l|}{\textbf{Inhalt}} \\
    \hline
    \endhead
    \hline
    \endfoot
    \endlastfoot
\texttt{fbr-<branch>-<arch>} &
    Hierhin wird das uClibc-Buildroot entpackt. \texttt{<branch>} steht
    hierbei für den Entwicklungszweig (z.B. \texttt{trunk}), aus dem das FBR
    stammt. Ist der Ursprung des FBRs ein entpacktes \texttt{src}-Paket, wird
    \texttt{fbr-custom} benutzt. \texttt{<arch>} steht für die jeweilige
    Prozessorarchitektur (z.B. \texttt{x86} oder \texttt{x86\_64}). Mehr zu
    diesem Verzeichnis steht weiter unten. \\
\hline
\texttt{dl}                            &
    Hier werden die heruntergeladenen Archive gespeichert. \\
\hline
\texttt{own}                       &
    Hier können eigene FBR-Pakete abgelegt werden, die ebenfalls übersetzt
    werden sollen. \\
\hline
\end{longtable}

Unterhalb des Buildroot-Verzeichnisses
\texttt{\~{}/.fbr/fbr-<branch>-<arch>/buildroot} sind die folgenden
Verzeichnisse interessant:

\begin{longtable}{|l|p{10cm}|}
    \hline
    \multicolumn{1}{|l}{\textbf{Verzeichnis}} &
    \multicolumn{1}{|l|}{\textbf{Inhalt}} \\
    \hline
    \endhead
    \hline
    \endfoot
    \endlastfoot
\texttt{output/sandbox} &
    In diesem Verzeichnis gibt es für jedes FBR-Paket ein Unterverzeichnis, das
    die Dateien des FBR-Pakets nach dem Übersetzungsvorgang aufnimmt. In dem
    Verzeichnis \texttt{output/sandbox/<paket>/target} befinden sich dabei die
    Dateien, die für den fli4l-Router vorgesehen sind. In dem Verzeichnis
    \texttt{output/sandbox/<paket>/staging} hingegen befinden sich Dateien, die
    zum Übersetzen \emph{anderer} FBR-Pakete, die dieses FBR-Paket benötigen,
    erforderlich sind. \\
\hline
\texttt{output/target} &
    In diesem Verzeichnis werden \emph{alle} übersetzen Programme für den
    fli4l-Router abgelegt. Dieses Verzeichnis spiegelt somit die
    Verzeichnisstruktur auf dem fli4l-Router wider. Mit Hilfe von
    \texttt{chroot} kann man in dieses Verzeichnis wechseln und die übersetzten
    Programme ausprobieren.\footnote{Dies ist an gewisse Voraussetzungen
    geknüpft, siehe hierzu den Abschnitt \jump{sec:src:test}{``Testen eines
    übersetzten Programms''}.} \\
\hline
\end{longtable}

\subsubsection{Allgemeine Einstellungen}

Die Arbeitsweise von \texttt{fbr-make} kann durch verschiedene
Umgebungsvariablen beeinflusst werden:

\begin{longtable}{|l|p{10cm}|}
    \hline
    \multicolumn{1}{|l}{\textbf{Variable}} &
    \multicolumn{1}{|l|}{\textbf{Beschreibung}} \\
    \hline
    \endhead
    \hline
    \endfoot
    \endlastfoot
\texttt{FBR} &
    Gibt den Pfad zum FBR explizit an. Standardmäßig wird der Pfad
    \texttt{\~{}/.fbr/fbr-<branch>-<arch>} (s.o.) verwendet. \\
\hline
\texttt{FBR\_BASEDIR} &
    Gibt den Basispfad zum FBR explizit an. Standardmäßig wird der Pfad
    \texttt{\~{}/.fbr} (s.o.) verwendet. Diese Variable wird ignoriert, falls
    die Umgebungsvariable \texttt{FBR} gesetzt wird. \\
\hline
\texttt{FBR\_DLDIR} &
    Gibt das Verzeichnis an, das die Quellarchive enthält. Standardmäßig wird
    der Pfad \texttt{\$\{FBR\}/../dl} (also z.B. \texttt{\~{}/.fbr/dl})
    verwendet. \\
\hline
\texttt{FBR\_BRANCH} &
    Gibt explizit den Namen des Branches an, unter dem die Pakete unterhalb von
    \texttt{\~{}/.fbr} (s.o.)gebaut werden. Diese Variable wird ignoriert, falls
    die Umgebungsvariable \texttt{FBR} gesetzt wird. \\
\hline
\texttt{FBR\_CATEGORY} &
    Gibt explizit den Namen der Category an, unter dem die Pakete unterhalb von
    \texttt{\~{}/.fbr} (s.o.)gebaut werden. Diese Variable wird ignoriert, falls
    die Umgebungsvariable \texttt{FBR} gesetzt wird. \\
\hline
\texttt{FBR\_OWNDIR} &
    Gibt das Verzeichnis an, das die eigenen Pakete enthält. Standardmäßig wird
    der Pfad \texttt{\$\{FBR\}/../own} (also z.B. \texttt{\~{}/.fbr/own})
    verwendet. \\
\hline
\texttt{FBR\_TIDY} &
    Wenn diese Variable den Wert ``y'' enthält, werden Zwischenergebnisse, die
    während des Bauens der FBR-Pakete entstehen, unmittelbar nach der
    Installation in das Verzeichnis \texttt{output/target} gelöscht. Das spart
    viel Speicherplatz und ist eigentlich immer empfehlenswert, wenn man nach
    dem Bauen von Paketen nicht den Drang verspürt, in \texttt{output/build/...}
    hineinzuschauen. Falls diese Variable den Wert ``k'' enthält, werden nur
    die Zwischenergebnisse in den diversen Linux-Kernel-Verzeichnissen entfernt,
    weil dies verhältnismäßig viel Platz spart, ohne dass dadurch Funktionalität
    verloren geht. Alle anderen Belegungen (oder wenn die Variable gänzlich
    fehlt) sorgen dafür, dass alle Zwischenergebnisse erhalten bleiben.\\
\hline
\texttt{FBR\_ARCH} &
    Diese Variable gibt die Prozessorarchitektur an, für die das FBR (bzw.
    einzelne FBR-Pakete) gebaut werden sollen. Fehlt sie, wird \texttt{x86}
    angenommen. Die unterstützten Architekturen sind weiter unten zu finden.\\
\hline
\end{longtable}

Momentan unterstützt das FBR die folgenden Architekturen:

\begin{longtable}{|l|p{10cm}|}
    \hline
    \multicolumn{1}{|l}{\textbf{Architektur}} &
    \multicolumn{1}{|l|}{\textbf{Beschreibung}} \\
    \hline
    \endhead
    \hline
    \endfoot
    \endlastfoot
\texttt{x86} &
    Intel x86-Architektur (32-Bit), auch IA-32 genannt. \\
\hline
\texttt{x86\_64} &
    AMD x86-64-Architektur (64-Bit), von Intel auch Intel 64 oder EM64T genannt. \\
\hline
\end{longtable}

\subsubsection{Übersetzen aller FBR-Pakete}

Wenn Sie \texttt{fbr-make} mit dem Argument \texttt{world} ausführen,
dauert es je nach verwendetem Rechner und verwendeter Internetanbindung
mehrere Stunden, bis alle Quellarchive heruntergeladen und übersetzt worden
sind.\footnote{Das Herunterladen der Quellarchive wird natürlich nur einmal
durchgeführt, so lange Sie das FBR nicht aktualisieren und dadurch neuere
Paketversionen andere Quellarchive benötigen.}

\subsubsection{Übersetzen des Toolchains}

Wenn Sie \texttt{fbr-make} mit dem Argument \texttt{toolchain} ausführen,
werden alle FBR-Pakete heruntergeladen und übersetzt, die für das Bauen der
eigentlichen fli4l-Binärprogramme benötigt werden (also Übersetzer, Binder,
uClibc-Bibliothek etc.). Normalerweise wird dieses Kommando nicht benötigt,
da alle FBR-Pakete vom Toolchain abhängig sind und somit diese
Toolchain-Programme ohnehin heruntergeladen und gebaut werden.

\subsubsection{Übersetzen eines einzelnen FBR-Pakets}

Wollen Sie hingegen nur ein bestimmtes FBR-Paket übersetzen (etwa die Programme
für ein selbst entwickeltes OPT), können Sie den Namen des FBR-Pakets bzw. die
Namen mehrerer FBR-Pakete dem \texttt{fbr-make}-Programm mitgeben (etwa
\texttt{fbr-make openvpn} zum Herunterladen und Übersetzen der
OpenVPN-Programme). Dabei werden alle nötigen Abhängigkeiten ebenfalls
heruntergeladen und übersetzt.

\subsubsection{Erneutes Übersetzen eines einzelnen FBR-Pakets}

Möchten Sie ein bestimmtes FBR-Paket erneut übersetzen (warum auch immer),
müssen Sie zuerst die Informationen im FBR über den vorher stattgefundenen
Übersetzungsvorgang entfernen. Dazu können Sie den Befehl
\texttt{fbr-make <paket>-clean} (z.B. \texttt{fbr-make openvpn-clean})
verwenden. Dabei werden die Informationen all jener FBR-Pakete, die von dem
angegebenen FBR-Paket abhängig sind, ebenfalls zurückgesetzt, so dass sie beim
nächsten \texttt{fbr-make world} ebenfalls neu übersetzt werden.

\subsubsection{Erneutes Übersetzen aller FBR-Pakete}

Möchten Sie das komplette FBR neu übersetzen (z.B. weil Sie es als
Benchmark-Programm für Ihr neues High-End-Entwicklersystem nutzen wollen ;-),
können Sie mit Hilfe des Kommandos \texttt{fbr-make clean} nach der Bestätigung
einer Sicherheitsabfrage alle Artefakte entfernen, die während des letzten
FBR-Baus erzeugt worden sind.\footnote{Es wird das gesamte Verzeichnis
\texttt{\~{}/.fbr/fbr-<branch>-<arch>/buildroot/output} entfernt.} Dies ist auch
nützlich, um belegten Plattenspeicher freizugeben.

\marklabel{sec:src:test}{\subsection{Testen eines übersetzten Programms}}

Ist ein Programm mit \texttt{fbr-make} übersetzt worden, kann es auf dem
Entwicklungsrechner auch getestet werden. Ein solcher Test funktioniert
natürlich nur, wenn die Prozessorarchitektur des Entwicklerrechners mit der
Prozessorarchitektur des fli4ls, für welchen die Programme übersetzt werden
sollen, übereinstimmt. (Es ist z.B. nicht möglich,
\texttt{x86\_64}-fli4l-Programme auf einem \texttt{x86}-Betriebssystem
auszuführen.) Ist diese Voraussetzung erfüllt, kann man mit

\begin{example}
\begin{Verbatim}[commandchars=\\\{\}]
\textbf{chroot ~/.fbr/fbr-<branch>-<arch>/buildroot/output/target /bin/sh}\enter
\end{Verbatim}
\end{example}

\noindent in das fli4l-Zielverzeichnis wechseln und dort das/die
übersetzte(n) Programm(e) direkt ausprobieren. Beachten Sie jedoch bitte, dass
Sie für \texttt{chroot} Administrator-Rechte benötigen und daher je nach
Vorliebe und Systemkonfiguration die Dienste von \texttt{sudo} oder \texttt{su}
in Anspruch nehmen müssen! Auch müssen Sie zumindest das FBR-Paket
\texttt{busybox} übersetzt haben (via \texttt{fbr-make busybox}), damit Sie in
der \texttt{chroot}-Umgebung eine Shell zur Verfügung haben. Ein kleines
Beispiel:

\begin{example}
\begin{Verbatim}[commandchars=\\\{\}]
$ \textbf{sudo chroot ~/.fbr/fbr-trunk-x86/buildroot/output/target /bin/sh}\enter
Passwort:\textbf{(Ihr Passwort)}\enter


BusyBox v1.22.1 (fli4l) built-in shell (ash)
Enter 'help' for a list of built-in commands.

# \textbf{ls}\enter
THIS_IS_NOT_YOUR_ROOT_FILESYSTEM  mnt
bin                               opt
dev                               proc
etc                               root
home                              run
img                               sbin
include                           share
lib                               sys
lib32                             tmp
libexec                           usr
man                               var
media                             windows
# \textbf{bc --version}\enter
bc 1.06
Copyright 1991-1994, 1997, 1998, 2000 Free Software Foundation, Inc.
# \textbf{echo "42 - 23" | bc}\enter
19
#
\end{Verbatim}
\end{example}

\marklabel{sec:debugging}{\subsection{Entwanzen eines übersetzten Programms}}

Macht ein Programm auf dem fli4l Probleme, mit anderen Worten: es stürzt ab,
dann hat man die Möglichkeit, den Programmzustand unmittelbar vor dem Absturz
nachträglich zu analysieren (auch ``Post-Mortem Debugging'' genannt). Hierzu
muss man zuerst in der Konfiguration des \texttt{base}-Pakets
\var{DEBUG\_ENABLE\_CORE='yes'} aktivieren. Wird dann beim Absturz ein
Speicherabbild in \texttt{/var/log/dumps/core.<PID>} generiert, wobei ``PID''
die Prozessnummer des abgestürzten Prozesses ist, dann kann man den Zustand des
Programms auf einem Linux-Rechner mit einem voll übersetzten FBR folgendermaßen
analysieren. Im folgenden Beispiel ist das zu analysierende Programm
\texttt{/usr/sbin/collectd}, das sich mit einem SIGBUS verabschiedet hatte. Das
Speicherabbild liegt dabei in \texttt{/tmp/core.collectd}.

\begin{example}
\begin{Verbatim}[commandchars=\\\%!]
fli4l@eisler:~$ \textbf%.fbr/fbr-trunk-x86/buildroot/output/host/usr/bin/i586-linux-gdb!\enter
GNU gdb (GDB) 7.5.1
Copyright (C) 2012 Free Software Foundation, Inc.
License GPLv3+: GNU GPL version 3 or later <http://gnu.org/licenses/gpl.html>
This is free software: you are free to change and redistribute it.
There is NO WARRANTY, to the extent permitted by law.  Type "show copying"
and "show warranty" for details.
This GDB was configured as "--host=x86_64-unknown-linux-gnu --target=i586-buildr
oot-linux-uclibc".
For bug reporting instructions, please see:
<http://www.gnu.org/software/gdb/bugs/>.
(gdb) \textbf%set sysroot /project/fli4l/.fbr/fbr-trunk-x86/buildroot/output/target!\enter
(gdb) \textbf%set debug-file-directory /project/fli4l/.fbr/fbr-trunk-x86/buildroot/outpu!
\textbf%t/debug!\enter
(gdb) \textbf%file /project/fli4l/.fbr/fbr-trunk-x86/buildroot/output/target/usr/sbin/co!
\textbf%llectd!\enter
Reading symbols from /project/fli4l/.fbr/fbr-trunk-x86/buildroot/output/target/u
sr/sbin/collectd...Reading symbols from /project/fli4l/.fbr/fbr-trunk-x86/buildr
oot/output/debug/.build-id/8b/28ab573be4a2302e1117964edede2e54ebbdbf.debug...don
e.
done.
(gdb) \textbf%core /tmp/core.collectd!\enter
[New LWP 2250]
[New LWP 2252]
[New LWP 2259]
[New LWP 2257]
[New LWP 2255]
[New LWP 2232]
[New LWP 2235]
[New LWP 2238]
[New LWP 2242]
[New LWP 2244]
[New LWP 2245]
[New LWP 2231]
[New LWP 2243]
[New LWP 2251]
[New LWP 2248]
[New LWP 2239]
[New LWP 2229]
[New LWP 2249]
[New LWP 2230]
[New LWP 2247]
[New LWP 2233]
[New LWP 2256]
[New LWP 2236]
[New LWP 2246]
[New LWP 2240]
[New LWP 2241]
[New LWP 2237]
[New LWP 2234]
[New LWP 2253]
[New LWP 2254]
[New LWP 2258]
[New LWP 2260]
Failed to read a valid object file image from memory.
Core was generated by `collectd -f'.
Program terminated with signal 7, Bus error.
#0  0xb7705f5d in memcpy ()
   from /project/fli4l/.fbr/fbr-trunk-x86/buildroot/output/target/lib/libc.so.0
(gdb) \textbf%backtrace!\enter
#0  0xb7705f5d in memcpy ()
   from /project/fli4l/.fbr/fbr-trunk-x86/buildroot/output/target/lib/libc.so.0
#1  0xb768a251 in rrd_write (rrd_file=rrd_file@entry=0x808e930, buf=0x808e268,
    count=count@entry=112) at rrd_open.c:716
#2  0xb76834f3 in rrd_create_fn (
    file_name=file_name@entry=0x808d2f8 "/data/rrdtool/db/vm-fli4l-1/cpu-0/cpu-i
nterrupt.rrd.async", rrd=rrd@entry=0xacff2f4c) at rrd_create.c:727
#3  0xb7683d7b in rrd_create_r (
    filename=filename@entry=0x808d2f8 "/data/rrdtool/db/vm-fli4l-1/cpu-0/cpu-int
errupt.rrd.async", pdp_step=pdp_step@entry=10, last_up=last_up@entry=1386052459,
    argc=argc@entry=16, argv=argv@entry=0x808cf18) at rrd_create.c:580
#4  0xb76b77fd in srrd_create (
    filename=0xacff33f0 "/data/rrdtool/db/vm-fli4l-1/cpu-0/cpu-interrupt.rrd.asy
nc",
    pdp_step=10, last_up=1386052459, argc=16, argv=0x808cf18) at utils_rrdcreate
.c:377
#5  0xb76b78cb in srrd_create_thread (targs=targs@entry=0x808bab8)
    at utils_rrdcreate.c:559
#6  0xb76b7a8f in srrd_create_thread (targs=0x808bab8) at utils_rrdcreate.c:491
#7  0xb7763430 in ?? ()
   from /project/fli4l/.fbr/fbr-trunk-x86/buildroot/output/target/lib/libpthread
.so.0
#8  0xb775e672 in clone ()
   from /project/fli4l/.fbr/fbr-trunk-x86/buildroot/output/target/lib/libpthread
.so.0
(gdb) \textbf%frame 1!\enter
#1  0xb768a251 in rrd_write (rrd_file=rrd_file@entry=0x808e930, buf=0x808e268,
    count=count@entry=112) at rrd_open.c:716
716         memcpy(rrd_simple_file->file_start + rrd_file->pos, buf, count);
(gdb) \textbf%print (char*) buf!\enter
$1 = 0x808e268 "RRD"
(gdb) \textbf%print rrd_simple_file->file_start!\enter
value has been optimized out
(gdb) \textbf%list!\enter
711         if((rrd_file->pos + count) > old_size)
712         {
713             rrd_set_error("attempting to write beyond end of file");
714             return -1;
715         }
716         memcpy(rrd_simple_file->file_start + rrd_file->pos, buf, count);
717         rrd_file->pos += count;
718         return count;       /* mimmic write() semantics */
719     #else
720         ssize_t   _sz = write(rrd_simple_file->fd, buf, count);
(gdb) \textbf%list 700!\enter
695      * rrd_file->pos of rrd_simple_file->fd.
696      * Returns the number of bytes written or <0 on error.  */
697
698     ssize_t rrd_write(
699         rrd_file_t *rrd_file,
700         const void *buf,
701         size_t count)
702     {
703         rrd_simple_file_t *rrd_simple_file = (rrd_simple_file_t *)rrd_file->
pvt;
704     #ifdef HAVE_MMAP
(gdb) \textbf%print *(rrd_simple_file_t *)rrd_file->pvt!\enter
$2 = {fd = 9, file_start = 0xa67d0000 <Address 0xa67d0000 out of bounds>,
  mm_prot = 3, mm_flags = 1}
\end{Verbatim}
\end{example}

Hier sieht man nach etwas ``Wühlen'', dass sich in dem
\texttt{rrd\_simple\_file\_t}-Objekt ein ungültiger Zeiger befindet
(``Address ... out of bounds''). Im weiteren Debugging-Verlauf wurde deutlich,
dass ein gescheiterter \texttt{posix\_fallocate}-Aufruf die Ursache für den
Programmabsturz war.

Wichtig hierbei ist, dass \emph{alle} anzugebenden Pfade voll qualifiziert sind
(\texttt{/project/...}) und dass man auch keine ``Abkürzungen'' (etwa
\texttt{\~{}/...}) verwendet. Wenn man dies nicht beachtet, kann es passieren,
dass \texttt{gdb} die Debug-Informationen zur Anwendung und/oder zu den
verwendeten Bibliotheken nicht findet. Die Debug-Informationen sind nämlich
ais Platzgründen nicht direkt in dem untersuchten Programm enthalten, sondern in
einer separaten Datei unterhalb des Verzeichnisses
\texttt{\~{}/.fbr/fbr-<branch>-<arch>/buildroot/output/debug/} gespeichert.

\subsection{Informationen über das FBR}

\subsubsection{Anzeige der Hilfe}

Was \texttt{fbr-make} alles für Sie tun kann, können Sie sich mit dem Kommando
\texttt{fbr-make help} ausgeben lassen.

\subsubsection{Anzeige von Programminformationen}

Sie können sich alle verfügbaren FBR-Pakete sowie deren Versionen anschauen,
indem Sie das Kommando \texttt{fbr-make show-versions} nutzen:

\begin{example}
\begin{Verbatim}[commandchars=\\\{\}]
$ \textbf{fbr-make show-versions}\enter
Configured packages

acpid 2.0.20
actctrl 3.25+dfsg1
add-days undefined
[...]
\end{Verbatim}
\end{example}

\subsubsection{Anzeige von Bibliotheksabhängigkeiten}

Mit Hilfe des Kommandos \texttt{fbr-make links-against <soname>} können Sie sich
alle Dateien in \texttt{\~{}/.fbr/fbr-<branch>-<arch>/buildroot/output/target}
anzeigen lassen, die gegen eine Bibliothek mit dem Bibliotheksnamen
\texttt{soname} gebunden sind. Um beispielsweise alle Programme und Bibliotheken
zu identifizieren, welche die \texttt{libm} (Bibliothek mit mathematischen
Funktionen) verwenden, nutzen Sie das Kommando \texttt{fbr-make links-against
libm.so.0}, da \texttt{libm.so.0} der Bibliotheksname der libm-Bibliothek ist.
Eine mögliche Ausgabe ist:

\begin{example}
\begin{Verbatim}[commandchars=\\\{\}]
$ \textbf{fbr-make links-against librrd_th.so.4}\enter
Executing plugin links-against
Files linking against librrd_th.so.4
collectd usr/lib/collectd/rrdcached.so
collectd usr/lib/collectd/rrdtool.so
rrdtool usr/bin/rrdcached
\end{Verbatim}
\end{example}

Dabei steht in der ersten Spalte der Paketname und in der zweiten der (relative)
Pfad zu der Datei, die gegen die angegebenene Bibliothek gebunden ist.

Um den Bibliotheksnamen für eine Bibliothek herauszufinden, können Sie
\texttt{readelf} wie folgt nutzen:

\begin{example}
\begin{Verbatim}[commandchars=\\\{\}]
$ \textbf{readelf -d ~/.fbr/fbr-trunk-x86/buildroot/output/target/lib/libm-0.9.33.2.so |}\enter
> \textbf{grep SONAME}\enter
 0x0000000e (SONAME)                     Library soname: [libm.so.0]
\end{Verbatim}
\end{example}

\subsubsection{Anzeige von Versionsänderungen}

(Nur) für fli4l-Team-Entwickler mit Schreibzugriff auf das fli4l-SVN-Repository
ist das Kommando \texttt{fbr-make version-changes} interessant. Es listet alle
FBR-Pakete auf, deren Version lokal modifiziert wurde, deren Version in der
Arbeitskopie also von der Repository-Version abweicht. Damit kann der Entwickler
sich einen Überblick über aktualisierte FBR-Pakete verschaffen, etwa bevor er
die Änderungen ins Repo schreibt. Eine mögliche Ausgabe ist:

\begin{example}
\begin{Verbatim}[commandchars=\\\{\}]
$ \textbf{fbr-make version-changes}\enter
Executing plugin version-changes
Package version changes
KAMAILIO: 4.0.5 --> 4.1.1
\end{Verbatim}
\end{example}

Hier sieht man sofort, dass das \texttt{kamailio}-FBR-Paket von der Version
4.0.5 auf die Version 4.1.1 aktualisiert worden ist.

\subsection{Ändern der FBR-Konfiguration}

\subsubsection{Rekonfiguration des FBRs}

Mit Hilfe des Kommandos \texttt{fbr-make buildroot-menuconfig} ist es zum einen
möglich, die zu übersetzenden FBR-Pakete auszuwählen. Das ist nützlich, wenn Sie
andere FBR-Pakete für den fli4l übersetzen möchten, die standardmäßig nicht
aktiviert sind, aber im uClibc-Buildroot integriert sind, oder wenn Sie
eigene FBR-Pakete aktivieren wollen. Zum anderen können andere, globale
Eigenschaften des FBRs verändert werden, etwa die Version des verwendeten
GCC-Compilers. Beim erfolgreichen Beenden des Konfigurationsmenüs wird die neue
Konfiguration im Verzeichnis \texttt{src/fbr/buildroot/.config} gespeichert.

\achtung{Beachten Sie jedoch bitte, dass solche Änderungen der
Toolchain-Konfiguration offiziell \emph{nicht} unterstützt werden, weil die
resultierenden Binärprogramme mit hoher Wahrscheinlichkeit inkompatibel mit der
offiziellen fli4l-Distribution werden. Wenn Sie also Binärprogramme für Ihr
eigenes OPT benötigen und dieses OPT veröffentlichen wollen, sollten Sie keine
Toolchain-Einstellung verändern!}

\subsubsection{Rekonfiguration der uClibc-Bibliothek}

Mit dem Kommando \texttt{fbr-make uclibc-menuconfig} kann die Funktionalität der
verwendeten uClibc-Bibliothek verändert werden. Beim erfolgreichen Beenden
des Konfigurationsmenüs wird die neue Konfiguration in
\texttt{src/fbr/buildroot/package/uclibc/uclibc.config} gespeichert.

\achtung{Wie im letzten Abschnitt gilt auch hier: Eine Änderung ist mit hoher
Wahrscheinlichkeit nicht kompatibel mit der offiziellen fli4l-Distribution und
wird somit nicht unterstützt!}

\subsubsection{Rekonfiguration der Busybox}

Mit Hilfe des Kommandos \texttt{fbr-make busybox-menuconfig} kann die Busybox
in ihrer Funktionalität angepasst werden. Beim erfolgreichen Beenden
des Konfigurationsmenüs wird die neue Konfiguration in
\texttt{src/fbr/buildroot/package/busybox/busybox-<Version>.config}
gespeichert.

\achtung{Auch hier gilt: Eine Änderung ist
höchstwahrscheinlich nicht kompatibel mit der offiziellen fli4l-Distribution und
wird somit nicht unterstützt! Allenfalls das Ergänzen der Busybox um neue
Applets ist problemlos, solange Sie die so modifizierte Busybox nur auf Ihren
fli4l-Routern einsetzen (und nicht vom Nutzer Ihres OPTs den Einsatz einer
derart modifizierten Busybox verlangen).}

\subsubsection{Rekonfiguration der Linux-Kernelpakete}

Mit \texttt{fbr-make linux-menuconfig} bzw.
\texttt{fbr-make linux-<version>-menuconfig} kann die Konfiguration aller
aktivierten Kernel-Pakete bzw. eines bestimmten Kernel-Pakets vorgenommen
werden. Beim erfolgreichen Beenden des Konfigurationsmenüs wird die neue
Konfiguration in
\texttt{src/fbr/buildroot/linux/linux-<version>/dot-config-<arch>}
gespeichert.\footnote{Dies gilt nur für den Standard-Kernel. Für Varianten eines Kernelpakets wird stattdessen eine
\texttt{diff}-Datei in
\texttt{src/fbr/buildroot/linux/linux-<version>/linux-<version>\_<variante>/dot-config-<arch>.diff}
abgelegt.}

\achtung{Wie im letzten Abschnitt gilt auch hier: Eine Änderung ist
höchstwahrscheinlich nicht kompatibel mit der offiziellen fli4l-Distribution und
wird somit nicht unterstützt! Allenfalls das Ergänzen des Linux-Kernels um neue
Module ist problemlos, solange Sie den so modifizierten Kernel nur auf Ihren
fli4l-Routern einsetzen (und nicht vom Nutzer Ihres OPTs den Einsatz eines
derart modifizierten Kernels verlangen).}

\subsection{Aktualisierung des FBRs}

Jedem der beschriebenen Kommandos geht eine Prüfung des FBRs auf Aktualität
voraus. Wird eine Diskrepanz zwischen den Quellen, in denen sich
\texttt{fbr-make} befindet (also entpacktes \texttt{src}-Paket oder
SVN-Arbeitskopie) und dem FBR in
\texttt{\~{}/.fbr/fbr-<branch>-<arch>/buildroot} entdeckt, wird letzteres auf
den neuesten Stand gebracht. Dabei werden neu dazugekommene FBR-Pakete
integriert sowie alte, nicht mehr enthaltene FBR-Pakete gelöscht. Auch die
Konfigurationen werden verglichen: FBR-Pakete mit veränderter Konfiguration
sowie alle davon abhängigen FBR-Pakete werden neu gebaut. Dies stellt sicher,
dass das FBR auf Ihrem Computer immer dem der fli4l-Entwickler entspricht (mit
Ausnahme Ihrer eigenen FBR-Pakete unterhalb von \texttt{\~{}/.fbr/own/}).
\textbf{Das bedeutet jedoch auch, dass Änderungen am offiziellen Teil der
Buildroot-Konfiguration bei der nächsten Aktualisierung verloren gehen!} Auch
deshalb ist eine Rekonfiguration des FBRs (s.o.) somit nicht zu empfehlen,
zumindest nicht, wenn Sie mit \texttt{src}-Paketen anstatt mit SVN-Arbeitskopien
arbeiten. (Bei der Aktualisierung einer SVN-Arbeitskopie werden Ihre lokalen
Konfigurationsänderungen und die Änderungen im SVN-Repository zusammengeführt,
so dass das Problem der verlorenen Konfiguration hier nicht auftritt.) Hingegen
können Sie Ihre eigenen FBR-Pakete problemlos umkonfigurieren, ohne dass es bei
einer Aktualisierung zu Datenverlusten kommt.

\subsection{Eigene Programme ins FBR einbinden}

Die Übersetzung der einzelnen FBR-Pakete wird über kleine Makefiles
gesteuert. Will man eigene FBR-Pakete entwickeln, so muss man ein Makefile sowie
eine Konfigurationsbeschreibung unter \texttt{\~{}/.fbr/own/<paket>/} ablegen.
Wie diese aufgebaut sind und wie man daraus abgeleitet eigene Makefiles
schreiben kann, wird in der Dokumentation des uClibc-Buildroots unter
\altlink{http://buildroot.uclibc.org/downloads/manual/manual.html\#adding-packages}
ausführlich beschrieben.
