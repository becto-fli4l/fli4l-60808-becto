% Do not remove the next line
% Synchronized to r29817

\marklabel{sec:optppp}
{
\section {PPP - Connecter un ordinateur via le port série}
}
\configlabel{OPT\_PPP}{OPTPPP}

    En mettant la variable \var{OPT\_\-PPP='yes'} sur yes, il sera
    possible d'utiliser un PC via le port série pour se connecter au réseau local.
    Ceci peut être utile pour intégrer dans le réseau par ex. un ordinateur
    portable, qui n'a pas de carte réseau. Voici, une documentation sur
    l'interface série pour PC-client

\begin{description}
\config{PPP\_DEV}{PPP\_DEV}{PPPDEV}

        On indiquer ici, le port série de fli4l. Les valeurs suivantes
        sont permises~:

        \begin{tabular}[h]{l l}
          'com1' &           Port-COM1 (en minuscules!) \\
          'com2' &           Port-COM2 (en minuscules!) \\
        \end{tabular}

\config{PPP\_SPEED}{PPP\_SPEED}{PPPSPEED}

        On indiquer ici la vitesse de transmission (bit/sec). 38400 est
        supporté par les anciennes interfaces. Il peut y avoir éventuellement
        des problèmes si l'on paramètre les taux trop élevés de 57600, voire 115200 bit/s.

\begin{example}
\begin{verbatim}
    Exemple : PPP_SPEED='38400'
\end{verbatim}
\end{example}



\configlabel{PPP\_IPADDR}{PPPIPADDR}
\configlabel{PPP\_PEER}{PPPPEER}
\configvar {PPP\_IPADDR PPP\_PEER}

        On paramètre dans la variable \var{PPP\_\-IPADDR} l'adresse IP
        du routeur fli4l pour la connextion au port-COM, par ex. '192.168.4.1'
        et dans la variable \var{PPP\_\-PEER} l'adresse IP du PC-client,
        par exemple '192.168.4.2'.


\configlabel{PPP\_NETWORK}{PPPNETWORK}
\configlabel{PPP\_NETMASK}{PPPNETMASK}
\configvar {PPP\_NETWORK PPP\_NETMASK}

        On paramètre dans la variable \var{PPP\_\-NETWORK} le réseau utilisé,
        par ex. '198.168.4.0' et dans la variable \var{PPP\_\-NETMASK} Masque
        de sous-réseau utilisé, par ex. '255.255.0.0'. Ces deux variables sont
        complémentaires au paquetage 'samba\_lpd'.

    \wichtig{Il faut faire attention aux points suivants~:}
    \begin{enumerate}
    \item  Les adresses IP ne doivent \textbf{pas} provenir de la plage
      d'adresses du réseau Ethernet LAN, mais il faut avoir pour utiliser
      la configuration Point-to-Point une plage d'adresses réseau propre~!


     \item  2.	Pour que le PC-client puisse aussi avoir une connexion Internet
       il faut ajouter, un mini-réseau du type PPP et aussi de masquer le LAN
       (ou réseau local). Le réseau doit être également enregistré dans
       \jump{MASQNETWORK}{\var{MASQ\_NETWORK}} (voir ce chapitre).


     \item  3.	En plus, vous devez ajouter le PC-client dans la table d'hôte
       pour le serveur DNS du routeur fli4l.
    \end{enumerate}

       La raison est la suivante~:

       Si vous souhaitez utiliser le PC-client avec telnet ou ftp pour se connecter,
       le démon du routeur fli4l fait du Reverse-DNS-Lookup, afin de d'établir
       avec le client une connecter au réseau local. Si le PC-client ne figure pas
       dans la table d'hôte, fli4l établit une connexion Internet, pour rechercher
       le nom du client sur Iternet. C'est pour cela qu'il faut absolument écrire
       le PC-client dans la table d'hôte du routeur fli4l.

       Exemple de configuration avec liaison PPP sur le port série~:

\begin{example}
\begin{verbatim}
        PPP_DEV='com1'
        PPP_SPEED='38400'
        PPP_IPADDR='192.168.4.1'
        PPP_PEER='192.168.4.2'
        PPP_NETWORK='192.168.4.0'
        PPP_NETMASK='255.255.255.0'
\end{verbatim}
\end{example}

           Et en plus dans le fichier config/base.txt~:

\begin{example}
\begin{verbatim}
        MASQ_NETWORK='192.168.6.0/24 192.168.4.0/24'    # ancienne configuration
        PF_POSTROUTING_1='192.168.6.0/24 192.168.4.0/24' MASQUERADE    # nouvelle configuration
\end{verbatim}
\end{example}

           ci-dessus le premier numéro du réseau concerne le LAN-Ethernet,
           et le second concerne le réseau PPP.

           Si vous ne possédez pas de circuit ISDN sur le réseau 192.168.x.x
           qui doit être accessible de l'interne (chez moi - à cause de mon entreprise
           j'ai une telle connexion supplémentaire). En résumer, il faudrait que
           les deux réseaux soient \jump{MASQNETWORK}{\var{MASQ\_NETWORK}} on peut
           simplifier les règles pour le firewall (ou pare-feu).

           Par conséquent, le mieux~:

\begin{example}
\begin{verbatim}
        MASQ_NETWORK='192.168.0.0/16'    # ancienne configuration
        PF_POSTROUTING_1='192.168.0.0/16' MASQUERADE    # nouvelle configuration
\end{verbatim}
\end{example}

           C'est-à-dire~: "Masquer tout ce qui concerne et qui commence par 192.168".


           Et enfin il faut toujours configurer le DNS, par exemple~:

\begin{example}
\begin{verbatim}
        HOST_5='192.168.4.2 serial-pc'
\end{verbatim}
\end{example}

           Ne pas oublier d'incrémenter la variable \jump{HOSTN}{\var{HOST\_\-N}}~!

           Si le PC-client est un ordinateur Windows, vous devez configurer la carte
           d'accès distant pour une connexion PPP, pour pouvoir accéder au routeur fli4l.

           Lorsque vous utilisez un ordinateur sous Linux, le mieux est de créer un script
           shell qui sera installé sur le PC-client (par exemple /usr/local/bin/ppp-on)~:
\begin{example}
\begin{verbatim}
    #! /bin/sh
    dev='/dev/ttyS0'                    # COM1, für COM2: ttyS1
    speed='38400'                       # Speed
    options='defaultroute crtscts'      # Options
    myip='192.168.4.2'                  # IP-Adresse Notebook
    fli4lip='192.168.4.1'               # IP-Adresse fli4l-Router
    pppd $dev $speed $options $myip:$fli4lip &
\end{verbatim}
\end{example}

        Si vous avez des problèmes avec le démon pppd


        L'ordinateurs-fli4l doit également être enregistré dans le serveur de DNS
        sur le PC-client, si on souhaite se connecter à Internet. Il doit être
        enregistré dans le fichier /etc/resolv.conf du PC-client, entrer les deux
        lignes suivantes~: le Nom de domaine et adresse IP Ethernet du routeur fli4l.

        Exemple~:

\begin{example}
\begin{verbatim}
    search domain.de
    nameserver 192.168.1.4
\end{verbatim}
\end{example}

        Les valeurs correspondantes aux "Domain.de" et "192.168.1.4" sont à remplacer
        par vos paramètres. Important~: L'adresse IP doit être celle de la carte Ethernet
        du routeur-fli4l!


        La liaison série se fait avec un \jump{NULLMODEMKABEL}{cable-Nullmodem}. Des
        informations à ce sujet sont dans l'annexe de la documentation Base.

        Walter Oliver a rédigé un Howto (ou un guide), pour configurer le pc-client
        Windows avec le protocole PPP qui peut être lu ici~:

            \altlink{http://www.fli4l.de/hilfe/howtos/basteleien/opt-ppp-howto/}


\end{description}
