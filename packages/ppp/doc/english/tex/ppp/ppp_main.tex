% Synchronized to r29817
\marklabel{sec:optppp}
{
\section {PPP - Connection Of Computers Via Serial Interface}
}
\configlabel{OPT\_PPP}{OPTPPP}

    With \var{OPT\_\-PPP='yes'} a computer can be connected via its serial 
    interface. This may be useful i.e. for connecting computers to a network 
    that does not have a network interface card. The computer connected to the serial
    interface is called client PC below.

\begin{description}
\config{PPP\_DEV}{PPP\_DEV}{PPPDEV}

        Specify fli4l's serial interface the client PC is connected to. Possible 
        values are:

        \begin{tabular}[h]{l l}
          'com1' &           COM1 port (lower cases only!) \\
          'com2' &           COM2 port (lower cases only!) \\
        \end{tabular}

\config{PPP\_SPEED}{PPP\_SPEED}{PPPSPEED}

        Set transfer rate here (bit/sec). 38400 is supported by older interfaces too. 
        Problems may occur when using rates above this.

\begin{example}
\begin{verbatim}
    Example: PPP_SPEED='38400'            
\end{verbatim}
\end{example}



\configlabel{PPP\_IPADDR}{PPPIPADDR}
\configlabel{PPP\_PEER}{PPPPEER}
\configvar {PPP\_IPADDR PPP\_PEER}

        \var{PPP\_\-IPADDR} holds fli4l's  IP address on the COM port, i.e. '192.168.4.1'. 
        In variable \var{PPP\_\-PEER} the IP address of the client PC is set, i.e. '192.168.4.2'.

        
\configlabel{PPP\_NETWORK}{PPPNETWORK}
\configlabel{PPP\_NETMASK}{PPPNETMASK}
\configvar {PPP\_NETWORK PPP\_NETMASK}        
        
        \var{PPP\_\-NETWORK} holds the network used and variable \var{PPP\_\-NETMASK} 
        the netmask. These two variables are used by the extra package 'samba\_lpd'.

    \wichtig{Consider the following:}
    \begin{enumerate}
    \item Those IP addresses may \textbf{not} originate from the network address range of the 
      ethernet LANs. Point-To-Point-Configuration must use a separate network address range!

    
     \item  For the Client PC's connection to the Internet this mini-PPP-network has to be 
       masked in the same way as the LAN. The net has to be entered additionally in 
       \jump{MASQNETWORK}{\var{MASQ\_NETWORK}} (see next chapter) as well.

    
     \item  The client PC should be added to the host table of the DNS server.
    \end{enumerate}

       Causes:

       If telnet or ftp should be used from the client PC to the fli4l router 
       the daemons concerned on fli4l do a reverse DNS lookup to resolve the 
       client PC. If it is not found in the host table a connection to the Internet 
       will be established to do this - which is of course useless. To avoid
       this enter the client PC in the host table.

       Example for PPP Configuration over the serial interface:

\begin{example}
\begin{verbatim}
        PPP_DEV='com1'
        PPP_SPEED='38400'
        PPP_IPADDR='192.168.4.1'
        PPP_PEER='192.168.4.2'
        PPP_NETWORK='192.168.4.0'
        PPP_NETMASK='255.255.255.0'
\end{verbatim}
\end{example}

           and further on in config/base.txt:

\begin{example}
\begin{verbatim}
        MASQ_NETWORK='192.168.6.0/24 192.168.4.0/24'
\end{verbatim}
\end{example}
           
           The first network number is the one of the ethernet LAN and the 
           second the one of the PPP network.
           
           As long as there is no other net in the 192.168. range in use 
           both nets can be joined in \jump{MASQNETWORK}{\var{MASQ\_NETWORK}}. 
           This simplifies configuring firewall rules.

           In this case use:

\begin{example}
\begin{verbatim}
        MASQ_NETWORK='192.168.0.0/16'
\end{verbatim}
\end{example}

           this masks every net starting with 192.168.


           Last thing is to adapt DNS Configuration. Example:

\begin{example}
\begin{verbatim}
        HOST_5='192.168.4.2 serial-pc'
\end{verbatim}
\end{example}

           Do not forget to increment \jump{HOSTN}{\var{HOST\_\-N}}!
           
           If the client PC is running Windows you have to configure 
           the dial-up adapter for a PPP connection to the fli4l router.
           
           If a linux client is used create a shell script on the client (i.e.
           /usr/local/bin/ppp-on):
\begin{example}
\begin{verbatim}
    #! /bin/sh
    dev='/dev/ttyS0'                    # COM1, for COM2: ttyS1
    speed='38400'                       # speed
    options='defaultroute crtscts'      # options
    myip='192.168.4.2'                  # IP address client
    fli4lip='192.168.4.1'               # IP address fli4l router
    pppd $dev $speed $options $myip:$fli4lip &
\end{verbatim}
\end{example}

        In case of problems: man pppd

        
        The fli4l router has to be used as the DNS server on the client 
        if a connection to the Internet is desired. Add two lines to /etc/resolv.conf 
        on the client: the domain used and the ethernet IP address of the fli4l router 
        as name server.

        Example:

\begin{example}
\begin{verbatim}
    search domain.de
    nameserver 192.168.1.4
\end{verbatim}
\end{example}
        
        ``domain.de'' res. ``192.168.1.4'' have to be changed to your needs. 
        Important: The IP address has to be the one of fli4l's ethernet card!

        
        A so called \jump{NULLMODEMKABEL}{null modem cable} is used as the physical connection. 
        See appendix for package base for pin wirings.
        
        A (german) Howto for connecting a Windows client with serial PPP can be found at:
        
            \altlink{http://www.fli4l.de/hilfe/howtos/basteleien/opt-ppp-howto/}


\end{description}
