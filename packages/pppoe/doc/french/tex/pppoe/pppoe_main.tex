% Do not remove the next line
% Synchronized to r29817

\section{DSL~- DSL pour PPPoE, Fritz!DSL et PPTP}

fli4l supporte la connexion DSL avec trois variantes différentes~:

\begin{itemize}
\item PPPoE (Modem-DSL externe, avec un raccordement Ethernet, il supporte
  le protocole pppoe)
\item PPTP (Modem externe, avec un raccordement Ethernet, il supporte
  le protocole pptp)
\item Fritz!DSL (DSL avec adaptateur-DSL de AVM)
\end{itemize}

On peut choisir une seule connexion DSL, l'exploitation de plusieurs
connexions différentes n'est malheureusement pas encore possible.

La configuration de ces différentes variantes se ressemble, c'est pour
cela que nous allons décrire d'abord les variables générales, ensuite
nous évoquerons séparément les variables spécifiques avec leurs options.
Imond administre l'accés-DSL comme un Circuit\footnote {Malheureusement
en ce moment un seul Circuit-DSL est possible~-- Si vous voulez utiliser
plusieurs Circuits, il faut construire plusieurs média}, lorsque
l'on active l'une des variante-DSL vous pouvez activer imond
(\jump{OPTIMOND}{voir \var{OPT\_IMOND}}).

\subsection {Variables de configuration générales}

Les paquetages utilisent les mêmes variables de configuration, ils se
distinguent uniquement par leurs noms de paquetages placés en tête. Par ex.
dans tous les paquets on demande le nom d'utilisateur, la variable
s'appelle selon le paquetage \var{PPPOE\_USER}, \var{PPTP\_USER} ou
\var{FRITZDSL\_USER}. Par la suite, les variables seront décrites sans
leur préfixe, le préfixe manquant sera remplacé par une étoile donc
dans l'exemple concret la partie manquante est PPPOE (cela est valable
pour tous les autres préfixes).

\begin{description}
\configlabel{PPPOE\_NAME}{PPPOENAME}
\configlabel{PPTP\_NAME}{PPTPNAME}
\configlabel{FRITZDSL\_NAME}{FRITZDSLNAME}
\item[*\_NAME]

Dans cette variable nous devons donner un nom pour le circuit~- un maximun
de 15 caractères. il sera affiché dans le Client-imonc. Le nom ne doit
pas contenir d'espaces.

Exemple~: \var{PPPOE\_\-NAME}='DSL'

\configlabel{PPPOE\_USEPEERDNS}{PPPOEUSEPEERDNS}
\configlabel{PPTP\_USEPEERDNS}{PPTPUSEPEERDNS}
\configlabel{FRITZDSL\_USEPEERDNS}{FRITZDSLUSEPEERDNS}
\item[*\_USEPEERDNS]

On détermine ici si les fournisseurs d'accès Internet utilisent un serveur
de nom (ou DNS), nous devons enregistrer ce serveur de nom dans notre réseau
local pour la durée de connexion Internet. 

Logiquement cette option est uniquement utilisée pour la connexion au
fournisseur d'accès Internet. En même temps, presque tous les Fournisseurs
supportent ce type de transfert.

Vous devez enregistrer les adresses-IP du serveur de nom (ou DNS) fournies par votre
FAI dans le fichier base.txt dans la variable \var{DNS\_\-FORWARDERS} et vous
devez supprimer le serveur de nom qui est configuré sur votre PC du réseau
local. Ensuite vous devez mettre à la place, l'adresse IP de votre routeur. Avec
ce réglage la résolution des noms ne se perd pas dans le cache du serveur de nom.

Cette option offre l'avantage de toujours pouvoir travailler avec un serveur de
nom le plus proche, dans la mesure où le fournisseur d'accès à une adresse IP
correcte~- ainsi, la résolution de nom sera plus rapide.

En cas d'une défaillance d'un serveur de DNS du fournisseur d'accès, en régle
général, on pourra corriger plus rapidement la transmission des adresses des
serveurs DNS du fournisseur d'accès.

Malgré tout, il est nécessaire d'indiquer un serveur de nom valide dans la
variable \var{DNS\_\-FORWARDERS} du fichier base.txt pour se connecter,
autrement lors de la première connexion Internet la demande ne pourra pas être
résolue correctement. En outre, la configuration originale du serveur de nom
local est restaurée à la fin de la connexion.

Configuration standard~: \var{*\_\-USEPEERDNS}='yes'

\configlabel{PPPOE\_DEBUG}{PPPOEDEBUG}
\configlabel{PPTP\_DEBUG}{PPTPDEBUG}
\configlabel{FRITZDSL\_DEBUG}{FRITZDSLDEBUG}
\item[*\_DEBUG]

Pour avoir des informations de débogages supplémentaires par exemple sur pppd,
vous devez activer la variable \var{*\_\-DEBUG} régler celle-ci sur 'yes'.
Ces informations supplémentaires sur pppd seront enregistrées avec interface syslogd

IMPORTANT~: pour que le démon syslogd fonctionne, il faut activer la variable
\var{OPT\_\-SYSLOGD} régler celle-ci sur 'yes'

\configlabel{PPPOE\_USER}{PPPOEUSER}
\configlabel{PPTP\_USER}{PPTPUSER}
\configlabel{FRITZDSL\_USER}{FRITZDSLUSER}
\configlabel{PPPOE\_PASS}{PPPOEPASS}
\configlabel{PPTP\_PASS}{PPTPPASS}
\configlabel{FRITZDSL\_PASS}{FRITZDSLPASS}
\item[*\_USER, *\_PASS]

Dans ces variables on indique les données du Fournisseur d'accès.
\var{*\_USER} l'identification de l'utilisateur, \var{*\_PASS} le mot de passe.

\emph{ATTENTION~:} pour un accès au FAI T-Online (pour l'Allemagne) il est à noter~:

Le nom d'utilisateur AAAAAAAAAAAATTTTTT\#MMMM est composé, du numéro
co-utilisateur, puis du numéro T-online de 12 chiffres et de
l'identification. Le dernier chiffre du numéro T-Online doit se terminer
par '\#' si le numéro de T-Online ne comporte pas les 12 chiffres.

Avec ca si cela ne fonctionne pas~! (évidemment cela peut provenir du centrale
téléphonique), le caractère '\#' doit être placé entre le numéro T-Online
et l'identification.

Évidemment si le (numéro T-Online comporte les 12 chiffres) il n'y a pas
besoin de mettre le caractère '\#'.

Le nom d'utilisateur de T-Online doit se terminé par '@t-online.de'

Exemple~:

\begin{example}
\begin{verbatim}
        PPPOE_USER='111111111111222222#0001@t-online.de'
\end{verbatim}
\end{example}

Des infos sur la configuration des autres fournisseurs d'accès se
trouvent dans la FAQ~:
\begin{itemize}
\item \altlink{http://extern.fli4l.de/fli4l_faqengine/faq.php?list=category&catnr=3&prog=1}
\end{itemize}

\configlabel{PPPOE\_HUP\_TIMEOUT}{PPPOEHUPTIMEOUT}
\configlabel{PPTP\_HUP\_TIMEOUT}{PPTPHUPTIMEOUT}
\configlabel{FRITZDSL\_HUP\_TIMEOUT}{FRITZDSLHUPTIMEOUT}
\item[*\_HUP\_TIMEOUT]

On peut paramétrer dans cette variable le Timeout (ou le temps d'arrêt)
en seconde, s'il y a aucune transmission sur le circuit DSL 
(ou Internet), la connexion se coupera. Vous disposez des paramètres,
pour régler le délai '0' ou aucun délai 'never'~- Si vous sélectionnez
'never' le routeur a en plus une nouvelle contrainte, c'est de raccrocher
immédiatement. Les modifications dans mode Dial ne sont pas possibles~--
Le réglage doit être sur 'auto' et doit y rester. Le paramètre 'never'
est uniquement utilisé pour PPPOE et FRITZDSL.

\configlabel{PPPOE\_CHARGEINT}{PPPOECHARGEINT}
\configlabel{PPTP\_CHARGEINT}{PPTPCHARGEINT}
\configlabel{FRITZDSL\_CHARGEINT}{FRITZDSLCHARGEINT}
\item[*\_CHARGEINT]

On utilise cette variable pour placer une unité de temps~: on paramètre
ici le coût par unité téléphonique en seconde. Pour calculer le prix
total des communications.

En Allemagne les FAI les plus gros, facture l'unité Tél. exactement à la
minute, le paramètre correct dans la variable est donc '60'. Certain
FAI facture l'unité exactement en seconde dans ce cas la variable
\var{*\_CHARGEINT} sera de '1'.

Malheureusement, les unités de temps DSL ne son pas exploitée pleinement,
comme avec ISND. c'est après le temps paramétré ici \var{*\_HUP\_\-TIMEOUT}
que la transmission se coupe.

C'est pour cette raison que la variable \var{*\_\-CHARGEINT} sert
uniquement à calculer le prix des communications

\configlabel{PPPOE\_TIMES}{PPPOETIMES}
\configlabel{PPTP\_TIMES}{PPTPTIMES}
\configlabel{FRITZDSL\_TIMES}{FRITZDSLTIMES}
\item[*\_TIMES]

Dans cette variable on paramètre temps d'activation et d'arrêt de la connexion,
ainsi que le prix de l'unité Tél. Il est possible d'activer des circuits
'Default-Route' différents et aussi l'utilisation de (Least-Cost-Routing).
Le contrôle du routage s'effectue avec le démon (ou programme) imond.

Structure de la variable~:

\begin{example}
\begin{verbatim}
        PPPOE_TIMES='times-1-info [times-2-info] ...'
\end{verbatim}
\end{example}

Il y a dans chaque times-?-info 4 sous-paramètres~- cet sous-paramètres sont
séparés par deux points (':').

\begin{enumerate}
\item Sous-paramètre~: W1-W2

  On indique ici les périodes des jours ouvrables, par ex. Mo-Fr ou Sa-Su,
  il est possible décrire les jours en Anglais ou en Allemand. Si l'on
  paramètre un seul jour, il sera écrit W1-W1 par ex. Su-Su.

\item Sous-paramètre~: hh-hh

  On indique ici la période horaire, par ex. 09-18 ou 18-09. De 18-09 est
  synonyme à 18-24 plus 00-09. de 00-24 correspond à toute une journée.


\item Sous-paramètre~: Charge

  On indique ici le prix par minute de connexion ou par unité téléphonique
  en euro, par ex. 0.032 correspond à 3.2 Centimes par minute. Les unités
  téléphonique, sont calculées en tenant compte du temps de conversion pour
  un coût réel, et seront alors affiché dans le client-imonc.

\item Sous-paramètre~: LC-Default-Route

  Le contenu de ce sous-paramètre peut être Y ou N. cela signifie~:

  \begin{tabular}[h!]{lp{11cm}}
    Y: & Autorise la plage horaire et LC-Routing (ou calcul des frais)
      avec Default-Route (ou routage par défaut).\\

    N: & Autorise la plage horaire et le calcul des frais Tél automatiquement
      avec LC-Routing, il n'est pas utilisé pour autre chose. \\
            \end{tabular}

\end{enumerate}

Exemple (lire cette exemple comme une seul ligne):

\begin{example}
\begin{verbatim}
        PPPOE_TIMES='Mo-Fr:09-18:0.049:N
                     Mo-Fr:18-09:0.044:Y
                     Sa-Su:00-24:0.039:Y'
\end{verbatim}
\end{example}

        \wichtig{les paramètres de la variable \var{*\_TIMES} doit
          couvrir toute la semaine. si ce n'est pas le cas, aucune
          connexion valide ne peut se produire.}

        \emph{Si vous avez placé le paramètre ("Y") pour LC-Default-Route-Circuits
          et que vous n'avez pas réglé la semaine complète, Il aura des
          interruptions dans la période de la semaine avec Default-Route.
          Alors il sera impossible de surfer sur Internet pendant cet périodes~!}


Encore un exemple simple~:

\begin{example}
\begin{verbatim}
        PPPOE_TIMES='Mo-Su:00-24:0.0:Y'
\end{verbatim}
\end{example}

Cette exemple est pour ceux qui utilise un Flatrate (ou forfait d'accès
internet illimité).

Encore une derrière remarque pour le LC-Routing~: \emph{Les jours fériés
sont traités comme un dimanche.}

\configlabel{PPPOE\_FILTER}{PPPOEFILTER}
\configlabel{PPTP\_FILTER}{PPTPFILTER}
\configlabel{FRITZDSL\_FILTER}{FRITZDSLFILTER}
\item[*\_FILTER]

fli4l se coupe automatiquement, si aucun transfert de donné ne se fait sur
l'interface pppoe pendant le temps paramétré dans timeout. Malheureusement,
l'interface, évalue également les transferts de données, qui viennent de
l'extérieur, par exemple, des tentatives de connexion avec des Clients-P2P
comme eDonkey. A présent on est contacté en permanence par d'autres clients
externes, il peut arriver que le routeur fli4l ne se coupe jamais.

Avoir la possibilité de filtrer le trafic et de couper la connexion, même si 
quelqu'un tente de se connecter.

Si on paramètre la variable \var{*\_\-FILTER} sur yes. seul le trafic produit
par sa propre machine est générée, le trafic qui vient de l'extérieur est
ignoré complètement. Lorsque le trafic entrant est généré, le routeur se
trouvant entre Internet et les ordinateurs réagit, et rejette celui-ci
par ex. une demande de connexion, de plus les quelques paquets sortant seront
ignorés. On peut voir le fonctionnement exact ici~:
\begin{itemize}
\item \altlink{http://www.fli4l.de/hilfe/howtos/basteleien/hangup-problem-loesen/} et
\item \altlink{http://web.archive.org/web/20061107225118/http://www.linux-bayreuth.de/dcforum/DCForumID2/46.html}.
\end{itemize}

Une description plus précise se trouve dans l'annexe sur l'expression et
l'intégration des codes de filtrages~- C'est intéressant, uniquement si
vous voulez effectuer des modifications voir à la fin de la doc.

\configlabel{FRITZDSL\_MTU}{FRITZDSLMTU}
\configlabel{FRITZDSL\_MRU}{FRITZDSLMRU}
\configlabel{PPPOE\_MTU}{PPPOEMTU}
\config{*\_MTU *\_MRU}{PPPOE\_MRU}{PPPOEMRU}

  Ces variables sont optionnelles, ont peut paramétrer le \textbf{MTU}
  (maximum transmission unit) et le \textbf{MRU} (maximum receive unit).
  Optionnelle signifie que les variables ne sont peut être pas dans le
  fichier de configuration, elles sont à insérer par l'utilisateur si besoin~! \\
  normalement le MTU et le MRU sont réglés à 1492. Ce réglage doit être
  modifier uniquement dans des cas exceptionnels~! Ces variables ne sont pas
  utilisées pour OPT \_PPTP.

\configlabel{FRITZDSL\_NF\_MSS}{FRITZDSLNFMSS}
\config{*\_NF\_MSS}{PPPOE\_NF\_MSS}{PPPOENFMSS}

 Avec certains fournisseurs d'accès les effets suivants se produisent~:
  \begin{itemize}
  \item Le navigateur Web reçoit un lien, mais après plus rien ne se passe.
  \item Fonctionne avec des petits Mail, mais pas avec des plus gros Mail.
  \item Avec la fonction ssh, coupure du scp après avoir établi une connexion.
  \end{itemize}

  Afin d'éviter ces problèmes, configurez fli4l avec le MTU par défaut.
  Dans certains cas, ce n'est pas suffisant, c'est pourquoi fli4l permet
  explicitement de composer avec MSS (message segment size) et d'indiquer
  une valeur fournie par le fournisseur d'accés. Si le FAI ne fournie rien,
  vous pouvez essayer 1412 c'est une bonne valeur de départ; la valeur est
  de 40 octets de moins par rapport à la valeur MTU ($mss = mtu~- 40$). Cette
  variable est optionnelle, ce qui signifie, que la variable n'est peut être
  pas dans le fichier de configuration, elle est à insérer par l'utilisateur
  en cas de besoin~! Ces variables ne sont pas utilisées pour OPT \_PPTP.

\end{description}
\marklabel{sec:pppoe}
{
\subsection {OPT\_PPPOE~- DSL avec PPPoE}
}

En règle générale, pour la communication via ADSL, les paquets PPPoE sont
nécessaires, parce que les fournisseurs d'accés ne fournissent pas de bon
routeur, mais simplement un modem DSL. Entre le routeur-fli4l et le modem
fournie, on utilise le protocole PPP, en particulier sur le réseau éthernet.

Une ou deux cartes éthernet peuvent être installées dans votre routeur-fli4l,
pour obtenir dans le cas échéant~:
\begin{itemize}
\item Une seul carte avec IP pour le LAN et le protocole PPP pour le modem-DSL
\item Deux cartes~: une avec IP pour le LAN, et l'autre avec PPP pour le modem-DSL
\end{itemize}

La meilleure option est la solution avec les deux cartes Ethernet. Parce que
les deux protocoles~- IP et PPPoE~- sont séparés l'un de l'autre.

Mais la méthode avec une carte Ethernet fonctionne aussi. Dans ce cas, le
Modem-DSL-T est simplement raccordé au Hub ou switch du réseau. Vous pouvez
avoir éventuellement une légère pertes de transmission lors du débit maximum.

Si vous avez des problèmes de communication entre le modem et la carte réseau,
vous pouvez essayer de ralentir la vitesse de la carte réseau, en passant
éventuellement en mode Half-Duplex. Toutes les cartes réseaux PCI peut être
configurées pour fonctionner dans les différents modes, mais seulement quelques
cartes ISA. Soit vous utilisez le programme ethtool qui est dans le paquetage
advanced\_networking, soit vous créez un support de démarrage sous DOS avec les
outils de configuration de la carte. Démarrer fli4l avec ce support et exécuter
l'outil de configuration de la carte pour choisir et enregistrer le mode de
fonctionnement plus lent pour la carte. Le programme de configuration et le pilote
sont généralement fournie sur une disquette à l'achat de la cartes, ils peuvent
aussi être téléchargable depuis le site Web du fabricant. Éventuellement, vous
pouvez rechercher et trouver des informations sur les cartes dans wiki~:
\begin{itemize}
\item \altlink{https://ssl.nettworks.org/wiki/display/f/Netzwerkkarten}
\end{itemize}

Si vous utilisez deux cartes, la première sera pour le LAN (ou réseau local) et
la seconde pour la connexion au modem DSL.

Seul la première carte, doit être paramétrée avec une adresse-IP.

En d'autres termes~:

\begin{example}
\begin{verbatim}
        IP_NET_N='1'                # Seule *Une* carte avec l'adresse IP~!
        IP_NET_1xxx='...'              # Les paramètres habituels
\end{verbatim}
\end{example}

Dans la variable \var{PPPOE\_\-ETH} on indique 'eth1' et pour la deuxième carte
Ethernet on définie *aucun* paramètre dans la variable \var{IP\_\-NET\_\-2-xxx}.

\begin{description}
\config{OPT\_PPPOE}{OPT\_PPPOE}{OPTPPPOE} \sloppypar{Activé pour la prise
  en charge PPPoE. Configuration Standard~: \var{OPT\_PPPOE}='no'.}

\config{PPPOE\_ETH}{PPPOE\_ETH}{PPPOEETH}

Nom des Interfaces-Ethernet

\begin{tabular}[h!]{ll}
  'eth0' & 1. Carte-Ethernet\\
  'eth1' & 2. Carte-Ethernet\\
  ...    &  ...\\
\end{tabular}

Configuration Standard~: \var{PPPOE\_\-ETH}='eth1'

\config{PPPOE\_TYPE}{PPPOE\_TYPE}{PPPOETYPE}

\emph{PPPOE} sert à transférer les paquets-PPP directement sur la ligne
ethernet. C. à d. dans la première étape les paquets-ppp de données sont
transmis par le Démon-ppp, puis dans la deuxième étape ils sont, transformé
en paquets-pppoe et envoyés sur Ethernet pour arriver sur le modem-DSL.
Dans la deuxième étape les paquets sont empaquetés par le Démon-pppoe ou par
le Kernel. Au moyen de la variable \var{PPPOE\_TYPE} on défini la manière
d'empaquetage des paquets-pppoe.

\begin{table}[h!]
  \centering
  \begin{tabular}{|l|p{10cm}|}
    \hline
    Valeur & Description \\
    \hline
    async & Les paquets sont créés avec Démon-pppoe; La
    communication entre \emph{pppd} et \emph{pppoed} est asynchrone. \\
    sync & Les paquets sont créés avec Démon-pppoe; La
    communication entre \emph{pppd} et \emph{pppoed} est synchrone.
    Cela conduit à une plus grande efficacité pour la communication,
    et une moindre charge du processeur. \\
    in\_kernel & Les paquets-ppp sont directement transformés par le
    Kernel-Linux, en paquet-pppoe. Il est alors inutile de communiquer
    avec le deuxième démon, donc une économie sur la transformation
    des paquets, et une moindre charge du processeur. \\
    \hline
  \end{tabular}
  \caption{Type de création de paquet-pppoe}
  \label{tab:pppoe-type}
\end{table}

Un utilisateur a fait une comparaison des différentes variantes avec un
ordinateur Fujitsu Siemens PCD-H P75, voici le tableau
\ref{tab:pppoe-type-load} Présentation des résultats\footnote{Les chiffres ont
été prélevés d'un message des newsgroups spline.fli4l et n'a été objet 
d'aucun examen. L'article du Message portait ID 
$<$caf9fk\$ala\$1@bla.spline.inf.fu-berlin.de$>$.}.
\begin{table}[h!]
  \centering
  \begin{tabular}[h]{|l|l|l|l|}
    \hline
    fli4l & NIC & Bande passante (en aval) & Charge du CPU \\
    \hline
    2.0.8 & rtl8029 + rtl8139 & 310 kB/s  &100\% \\
    2.0.8 & 2x 3Com Etherlink III & 305 kB/s & 100\% \\
    2.0.8 & SMC + 3Com Etherlink III & 300 kB/s & 100\% \\
    2.1.7 & SMC + 3Com Etherlink III & 375 kB/s & 40\%\\
    \hline
  \end{tabular}
  \caption{Bande passante et charge CPU pour pppoe}
  \label{tab:pppoe-type-load}
\end{table}

\config{PPPOE\_HUP\_TIMEOUT}{PPPOE\_HUP\_TIMEOUT}{PPPOEHUPTIMEOUT2}

Si on utilise in\_kernel pour le Type-PPPoE et dialmode auto, on peut
indiquer 'never' dans Timeout (ou temps d'arrêt). Le routeur ne raccrochera plus et se
reconnectera automatiquement après une déconnexion au FAI, Toute
modification ultérieure dans dialmodes ne sera plus possible.

\end{description}

\marklabel{sec:pppoecirc}
{
\subsection {OPT\_PPPOE\_CIRC~- Plusieurs Circuits-DSL avec PPPoE (Expérimental)}
}
\configlabel{OPT\_PPPOE\_CIRC}{OPTPPPOECIRC}
\configlabel{PPPOE\_CIRC\_N}{PPPOECIRCN}
\configlabel{PPPOE\_CIRC\_x\_NAME}{PPPOECIRCxNAME}
\configlabel{PPPOE\_CIRC\_x\_TYPE}{PPPOECIRCxTYPE}
\configlabel{PPPOE\_CIRC\_x\_USEPEERDNS}{PPPOECIRCxUSEPEERDNS}
\configlabel{PPPOE\_CIRC\_x\_ETH}{PPPOECIRCxETH}
\configlabel{PPPOE\_CIRC\_x\_USER}{PPPOECIRCxUSER}
\configlabel{PPPOE\_CIRC\_x\_PASS}{PPPOECIRCxPASS}
\configlabel{PPPOE\_CIRC\_x\_DEBUG}{PPPOECIRCxDEBUG}
\configlabel{PPPOE\_CIRC\_x\_HUP\_TIMEOUT}{PPPOECIRCxHUPTIMEOUT}
\configlabel{PPPOE\_CIRC\_x\_CHARGEINT}{PPPOECIRCxCHARGEINT}
\configlabel{PPPOE\_CIRC\_x\_TIMES}{PPPOECIRCxTIMES}
\configlabel{PPPOE\_CIRC\_x\_FILTER}{PPPOECIRCxFILTER}
\configlabel{PPPOE\_CIRC\_x\_MRU}{PPPOECIRCxMRU}
\configlabel{PPPOE\_CIRC\_x\_MTU}{PPPOECIRCxMTU}

Si l'on veut gérer plusieurs accès DSL, on peut le faire avec la variable
\var{OPT\_\-PPPOE\_\-CIRC}. Si on place cette variable sur \emph{yes},
on peut définir plusieurs Circuits-PPPOE. On détermine le nombre dans la
variable \var{PPPOE\_CIRC\_N}, les options sont identiques aux variables
\var{OPT\_PPPOE} précédente, Il faut simplement rajouter \emph{CIRC\_x},
par exemple \var{PPPOE\_CIRC\_x\_NAME} au lieu de \var{PPPOE\_NAME}.

\subsection {OPT\_FRITZDSL~- DSL avec Carte-Fritz!DSL}

Ici la connexion Internet est activé pour une carte Fritz!DSL.
On utiliser Fritz!Card DSL de AVM pour se connecter à Internet. Les 
pilotes de ces cartes ne sont pas sous licence GPL, c'est pour cela qu'il
ne sont pas livrés avec le paquetage DSL. Il est nécessaire de télécharger
ces pilotes à cette adresse \altlink{http://www.fli4l.de/fr/telechargement/version-stable/pilote-avm/}
et de les décompresser dans le répertoire fli4l.\\ 
Ces pilotes sont trop volumineux pour une disquette, il est absolument
nécessaire d'installer fli4l sur un disque dur, si l'on veut utiliser
ces pilotes.

la réalisation et la prise en charge des circuits pour les cartes
Fritz!Card DSL a été rendue possible avec le soutien amical de
Stefan Uterhardt ( \email{zer0@onlinehome.de} ).


\begin{description}

  \config{OPT\_FRITZDSL}{OPT\_FRITZDSL}{OPTFRITZDSL}

  \sloppypar{Activez cette variable pour la prise en charge des Fritz!Card DSL.
    Configuration Standard~: \var{OPT\_FRITZDSL}='no'.}

\config{FRITZDSL\_TYPE}{FRITZDSL\_TYPE}{FRITZDSLTYPE}

Il existe plusieurs cartes Fritz!, avec lesquelles une connexion-DSL
peut s'effectuer. On paramètre le type de carte dans la variable
\var{FRITZDSL\_TYPE}, les différents types disponible se trouvent
dans le tableau \ref{tab:fritz-karten}.

\begin{table}[htb]
  \centering
  \begin{tabular}{l|l}
    Type de carte & Application \\
    \hline
    fcdsl & Fritz!Card DSL \\
    fcdsl2 & Fritz!Card DSLv2\\
    fcdslsl & Fritz!Card DSL SL\\
    fcdslusb & Fritz!Card DSL USB\\
    fcdslslusb & Fritz!Card DSL SL USB\\
    fcdslusb2 & Fritz!Card DSL USBv2
  \end{tabular}
  \caption{Cartes-Fritz}
  \label{tab:fritz-karten}
\end{table}

        Configuration Standard~:

\begin{example}
\begin{verbatim}
        FRITZDSL_TYPE='fcdsl'
\end{verbatim}
\end{example}

\config{FRITZDSL\_PROVIDER}{FRITZDSL\_PROVIDER}{FRITZDSLPROVIDER}

On règle avec cette variable le type de serveur. Les options possibles sont~:\\
U-R2, ECI, Siemens, Netcologne, oldArcor, Switzerland, Belgium,
Austria1, Austria2, Austria3, Austria4

En Allemagne il s'agit presque toujours de UR-2. Siemens et ECI sont
uniquement utilisées pour d'ancien connexion.\\
Pour la Suisse et la Belgique Les options sont très explicites et pour
l'Autriche, il faut essayer.\\
Si quelqu'un a une meilleure option possible pour l'Autriche, vous pouvez
la communiquer merci.

        Configuration Standard~:

\begin{example}
\begin{verbatim}
        FRITZDSL_PROVIDER='U-R2'
\end{verbatim}
\end{example}

\end{description}

\subsection {OPT\_PPTP~- DSL avec PPTP pour Autriche/les Pays-Bas (EXPÉRIMENTAL)}

En Autriche (et dans d'autres pays européens), au lieu utiliser
PPPoE ils utilisent Protocole-PPTP. Là aussi, une carte Ethernet
est connectée au Modem-PPTP.

C'est à partir de la version 2.0 que l'accès au circuit avec le protocole
PPTP a été réalisé~- Avec le soutien amical de Rudolf Hämmerle
(\email{rudolf.haemmerle@aon.at}).

Deux cartes Ethernet sont utilisées pour la connexion PPTP. Cela
devrait être la première carte pour le raccordement au réseau
local (ou LAN), et la seconde pour la connexion au modem-DSL.

Seul la première carte, est paramétrée avec une adresse-IP.

En d'autres termes~:

\begin{example}
\begin{verbatim}
        IP_NET_N='1'                   # Seule *Une* carte avec l'adresse IP~!
        IP_NET_1xxx='...'              # Les paramètres habituels
\end{verbatim}
\end{example}

Dans la variable \var{PPTP\_\-ETH} on indique 'eth1' et pour la deuxième carte
Ethernet on définie *aucun* paramètre dans la variable \var{IP\_\-NET\_\-2-xxx}.

\begin{description}

  \config{OPT\_PPTP}{OPT\_PPTP}{OPTPPTP}

  \sloppypar{Activé pour la prise en charge avec une connexion PPTP.
  Configuration Standard~: \var{OPT\_PPTP}='no'.}

\config{PPTP\_ETH}{PPTP\_ETH}{PPTPETH}

Nom des Interfaces-Ethernet

\begin{tabular}[h!]{ll}
  'eth0' & 1. Carte-Ethernet\\
  'eth1' & 2. Carte-Ethernet\\
   ...   &  ...\\
\end{tabular}

Configuration Standard~: \var{PPTP\_\-ETH}='eth1'

\config{PPTP\_MODEM\_TYPE}{PPTP\_MODEM\_TYPE}{PPTPMODEMTYPE}

Il existe différents types de modems PPTP, pour effectuer une connexion-pptp.
Le type de modem est paramétré dans la variable \var{PPTP\_\-MODEM\_\-TYPE},
dans le tableau \ref{tab:pptp-modemtypen} sont énuméré les différents types
pour les internautes.

\begin{table}[htb]
  \centering
  \begin{tabular}{l|l}
    Type-Modem & Utilisation \\
    \hline
    bbaa & Autriche \\
    bcaa & Autriche \\
    xdsl & Autriche, Inode xDSL@home \\
    mxstream & les Pays-Bas
  \end{tabular}
  \caption{Type de Modem-PPTP}
  \label{tab:pptp-modemtypen}
\end{table}

        Configuration Standard~:

\begin{example}
\begin{verbatim}
        PPTP_MODEM_TYPE='bcaa'
\end{verbatim}
\end{example}

\end{description}

\subsubsection{Inode xDSL@home}

Le déploiement, et la mise en place pour la prise en charge de Inode xDSL@home
est décrits dans l'assistance Inode\footnote{Voir
\altlink{http://www6.inode.at/support/internetzugang/xdsl_home/konfiguration_ethernet_linux.html}}.

Pour l'instant, il y a peut être encore des problèmes avec le renouvellement
du bail avec l'interface dhcp (L'adresse IP de l'interface dhcp qui est attribué
automatiquement doit être renouvelée régulièrement) et la coupure de l'accès du circuit avec
imonc, cela ne fonctionne pas toujours correctement. Ici toute aide par patch
ou autre dispositif est le bien venu, les utilisateurs peuvent aussi donner
leurs avis.

Avec xdsl, il y a deux autres options pour pptp~:

\begin{description}
\config{PPTP\_CLIENT\_REORDER\_TO}{PPTP\_CLIENT\_REORDER\_TO}{PPTPCLIENTREORDERTO}

Le client-pptp, qui utilise xdsl, peut régler les paquets entre la mémoire
tampon et son PC. Normalement, le paquet attend 0,3 s avant d'être envoyer
au PC. Grâce à cette variable, on peut faire varier le temps de 0,00
(pas de mémoire tampon) à 10.00. Le temps doit toujours être indiqué
entre deux postes.

\config{PPTP\_CLIENT\_LOGLEVEL}{PPTP\_CLIENT\_LOGLEVEL}{PPTPCLIENTLOGLEVEL}

Dans cette variable on indique, le nombre d'enregistrement possible
que produit le client-pptp pour Debug. 0 (peu), 1 (défaut) et 2 (beaucoup).

\end{description}

\subsection {OPT\_POESTATUS~- Moniteur pour l'état du PPPoE avec la console-fli4l}
\configlabel{OPT\_POESTATUS}{OPTPOESTATUS}

    Thorsten Pohlmann a développé un moniteur pour l'état du PPPoE pour
    les connexions DSL.

    Si vous paramétrez la variable \var{OPT\_\-POESTATUS}='yes' vous pouvez
    consulté le l'état de la DSL sur le 3ème écran fli4l à tout moment. Vous
    passez cette écran avec la combinaison de touches ALT-F3, vous pouvez
    revenir arrière.
    1. Ecran fli4l avec ALT-F1.

