% Last Update: $Id$
% ---------------------------------------------------------------------------------
% ifrename_appendix.tex:
% ---------------------------------------------------------------------------------
% creation date: 2007/04/14 - <tobias@tb-home.de>
% last modified: 2011/03/18 - <tobias@tb-home.de>
% ---------------------------------------------------------------------------------


\section {Entwicklungshintergrund/Problemstellung}

Author: Tobias Becker\\
Date: 2009-08-02\\
To: fli4l\\
Subject: [fli4l] persistente/konsistente Netzwerkbezeichnungen - RemoteLocation\\
\\
Hallo NG,\\
\\
vielleicht kann sich jemand zu diesem Thema äußern - es geht um persistente, bzw. konsistente Netzwerkbezeichnungen unter dem akuellen fli4l Kernel, bzw.
Version.\\
\\
Bei großen Distributionen wird dies z.T. von udev geregelt, dort gibt es z.B. nach Erkennen einer neuen NIC eine entsprechende Regel (persistent-net-rules
unter /etc/udev). Man kann dann die generierte Regel modifizieren, um z.B. der NIC so einen entsprechenden Namen zuzuweisen -> aus eth0 wird dann z.B. eth0-uplink.\\
\\
Betreibt man im fli4l Netzwerkkarten die verschiedene Treiber benötigen (1 NIC pro Treiber), ist man auf der sicheren Seite, denn die NIC wird beim booten immer mit der gleichen IP bestückt. Betreibt man aber eine Karte mit mehreren physikalischen Ports (mehrere NICS pro Treiber/Quadport o. ä) ist die Bestückung mit IP-Adressen dem Zufallsprinzip unterworfen, im Idealfall ändert sich nur die komplette Reihenfolge, d.h. aus eth0 wird eth4 und umgekehrt.\\
\\
Betreibt man einen fli4l mehrere Kilometer entfernt, kann man sich nach einem Reboot, aussperren, usw.\\
\\
Bis dato konnte ich in der base-Konfiguration keine Einstellmöglichkeit finden, wo man zumindest die NIC anhand eines eindeutigen Parameters (idealerweise MAC-Adresse,
bzw. pci-bus, usw.) definiert (ich meine nicht die ethervariable bei verwendung von dhcp-providern, wäre ja auch schwachsinn) -> aus dem grund habe ich mir ein ifrename-paket aus den quellen der wireless-tools erstellt, aber vielleicht gibt es ja noch eine Möglichkeit das direkt im fli4l-base Paket zu lösen, dann muss man nach einem Update dieses nicht ständig neu anpassen.\\

[...]


\section {Verwendung mit KVM/XEN}
	-ToDo: Erstellung HowTo KVM/XEN 



