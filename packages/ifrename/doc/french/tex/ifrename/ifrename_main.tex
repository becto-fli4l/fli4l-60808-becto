% Do not remove the next line
% Synchronized to r30214
% -------------------------------------------------------------------------------
% ifrename_main.tex: opt tex_documentation
% -------------------------------------------------------------------------------
% creation date: 2007/04/14 - <tobias@tb-home.de>
% last modified: 2011/04/17 - <tobias@tb-home.de>
% -------------------------------------------------------------------------------


\section {Résumé}

\textbf{ifrename} - Renomme les interfaces réseau basées sur différents
critères statiques\\


Ifrename est un outil vous permettant d'attribuer un nom cohérent à chacune
de vos interfaces réseau.\\
\\
Par défaut, les noms des interfaces sont donnés dynamiquement, chaque interface
réseau sera assigné par le premier nom disponible (eth0, eth1 ...). L'ordre
des interfaces réseau peuvent varier. Pour les interfaces intégrées,
l'attribution des noms varies, le temps de boot du Kernel. Pour les interfaces
amovibles, des noms varies selon l'ordre de branchement par l'utilisateur.\\
\\
Ifrename permet à l'utilisateur de décider du nom de interface réseau. Ifrename
peut utiliser différent sélections pour renommer les interfaces réseau qui sont
installées sur le système, la sélection plus courante est l'adresse MAC de
l'interface.\\
\\
Ifrename doit être exécuté avant que les interfaces soient actualisées, ce qui
explique pourquoi il est surtout utile dans divers scripts (init, hotplug),
mais est rarement utilisé directement par l'utilisateur. Par défaut, ifrename
renomme toutes les interfaces présents dans le système en utilisant
les correspondances définies dans le fichier /etc/iftab\\

\vspace{15pt}
\hrule
\achtung {ATTENTION~:\\
Ifrename a été développé pour une utilisation sur un routeur Ethernet pur,
il n'a pas été testé sur une configuration avec une connexion Internet via
un modem branché au routeur.\\
\\
Pour cette raison, sur ce point l'utilisateur sera entièrement responsable
de l'utilisation de ce paquetage.\\
\\
Il est donc utile de tester d'abord le fonctionnement de cette OPT dans un
environnement virtuel, au lieu de peut-être faire un système de production
inutilisable~!}

\hrule
\vspace{15pt}

\section {Configuration}

\begin{description}
\config{OPT\_IFRENAME:}{OPT\_IFRENAME}{OPTIFRENAME}{
Avec les valeurs yes/no~- vous activez ou vous déactivez le paquetage}

\config{IFRENAME\_DEBUG:}{IFRENAME\_DEBUG}{IFRENAMEDEBUG}{
Avec les valeurs yes/no~- vous affichez ou pas des messages supplémentaire de
débogage sur la console}

\config{IFRENAME\_ETH\_N:}{IFRENAME\_ETH\_N}{IFRENAMEETHN}{
Vous indiquez ici le nombre d'interface Ethernet qui doit être fermé, avant
l'activation de ifrename}

\config{IFRENAME\_ETH\_x\_MAC:}{IFRENAME\_ETH\_x\_MAC}{IFRENAMEETHxMAC}{
Vous indiquez ici l'adresse MAC de interface Ethernet qui doit être renommée.}

\config{IFRENAME\_ETH\_x\_NAME:}{IFRENAME\_ETH\_x\_NAME}{IFRENAMEETHxNAME}{
Vous indiquez ici le nouveau nom de interface Ethernet}

\end{description}
