% Last Update: $Id$
\section{CHRONY - Benachrichtigung anderer Applikationen über Timewarps}

Stellt chrony fest, dass die Uhr sehr weit von der tatsächlichen
Uhrzeit abweicht, korrigiert chrony die Zeit in einem grossen
Schritt und führt Scripts aus, um andere Anwednungen von diesem Zeitsprung zu informieren.  Um z.B. den Imond von einem Zeitsprung zu informieren, macht
chrony folgendes:

\begin{enumerate}
\item Scripte ins Archiv aufnehmen

Chrony fügt dem Archiv zwei Files hinzu:

\begin{verbatim}
imond  yes  etc/chrony.d/timewarp.sh mode=555 flags=sh
imond  yes  etc/chrony.d/timewarp100.imond mode=555 flags=sh
\end{verbatim}

timewarp.sh führt alle Scripts im gleichen Verzeichnis aus, die dem
Namen timewarp$<$3 ziffern$>$.$<$name$>$ entsprechen.

\item Script zur Verfügung stellen

chrony nimmt folgendes Script mit ins Archiv auf:

\begin{verbatim}
# inform imond about time warp
imond-stat "adjust-time $timewarp 1"
\end{verbatim}

Damit wird der imond über den Zeitsprung informiert und kann seine
interne Zeitbasis anpassen.
\end{enumerate}
